\chapter{Conceitos Preliminares}\label{cap1}

Lógica é um \emph{sistema formal\/} para descrever e raciocinar sobre o mundo. Para isso, precisamos de uma \emph{linguagem formal\/} e de um conjunto de \emph{regras de inferência\/} ou regras de dedução, que possibilitem inferir {\emph conclusões\/}, a partir de um dado conjunto de \emph{hipóteses\/}, ou premissas. O objetivo deste primeiro capítulo é esclarecer o significado desses conceitos, que serão fundamentais ao longo de todo o texto.

Para tornar mais interessante o nosso estudo de Lógica e colocar a teoria em um contexto mais prático, vamos utilizar, ao longo do curso, o assitente de provas \textbf{Coq}.  Uma breve introdução ao \textbf{Coq} é também apresentada neste capítulo.

\section{Linguagens Formais}

Uma \emph{linguagem formal\/} \'e um conjunto (possivelmente infinito) de sentenças, definidas sobre um dado \emph{alfabeto\/}, ou conjunto finito de símbolos. Cada palavra da linguagem é simplesmente uma sequência finita de símbolos do alfabeto. As sentenças de uma linguagem formal são também chamadas de \emph{termos\/}, ou \emph{fórmulas}, da linguagem.
  
A \emph{sintaxe\/} de uma linguagem formal é especificada por um conjunto finito de regras bem definidas, que determinam a estrutura de suas sentenças. Para compreender o significado dessas sentenças, é necessário definir também a \emph{semântica\/} da linguagem, que é usualmente especificada com base nas regras de definição da sintaxe da linguagem. 

Nas seções a seguir apresentamos alguns exemplos de definição da sintaxe e semântica de algumas linguagens simples, com o objetivo de esclarecer melhor esses conceitos. 

\subsection{Sintaxe}\label{cap1:syn}

Uma linguagem finita pode ser especificada simplesmente pela enumera\c{c}\~ao de todas as suas  sentenças. Entretanto, isso não é muito prático, se a linguagem tem um grande número de sentenças, e certamente não é possível, no caso de linguagens infinitas. A sintaxe de uma linguagem formal é, portanto, usualmente definida na forma de um conjunto finito de regras, que especificam como podem ser construídas as sentenças dessa linguagem.

%O estudo de linguagens formais possui uma extensiva literatura e \'e o objeto de estudo da disciplina %Fundamentos Te\'oricos da Computa\c{c}\~ao.
%Neste cap\'itulo, iremos apenas apresentar alguns exemplos que visam ilustrar como definir conjuntos %em termos de regras.

\begin{Definition}[\textbf{Valores Booleanos}]
  O conjunto de valores lógicos, ou conjunto Booleano, contém apenas dois elementos -- $\{T,F\}$ -- que representam, respectivamente, verdade e falsidade. O nome \emph{Booleano} é uma homenagem ao matemático e lógico inglês George Boole  (1815--1864), quem primeiro desenvolveu a Lógica Booleana, que constitui a base para os circuitos dos atuais computadores digitais. O conjunto $\Bool$, dos termos que denotam valores booleanos, pode ser definido pelas seguintes regras:
  \[
      \begin{array}{l}
        T \in\Bool\\
        F \in\Bool
      \end{array}
  \]
\end{Definition}

Como o conjunto dos valores Booleanos \'e finito, sua defini\c{c}\~ao por meio de regras \'e imediata: basta enumerar uma regra que especifique a pertinência de cada um dos seus elementos. 

\begin{Definition}[\textbf{N\'umeros Naturais}]
O conjunto $\Nat$, dos termos que representam n\'umeros naturais, pode ser definido pelas seguintes regras:
\[
   \begin{array}{l}
     \zero \in \Nat\\
     \text{se $n \in \Nat$ ent\~ao $\suc\, n\in\Nat$}
   \end{array}
\]
\end{Definition}
As regras acima constituem uma definição \emph{indutiva}, ou \emph{recursiva\/}, do conjunto $\Nat$, no sentido de que alguns elementos do conjunto $\Nat$ são definidos em termos de elementos do prórpio conjunto $\Nat$. A primeira regra especifica que o termo $\zero$ pertence ao conjunto $\Nat$. A segunda regra especifica como são construídos elementos de $\Nat$, a partir de elementos do próprio conjunto $\Nat$:  se $n$ \'e um termo pertencente ao conjunto $\Nat$, ent\~ao o termo $\suc\,n$ tamb\'em pertence a $N$. Desse modo, os termos do conjunto $\Nat$ são:
\[ \{\zero,\,\suc\,\zero,\,\suc\,(\suc\,\zero),\, \suc\,(\suc\,(\suc\,\zero)),\,...\} \] 

\begin{Definition}[\textbf{Listas}]
O conjunto $\TList\, \TType$, dos termos que representam listas de elementos de um dado conjunto $\TType$, pode ser definido, indutivamente, pelas seguintes regras:
  \[
  \begin{array}{l}
    [\,] \in \TList\,\TType \\
    \text{se $t \in \TType$ e $ts \in \TList\,\TType$ ent\~ao  $t :: ts \in \TList\,\TType$}
  \end{array}
  \]
\end{Definition}
A primeira regra especifica que o termo $[\,]$ pertence ao conjunto $\TList\,\TType$. O termo $[\,]$  denota a lista vazia, ou seja, que n\~ao possui nenhum elemento. A segunda regra especifica como formar termos que representam listas não vazias: dada uma lista $ts \in \TList\,\TType$ e um elemento $t \in\TType$, o termo $t :: ts$ representa a lista cujo primeiro elemento, ou \emph{cabeça} da lista, é $t$, e cujo restante, ou \emph{cauda} da lista, é a lista $ts$. 

Note que listas são definidas de maneira independente do conjunto dos termos que constituem seus elementos -- dizemos que a definição de listas \'e \emph{polim\'orfica\/} em rela\c{c}\~ao ao conjunto de seus elementos. De acordo com a definição acima, temos, por exemplo, que as seguintes listas pertencem ao conjunto $\TList\,\Bool$:
\begin{itemize}
  \item $[\,]$, que representa a lista vazia.
  \item $T :: [\,]$, que representa a lista contendo apenas o elemento $T$, isto é, a lista cuja cabe\c{c}a \'e $T$ e cuja cauda é a lista $[\,]$. Esta lista é também representada, abreviadamente, como $[T]$.
  \item $F :: T :: [\,]$, representada abreviadamente como $[F,T]$, é a lista cuja cabe\c{c}a \'e $F$ e cuja cauda é a lista $T :: [\,]$ (ou seja, $[T]$).
\end{itemize}
Como exemplo de elemento do conjunto $\TList\,\Nat$ temos, por exemplo, a lista $zero :: (suc\,\,zero) :: [\,]$.

\begin{Definition}[\textbf{Express\~oes Aritm\'eticas}]\label{def:arithexp}
 A linguagem $\TExp$ é um conjunto dos termos que representam expressões ariméticas sobre números naturais, envolvendo apenas os operadores de adição e multiplicação, representados, respectivamente, pelos termos $\tplus$ e $\ttimes$. A linguagem $\TExp$ pode ser definida, indutivamente, pelas seguintes regras:
  \[
  \begin{tabular}{l}
    se $n\in\Nat$, ent\~ao $\tconst\,n\in\TExp$ \\
    se $e_1 \in \TExp$  e  $e_2 \in \TExp$ ent\~ao $\tplus\,e_1\,e_2\in\TExp$\\
    se $e_1 \in \TExp$  e  $e_2 \in \TExp$ ent\~ao $\ttimes\,e_1\,e_2\in\TExp$
    \end{tabular}
  \]
Informalmente, as regras acima nos dizem que todo elemento de $\Nat$ constitui uma constante em $\TExp$, e que os operadores $\tplus$ e $\ttimes$ podem ser usados para construir termos que representam expressões aritméticas mais complexas, a partir de quaisquer dois termos $e_1,e_2\in\TExp$.
\end{Definition}

\begin{Example}
 Os exemplos a seguir ilustram termos da linguagem $\TExp$, mostrando a interpretação intuitiva de cada termo. Uma semântica para essa linguagem será definida formalmente na seção ~\ref{cap1:sem}. 
  \begin{itemize}
    \item $\tconst (\suc\, \zero)$ é um termo que representa n\'umero natural $1$.
    \item $\tplus\, (\tconst (\suc\, \zero))\, (\tconst (\suc \,\zero))$ é um termo que representa a express\~ao $1 + 1$.
  \end{itemize}
 \end{Example}

As defini\c{c}\~oes anteriores especificam formalmente a sintaxe de quatro linguagens: $\Bool$, $\Nat$, $\TList\,\TType$ e $\TExp$.  Entretanto, o significado dos termos dessas linguagens apenas foi apresentado de maneira informal.  Na se\c{c}\~ao a seguir, mostramos como atribuir significado aos termos de uma linguagem formalmente.

\subsection{Sem\^antica}\label{cap1:sem}
A sintaxe de uma linguagem apenas define o conjunto de termos da linguagem, especificando a  estrutura desses termos, mas não atribui a eles qualquer significado. Para isso, é preciso também 
definir a \emph{semântica} da linguagem.

A sem\^antica de uma linguagem $L$ pode ser definida na forma de uma função, cujo domínio é $L$ e cujo contra-domínio é alguma estrutura matemática $S$ apropriada.\footnote{Assumimos que o leitor tem familiaridade com os conceitos de dom\'inio e contra-dom\'inio de uma função, que serão, entretanto, revisados no cap\'itulo~\ref{}}.  Tal função é usualmente representada na forma $\llbracket\,\rrbracket\,:\,L \rightarrow S$. Essa maneira de definir semântica é ilustrada a seguir, por exemplos que provêm definições semânticas para as linguagens $\Bool$, $\Nat$, $\TExp$ e $\TList\,\TType$, cujas sintaxes foram definidas na se\c{c}\~ao~\ref{cap1:syn}.

\begin{Definition}[\textbf{Valores Booleanos}]
Uma poss\'ivel sem\^antica para termos do conjunto $\Bool$ \'e dada pela fun\c{c}\~ao $\llbracket\,\rrbracket : \Bool \rightarrow \{0,1\}$, definida como:
\[
\begin{array}{lcl}
\llbracket T \rrbracket & = & 1\\
\llbracket F \rrbracket & = & 0\\
\end{array}
\]
\end{Definition}
Essa não é a \'unica poss\'ivel maneira de interpretar os termos de $\Bool$. 
Outra poss\'ivel interpretação é dada pela função $\llbracket\,\rrbracket : \Bool \rightarrow \mathbb{Z}$, definida a seguir:
\[
\begin{array}{lcl}
\llbracket T \rrbracket & = & \{k\in\mathbb{Z}\,|\,k\neq 0\}\\
\llbracket F \rrbracket & = & \{0\}
\end{array}
\]
Essa interpretação associa, ao termo $T$, o conjunto de todos números inteiros diferentes de $0$; e associa o valor $0$ ao termo $F$. Essa interpretação é tal como usado nas linguagens de programa\c{c}\~ao C/C++.


\begin{Definition}[\textbf{N\'umeros Naturais}]\label{def:sem:Nat}
A linguagem $\Nat$ pode ser interpretada, de modo natural, sobre o conjunto dos números naturais $\mathbb{N}$: o número $0$ é associado ao termo $\zero$, e o número $k$ ao termo que contém $k$ ocorr\^encias da constante $\suc$. Essa semântica é especificada pela função $\llbracket\,\rrbracket : \Nat \rightarrow \mathbb{N}$, a seguir. Note que essa função é definida recursivamente, sobre a estrutura da sintaxe de $\Nat$. 
\[
\begin{array}{lcl}
\llbracket \zero \rrbracket & = & 0\\
\llbracket \suc\,\,n\rrbracket & = & \llbracket n \rrbracket + 1
\end{array}
\]
\end{Definition}
Para que uma definição recursiva determine uma função $f : D \rightarrow C$, essa definição deve satisfazer aos dois seguintes crit\'erios:
\begin{enumerate}

\item \emph{Totalidade\/}: Isso significa que a definição recursiva deve associar, a cada elemento do domínio $D$, algum elemento do contradomínio $C$.  Uma maneira de garantir totalidade de uma definição recursiva da semântica de uma linguagem $L$ é especificar uma equa\c{c}\~ao de definição para cada regra de construção de termos da sintaxe de $L$. 


\item \emph{Terminação\/}: Para garantir terminação, a definição recursiva deve incluir \emph{casos base\/}, ou seja, casos em que a definição é dada diretamente, e não em termos de si própria. Além disso, em todos os demais casos -- ditos \emph{recursivos\/}, ou \emph{indutivos\/} --  a definição é especificada em termos valores que se aproximam cada vez mais dos casos base. No caso da definição recursiva de uma semântica de uma linguagem $L$, a terminação pode ser garantida se a semântica de qualquer termo de $L$ é definida apenas em função da semântica de seus subtermos. 
\end{enumerate}

A definição \ref{def:sem:Nat} determina, de fato, a função semântica $\llbracket\,\rrbracket : \Nat \rightarrow \mathbb{N}$: 1) satisfaz ao critério de totalidade, uma vez que inclui uma equação de definição da semântica para regra da sintaxe de $\Nat$; 2) satisfaz o critétio de terminação, pois inclui um caso base, que corresponde à definição do significado do termo $\zero$, e, no outro caso, define o significado de um termo com $k$ ocorrências da constante $\suc$,  em termos do significado do subtermo com $(k-1)$ ocorrências de $\suc$, ou seja, uma ocorrência a menos de $\suc$ do que no termo original.

A linguagem $\TList\,\TType$ pode também ser interpretada sobre alguma estrutura matemática apropriada. Uma possível definição da  semântica de $\TList\,\TType$ será  apresentada mais adiante. Por enquanto, vamos tomar como base nossa interpretação informal dos termos dessa linguagem como listas, e vamos mostrar, por meio de alguns exemplos, como podemos definir funções, sobre listas representadas por esses termos. Tal como no caso da função semântica, a definição de cada uma dessas funções é também baseadaa na estrutura sintática dos termos da linguagem.

\begin{Definition}[\textbf{Comprimento de uma lista}]
A função $\length : \TList\,\TType \rightarrow \mathbb{N}$, recebe como argumento uma lista, representada como um termos de $\TList\,\TType$, e retorna o número de elementos, ou seja, o comprimento, da lista dada. Essa função pode ser definida recursivamente, sobre a estrutura sintática dos termos de $\TList\,\TType$, do seguinte modo:
\[
\begin{array}{lclr}
  \length\,\,[\,] & = & 0 & (1)\\
  \length\,\,(t :: ts) & = & 1 + \length\,\, ts & (2)
\end{array}
\]
Note que a definição acima determina a função $\length : \TList\,\TType \rightarrow \mathbb{N}$, pois: 1) a definição \'e total, já que existe uma equação para cada regra da definição da sintaxe de $\TList\,\TType$; e 3) termina sempre, pois inclui um caso base (a definição de $\length$ para a lista vazia) e, no caso recursivo, a definição para uma lista da forma $(t :: ts)$ é dada apenas em termos da sublista $ts$, que possui um elemento a menos do que a lista original $(t::ts)$.
\end{Definition}

\begin{Example}
A definição acima pode ser usada para calcular o n\'umero de elementos de uma lista, conforme mostra, passo a passo, o exemplo a seguir, que ilustra o cálculo de $\length\,\,(T :: (F :: (T :: [\,])))$:
\[
\begin{array}{llcl}
& \length\,\,(T :: (F :: (T :: [\,]))) &  & \\
= &1 + \length\,\,(F :: (T :: [\,]))  & & \{\text{pela equa\c{c}\~ao }(2)\}\\
= &1 + (1 + \length\,\,(T :: [\,]))  &  & \{\text{pela equa\c{c}\~ao }(2)\}\\
= &1 + (1 + (1 + \length\,\,[\,]))  &  & \{\text{pela equa\c{c}\~ao }(2)\}\\
= &1 + (1 + (1 + 0))  &  & \{\text{pela equa\c{c}\~ao }(1)\}\\
= &3                  &   & 
\end{array}
\]
\end{Example}
Cada passo do cálculo acima simplesmente reescreve a express\~ao, de acordo com as equa\c{c}\~oes de definição de $\length$. Por exemplo, a expressão $\length\,\,(T :: [\,])$  pode ser reescrita como $1 + \length\, [\,]$, de acordo com a equa\c{c}\~ao $(2)$.

\begin{Definition}[\textbf{Concatena\c{c}\~ao de listas}]\label{def:concat:lists}
A concatenação de duas listas $xs$ e $ys$, denotada como $xs\text{ ++ }ys$, resulta em uma nova lista, 
que consiste da segunda lista justaposta ao final da primeira. Por exemplo,  $(T :: F :: [\,])\,\text{++}\,(F :: F :: [\,])$ resulta na lista $(T :: F :: F :: F :: [\,])$. A operação de concatenação de listas pode ser definida, recursivamente, do seguinte modo:
 \[
  \begin{array}{lclr}
    [\,] \text { ++ } ys & = & ys & (1)\\
    (x :: xs) \text{ ++ }  ys & = & x :: (xs\text{ ++ } ys) & (2)
  \end{array}
  \]
Note que a recursão é definida sobre a estrutura sint\'atica da lista dada como primeiro par\^ametro. A equa\c{c}\~ao $(1)$ especifica que, se o primeiro parâmetro é a lista vazia, ent\~ao o resultado da concatena\c{c}\~ao \'e a segunda lista. A equa\c{c}\~ao $(2)$ apenas se aplica se o primeiro parâmetro é uma lista não vazia, isto é, da forma $(x :: xs)$. Neste caso, o resultado \'e obtido inserindo-se o primeiro elemento $x$, da lista $(x :: xs)$, no in\'icio da lista resultante de se concatenar a cauda $xs$, da primeira lista, com a segunda lista $ys$.
\end{Definition}

\begin{Example}
O cálculo do valor de $(T :: F :: [\,])\,\text{++}\,(F :: F :: [\,])$ é ilustrado, passo a passo, a seguir:
\[
\begin{array}{llcl}
& (T :: F :: [\,]) \text{ ++ } (F :: F ::[\,])       & & \\
= & T :: ((F :: [\,]) \text{ ++ } (F :: F ::[\,])) &  & \text{pela equa\c{c}\~ao }(2)\\
= & T :: (F :: ([\,] \text{ ++ } (F :: F ::[\,]))  &  & \text{pela equa\c{c}\~ao }(2)\\
= & T :: (F :: (F :: F ::[\,]))                          &  & \text{pela equa\c{c}\~ao }(1)\\
\end{array}
\]
\end{Example}

\begin{Definition}[\textbf{Expressões aritméticas}]\label{def:sem:Exp}
Uma maneira natural interpretar a linguagem $\TExp$ é ver cada termo como uma  expressão arimética sobre números naturais, envolvendo os operadores de adição ($+$) e multiplicação ($\times$). Com essa interpretação em mente, vamos definir uma função $\eval : \TExp \rightarrow \mathbb{N}$, que mapeia cada termo de $\TExp$ no valor resultante da avaliação da expressão aritmética correspondente. Essa função pode ser definida recursivamente, sobre a estrutura sintática dos termos de $\TExp$, tal como mostrado a seguir. A definição usa da função  $\llbracket\,\rrbracket : \Nat \rightarrow \mathbb{N}$, definida anteriormente.
\[
\begin{array}{lcl}
\eval\, (\tconst\, n)  & = & \llbracket n \rrbracket \\
\eval\, (\tplus\,e_1\,e_2) & = & \eval\,e_1 + \eval\,e_2 \\
\eval\, (\ttimes\,e_1\,e_2) & = & \eval\,e_1 \times \eval\,e_2 \\
\end{array}
\]
\end{Definition}
A função $\eval$ determina o valor da expressão aritmética representada por um termo da linguagem $\TExp$, conforme ilustrado pelo exemplo a seguir:
\[ \small{
\begin{array}{ll}
&\eval\, (\ttimes\,(\tplus\,(\tconst (\suc (\suc\,\zero)))\, (\tconst (\suc\,\zero)))\, (\tconst (\suc (\suc\,\zero)))) \\
= & \eval\, (\tplus\,(\tconst (\suc (\suc\,\zero))) (\tconst (\suc\,\zero))) \times \eval\, (\tconst (\suc (\suc\, \zero)))  \\
=  & (\eval\, (\tconst (\suc (\suc\,\zero))) + \eval\,  (\tconst (\suc\,\zero))) \times \eval\, (\tconst (\suc (\suc\, \zero))) \\
=  & (\llbracket \suc (\suc\,\zero) \rrbracket + \llbracket  \suc\,\zero \rrbracket) \times \llbracket \suc (\suc\,\zero) \rrbracket \\
=  & (2 + 1) \times 2 \\
= & 6 
\end{array}}
\]

Todas as defini\c{c}\~oes apresentadas nesta se\c{c}\~ao podem ser implementadas em uma linguagem de programação de uso geral. Em linguagens funcionais, a implementação seria praticamente uma transcrição das definições apresentadas. Neste texto, vamos utilizar o assistente de provas Coq para implementação das definições apresentadas. Mais adiante, vamos também utilizar Coq para provar propriedades das funções definidas. 

 Na se\c{c}\~ao \ref{cap1:coq}, a seguir, introduzimos alguns conceitos básicos sobre Coq. Esses conceitos serão necess\'arios para que você possa descrever, em Coq, a sintaxe e a semântica de linguagens, assim como definir funções sobre a termos de uma linguagem, com base na estrutura sint\'atica de termos, tal como ilustrado pelos exemplos apresentados anteriormente.

\subsection{Exerc\'icios}

\begin{enumerate}
  \item Apresente a execu\c{c}\~ao passo a passo das seguintes express\~oes:
  \begin{enumerate}
    \item $\llbracket \suc\, (\suc\, (suc\,\,zero))\rrbracket$ 
    \item $\length\, (\zero :: \zero :: [\,])$
    \item $(\zero :: (\suc\,\, \zero) ::[\,])\text{ ++ }((\suc\,\,\zero) :: \zero :: [\,])$
  \end{enumerate}
  \item A operação de concatenação de duas listas foi definida recursivamente na Definição \ref{def:concat:lists}. Justifique que essa definição é uma fun\c{c}\~ao, mostrando que ela satisfaz  os critérios de totalidade e termina\c{c}\~ao.
  \item Dê uma defini\c{c}\~ao recursiva para a função $\textit{nsubtermos }\TExp \rightarrow \mathbb{N}$ que, dada uma express\~ao aritm\'etica, representada como um termos de $\TExp$, retorna o número de subtermos dessa expressão. Justifique que sua definição é uma função, mostrando que ela satisfaz os critérios de totalidade e termina\c{c}\~ao.
  \item O objetivo deste exerc\'icio \'e que você compreenda porque defini\c{c}\~oes recursivas que n\~ao atendem os crit\'erios de totalidade e termina\c{c}\~ao n\~ao podem ser consideradas fun\c{c}\~oes. 
        Apresente exemplos de defini\c{c}\~oes recursivas que: 1) n\~ao atendem o crit\'erio de 
        totalidade e 2) n\~ao atendem o crit\'erio de termina\c{c}\~ao. Por que essas definições n\~ao podem ser fun\c{c}\~oes?        
\end{enumerate}

\section{Introdu\c{c}\~ao ao Assistente de Provas Coq}\label{cap1:coq}

Um assistente de provas \'e uma linguagem de programa\c{c}\~ao que permite a elabora\c{c}\~ao de programas e provas matem\'aticas.
Neste trabalho, n\~ao pretendemos de forma alguma fornecer um tutorial para a utiliza\c{c}\~ao de Coq. Existem diversos bons livros
para isso, como por exemplo \cite{coqart,Pierce12,Coqrefman}. Neste texto, descreveremos apenas os recursos necess\'arios de Coq para
o aprendizado dos conceitos desta apostila sem mencionar os detalhes te\'oricos necess\'arios para uma completa compreens\~ao dessa ferramenta.

\subsection{Representando Sintaxe em Coq}

Representaremos defini\c{c}\~oes de sintaxe de termos utilizando tipos de dados em Coq. O trecho de c\'odigo Coq a seguir ilustra um
tipo de dados que denota o conjunto $\mathcal{B}$ de valores booleanos.

\begin{lstlisting}
Inductive Bool : Set :=
   | T : Bool
   | F : Bool.
\end{lstlisting}

Novos tipos de dados em Coq s\~ao declarados utilizando a palavra chave \texttt{Inductive}. Al\'em do nome do novo tipo (\texttt{Bool}), sempre
devemos especificar o universo ao qual o tipo pertence (no trecho anterior, o tipo \texttt{Bool} \'e definido como sendo do universo 
\texttt{Set}, que representam valores sobre os quais podemos efetuar c\'alculos) e os construtores de valores deste tipo. Construtores s\~ao
a terminologia utilizada em Coq para representar regras de forma\c{c}\~ao de elementos de um dado tipo. No exemplo anterior, o tipo \texttt{Bool}
possui dois construtores: \texttt{T} e \texttt{F}, que representam os \'unicos elementos deste tipo.

A defini\c{c}\~ao da sintaxe de n\'umeros naturais, conjunto $\mathcal{N}$, \'e feita de maneira similar:
\begin{lstlisting}
Inductive Nat : Set :=
   | Zero : Nat
   | Suc  : Nat -> Nat.
\end{lstlisting}
O tipo \texttt{Nat} \'e definido usando dois construtores. O primeiro, \texttt{Zero}, representa a constante $zero\in\mathcal{N}$. O construtor
\texttt{Suc} \'e o primeiro exemplo de um tipo funcional em Coq. O tipo \texttt{Nat -> Nat} denota uma fun\c{c}\~ao que recebe como par\^ametro
um valor de tipo \texttt{Nat} e retorna um valor de \texttt{Nat}. Sendo assim, o constructor \texttt{Suc} recebe como par\^ametro um valor de
tipo \texttt{Nat} e retorna outro valor de \texttt{Nat}. Devido ao tipo de \texttt{Suc}, temos que \texttt{Nat} \'e um tipo recursivo, isto \'e,
ele \'e definido em termos de si pr\'oprio. Como exemplo, qualquer valor de \texttt{Nat} que seja formado pela constante \texttt{Suc}, deve
possuir como subtermo outro valor de \texttt{Nat}. O n\'umero $3$ corresponde
 ao seguinte elemento de $\mathcal{N}$: $suc\,\,(suc\,\,(suc\,\,zero))$ e a seguinte defini\c{c}\~ao em Coq:
\begin{lstlisting}
Definition three : Nat := Suc (Suc (Suc Zero)).
\end{lstlisting}
A palavra chave \texttt{Definition} permite a defini\c{c}\~ao de fun\c{c}\~oes n\~ao recursivas sobre tipos quaisquer. No trecho de c\'odigo
anterior, \texttt{three} representa a fun\c{c}\~ao constante que sempre retorna \texttt{Suc (Suc (Suc Zero))} como resultado.
Por\'em, como \texttt{five} (apresentada a seguir) nos mostra, podemos utilizar uma defini\c{c}\~ao para criar outra, desde que isso 
n\~ao envolva recurs\~ao.
\begin{lstlisting}
Definition five : Nat := Suc (Suc three).
\end{lstlisting}

A seguinte defini\c{c}\~ao denota o conjunto de listas sobre um certo tipo \texttt{T}, ou seja, c\'odigo Coq para o conjunto 
\textit{List $\mathcal{T}$}. 
\begin{lstlisting}
Inductive List (T : Set) : Set :=
  | nil  : List T
  | cons : T -> List T -> List T.
\end{lstlisting}
No c\'odigo Coq correpondente a este cap\'itulo, definimos algumas nota\c{c}\~oes para facilitar a cria\c{c}\~ao de valores de tipo lista.
Estas nota\c{c}\~oes permitem escrever \texttt{[]} ao inv\'es de \texttt{nil} e \texttt{t :: ts} ao inv\'es de \texttt{Cons t ts}. Como
exemplo, as seguintes listas s\~ao id\^enticas, por\'em, \texttt{myList} usa as nota\c{c}\~oes e \texttt{myList'} usa os construtores
definidos para o tipo \texttt{List}.
\begin{lstlisting}
Definition myList' : List Bool := Cons T (Cons F nil).
Definition myList : List Bool := T :: F :: [].
\end{lstlisting}

Finalmente, o seguinte tipo define a  sintaxe de express\~oes aritm\'eticas, equivalente ao conjunto $\mathcal{E}$:
\begin{lstlisting}
Inductive Exp : Set :=
  | Const : Nat -> Exp
  | Plus  : Exp -> Exp -> Exp
  | Times : Exp -> Exp -> Exp.
\end{lstlisting}
Assim como fizemos para listas, definidos nota\c{c}\~oes para facilitar a escrita de valores do tipo \texttt{Exp} que s\~ao ilustradas a seguir:
\begin{lstlisting}
Definition myExp' : Exp := 
      Plus (Const Zero) (Const (Suc Zero)).
Definition myExp : Exp := 
      (Const Zero) :+ (Const (Suc Zero)).
\end{lstlisting}
As nota\c{c}\~oes definidas para express\~oes visam evitar o uso dos construtores \texttt{Plus} e \texttt{Times} substituindo-os pelos s\'imbolos
\texttt{:+} e \texttt{:*}, respectivamente.

\subsection{Representanto Sem\^antica em Coq}

Conforme apresentado na se\c{c}\~ao \ref{cap1:sem}, defini\c{c}\~oes sem\^anticas s\~ao especificadas por recurs\~ao sobre a estrutura sint\'atica
dos termos do conjunto em quest\~ao. Al\'em disso, para que defini\c{c}\~oes recursivas sejam consideradas fun\c{c}\~oes estas devem atender
os crit\'erios de totalidade e termina\c{c}\~ao.

Em Coq, toda fun\c{c}\~ao (recursiva ou n\~ao) deve obedecer os crit\'erios de totalidade e termina\c{c}\~ao. Se uma defini\c{c}\~ao 
n\~ao atender esses crit\'erios, ser\'a emitida uma mensagem de erro. Essas restri\c{c}\~oes s\~ao fundamentais para que Coq possa ser 
utilizado como um verificador de provas.

A primeira defini\c{c}\~ao sem\^antica apresentada ser\'a para o tipo
\texttt{Bool}. Como vimos na sec\c{c}\~ao \ref{cap1:sem}, a
fun\c{c}\~ao sem\^antica para o conjunto $\mathcal{B}$ \'e tal que
ao valor $T$ atribui-se $1$ e $0$ para $F$, ou seja, esta fun\c{c}\~ao
possui como dom\'inio o conjunto de valores booleanos, $\mathcal{B}$
e contra-dom\'inio o conjunto dos n\'umeros naturais, $\mathbb{N}$. 
A seguinte fun\c{c}\~ao apresenta a sem\^antica para o tipo \texttt{Bool}:
\begin{lstlisting}
Definition BoolSem (b : Bool) : nat :=
  match b with
    | T => 1
    | F => 0
  end.
\end{lstlisting}
O c\'odigo anterior mostra a fun\c{c}\~ao \texttt{BoolSem} que recebe
como par\^ametro um valor de tipo \texttt{Bool}, de nome \texttt{b}, e
retorna um n\'umero natural, um valor de tipo \texttt{nat}. Como a
sem\^antica de valores booleanos depende de sua sintaxe ( $1$ para $T$
e $0$ para $F$) devemos usar casamento de padr\~ao para analisar a
estrutura sint\'atica do par\^ametro \texttt{b} para, ent\~ao,
retornar o resultado apropriado. O casamento de padr\~ao, em Coq, \'e
feito usando a constru\c{c}\~ao \texttt{match}, que permite definir um
conjunto de equa\c{c}\~oes, uma para cada possibilidade sint\'atica do
valor analisado. Note que, Coq exige que todo \texttt{match}
obede\c{c}a o crit\'erio de totalidade, isto é, as equa\c{c}\~oes
devem cobrir todas as possibilidades do tipo do valor analisado.

Para a fun\c{c}\~ao \texttt{BoolSem} a totalidade \'e facilmente
atingida estabelecendo uma equa\c{c}\~ao para cada um dos valores de
\texttt{Bool}.

O pr\'oximo exemplo define a sem\^antica do tipo \texttt{Nat}. Como
este tipo \'e recursivo, sua fun\c{c}\~ao sem\^antica tamb\'em ser\'a.
Em Coq, definimos fun\c{c}\~oes recursivas utilizando o comando
\texttt{Fixpoint}.
 
\begin{lstlisting}
Fixpoint NatSem (n : Nat) : nat :=
  match n with
    | Zero  => 0
    | Suc n' => 1 + NatSem n'
  end.
\end{lstlisting}
Como o comando \texttt{Fixpoint} permite defini\c{c}\~oes recursivas,
Coq exige que estas satisfa\c{c}am os crit\'erios de termina\c{c}\~ao
e totalidade. Note que o \texttt{match} que define a fun\c{c}\~ao
\texttt{NatSem} \'e total (pois, todo valor de \texttt{Nat} ou \'e
igual a \texttt{Zero} ou igual \texttt{Suc n}, para algum \texttt{n}
de tipo \texttt{Nat}) e termina, pois a chamada recursiva de
\texttt{NatSem} diminui o n\'umero de constantes \texttt{Suc}
presentes no termo.

Na equa\c{c}\~ao para a constante \texttt{Suc} em \texttt{NatSem},
temos a presen\c{a} de uma vari\'avel no lado esquerdo. Durante o
casamento de padr\~ao, vari\'aveis s\~ao associadas a valores. Como um
exemplo, considere executar \texttt{NatSem (Suc (Suc Zero))}. Neste
caso, temos que a equa\c{c}\~ao:
\begin{lstlisting}
Suc n' => 1 + NatSem n'
\end{lstlisting}
ser\'a executada. Durante o processo de casamento de padr\~ao, temos
que o valor de n' ser\'a \texttt{Suc Zero}, j\'a que o termo analisado
pelo \texttt{match} foi \texttt{Suc (Suc Zero)}. Note que a estrutura
do padr\~ao da equa\c{c}\~ao (\texttt{Suc n'})  faz com que o subtermo
\texttt{Suc Zero} de \texttt{Suc (Suc Zero)} seja atribu\'ido \`a
vari\'avel \texttt{n'}.

As fun\c{c}\~oes \texttt{length} e \texttt{app} definem as
opera\c{c}\~oes de c\'alculo do n\'umero de elementos e
concatena\c{c}\~ao sobre listas. O c\'odigo Coq correspondente a ambas
\'e apresentado a seguir:

\begin{lstlisting}
Fixpoint length {T} (l : List T) : nat :=
  match l with
    | []      => 0
    | t :: ts => 1 + length ts
  end.

Fixpoint app {T} (l l' : List T) : List T :=
  match l with
    | []      => l'
    | t :: ts => t :: (app ts l')
  end.
\end{lstlisting}
A defini\c{c}\~ao destas fun\c{c}\~oes \'e simplesmente uma
tradu\c{c}\~ao do que foi apresentado na se\c{c}\~ao \ref{cap1:sem}
para Coq.

Com isso, terminamos nossa introdu\c{c}\~ao \`a utiliza\c{c}\~ao de
Coq para definir sintaxe e sem\^antica de linguagens formais. Nos
capítulos seguintes, maiores detalhes da ferramenta ser\~ao
apresentados de forma a podermos utiliz\'a-la como um recurso
did\'atico adicional para o aprendizado do conte\'udo de Matem\'atica Discreta.

\subsection{Exerc\'icios}

\begin{enumerate}
    \item Apresente uma fun\c{c}\~ao em Coq que calcula o valor de uma express\~ao aritm\'etica (defini\c{c}\~ao \ref{def:arithexp}).
        Sua solu\c{c}\~ao deve possuir como par\^ametro um valor de
        tipo \texttt{Exp}. O tipo de retorno de sua fun\c{c}\~ao deve
        ser \texttt{nat}, o tipo correspondente a n\'umeros naturais
        em Coq. 
    \item Defina fun\c{c}\~oes que, a partir de uma lista fornecida
      como par\^ametro, retornem a cabe\c{c}a e a cauda desta.
\end{enumerate}

\section{Notas Bibliogr\'aficas}

Conceitos de sintaxe e sem\^antica s\~ao pervasivos na Ci\^encia da
Computa\c{c}\~ao. Estes s\~ao presentes em livros de sem\^antica
formal \cite{Winskel93}, teoria de linguagens \cite{Hopcroft06} e
constru\c{c}\~ao de compiladores \cite{Aho86}.

Coq \'e, atualmente, o mais maduro assistente de provas baseado em
teoria de tipos e \'e empregado com sucesso em diversos projetos
industriais e cient\'ificos. Diversos livros fornecem
uma introdu\c{c}\~ao \`a utiliza\c{c}\~ao dessa ferramenta al\'em dos
prop\'ositos deste texto. O leitor interessado pode aprofundar seus
conhecimentos usando \cite{coqart,Pierce12,Coqrefman}.
