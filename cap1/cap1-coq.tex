\section{Introdu\c{c}\~ao ao Assistente de Provas Coq}\label{cap1:coq}

Um assistente de provas \'e uma ferramenta de software para auxílio no desenvolvimento de provas de teoremas, provas de propriedades de programas ou de circuitos integrados, etc. Essa ferramenta inclui uma linguagem de programa\c{c}\~ao, que possibilita descrever  algoritmos e provas matem\'aticas. Nesta seção, introduzimos alguns conceitos relacionados à construção de programas e provas em Coq, com o objetivo de que você possa utilizar essa ferramenta para a implementação de provas apresentadas ao longo deste texto. Para uma descrição detalhada sobre o ambiente de desenvolvimento de Coq (CoqIDE), ou sobre sua linguagem de programação (Gallina), consulte a documentação disponível na página web \url{coq.inria.fr}, ou a bibliografia sugerida no final deste capítulo.

\subsection{Representando Sintaxe em Coq}

A sintaxe de termos de uma linguagem pode ser representada, em Coq, por um tipo de dados da linguagem. O trecho de c\'odigo Coq a seguir ilustra um tipo de dados que denota o conjunto dos valores Booleanos, tal como a linguagem $\mathcal{B}$, definida anteriormente.

\begin{lstlisting}
Inductive Bool : Set :=
   | T : Bool
   | F : Bool.
\end{lstlisting}

Uma declaração de tipo de dados, em Coq, começa com a palavra chave \texttt{Inductive}, seguida do nome do tipo de dados e do universo ao qual o tipo de dados pertence, separados pelo símbolo \texttt{:}. Esse cabeçalho da declaração termina com o símbolo de definição \texttt{:=}, e é seguido da definição dos contrutores de valores do tipo de dados que está sendo definido. No exemplo acima, o nome do tipo de dados é \texttt{Bool}. O universo a que ele pertence é \texttt{Set}, que representa valores sobre os quais se podem efetuar c\'alculos. Os construtores de valores do tipo \texttt{Bool} são \texttt{T} e \texttt{F}, que representam os únicos elementos desse tipo. Cada construtor do tipo de dado corresponde a uma regra de definição da sintaxe de elementos de um dado tipo. Note que, para cada construtor, é também especificado o seu tipo.

A sintaxe do conjunto de n\'umeros naturais $\Nat$ \'e feita de maneira similar:
\begin{lstlisting}
Inductive Nat : Set :=
   | Zero : Nat
   | Suc  : Nat -> Nat.
\end{lstlisting}
O tipo \texttt{Nat} tem dois construtores. O primeiro, \texttt{Zero}, representa a constante $\zero\in \Nat$. O construtor \texttt{Suc} tem tipo \texttt{Nat -> Nat}, o que significa que  \texttt{Suc} é uma função, que recebe como par\^ametro um valor de tipo \texttt{Nat} e retorna um valor de tipo \texttt{Nat}.

O tipo de \texttt{Suc} indica que \texttt{Nat} \'e um tipo de dado recursivo, isto \'e, ele \'e definido em termos de si pr\'oprio. Observe que qualquer valor de \texttt{Nat}, que seja formado pela constante \texttt{Suc}, possui como subtermo outro valor de tipo \texttt{Nat}. Por exemplo, o n\'umero $3$ corresponde ao seguinte elemento de $\Nat$: $\suc\,\,(\suc\,\,(\suc\,\,\zero))$ e à seguinte defini\c{c}\~ao em Coq:
\begin{lstlisting}
Definition three : Nat := Suc (Suc (Suc Zero)).
\end{lstlisting}
A palavra chave \texttt{Definition} introduz uma defini\c{c}\~ao de função n\~ao recursiva, sobre tipos quaisquer. Neste trecho de c\'odigo, \texttt{three} representa a fun\c{c}\~ao constante que sempre retorna \texttt{Suc (Suc (Suc Zero))} como resultado.

Uma função pode ser usada para definir outra função, tal como ilustra o exemplo a seguir, que define \texttt{five}, usando \texttt{three}:
\begin{lstlisting}
Definition five : Nat := Suc (Suc three).
\end{lstlisting}

O conjunto de listas $\TList\,\TType$ pode ser definido, em Coq, como no exemplo a seguir:
\begin{lstlisting}
Inductive List (T : Set) : Set :=
  | nil  : List T
  | cons : T -> List T -> List T.
\end{lstlisting}

No código Coq correspondente a este capítulo, definimos uma nota\c{c}\~ao mais conveniente para valores de tipo lista, de maneira que se possa escrever \texttt{[]}, ao inv\'es de \texttt{nil}, e \texttt{t :: ts}, ao inv\'es de \texttt{Cons t ts}. Desse modo, as \texttt{myList} e \texttt{myList'}, a seguir, s\~ao id\^enticas, mas \texttt{myList} usa a nova notação e \texttt{myList'} usa os construtores definidos para o tipo \texttt{List}.
\begin{lstlisting}
Definition myList' : List Bool := Cons T (Cons F nil).
Definition myList  : List Bool := T :: F :: [].
\end{lstlisting}

Finalmente, a  sintaxe de termos do conjunto $\TExp$, que representam express\~oes aritm\'eticas, pode ser definida, em Coq, do seguinte modo:
\begin{lstlisting}
Inductive Exp : Set :=
  | Const : Nat -> Exp
  | Plus  : Exp -> Exp -> Exp
  | Times : Exp -> Exp -> Exp.
\end{lstlisting}


Assim como para o caso de listas, definiremos uma notação mais conveniente para que se possa escrever valores do tipo \texttt{Exp}, utilizando os operadores \texttt{(:+)} e \texttt{(:*)}, em lugar dos construtores \texttt{Plus} e \texttt{Times}, respectivamente. Desse modo, as expressões \texttt{myExp'} e \texttt{myExp}, definidas a seguir, são idênticas, exceto que \texttt{myExp} usa a nova notação e \texttt{myExp'} usa os construtores originais.
\begin{lstlisting}
Definition myExp' : Exp :=
      Plus (Const Zero) (Const (Suc Zero)).
Definition myExp : Exp :=
      (Const Zero) :+ (Const (Suc Zero)).
\end{lstlisting}

\subsection{Representando Sem\^antica em Coq}

Conforme vimos na se\c{c}\~ao \ref{cap1:sem}, a sem\^antica de uma linguagem pode ser definida indutivamente, sobre a estrutura sint\'atica dos termos dessa linguagem. Além disso, para que essa defini\c{c}\~ao recursiva determine uma fun\c{c}\~ao semântica, ela deve satisfazer os crit\'erios de totalidade e termina\c{c}\~ao.

Em Coq, toda fun\c{c}\~ao (recursiva ou n\~ao) deve satisfazer os crit\'erios de totalidade e termina\c{c}\~ao. Essas restri\c{c}\~oes s\~ao fundamentais para que Coq possa ser
utilizado como um verificador de provas. Se uma defini\c{c}\~ao n\~ao satisfaz  esses crit\'erios, é emitida uma mensagem de erro.

Como primeiro exemplo, vamos escrever, em Coq, uma definição de semântica para o tipo
\texttt{Bool}. Como vimos na sec\c{c}\~ao \ref{cap1:sem}, podemos definir a sem\^antica para o conjunto $\Bool$ como uma função $\llbracket\,\rrbracket : \Bool \rightarrow \mathbb{N}$, que associa a $T$ o valor $1$, e associa a $F$ o valor $0$.
Essa função semântica pode ser definida, em Coq, sobre o tipo \texttt{Bool}, do seguinte modo:
\begin{lstlisting}
Definition BoolSem (b : Bool) : nat :=
  match b with
    | T => 1
    | F => 0
  end.
\end{lstlisting}
Esse trecho de c\'odigo define a fun\c{c}\~ao \texttt{BoolSem}, que recebe
como par\^ametro um valor \texttt{b}, de tipo \texttt{Bool}, e
retorna um n\'umero natural, isto é, um valor de tipo \texttt{nat}. Como a
sem\^antica é definida com base na sintaxe de valores do tipo \texttt{Bool}, é usado \emph{casamento de padr\~ao\/} para analisar a estrutura sint\'atica do par\^ametro \texttt{b}.  e ent\~ao retornar o resultado apropriado. Em Coq, o casamento de padr\~ao \'e feito usando a constru\c{c}\~ao \texttt{match}, que permite definir um conjunto de equa\c{c}\~oes, uma para cada possibilidade sint\'atica do valor analisado. Note que Coq exige que todo \texttt{match} obede\c{c}a o crit\'erio de totalidade, isto é, as equa\c{c}\~oes devem cobrir todas as possibilidades de valores do tipo analisado.

Para a fun\c{c}\~ao \texttt{BoolSem} a totalidade \'e facilmente obtida estabelecendo-se uma equa\c{c}\~ao para cada um dos valores do tipo \texttt{Bool}.

O exemplo a seguir define a sem\^antica do tipo \texttt{Nat}. Como
este tipo \'e recursivo, sua fun\c{c}\~ao sem\^antica é tamb\'em definida recursivamente.
Em Coq, definimos fun\c{c}\~oes recursivas utilizando o comando \texttt{Fixpoint}.

\begin{lstlisting}
Fixpoint NatSem (n : Nat) : nat :=
  match n with
    | Zero  => 0
    | Suc n' => 1 + NatSem n'
  end.
\end{lstlisting}

A função semântica \texttt{NatSem} tem como parâmetro um valor \texttt{n}, do tipo \texttt{Nat}. Conforme especificado pelo casamento de padrão usado na definição de \texttt{NatSem}, esse valor pode ser de um dos dois seguintes tipos: \texttt{Zero}, ou \texttt{Suc n'}, para algum valor \texttt{n'}, do tipo \texttt{Nat}. Uma variável usada em um casamento de padrão, tal como \texttt{n'}, no padrão \texttt{Suc n'}, casa com qualquer valor. Por exemplo, na avaliação da expressão \texttt{NatSem (Suc (Suc Zero))}, o argumento \texttt{(Suc (Suc Zero))} casa com o padrão \texttt{Suc n'}, sendo o valor \texttt{(Suc Zero)} associado à variável \texttt{n'}. Desse modo, na chamada recursiva, o argumento de \texttt{NatSem} será \texttt{(Suc Zero)}.

Note que a definição recursiva de \texttt{NatSem} \'e total, pois todo valor de tipo \texttt{Nat}, ou \'e igual a \texttt{Zero}, ou é igual a \texttt{Suc n}, para algum \texttt{n} de tipo \texttt{Nat}. Além disso, a recursão sempre termina, pois, na chamada recursiva, o  n\'umero de constantes \texttt{Suc} presentes no termo passado como argumento é decrescido de 1, eventualmente reduzindo o termo ao caso base \texttt{Zero}.

Os exemplos a seguir ilustram a implementação, em Coq, de funções definidas sobre listas. A função \texttt{length} recebe como argumento uma lista e retorna o número de elementos dessa lista. A função \texttt{app} recebe como argumentos duas listas e retorna a lista obtida concatenando-se a segunda lista depois da primeira.

\begin{lstlisting}
Fixpoint length {T} (l : List T) : nat :=
  match l with
    | []      => 0
    | t :: ts => 1 + length ts
  end.

Fixpoint app {T} (l l' : List T) : List T :=
  match l with
    | []      => l'
    | t :: ts => t :: (app ts l')
  end.
\end{lstlisting}

Observe que a defini\c{c}\~ao de cada uma dessas fun\c{c}\~oes é uma simples
tradu\c{c}\~ao, para Coq, da definição apresentada na se\c{c}\~ao \ref{cap1:sem}.

Como último exemplo, vamos definir a função \texttt{head}, que recebe como parâmetro uma lista e retorna a cabeça da lista. No caso em que a lista passada como argumento é vazia, a cabeça da lista não é definida. Para que a definição satisfaça o critério de totalidade, vamos definir a função \texttt{head} com um primeiro parâmetro adicional -- \texttt{default} -- que especifica o valor a ser retornado pela função no caso em que a lista passada como argumento é vazia. A definição da função \texttt{head}, em Coq, é apresentada a seguir.

\begin{lstlisting}
Definition head {T} (default : T) (l : List T) : T :=
  match l with
  | nil => default
  | h :: t => h
  end.
\end{lstlisting}

Essa breve introdução a Coq ilustra como definir sintaxe e semântica de linguagens neste sistema. Nos próximos capítulos vamos abordar um pouco mais sobre os recursos dessa ferramenta, de maneira que ela possa ser usada como um recurso did\'atico adicional para o aprendizado do conte\'udo de Matem\'atica Discreta.

\subsection{Exerc\'icios}

\begin{enumerate}
    \item Escreva, em Coq, uma definição da fun\c{c}\~ao \texttt{evalExp}, que calcula o valor de uma express\~ao aritm\'etica. Essa função tem como parâmetro um valor do tipo \texttt{Exp} e retorna como resultado um valor do tipo \texttt{nat}, o tipo correspondente a n\'umeros naturais, em Coq.
    \item Defina, em Coq, a função \texttt{tail}, que recebe como parâmetro uma lista, e retorna a sua cauda. Note que, se a lista passada como argumento for vazia, isto é \texttt{nil} (ou \texttt{[\,]}), a cauda é a própria lista vazia.
        \item Defina, em Coq, a função \texttt{last}, que recebe como parâmetro uma lista, e retorna o último elemento da lista. Assim como no caso da função \texttt{head}, essa função não seria definida no caso em que a lista é vazia. Esse problema deve ser evitado tal como foi feito no caso da definição da função \texttt{head}, ou seja, incluindo na definição um parâmetro adicional, que especifica o valor a ser retornado pela função no caso em que a lista passada como argumento for vazia.

\end{enumerate}
