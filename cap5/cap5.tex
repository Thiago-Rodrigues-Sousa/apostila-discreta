\chapter{Teoria de Conjuntos}\label{cap5}

\epigraph{Ninguém deveria nos expulsar do paraíso que Cantor criou.}{David Hilbert, Matemático Alemão sobre a Teoria de
  Conjuntos criada por Georg Cantor.}

\section{Motivação}

De maneira simplista, pode-se dizer que o alicerce fundamental da
matemática é a teoria de conjuntos. Isto se torna mais e mais evidente
a medida que você avança por cursos mais avançados de matemática, já
que a teoria de conjuntos é uma linguagem projetada para descrever e
explicar todos os tipos de estruturas matemáticas.

Em se tratando de computação, a teoria de conjuntos possui um papel
importante no projeto de estruturas de dados e bancos de
dados. Primeiramente, diversas estruturas
eficientes são implementações de um tipo abstrato de dados que
define operações sobre conjuntos. Por sua vez, toda a teoria de bancos
de dados relacionais é baseada em operações básicas sobre conjuntos.

O objetivo deste capítulo é apresentar a teoria de conjuntos e como
esta pode ser utilizada para descrever propriedades de objetos
matemáticos.

\section{Introdução aos Conjuntos}

Não apresentaremos uma definição formal do que é um conjunto. Isto se
deve ao fato de que  a teoria de conjuntos foi concebida com o intuito
de ser a fundamentação teórica de toda a matemática. Isto é, em
princípio, todos os objetos matemáticos são definidos em termos de
conjuntos.

Conjuntos nada mais são que uma coleção de objetos denominados
\emph{elementos}. Porém, existem algumas restrições para considerarmos
uma coleção de objetos um conjunto. A primeira diz respeito a
\emph{ordem}. Em um conjunto a ordem dos elementos é irrelevante. A
segunda é sobre a \emph{multiplicidade}. Esta especifica que em um
conjunto qualquer há somente uma ocorrência de um certo valor, isto é,
não é permitido que um elemento apareça mais de uma vez em um mesmo
conjunto.

Uma vez que elementos podem ocorrer uma única vez em um conjunto,
podemos dizer que a operação de determinar se um elemento está ou não
em um conjunto possui um valor lógico (isto é, verdadeiro ou
falso). Se $A$ é um conjunto e $x$ um elemento, representamos por $x
\in A$ o fato de $x$ ser um elemento do conjunto $A$. Representamos que $x$ não
é um elemento de $A$ por $x \not\in A$. Note que a seguinte
equivalência é verdadeira: $x\not\in A \equiv \neg (x \in A)$.

Denominamos por \emph{cardinalidade} ou tamanho o número de elementos
de um conjunto. Se $A$ é um conjunto, representamos por $|A|$ o número
de elementos de $A$.

Existe um único conjunto $A$ tal que $|A| = 0$. Este é conhecido como
conjunto vazio e é representado como $\{\}$ ou $\emptyset$.

Adotaremos, como convenção, que conjuntos
serão sempre representados por letras maiúsculas e elementos por
letras minúsculas.


\section{Descrevendo Conjuntos}

Existem diversas maneiras para se descrever conjuntos. Apresentaremos,
de maneira suscinta três maneiras: enumeração, \emph{set
  comprehension}\footnote{Infelizmente, não conheço uma tradução para
  este termo. Por isso, mantive o nome original.} e por recursão.

\subsection{Enumeração}

Definimos um conjunto por enumeração simplesmente listando seus
elementos. Este é um método conveniente para conjuntos finitos que
possuam poucos elementos.

O exemplo a seguir mostra alguns conjuntos
definidos por enumeração.

\begin{Example}
Abaixo apresentamos alguns conjuntos definidos por enumeração:
\[
\begin{array}{lcl}
   V & = & \{a,e,i,o,u\} \\
   P & = & \{\text{arara, pelicano, pardal}\}\\
   X & = & \{2,4,6,8\} \\
   J & = & \{\}\\
  L & = & \{1, \{1\}\} \\
\end{array}
\]
A cardinalidade de cada um deles é:

\[
\begin{array}{lcl}
   |V| & = & 5 \\
   |P| & = & 3\\
   |X| & = & 4 \\
   |J| & = & 0\\
  |L| & = &2 \\
\end{array}
\]
Note que as seguintes proposições sobre pertinência nestes conjuntos
são verdadeiras:
\[
\begin{array}{l}
a \in V \\
\text{arara} \in P \\
2 \in X\\
\{1\} \in L\\
1 \in L\\
\end{array}
\]
Por sua vez, as seguintes proposições são falsas considerando os
conjuntos anteriores:
\[
\begin{array}{l}
p \in V \\
\text{vaca}\in P \\
7 \in X \\
2 \in L \\
\{1,\{1\}\} \in \{1,\{1\}\}
\end{array}
\]
\end{Example}

\subsection{Set Comprehension}

A notação de set comprehension\footnote{Manteremos o nome sem tradução
    por não conhecer um termo em língua portuguesa para este tipo de
    notação matemática.} permite-nos especificar um conjunto em
termos de uma propriedade que descreve quais são os elementos deste.
De maneira simples, temos que um set comprehension é
representado da seguinte maneira:
\[\{x  \in X\,|\,p(x)\}\]
em que $x$ é uma variável  (ou uma expressão), $X$ um conjunto e $p(x)$ é uma fórmula
da lógica de predicados. Esta maneira de descrever conjuntos é útil
para descrever conjuntos com muitos elementos ou infinitos.

É importante notar que ocorrências da variável $x$ em $p(x)$ no set
comprehension $\{x\in X\,|\,p(x)\}$ são consideradas ligadas por
Podemos caracterizar a pertinência a um conjunto definido usando set
comprehension utilizando a seguinte equivalência:

\[y\in\{x\in X \,|\, p(x)\} \equiv y\in X \land p(y)\]

O leitor atento deve ter percebido que a equivalência anterior nada
mais é que a aplicação da substituição $[x\mapsto y]$ a fórmula
especificada no set comprehension.

\begin{Example}
Considere a tarefa de representar os conjuntos de todos os números naturais
pares e de todos os números naturais múltiplos de 3. Poderíamos
representar estes conjuntos da seguinte maneira:
\[
\begin{array}{lcr}
P & = &\{0,2,4,6,...\}\\
T & = & \{0,3,6,9,...\}\\
\end{array}
\]
Apesar da estrutura parecer óbvia, o uso de ``...'' deve ser evitado
por este permitir ambiguidades na interpretação de um conjunto. Sem
saber que o conjunto $T$ representa os números naturais múltiplos de
3, como saber se $173$ pertence ou não a este conjunto?

Para evitar este tipo de ambiguidades, podemos utilizar set
comprehensions para definir conjuntos infinitos de maneira precisa. Os
conjuntos anteriores podem ser representados da seguinte maneira:
\[
\begin{array}{lcr}
P & = &\{x\in\mathbb{N}\,|\,\exists y. y\in \mathbb{N}\land x =
2y\}\\
T & = & \{x\in\mathbb{N}\,|\ \exists y. y\in\mathbb{N}\land x =
3y\}\\
\end{array}
\]

Podemos representar os fatos de que $a \in P$ e $b \in T$ como as
seguintes fórmulas:
\[
\begin{array}{l}
a \in \mathbb{N} \land \exists y. y\in\mathbb{N} \land a = 2y\\
b \in \mathbb{N} \land \exists y. y\in\mathbb{N} \land b = 3y\\
\end{array}
\]

Note que como utilizarmos uma fórmula da lógica de predicados para
descrever elementos de um conjunto não há margem para interpretações
ambíguas.
\end{Example}

Devemos sempre especificar qual o conjunto de
``origem''\footnote{Considera-se como conjunto ``origem'' do set
  comprehension $\{x\in X\,|\,p(x)\}$ o conjunto $X$.} dos elementos
para os quais estamos definindo um conjunto utilizando set comprehension.
A não especificação do conjunto de origem permite a formulação de
paradoxos, como o conhecido paradoxo de Russell, descoberto por
Bertrand Russell no início do século XX.

\subsubsection{O Paradoxo de Russell}

Antes de apresentar o paradoxo de Russell formalmente, é útil
analisá-lo em um contexto mais simples, porém, equivalente.

\begin{Example}
Considere o seguinte problema:

\begin{quote}
``Considere uma cidade em que existe apenas um barbeiro e que este faz a
barba de todos que não fazem a própria barba. O barbeiro faz sua
própria barba?''
\end{quote}

Após refletir uma pouco sobre esta sentença, percebemos que esta é um
paradoxo, pois:
\begin{itemize}
  \item Se o barbeiro não faz a própria barba, ele deveria fazê-la, já
    que ele faz a barba apenas de quem não faz a própria barba.
  \item Porém se ele faz a própria barba, pela definição, ele não
    deveria fazê-la.
\end{itemize}

Ou seja, a sentença sobre o barbeiro desta cidade é um paradoxo.
\end{Example}

Russell percebeu que a definição usando set comprehension poderia
gerar um paradoxo similar ao apresentado no exemplo
anterior. A demonstração deste paradoxo é apresentada a seguir.

Seja $\mathcal{S}$ o conjunto de todos os conjuntos que não
são elementos de si próprios, isto é:
\[\mathcal{S} =\{X\,|\,X\not\in X\}\]
Evidentemente, temos que $\mathcal{S} \in \mathcal{S}$ ou $\mathcal{S}
\not\in \mathcal{S}$. Considere os seguintes casos:
\begin{itemize}
  \item Caso $\mathcal{S} \in \mathcal{S}$: Se $\mathcal{S} \in
    \mathcal{S}$, pela definição de $\mathcal{S}$, temos que
    $\mathcal{S} \not\in \mathcal{S}$, o que constitui uma contradição.
  \item Caso $\mathcal{S} \not\in \mathcal{S}$: Logo, pela definição
    de $\mathcal{S}$, temos que $\mathcal{S} \in \mathcal{S}$, o que
    constitui uma contradição.
\end{itemize}
Como ambos os casos cobrem todas as possibilidades, temos que
$\mathcal{S}\in\mathcal{S}$ não pode ser uma proposição lógica, já que
esta não pode ser determinada como verdadeira ou falsa.

\subsection{Conjuntos Definidos Recursivamente}

Conjuntos definidos por recursão são muito utilizados em computação
para a definição de estruturas de dados e algoritmos. Nesta seção,
veremos como definir conjuntos recursivamente.

Assim como toda definição recursiva, conjuntos
indutivos\footnote{Conjuntos definidos recursivamente são também conhecidos
  como conjuntos indutivos.} devem possuir casos base e passos
recursivos. Porém, apenas estes elementos são suficientes para
caracterizar uma definição de um conjunto. Adicionalmente, devemos
possuir uma regra, denominada \emph{regra de
  fechamento}\footnote{Normalmente, estas regras são denominadas como
  \emph{extremal rule}, em textos sobre teoria de conjuntos.} que
especifica que todos os elementos do conjunto definido são formados a
partir do(s) caso(s) base e de um número finito de usos do(s)
passo(s) recursivo(s).

Resumindo, para definir um conjunto recursivamente devemos especificar
três partes: casos base, passos recursivos e regra de fechamento:
\begin{itemize}
  \item Casos base consistem de afirmativas simples, como $1\in S$.
   \item Passos recursivos consistem de afirmativas envolvendo
     implicações e quantificadores universais, como \[\forall x. x\in
     S \to x + 1 \in S\]
   \item Regras de fechamento especificam que todo elemento do
     conjunto pode ser obtido a partir de um número finito de
     utilização das regras anteriores.
\end{itemize}

A seguir apresentaremos algumas definições de conjuntos definidos
recursivamente.

\begin{Example}
Como um primeiro exemplo de definição de um conjunto recursivo, vamos
considerar o conjunto dos números naturais,
$\mathbb{N}$. O leitor deve lembrar da definição do conjunto
$\mathcal{N}$ do capítulo \ref{cap1} em que apresentamos a sintaxe de
termos que representam números naturais utilizando a operação de
sucessor e uma constante para representar o número $0$.

Utilizaremos a
mesma idéia para a definição recursiva de $\mathbb{N}$:
\begin{itemize}
  \item Caso base: $0\in\mathbb{N}$.
  \item Passo recursivo: $\forall n. n\in\mathbb{N}\to n + 1 \in
    \mathbb{N}$
  \item Regra de fechamento: Todo $n\in\mathbb{N}$ pode se obtido por
    um número finito de aplicações das regras anteriores.
\end{itemize}
\end{Example}

\begin{Example}
Considere  $P=\{x\in\mathbb{N}\,|\,\exists
y.y\in\mathbb{N}\land x = 2y\}$, que obviamente representa o conjunto
de todos os números naturais pares. Podemos definí-lo recursivamente
se percebermos que este é formado por um caso base ($0\in P$) e que se
$n\in P$ então $n + 2\in P$, o que é definido formalmente a seguir:
\begin{itemize}
  \item Caso base: $0\in P$
  \item Passo recursivo: $\forall n. n\in P \to n + 2 \in P$
  \item Regra de fechamento: Todo $n\in P$ pode ser obtido por um
    número finito de aplicações de regras anteriores.
\end{itemize}
\end{Example}

A seguir apresentamos uma exemplo de definição maior: a
definição recursiva do conjunto dos números inteiros.

\begin{Example}
O conjunto dos números naturais possui uma definição direta porquê
este é infinito em apenas ``uma direção'', o que não acontece com o
conjunto $\mathbb{Z}$, que possui infinitos elementos positivos e negativos.
Então como definir este conjunto recursivamente?

Uma primeira tentativa seria algo similar a dizer que $0\in\mathbb{Z}$
e que se $n \in \mathbb{Z}$ então $n + 1$ e $n - 1$ também pertencem a
$\mathbb{Z}$. Mais formalmente:
\begin{itemize}
  \item Caso base: $0\in\mathbb{Z}$.
  \item Passo recursivo: $\forall x. x\in\mathbb{Z} \to x+1 \in
    \mathbb{Z} \land x - 1 \in\mathbb{Z}$
\end{itemize}

Apesar de correta, a definição anterior possui um inconveniente:
existe mais de uma maneira de deduzirmos que um certo número pertence
ou não a $Z$. Como exemplo, considere a seguinte derivação, de que
$-2\in\mathbb{Z}$:

\[
\infer[\andED]{-2\in\mathbb{Z}}
                    {
                      \infer[\impE]{0\in\mathbb{Z}\land -2
                        \in\mathbb{Z}}
                               {
                                 \infer[\forallE]{-1\in\mathbb{Z} \to
                                   0 \in\mathbb{Z} \land -2
                                   \in\mathbb{Z}}
                                   {\infer[\Id]{\forall n.n\in\mathbb{Z}\to n + 1
                                     \in\mathbb{Z}\land n - 1 \in
                                     \mathbb{Z}}{}}
                                    &
                                 \infer[\andED]{-1\in\mathbb{Z}}
                                           {
                                             \infer[\impE]{1\in\mathbb{Z}\land
                                             1\in\mathbb{Z}}
                                               {
                                                 \infer[\Id]{0\in\mathbb{Z}}{}
                                                 &
                                                 \infer[\forallE]{0\in\mathbb{Z}\to
                                                 1 \in\mathbb{Z}\land
                                                 -1\in\mathbb{Z}}{
                                                 \infer[\Id]{\forall
                                               n. n\in\mathbb{Z}\to n
                                               + 1 \in \mathbb{Z}\land
                                             n - 1 \in\mathbb{Z}}{}}
                                               }
                                           }
                               }
                    }
\]
Veja que da mesma maneira que concluímos que $-2\in\mathbb{Z}$,
poderíamos ter utilizado a regra de $\andEE$ para concluir que
$0\in\mathbb{Z}$. Isto mostra que existe mais de uma maneira de
deduzir que $0\in\mathbb{Z}$: uma é utilizando o caso base do conjunto
$\mathbb{Z}$ e outra é utilizando uma combinação de passo recursivo e
do caso base.

Em termos matemáticos, a definição anterior não é problemática. Porém,
definições que geram ``elementos repetidos'' são inconvenientes
computacionalmente pois, esta repetição pode denotar desperdício de
tempo de CPU ou de memória (para armazenar as repetições). Desta
forma, devemos sempre especificar conjuntos recursivos de maneira que
haja uma única maneira de representar qualquer elemento deste conjunto.

Uma definição equivalente que não possui o inconveniente de gerar
elementos repetidos é baseada na seguinte observação:
se $n\in\mathbb{Z}$ então tanto $n + 1$ quanto $-(n + 1)$
pertencem a $\mathbb{Z}$. Estes critérios serão utilizados na
definição recursiva de $\mathbb{Z}$, apresentada a seguir:
\begin{itemize}
  \item Caso base: $0\in\mathbb{Z}$.
  \item Caso recursivo: $\forall n. n\in\mathbb{Z}\land n\geq 0 \to (n
    + 1) \in \mathbb{Z} \land -(n + 1) \in\mathbb{Z}$
  \item Regra de fechamento: Todo $n\in\mathbb{Z}$ pode ser gerado por
    um número finito de aplicações das regras anteriores.
\end{itemize}
É útil que o leitor tente verificar que a derivação de
$-2\in\mathbb{Z}$, utilizando esta última definição não possui o
inconveniente de gerar elementos repetidos.
\end{Example}

As formas apresentadas de descrever conjuntos não são equivalentes
entre si. Evidentemente, não podemos utilizar enumeração para
representar conjuntos infinitos. Porém, mesmo as técnicas de set
comprehension e recursividade não são equivalentes. Como exemplo,
considere o seguinte conjunto:
\[\{x\in\mathbb{R}\,|\,0 \leq x \leq 1\}\]
Não é possível construir uma definição recursiva para este conjunto,
uma vez que, dado um número real $x$ não é possível determinar de
maneira única qual seria o ``sucessor''\footnote{Este uso da palavra
  sucessor é um abuso de linguagem.} de $x$ na reta real.

Outra maneira de descrever conjuntos é definindo-os utilizando
operações sobre conjuntos existentes. Este é o assunto da próxima
seção.

\subsection{Exercícios}

\begin{enumerate}
  \item Apresente uma definição recursiva do conjunto de números naturais
    ímpares.
  \item Apresente uma definição recursiva do conjunto de números
    inteiros múltiplos de 5.
  \item Uma sequência é um palíndromo se esta pode ser lida da mesma
    maneira da esquerda para direita quanto da direita para
    esquerda. Apresente uma definição recursiva do conjunto
    $\mathbb{P}$, que consiste de todos os palíndromos de bits
  (formados apenas pelos bits 0 e 1).
\end{enumerate}

\section{Operações Sobre Conjuntos}

Existem diversas operações que podem ser aplicadas a conjuntos. Seja
para criar outros conjuntos ou mesmo para compará-los. As próximas
subseções apresentam estas propriedades.

\subsection{Subconjuntos e Igualdade de Conjuntos}

Existem diversas relações entre conjuntos que são determinadas pelos
elementos que estes compartilham. Uma destas operações é a de
\emph{continência}. A expressão $A\subseteq B$, que  pode ser lida
como ``$A$ está contido em $B$'', é verdadeira se todo elemento de $A$
é também elemento de $B$. Esta idéia é definida formalmente a seguir:
\begin{Definition}[Continência]
Sejam $A$ e $B$ dois conjuntos quaisquer. Dizemos que $A \subseteq B$
se e somente se
\[\forall x. x\in A \to x \in B\]
\end{Definition}

Por sua vez, dizemos que dois conjuntos são iguais se estes possuem
exatamente os mesmos elementos. A seguinte definição formaliza este
conceito.

\begin{Definition}[Igualdade]
Sejam $A$ e $B$ dois conjuntos quaisquer. Dizemos que $A = B$ se e
somente se:
\[
\forall x. x\in A \leftrightarrow x \in B
\]
\end{Definition}
Utilizando álgebra para lógica de predicados e a definição de $A
\subseteq B$, podemos obter uma definição alternativa da igualdade de
conjuntos. Esta é demonstrada a seguir:
\[
\begin{array}{lc}
A = B & \equiv \\
\forall x. x \in A \leftrightarrow x \in B & \equiv\\
\forall x. (x\in A \to x \in B) \land (x \in B \to x \in A) & \equiv\\
(\forall x. x\in A \to x \in B) \land (\forall x . x\in B \to x \in A)
& \equiv\\
A\subseteq B\land B \subseteq A
\end{array}
\]
Logo, podemos concluir que $A = B$ é equivalente a $A \subseteq B
\land B\subseteq A$.

A definição da igualdade de conjuntos implica que  para dois conjuntos
serem considerados diferentes, um deles deve possuir pelo menos um
elemento que o outro não possui. Note que se $A \neq B$ e $A\subseteq
B$, podemos dizer que existe $y$ pertencente a $B$ tal que $y\not\in
A$. Esta noção é definida formalmente a seguir.

\begin{Definition}[Subconjunto Próprio]
Sejam $A$ e $B$ dois conjuntos quaisquer. Dizemos que $A$ é um
subconjunto próprio de $B$, $A\subset B$, se e somente se $A\subseteq
B$ e $A \neq B$.
\end{Definition}

\subsection{União, Interseção e Diferença de Conjuntos}

Nesta seção descreveremos formalmente operações que já devem ser
conhecidas pelo leitor.
\begin{itemize}
  \item A união de dois conjuntos $A$ e $B$, $A\cup B$, é o conjunto
    que contém todos os elementos que estão em $A$ ou em $B$ (ou
    ambos). Todo elemento de $A\cup B$ deve pertencer a $A$ ou $B$ ou
    ambos.
  \item A interseção de dois conjuntos $A$ e $B$, $A\cap B$, é o
    conjunto que contém todos os elementos que estão em $A$ e em $B$.
  \item A diferença de dois conjuntos $A$ e $B$, $A - B$, é o conjunto
    de todos os elementos que estão em $A$, mas não estão em $B$.
\end{itemize}

A seguir, definimos estas operações de maneira precisa.

\begin{Definition}[União, interseção e diferença]
Sejam $A$ e $B$ dois conjuntos quaisquer. Então:
\begin{itemize}
  \item $A\cup B = \{x \,|\, x\in A \lor x \in B\}$
  \item $A\cap B = \{x \,|\, x\in A \land x \in B\}$
  \item $A - B = \{x \,|\, x\in A \land x \not\in B\}$
\end{itemize}
\end{Definition}

\begin{Example}
Sejam $A =\{1,2,3\}$, $B = \{3,4,5\}$ e $C = \{4,5,6\}$. Então:
\[
\begin{array}{lcl}
  A \cup B & = & \{1,2,3,4,5\} \\
 A \cap B & = & \{3\} \\
 A - B & = & \{1,2\} \\
 A \cup C & = & \{1,2,3,4,5,6\}\\
 A\cap C & = & \emptyset \\
 A - C & = & \{1,2,3\}\\
\end{array}
\]
\end{Example}