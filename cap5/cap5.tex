\chapter{Teoria de Conjuntos}\label{cap5}

\epigraph{Ninguém deveria nos expulsar do paraíso que Cantor criou.}{David Hilbert, Matemático Alemão sobre a Teoria de
  Conjuntos criada por Georg Cantor.}

\section{Motivação}

De maneira simplista, pode-se dizer que o alicerce fundamental da
matemática é a teoria de conjuntos. Isto se torna mais e mais evidente
a medida que você avança por cursos mais avançados de matemática, já
que a teoria de conjuntos é uma linguagem projetada para descrever e
explicar todos os tipos de estruturas matemáticas.

Em se tratando de computação, a teoria de conjuntos possui um papel
importante no projeto de estruturas de dados e bancos de
dados. Primeiramente, diversas estruturas
eficientes são implementações de um tipo abstrato de dados que
define operações sobre conjuntos. Por sua vez, toda a teoria de bancos
de dados relacionais é baseada em operações básicas sobre conjuntos.

O objetivo deste capítulo é apresentar a teoria de conjuntos e como
esta pode ser utilizada para descrever propriedades de objetos matemáticos.

\section{Descrevendo Conjuntos}

Não apresentaremos uma definição formal do que é um conjunto. Isto se
deve ao fato de que  a teoria de conjuntos foi concebida com o intuito
de ser a fundamentação teórica de toda a matemática. Isto é, em
princípio, todos os objetos matemáticos são definidos em termos de
conjuntos.

Conjuntos nada mais são que uma coleção de objetos denominados
\emph{elementos}. Porém, existem algumas restrições para considerarmos
uma coleção de objetos um conjunto. A primeira diz respeito a
\emph{ordem}. Em um conjunto a ordem dos elementos é irrelevante. A
segunda é sobre a \emph{multiplicidade}. Esta especifica que em um
conjunto qualquer há somente uma ocorrência de um certo valor, isto é,
não é permitido que um elemento apareça mais de uma vez em um mesmo
conjunto.

Adotaremos, como convenção, que conjuntos
serão sempre representados por letras maiúsculas e elementos por
letras minúsculas.

Existem diversas maneiras para se descrever conjuntos. Apresentaremos,
de maneira suscinta três maneiras: enumeração, \emph{set
  comprehension}\footnote{Infelizmente, não conheço uma tradução para
  este termo. Por isso, mantive o nome original.} e por recursão.


\subsection{Enumeração}

Definimos um conjunto por enumeração simplesmente listando seus
elementos. Este é um método conveniente para conjuntos finitos que
possuam poucos elementos.

O exemplo a seguir mostra alguns conjuntos
definidos por enumeração.

\begin{Example}
Abaixo apresentamos alguns conjuntos definidos por enumeração:
\[
\begin{array}{lcl}
   V & = & \{a,e,i,o,u\} \\
   P & = & \{\text{arara, pelicano, pardal}\}\\
   X & = & \{2,4,6,8\} \\
   J & = & \{\}\\
  L & = & \{1, \{1\}\} \\
\end{array}
\]
\end{Example}

Uma vez que elementos podem ocorrer uma única vez em um conjunto,
podemos dizer que a operação de determinar se um elemento está ou não
em um conjunto possui um valor lógico (isto é, verdadeiro ou
falso). Se $A$ é um conjunto e $x$ um elemento, representamos por $x
\in A$ o fato de $x$ ser um elemento do conjunto $A$. Representamos que $x$ não
é um elemento de $A$ por $x \not\in A$. Note que a seguinte
equivalência é verdadeira: $x\not\in A \equiv \neg (x \in A)$.

\begin{Example}
Considere os seguintes conjunto $V = \{a,e,i,o,u\}$.
Temos que $a\in V$ e que $g \not\in V$.
\end{Example}

Evidentemente podemos definir conjuntos que possuem um número qualquer
de elementos. Denominamos por \textit{cardinalidade} o número de
elementos de um certo conjunto e representamos a cardinalidade de um
conjunto $A$ por $|A|$.

\begin{Example}
Considere os conjuntos:

\[
\begin{array}{lcl}
   V & = & \{a,e,i,o,u\} \\
   P & = & \{\text{arara, pelicano, pardal}\}\\
   X & = & \{2,4,6,8\} \\
   J & = & \{\}\\
  L & = & \{1, \{1\}\} \\
\end{array}
\]


A cardinalidade de cada um deles é:

\[
\begin{array}{lcl}
   |V| & = & 5 \\
   |P| & = & 3\\
   |X| & = & 4 \\
   |J| & = & 0\\
  |L| & = &2 \\
\end{array}
\]
\end{Example}

Note que o conjunto $J$ não possui elementos e, portanto, é denominado
\emph{conjunto vazio}. O conjunto vazio é usualmente representado pelo
símbolo $\emptyset$.

\subsection{Set Comprehension}

A notação de set comprehension permite-nos especificar um conjunto em
termos de uma propriedade que descreve quais são os elementos deste
conjunto. De maneira simples, temos que um set comprehension é
representado da seguinte maneira:
\[\{x\,|\,p(x)\}\]
em que $x$ é uma variável  (ou uma expressão) e $p(x)$ é uma fórmula
da lógica de predicados. Esta maneira de descrever conjuntos é útil
para descrever conjuntos com muitos elementos ou infinitos.

\begin{Example}
Considere a tarefa de representar os conjuntos de todos os números naturais
pares e de todos os números naturais múltiplos de 3. Poderíamos
representar estes conjuntos da seguinte maneira:
\[
\begin{array}{lcr}
P & = &\{0,2,4,6,...\}\\
T & = & \{0,3,6,9,...\}\\
\end{array}
\]
Apesar da estrutura parecer óbvia, o uso de ``...'' deve ser evitado
por este permitir ambiguidades na interpretação de um conjunto. Sem
saber que o conjunto $T$ representa os números naturais múltiplos de
3, como saber se $173$ pertence ou não a este conjunto?

Para evitar este tipo de ambiguidades, podemos utilizar set
comprehensions para definir conjuntos infinitos de maneira precisa. Os
conjuntos anteriores podem ser representados da seguinte maneira:
\[
\begin{array}{lcr}
P & = &\{x\,|\,x\in\mathbb{N}\land \exists y. y\in \mathbb{N}\land x =
2y\}\\
T & = & \{x\,|\,x\in\mathbb{N}\land \exists y. y\in\mathbb{N}\land x =
3y\}\\
\end{array}
\]
Note que como utilizarmos uma fórmula da lógica de predicados para
descrever elementos de um conjunto não há margem para interpretações
ambíguas. Pode-se, a partir da fórmula, determinar facilmente se um
elemento pertence ou não a um conjunto.
\end{Example}

Devemos sempre especificar qual o conjunto de ``origem'' dos elementos
para os quais estamos definindo um conjunto utilizando set comprehension.
A não especificação do conjunto de origem permite a formulação de
paradoxos, como o conhecido paradoxo de Russell, descoberto por
Bertrand Russell no início do século XX.

\subsubsection{O Paradoxo de Russell}

Antes de apresentar o paradoxo de Russell formalmente, é útil
analisá-lo em um contexto mais simples, mas equivalente.

\begin{Example}
Considere o seguinte problema:

\begin{quote}
``Considere uma cidade em que existe apenas um barbeiro e que este faz a
barba de todos que não fazem a própria barba. O barbeiro faz sua
própria barba?''
\end{quote}

Após refletir uma pouco sobre esta sentença, percebemos que esta é um
paradoxo, pois:
\begin{itemize}
  \item Se o barbeiro não faz a própria barba, ele deveria fazê-la, já
    que ele faz a barba apenas de quem não faz a própria barba.
  \item Porém se ele faz a própria barba, pela definição, ele não
    deveria fazê-la.
\end{itemize}

Ou seja, a sentença sobre o barbeiro desta cidade é um paradoxo.
\end{Example}