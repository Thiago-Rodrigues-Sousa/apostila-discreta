\chapter{Teoria de Conjuntos}\label{cap5}

\epigraph{Ninguém deveria nos expulsar do paraíso que Cantor criou
  para nós.}{David Hilbert, Matemático Alemão sobre a Teoria de
  Conjuntos criada por Georg Cantor.}

\section{Motivação}

De maneira simplista, pode-se dizer que o alicerce fundamental da
matemática é a teoria de conjuntos. Isto se torna mais e mais evidente
a medida que você avança por cursos mais avançados de matemática, já
que a teoria de conjuntos é uma linguagem projetada para descrever e
explicar todos os tipos de estruturas matemáticas.

Em se tratando de computação, a teoria de conjuntos possui um papel
importante no projeto de estruturas de dados e bancos de
dados. Primeiramente, diversas estruturas
eficientes são implementações de um tipo abstrato de dados que
define operações sobre conjuntos. Por sua vez, toda a teoria de bancos
de dados relacionais é baseada em operações básicas sobre conjuntos.

O objetivo deste capítulo é apresentar a teoria de conjuntos e como
esta pode ser utilizada para descrever propriedades de objetos matemáticos.

\section{Conjuntos, Informalmente}
