\chapter{Teoria de Conjuntos}\label{cap5}

\epigraph{Ninguém deveria nos expulsar do paraíso que Cantor criou.}{David Hilbert, Matemático Alemão sobre a Teoria de
  Conjuntos criada por Georg Cantor.}

\section{Motivação}

De maneira simplista, pode-se dizer que o alicerce fundamental da
matemática é a teoria de conjuntos. Isto se torna mais e mais evidente
a medida que você avança por cursos mais avançados de matemática, já
que a teoria de conjuntos é uma linguagem projetada para descrever e
explicar todos os tipos de estruturas matemáticas.

Em se tratando de computação, a teoria de conjuntos possui um papel
importante no projeto de estruturas de dados e bancos de
dados. Primeiramente, diversas estruturas
eficientes são implementações de um tipo abstrato de dados que
define operações sobre conjuntos. Por sua vez, toda a teoria de bancos
de dados relacionais é baseada em operações básicas sobre conjuntos.

O objetivo deste capítulo é apresentar a teoria de conjuntos e como
esta pode ser utilizada para descrever propriedades de objetos
matemáticos.

\section{Introdução aos Conjuntos}

Não apresentaremos uma definição formal do que é um conjunto. Isto se
deve ao fato de que  a teoria de conjuntos foi concebida com o intuito
de ser a fundamentação teórica de toda a matemática. Isto é, em
princípio, todos os objetos matemáticos são definidos em termos de
conjuntos.

Conjuntos nada mais são que uma coleção de objetos denominados
\emph{elementos}. Porém, existem algumas restrições para considerarmos
uma coleção de objetos um conjunto. A primeira diz respeito a
\emph{ordem}. Em um conjunto a ordem dos elementos é irrelevante. A
segunda é sobre a \emph{multiplicidade}. Esta especifica que em um
conjunto qualquer há somente uma ocorrência de um certo valor, isto é,
não é permitido que um elemento apareça mais de uma vez em um mesmo
conjunto.

Uma vez que elementos podem ocorrer uma única vez em um conjunto,
podemos dizer que a operação de determinar se um elemento está ou não
em um conjunto possui um valor lógico (isto é, verdadeiro ou
falso). Se $A$ é um conjunto e $x$ um elemento, representamos por $x
\in A$ o fato de $x$ ser um elemento do conjunto $A$. Representamos que $x$ não
é um elemento de $A$ por $x \not\in A$. Note que a seguinte
equivalência é verdadeira: $x\not\in A \equiv \neg (x \in A)$.

Denominamos por \emph{cardinalidade} ou tamanho o número de elementos
de um conjunto. Se $A$ é um conjunto, representamos por $|A|$ o número
de elementos de $A$.

Existe um único conjunto $A$ tal que $|A| = 0$. Este é conhecido como
conjunto vazio e é representado como $\{\}$ ou $\emptyset$.

Adotaremos, como convenção, que conjuntos
serão sempre representados por letras maiúsculas e elementos por
letras minúsculas.


\section{Descrevendo Conjuntos}

Existem diversas maneiras para se descrever conjuntos. Apresentaremos,
de maneira suscinta três maneiras: enumeração, \emph{set
  comprehension}\footnote{Infelizmente, não conheço uma tradução para
  este termo. Por isso, mantive o nome original.} e por recursão.

\subsection{Enumeração}

Definimos um conjunto por enumeração simplesmente listando seus
elementos. Este é um método conveniente para conjuntos finitos que
possuam poucos elementos.

O exemplo a seguir mostra alguns conjuntos
definidos por enumeração.

\begin{Example}
Abaixo apresentamos alguns conjuntos definidos por enumeração:
\[
\begin{array}{lcl}
   V & = & \{a,e,i,o,u\} \\
   P & = & \{\text{arara, pelicano, pardal}\}\\
   X & = & \{2,4,6,8\} \\
   J & = & \{\}\\
  L & = & \{1, \{1\}\} \\
\end{array}
\]
A cardinalidade de cada um deles é:

\[
\begin{array}{lcl}
   |V| & = & 5 \\
   |P| & = & 3\\
   |X| & = & 4 \\
   |J| & = & 0\\
  |L| & = &2 \\
\end{array}
\]
Note que as seguintes proposições sobre pertinência nestes conjuntos
são verdadeiras:
\[
\begin{array}{l}
a \in V \\
\text{arara} \in P \\
2 \in X\\
\{1\} \in L\\
1 \in L\\
\end{array}
\]
Por sua vez, as seguintes proposições são falsas considerando os
conjuntos anteriores:
\[
\begin{array}{l}
p \in V \\
\text{vaca}\in P \\
7 \in X \\
2 \in L \\
\{1,\{1\}\} \in \{1,\{1\}\}
\end{array}
\]
\end{Example}

\subsection{Set Comprehension}

A notação de set comprehension\footnote{Manteremos o nome sem tradução
    por não conhecer um termo em língua portuguesa para este tipo de
    notação matemática.} permite-nos especificar um conjunto em
termos de uma propriedade que descreve quais são os elementos deste.
De maneira simples, temos que um set comprehension é
representado da seguinte maneira:
\[\{x  \in X\,|\,p(x)\}\]
em que $x$ é uma variável  (ou uma expressão), $X$ um conjunto e $p(x)$ é uma fórmula
da lógica de predicados. Esta maneira de descrever conjuntos é útil
para descrever conjuntos com muitos elementos ou infinitos.

É importante notar que ocorrências da variável $x$ em $p(x)$ no set
comprehension $\{x\in X\,|\,p(x)\}$ são consideradas ligadas por
Podemos caracterizar a pertinência a um conjunto definido usando set
comprehension utilizando a seguinte equivalência:

\[y\in\{x\in X \,|\, p(x)\} \equiv y\in X \land p(y)\]

O leitor atento deve ter percebido que a equivalência anterior nada
mais é que a aplicação da substituição $[x\mapsto y]$ a fórmula
especificada no set comprehension.

\begin{Example}
Considere a tarefa de representar os conjuntos de todos os números naturais
pares e de todos os números naturais múltiplos de 3. Poderíamos
representar estes conjuntos da seguinte maneira:
\[
\begin{array}{lcr}
P & = &\{0,2,4,6,...\}\\
T & = & \{0,3,6,9,...\}\\
\end{array}
\]
Apesar da estrutura parecer óbvia, o uso de ``...'' deve ser evitado
por este permitir ambiguidades na interpretação de um conjunto. Sem
saber que o conjunto $T$ representa os números naturais múltiplos de
3, como saber se $173$ pertence ou não a este conjunto?

Para evitar este tipo de ambiguidades, podemos utilizar set
comprehensions para definir conjuntos infinitos de maneira precisa. Os
conjuntos anteriores podem ser representados da seguinte maneira:
\[
\begin{array}{lcr}
P & = &\{x\in\mathbb{N}\,|\,\exists y. y\in \mathbb{N}\land x =
2y\}\\
T & = & \{x\in\mathbb{N}\,|\ \exists y. y\in\mathbb{N}\land x =
3y\}\\
\end{array}
\]

Podemos representar os fatos de que $a \in P$ e $b \in T$ como as
seguintes fórmulas:
\[
\begin{array}{l}
a \in \mathbb{N} \land \exists y. y\in\mathbb{N} \land a = 2y\\
b \in \mathbb{N} \land \exists y. y\in\mathbb{N} \land b = 3y\\
\end{array}
\]

Note que como utilizarmos uma fórmula da lógica de predicados para
descrever elementos de um conjunto não há margem para interpretações
ambíguas.
\end{Example}

\begin{Example}
Considere o tarefa de definir o conjunto de todos os números naturais
que são quadrados perfeitos. Mais formalmente:
\[\{n^2\,|\,n\in\mathbb{N}\}\]
Note que se $y\in \{n^2\,|\,n\in\mathbb{N}\}$ deveríamos ser capazes
de deduzir que $y$ é um quadrado perfeito. A questão é que o conjunto
\[\{n^2\,|\,n\in\mathbb{N}\}\]
é equivalente a
\[\{x\,|\,\exists n. n\in\mathbb{N} \land x = n^2\}\]
que, permite-nos deduzir que $y$ é um quadrado perfeito.
\end{Example}

De maneira geral, se um conjunto é definido usando set comprehension
em que os elementos são especificados em termos de uma expressão ao
invés de uma simples variável, esta pode ser convertida em uma
definição equivalente em que os elementos são especificados somente
usando variáveis, como no exemplo anterior.

O mecanismo de set comprehension é bastante expressivo. Inclusive
devemos ter alguns cuidados para evitar a definição de paradoxos, como
o descrito na próxima seção.

% Devemos sempre especificar qual o conjunto de
% ``origem''\footnote{Considera-se como conjunto ``origem'' do set
%   comprehension $\{x\in X\,|\,p(x)\}$ o conjunto $X$.} dos elementos
% para os quais estamos definindo um conjunto utilizando set comprehension.
% A não especificação do conjunto de origem permite a formulação de
% paradoxos, como o conhecido paradoxo de Russell, descoberto por
% Bertrand Russell no início do século XX.

\subsubsection{O Paradoxo de Russell}

Antes de apresentar o paradoxo de Russell formalmente, é útil
analisá-lo em um contexto mais simples, porém, equivalente.

\begin{Example}
Considere o seguinte problema:

\begin{quote}
``Considere uma cidade em que existe apenas um barbeiro e que este faz a
barba de todos que não fazem a própria barba. O barbeiro faz sua
própria barba?''
\end{quote}

Após refletir uma pouco sobre esta sentença, percebemos que esta é um
paradoxo, pois:
\begin{itemize}
  \item Se o barbeiro não faz a própria barba, ele deveria fazê-la, já
    que ele faz a barba apenas de quem não faz a própria barba.
  \item Porém se ele faz a própria barba, pela definição, ele não
    deveria fazê-la.
\end{itemize}

Ou seja, a sentença sobre o barbeiro desta cidade é um paradoxo.
\end{Example}

Russell percebeu que a definição usando set comprehension poderia
gerar um paradoxo similar ao apresentado no exemplo
anterior. A demonstração deste paradoxo é apresentada a seguir.

Seja $\mathcal{S}$ o conjunto de todos os conjuntos que não
são elementos de si próprios, isto é:
\[\mathcal{S} =\{X\,|\,X\not\in X\}\]
Evidentemente, temos que $\mathcal{S} \in \mathcal{S}$ ou $\mathcal{S}
\not\in \mathcal{S}$. Considere os seguintes casos:
\begin{itemize}
  \item Caso $\mathcal{S} \in \mathcal{S}$: Se $\mathcal{S} \in
    \mathcal{S}$, pela definição de $\mathcal{S}$, temos que
    $\mathcal{S} \not\in \mathcal{S}$, o que constitui uma contradição.
  \item Caso $\mathcal{S} \not\in \mathcal{S}$: Logo, pela definição
    de $\mathcal{S}$, temos que $\mathcal{S} \in \mathcal{S}$, o que
    constitui uma contradição.
\end{itemize}
Como ambos os casos cobrem todas as possibilidades, temos que
$\mathcal{S}\in\mathcal{S}$ não pode ser uma proposição lógica, já que
esta não pode ser determinada como verdadeira ou falsa.

\subsection{Conjuntos Definidos Recursivamente}

Conjuntos definidos por recursão são muito utilizados em computação
para a definição de estruturas de dados e algoritmos. Nesta seção,
veremos como definir conjuntos recursivamente.

Assim como toda definição recursiva, conjuntos
indutivos\footnote{Conjuntos definidos recursivamente são também conhecidos
  como conjuntos indutivos.} devem possuir casos base e passos
recursivos. Porém, apenas estes elementos são suficientes para
caracterizar uma definição de um conjunto. Adicionalmente, devemos
possuir uma regra, denominada \emph{regra de
  fechamento}\footnote{Normalmente, estas regras são denominadas como
  \emph{extremal rule}, em textos sobre teoria de conjuntos.} que
especifica que todos os elementos do conjunto definido são formados a
partir do(s) caso(s) base e de um número finito de usos do(s)
passo(s) recursivo(s).

Resumindo, para definir um conjunto recursivamente devemos especificar
três partes: casos base, passos recursivos e regra de fechamento:
\begin{itemize}
  \item Casos base consistem de afirmativas simples, como $1\in S$.
   \item Passos recursivos consistem de afirmativas envolvendo
     implicações e quantificadores universais, como \[\forall x. x\in
     S \to x + 1 \in S\]
   \item Regras de fechamento especificam que todo elemento do
     conjunto pode ser obtido a partir de um número finito de
     utilização das regras anteriores.
\end{itemize}

A seguir apresentaremos algumas definições de conjuntos definidos
recursivamente.

\begin{Example}
Como um primeiro exemplo de definição de um conjunto recursivo, vamos
considerar o conjunto dos números naturais,
$\mathbb{N}$. O leitor deve lembrar da definição do conjunto
$\mathcal{N}$ do capítulo \ref{cap1} em que apresentamos a sintaxe de
termos que representam números naturais utilizando a operação de
sucessor e uma constante para representar o número $0$.

Utilizaremos a
mesma idéia para a definição recursiva de $\mathbb{N}$:
\begin{itemize}
  \item Caso base: $0\in\mathbb{N}$.
  \item Passo recursivo: $\forall n. n\in\mathbb{N}\to n + 1 \in
    \mathbb{N}$
  \item Regra de fechamento: Todo $n\in\mathbb{N}$ pode se obtido por
    um número finito de aplicações das regras anteriores.
\end{itemize}
\end{Example}

\begin{Example}
Considere  $P=\{x\in\mathbb{N}\,|\,\exists
y.y\in\mathbb{N}\land x = 2y\}$, que obviamente representa o conjunto
de todos os números naturais pares. Podemos definí-lo recursivamente
se percebermos que este é formado por um caso base ($0\in P$) e que se
$n\in P$ então $n + 2\in P$, o que é definido formalmente a seguir:
\begin{itemize}
  \item Caso base: $0\in P$
  \item Passo recursivo: $\forall n. n\in P \to n + 2 \in P$
  \item Regra de fechamento: Todo $n\in P$ pode ser obtido por um
    número finito de aplicações de regras anteriores.
\end{itemize}
\end{Example}

A seguir apresentamos uma exemplo de definição maior: a
definição recursiva do conjunto dos números inteiros.

\begin{Example}
O conjunto dos números naturais possui uma definição direta porquê
este é infinito em apenas ``uma direção'', o que não acontece com o
conjunto $\mathbb{Z}$, que possui infinitos elementos positivos e negativos.
Então como definir este conjunto recursivamente?

Uma primeira tentativa seria algo similar a dizer que $0\in\mathbb{Z}$
e que se $n \in \mathbb{Z}$ então $n + 1$ e $n - 1$ também pertencem a
$\mathbb{Z}$. Mais formalmente:
\begin{itemize}
  \item Caso base: $0\in\mathbb{Z}$.
  \item Passo recursivo: $\forall x. x\in\mathbb{Z} \to x+1 \in
    \mathbb{Z} \land x - 1 \in\mathbb{Z}$
\end{itemize}

Apesar de correta, a definição anterior possui um inconveniente:
existe mais de uma maneira de deduzirmos que um certo número pertence
ou não a $Z$. Como exemplo, considere a seguinte derivação, de que
$-2\in\mathbb{Z}$:

\[
\infer[\andED]{-2\in\mathbb{Z}}
                    {
                      \infer[\impE]{0\in\mathbb{Z}\land -2
                        \in\mathbb{Z}}
                               {
                                 \infer[\forallE]{-1\in\mathbb{Z} \to
                                   0 \in\mathbb{Z} \land -2
                                   \in\mathbb{Z}}
                                   {\infer[\Id]{\forall n.n\in\mathbb{Z}\to n + 1
                                     \in\mathbb{Z}\land n - 1 \in
                                     \mathbb{Z}}{}}
                                    &
                                 \infer[\andED]{-1\in\mathbb{Z}}
                                           {
                                             \infer[\impE]{1\in\mathbb{Z}\land
                                             -1\in\mathbb{Z}}
                                               {
                                                 \infer[\Id]{0\in\mathbb{Z}}{}
                                                 &
                                                 \infer[\forallE]{0\in\mathbb{Z}\to
                                                 1 \in\mathbb{Z}\land
                                                 -1\in\mathbb{Z}}{
                                                 \infer[\Id]{\forall
                                               n. n\in\mathbb{Z}\to n
                                               + 1 \in \mathbb{Z}\land
                                             n - 1 \in\mathbb{Z}}{}}
                                               }
                                           }
                               }
                    }
\]
Veja que da mesma maneira que concluímos que $-2\in\mathbb{Z}$,
poderíamos ter utilizado a regra de $\andEE$ para concluir que
$0\in\mathbb{Z}$. Isto mostra que existe mais de uma maneira de
deduzir que $0\in\mathbb{Z}$: uma é utilizando o caso base do conjunto
$\mathbb{Z}$ e outra é utilizando uma combinação de passo recursivo e
do caso base.

Em termos matemáticos, a definição anterior não é problemática. Porém,
definições que geram ``elementos repetidos'' são inconvenientes
computacionalmente pois, esta repetição pode denotar desperdício de
tempo de CPU ou de memória (para armazenar as repetições). Desta
forma, devemos sempre especificar conjuntos recursivos de maneira que
haja uma única maneira de representar qualquer elemento deste conjunto.

Uma definição equivalente que não possui o inconveniente de gerar
elementos repetidos é baseada na seguinte observação:
se $n\in\mathbb{Z}$ então tanto $n + 1$ quanto $-(n + 1)$
pertencem a $\mathbb{Z}$. Estes critérios serão utilizados na
definição recursiva de $\mathbb{Z}$, apresentada a seguir:
\begin{itemize}
  \item Caso base: $0\in\mathbb{Z}$.
  \item Caso recursivo: $\forall n. n\in\mathbb{Z}\land n\geq 0 \to (n
    + 1) \in \mathbb{Z} \land -(n + 1) \in\mathbb{Z}$
  \item Regra de fechamento: Todo $n\in\mathbb{Z}$ pode ser gerado por
    um número finito de aplicações das regras anteriores.
\end{itemize}
É útil que o leitor tente verificar que a derivação de
$-2\in\mathbb{Z}$, utilizando esta última definição não possui o
inconveniente de gerar elementos repetidos.
\end{Example}

As formas apresentadas de descrever conjuntos não são equivalentes
entre si. Evidentemente, não podemos utilizar enumeração para
representar conjuntos infinitos. Porém, mesmo as técnicas de set
comprehension e recursividade não são equivalentes. Como exemplo,
considere o seguinte conjunto:
\[\{x\in\mathbb{R}\,|\,0 \leq x \leq 1\}\]
Não é possível construir uma definição recursiva para este conjunto,
uma vez que, dado um número real $x$ não é possível determinar de
maneira única qual seria o ``sucessor''\footnote{Este uso da palavra
  sucessor é um abuso de linguagem.} de $x$ na reta real.

Outra maneira de descrever conjuntos é definindo-os utilizando
operações sobre conjuntos existentes. Este é o assunto da próxima
seção.

\subsection{Exercícios}

\begin{enumerate}
  \item Apresente uma definição recursiva do conjunto de números naturais
    ímpares.
  \item Apresente uma definição recursiva do conjunto de números
    inteiros múltiplos de 5.
  \item Uma sequência é um palíndromo se esta pode ser lida da mesma
    maneira da esquerda para direita quanto da direita para
    esquerda. Apresente uma definição recursiva do conjunto
    $\mathbb{P}$, que consiste de todos os palíndromos de bits
  (formados apenas pelos bits 0 e 1).
\end{enumerate}

\section{Operações Sobre Conjuntos}

Existem diversas operações que podem ser aplicadas a conjuntos. Seja
para criar outros conjuntos ou mesmo para compará-los. As próximas
subseções apresentam estas propriedades.

\subsection{Subconjuntos e Igualdade de Conjuntos}

Existem diversas relações entre conjuntos que são determinadas pelos
elementos que estes compartilham. Uma destas operações é a de
\emph{continência}. A expressão $A\subseteq B$, que  pode ser lida
como ``$A$ está contido em $B$'', é verdadeira se todo elemento de $A$
é também elemento de $B$. Esta idéia é definida formalmente a seguir:
\begin{Definition}[Continência]
Sejam $A$ e $B$ dois conjuntos quaisquer. Dizemos que $A \subseteq B$
se e somente se
\[\forall x. x\in A \to x \in B\]
\end{Definition}

Por sua vez, dizemos que dois conjuntos são iguais se estes possuem
exatamente os mesmos elementos. A seguinte definição formaliza este
conceito.

\begin{Definition}[Igualdade]
Sejam $A$ e $B$ dois conjuntos quaisquer. Dizemos que $A = B$ se e
somente se:
\[
\forall x. x\in A \leftrightarrow x \in B
\]
\end{Definition}
Utilizando álgebra para lógica de predicados e a definição de $A
\subseteq B$, podemos obter uma definição alternativa da igualdade de
conjuntos. Esta é demonstrada a seguir:
\[
\begin{array}{lc}
A = B & \equiv \\
\forall x. x \in A \leftrightarrow x \in B & \equiv\\
\forall x. (x\in A \to x \in B) \land (x \in B \to x \in A) & \equiv\\
(\forall x. x\in A \to x \in B) \land (\forall x . x\in B \to x \in A)
& \equiv\\
A\subseteq B\land B \subseteq A
\end{array}
\]
Logo, podemos concluir que $A = B$ é equivalente a $A \subseteq B
\land B\subseteq A$.

A definição da igualdade de conjuntos implica que  para dois conjuntos
serem considerados diferentes, um deles deve possuir pelo menos um
elemento que o outro não possui. Note que se $A \neq B$ e $A\subseteq
B$, podemos dizer que existe $y$ pertencente a $B$ tal que $y\not\in
A$. Esta noção é definida formalmente a seguir.

\begin{Definition}[Subconjunto Próprio]
Sejam $A$ e $B$ dois conjuntos quaisquer. Dizemos que $A$ é um
subconjunto próprio de $B$, $A\subset B$, se e somente se $A\subseteq
B$ e $A \neq B$.
\end{Definition}

\subsection{União, Interseção, Complemento e Diferença de Conjuntos}

Nesta seção descreveremos formalmente operações sobre conjuntos que já devem ser
conhecidas pelo leitor. Para todas as operações, considere que os
conjuntos $A$ e $B$ são subconjuntos de um conjunto universo $\mathcal{U}$.
\begin{itemize}
  \item A união de dois conjuntos $A$ e $B$, $A\cup B$, é o conjunto
    que contém todos os elementos que estão em $A$ ou em $B$ (ou
    ambos). Todo elemento de $A\cup B$ deve pertencer a $A$ ou $B$ ou
    ambos.
  \item A interseção de dois conjuntos $A$ e $B$, $A\cap B$, é o
    conjunto que contém todos os elementos que estão em $A$ e em $B$.
  \item O complemento de um conjunto $A$, $\overline{A}$, é o conjunto
    de todos elementos que pertencem ao conjunto universo
    $\mathcal{U}$ e não pertencem a $A$.
  \item A diferença de dois conjuntos $A$ e $B$, $A - B$, é o conjunto
    de todos os elementos que estão em $A$, mas não estão em $B$.
\end{itemize}

A seguir, definimos estas operações de maneira precisa.

\begin{Definition}[União, interseção e diferença]
Sejam $A$ e $B$ dois conjuntos quaisquer. Então:
\begin{itemize}
  \item $A\cup B = \{x \,|\, x\in A \lor x \in B\}$
  \item $A\cap B = \{x \,|\, x\in A \land x \in B\}$
  \item $\overline{A} = \{x\,|\,x\in\mathcal{U}\land x\not\in A\}$
  \item $A - B = \{x \,|\, x\in A \land x \not\in B\}$
\end{itemize}
Note que $\overline{A} = \mathcal{U} - A$. Dizemos que $A$ e $B$ são
\emph{disjuntos} se $A\cap B = \emptyset$
\end{Definition}

\begin{Example}
Sejam $A =\{1,2,3\}$, $B = \{3,4,5\}$, $C = \{4,5,6\}$ e $\mathcal{U} =\{1,2,3,4,5,6,7\}$. Então:
\[
\begin{array}{lcl}
  A \cup B & = & \{1,2,3,4,5\} \\
 A \cap B & = & \{3\} \\
 A - B & = & \{1,2\} \\
 A \cup C & = & \{1,2,3,4,5,6\}\\
 A\cap C & = & \emptyset \\
 A - C & = & \{1,2,3\}\\
\overline{A} & = & \{2,3,4,5,6,7\}\\
\end{array}
\]
\end{Example}

\subsection{Famílias de Conjuntos}

Damos o nome de família conjuntos que possuem como elementos outros
conjuntos contidos em um universo $\mathcal{U}$. Um exemplo de família
é o chamado conjunto potência ou conjunto das partes de um conjunto,
definido a seguir.

\begin{Definition}[Conjunto Potência]
Seja $A$ um conjunto qualquer. Denomina-se conjunto potência ou
conjunto das partes o conjunto de todos os subconjuntos de
$A$. Representamos este conjunto por $\mathcal{P}(A)$. Mais
formalmente, o conjunto potência é definido como:
\[
\mathcal{P}(A) = \{X\,|\, X \subseteq A\}
\]
\end{Definition}

\begin{Example}
Sejam $A = \{1,2\}$ e $B = \emptyset$. Temos que $\mathcal{P}(A) =
\{\emptyset,\{1\},\{2\},\{1,2\}\}$ e $\mathcal{P}(B) =
\{\emptyset\}$.
\end{Example}
Note que se $|A| = n$, para algum $n\in\mathbb{N}$, então
$|\mathcal{P}(A)| = 2^n$.

As operações de união e interseção de conjuntos se estendem
naturalmente para famílias de conjuntos. A definição destas operações
é apresentada a seguir.

\begin{Definition}[União e Interseção de Famílias] Seja $\mathcal{F}$
  uma família não vazia de conjuntos. A união e interseção da família
  $\mathcal{F}$ são definidas como:
\[
\begin{array}{lcl}
  \bigcup\mathcal{F} & = & \{x\,|\,\exists A. A \in\mathcal{F}\land x
  \in A\} \\
  \bigcap\mathcal{F} & = & \{x\,|\,\forall A. A \in\mathcal{F}\to x
  \in A\} \\
\end{array}
\]
\end{Definition}

\begin{Example}
Seja $\mathcal{F}=\{\{1,2,3\},\{2,3,4\},\{3,4,5\}\}$ uma família de
conjuntos. Temos que:
\[
\begin{array}{lclcl}
\bigcap\mathcal{F} & = & \{1,2,3\}\cap\{2,3,4\}\cap\{3,4,5\} & = &
\{3\}\\
\bigcup\mathcal{F} & = & \{1,2,3\}\cup\{2,3,4\}\cup\{3,4,5\} & = & \{1,2,3,4,5\}\\
\end{array}
\]
\end{Example}

Finalmente, uma notação alternativa para famílias de conjuntos são as
chamadas \emph{famílias indexadas}, que são definidas em termos de um
conjunto de índices.

\begin{Definition}[Famílias Indexadas]
Seja $I$ um conjunto não vazio, denominado conjunto de
índices. Denominamos por família indexada o conjunto
\[\mathcal{F}=\{A_i\,|\,i\in I\}\]
em que cada $A_i$ é definido em termos dos elementos do conjunto de
índices. A união e interseção de famílias indexadas é formalizada
como:
\[
\begin{array}{lcl}
\bigcup_{i\in I} A_i & = & \{x\,|\,\exists i. i\in I \land x\in
A_i\}\\
\bigcap_{i\in I} A_i & = & \{x\,|\,\forall i. i\in I \to x\in A_i\}\\
\end{array}
\]
\end{Definition}

\begin{Example}
Considere o seguinte conjunto de índices $I = \{1,2,3\}$ e a família
indexada $\mathcal{F}=\{A_i\,|\,i\in I\}$, em que
$A_i=\{i,i+1,i+2\}$. Temos:
\[
\begin{array}{lclcl}
\mathcal{F} & = & \{A_1,A_2,A_3\} & = &
\{\{1,2,3\},\{2,3,4\},\{3,4,5\}\}\\
\bigcup_{i\in\{1,2,3\}} & = & \{1,2,3,4,5\} \\
\bigcap_{i\in\{1,2,3\}} & = & \{3\} \\
\end{array}
\]
\end{Example}

\subsection{Exercícios}

\begin{enumerate}
	\item Represente as seguintes f\'ormulas expressas utilizando a linguagem da teoria de conjuntos
	      utilizando f\'ormulas da l\'ogica de predicados. Voc\^e poder\'a utilizar apenas
	      os seguintes s\'imbolos em suas respostas: $\in,\not\in,=,\neq,\land,\lor,\rightarrow,\leftrightarrow,\forall,\exists$.
	      Observe que n\~ao \'e permitido utilizar $\neg$, logo, voc\^e dever\'a utilizar equival\^encias alg\'ebricas para eliminar
	      as ocorr\^encias de $\neg$.
	\begin{enumerate}
		\item $\mathcal{F}\subseteq\mathcal{P}(A)$
		\item $A\subseteq\{2n\, |\, n\in\mathbb{N}\}$
		\item $\{n^2 + n + 1\,|\, n\in\mathbb{N}\}\subseteq\{2n + 1\,|\,n\in\mathbb{N}\}$
		\item $\mathcal{P}(\bigcup_{i\in I}A_i)\not\subseteq \bigcup_{i\in I}\mathcal{P}(A_i)$
		\item $x\in\bigcup\mathcal{F} - \mathcal{G}$
		\item $\{x\in B\,|\,x\not\in C\}\in\mathcal{P}(A)$
		\item $x\in\bigcap_{i\in I}(A_i\bigcup B_i)$
		\item $x\in (\bigcap_{i\in I} A_i)\cup(\bigcap_{i \in I} B_i)$
	\end{enumerate}
	\item Seja $I=\{2,3,4,5\}$ e para cada $i\in I$ considere que $A_i=\{i , i+1, i -1, 2i\}$.
	\begin{enumerate}
		\item Liste os elementos de $\mathcal{F} = \{A_i\,|\,i\in I\}$.
		\item Calcule $\bigcap_{i \in I} A_i$ e $\bigcup_{i \in I} A_i$.
	\end{enumerate}
	\item Mostre, utilizando equival\^encias alg\'ebricas da l\'ogica, que $x\in\mathcal{P}(A\cap B)$ \'e equivalente a
              $x\in\mathcal{P}(A)\cap\mathcal{P}(B)$, para qualquer $x$.
        \item Apresente exemplos de conjuntos $A$ e $B$ tais que $\mathcal{P}(A\cup B)\neq \mathcal{P}(A) \cup \mathcal{P}(B)$.
	\item Mostre que se $\mathcal{F} = \emptyset$  ent\~ao a f\'ormula $x\in\bigcup\mathcal{F}$ \'e
              equivalente a $F$ (contradi\c{c}\~ao).
	\item Mostre que se $\mathcal{F} = \emptyset$  ent\~ao a f\'ormula $x\in\bigcap\mathcal{F}$ \'e
              equivalente a $T$ (tautologia).
\end{enumerate}


\section{Leis Algébricas para Conjuntos}

Como operações sobre conjuntos são definidas usando fórmulas da lógica
(set comprehension), leis algébricas da lógica aplicam-se a expressões
envolvendo conjuntos. A tabela seguinte apresenta as principais
equivalências algébricas para conjuntos (em que $\circ\in\{\cap,\cup\}$).

\[
\begin{array}{|l|l|}
  \hline
  \begin{array}{lcl}
    A \circ A & \equiv & A\\
    A \circ B & \equiv & B \circ A \\
    (A \circ B) \circ C & \equiv & A \circ (B\circ C) \\
    A \cup (B\cap C) & \equiv & (A \cup B) \cap (A\cup C)\\
    A \cap (B\cup C) & \equiv & (A \cap B)\cup (A\cap C) \\
    A \cup \emptyset & \equiv & A \\
    A - B & \equiv & A \cap \overline{B}\\
  \end{array} &
  \begin{array}{lcl}
    A \cup \overline{A} & \equiv & \mathcal{U} \\
    \overline{A\cap B} & \equiv & \overline{A}\cup\overline{B}\\
    \overline{A\cup B} & \equiv & \overline{A}\cap\overline{B}\\
    A \cap \emptyset & \equiv & \emptyset\\
    A \cap \overline{A} & \equiv & \emptyset\\
    A \cap \mathcal{U} & \equiv A\\
  \end{array}
  \\ \hline
\end{array}
\]

Devido a similaridade das leis algébricas para conjuntos com as da
lógica, apresentaremos apenas alguns exemplos que ilustram sua utilização.

\begin{Example}
Considere a seguinte equivalência
\[
[A \cup (B\cap C)]\cap\{[\overline{A}\cup(B\cap C)]\cap \overline{(B\cap C)}\} \equiv \emptyset
\]
cuja demonstração apresentamos abaixo:
\[
\begin{array}{lc}
[A\cup (B\cap C)]\cap\{[\overline{A}\cup (B\cap
C)]\cap\overline{(B\cap C)}\} &\equiv \\
\lbrack A\cup (B\cap C) \rbrack \cap\{[\overline{A}\cap \overline{(B\cap
C)}]\cup\lbrack (B\cap C) \cap \overline{(B\cap C)}\rbrack\} & \equiv \\
\lbrack A\cup (B\cap C) \rbrack \cap\{[\overline{A}\cap \overline{(B\cap
C)}]\cup\emptyset\} & \equiv \\
\lbrack A\cup (B\cap C) \rbrack \cap[\overline{A}\cap \overline{(B\cap
C)}] & \equiv \\
\lbrack A\cup (B\cap C) \rbrack \cap \overline{[A\cup (B\cap
C)]} & \equiv \\
\emptyset
\end{array}
\]
\end{Example}
\begin{Example}
Considere a seguinte equivalência
\[
\lbrack C \cap (A\cup B) \rbrack \cup [(A\cup B)\cap \overline{C}]
\equiv A \cup B
\]
cuja demonstração apresentamos a seguir:
\[
\begin{array}{lc}
\lbrack C \cap (A\cup B) \rbrack \cup [(A\cup B)\cap \overline{C}] &
\equiv \\
\lbrack (A\cup B) \cap C \rbrack \cup [(A\cup B)\cap \overline{C}] &
\equiv \\
(A\cup B)\cap (C \cup \overline{C}) & \equiv\\
(A\cup B)\cap \mathcal{U} & \equiv\\
A\cup B
\end{array}
\]
\end{Example}

\subsection{Exercícios}

\begin{enumerate}
  \item Demonstre as seguintes equivalências algébricas para
    conjuntos.
  \begin{enumerate}
      \item $(A \cup B) \cap (A \cup \overline{B}) \equiv A$
      \item $A \cap (B \cup \overline{A}) \equiv B\cap A$
      \item $(A \cup B) - C \equiv (A - C) \cup (B - C)$
      \item $\overline{\lbrack (\overline{A} \cup \overline{B}) \cap
          \overline{A}\rbrack}  \equiv A$
  \end{enumerate}
\end{enumerate}

\section{Teoremas Envolvendo Conjuntos}

As técnicas de demonstração apresentadas no capítulo \ref{cap4} podem
ser utilizadas para provar diversos fatos da teoria de conjuntos. Para
isso, representaremos os fatos expressando fórmulas da teoria de
conjunto como fórmulas da lógica de predicados, utilizando as definições
apresentadas neste capítulo.

O restante desta seção apresenta diversos teoremas envolvendo
conjuntos e suas demonstrações. Inicialmente, consideraremos rascunhos
com um maior nível de detalhes para um maior entendimento do
leitor. Gradativamente, menos detalhes serão fornecidos até que
apresentemos somente o texto final para um dado teorema. Nestes casos,
recomendamos que o leitor ``preencha'' os detalhes omitidos ou mesmo
reconstrua todo o rascunho da demonstração em questão.

\begin{Example}
Como um primeiro exemplo, considere a tarefa de demonstrar o seguinte
teorema.
\begin{flushleft}
Sejam $A$, $B$ e $C$ conjuntos quaisquer. Então se $A\subseteq B$ e
$B\subseteq C$ então $A\subseteq C$.
\end{flushleft}
Primeiramente, temos que o teorema anterior possui como hipóteses que
$A$, $B$ e $C$ são conjuntos e a conclusão:
\[
A\subseteq B \land B \subseteq C \to A \subseteq C
\]
A partir disto, podemos montar uma versão inicial do rascunho deste
teorema:
\begin{flushleft}
  \begin{tabular}{ll}
        Hipóteses & Conclusão \\
        $A,B,C$ são conjuntos & $A\subseteq B \land B \subseteq C \to
        A \subseteq C$\\
  \end{tabular}
\end{flushleft}
Evidentemente, esta prova deverá iniciar utilizando a estratégia de
prova direta para implicação, que produz oa seguinte configuração do
rascunho:
\begin{flushleft}
  \begin{tabular}{ll}
        Hipóteses & Conclusão \\
        $A,B,C$ são conjuntos & $A \subseteq C$\\
        $A\subseteq B$ & \\
       $B \subseteq C$ & \\
  \end{tabular}
\end{flushleft}
Para demonstrar $A\subseteq C$, devemos expressá-la utilizando sua
definição usando lógica (o mesmo vale para as hipóteses):
\begin{flushleft}
  \begin{tabular}{ll}
        Hipóteses & Conclusão \\
        $A,B,C$ são conjuntos & $\forall x. x\in A \to x \in C$\\
        $\forall y. y\in A \to y \in B$ & \\
       $\forall z. z \in B \to z \in C$ & \\
  \end{tabular}
\end{flushleft}
Agora, utilizamos a estratégia de prova para o quantificador universal
e mais uma aplicação de prova direta, o que nos leva a:
\begin{flushleft}
  \begin{tabular}{ll}
        Hipóteses & Conclusão \\
        $A,B,C$ são conjuntos & $x \in C$\\
        $\forall y. y\in A \to y \in B$ & \\
       $\forall z. z \in B \to z \in C$ & \\
       $x$ arbitrário & \\
       $x\in A$ & \\
  \end{tabular}
\end{flushleft}
Neste ponto, podemos utilizar a estratégia de uso de hipóteses
envolvendo o quantificador universal (regra de eliminação deste
quantificador), obtendo:
\begin{flushleft}
  \begin{tabular}{ll}
        Hipóteses & Conclusão \\
        $A,B,C$ são conjuntos & $x \in C$\\
        $\forall y. y\in A \to y \in B$ & \\
       $\forall z. z \in B \to z \in C$ & \\
       $x$ arbitrário & \\
       $x\in A$ & \\
       $x\in A \to x\in B$ & \\
  \end{tabular}
\end{flushleft}
Usando uma vez a regra de eliminação da implicação, obtemos que $x\in
B$, conforme apresentado a seguir:
\begin{flushleft}
  \begin{tabular}{ll}
        Hipóteses & Conclusão \\
        $A,B,C$ são conjuntos & $x \in C$\\
        $\forall y. y\in A \to y \in B$ & \\
       $\forall z. z \in B \to z \in C$ & \\
       $x$ arbitrário & \\
       $x\in A$ & \\
       $x\in A \to x\in B$ & \\
       $x\in B$ & \\
  \end{tabular}
\end{flushleft}
Eliminando novamente o quantificador universal, obtemos:
\begin{flushleft}
  \begin{tabular}{ll}
        Hipóteses & Conclusão \\
        $A,B,C$ são conjuntos & $x \in C$\\
        $\forall y. y\in A \to y \in B$ & \\
       $\forall z. z \in B \to z \in C$ & \\
       $x$ arbitrário & \\
       $x\in A$ & \\
       $x\in A \to x\in B$ & \\
       $x\in B$ & \\
       $x\in B \to x \in C$ & \\
  \end{tabular}
\end{flushleft}
A demonstração é concluída por uma eliminação da implicação que
permite-nos deduzir que $x\in C$.

O texto para esta dedução é apresentado a seguir.
\begin{flushleft}
Suponha que $A,B,C$ são conjuntos quaisquer.\\
\verb|   |Suponha que $A\subseteq B$ e $B\subseteq C$.\\
\verb|      |Suponha $x$ arbitrário.\\
\verb|         |Suponha $x\in A$.\\
\verb|            |Como $x\in A$ e $A\subseteq B$, temos que $x\in B$.\\
\verb|            |Como $x\in B$ e $B\subseteq C$, temos que $x\in C$.\\
\verb|         |Logo, se $x\in A$ então $x\in C$.\\
\verb|      |Como $x$ é arbitrário, temos que $A\subseteq C$.\\
\verb|   |Portanto, se $A\subseteq B$ e $B\subseteq C$ então
$A\subseteq C$.\\
Logo, se $A,B,C$ são conjuntos e se $A\subseteq B$ e $B\subseteq C$
então $A\subseteq C$.
\end{flushleft}
\end{Example}
\begin{Example}
Considere o seguinte teorema:
\begin{flushleft}
Suponha $A,B$ e $C$ conjuntos tais que $A - B \subseteq C$. Se $x\in A
- C$ então $x \in B$.
\end{flushleft}
Neste teorema, temos como hipóteses que $A,B,C$ são conjuntos e $A - B
\subseteq C$. A conclusão deste é expressa pela seguinte fórmula:
\[
x\in A - C \to x\in B
\]
Inicialmente, o rascunho possui a seguinte forma:
\begin{flushleft}
\begin{tabular}{ll}
Hipóteses & Provar \\
$A,B,C$ são conjuntos & $x\in A - C \to x\in B$\\
 $A - B \subseteq C$
\end{tabular}
\end{flushleft}
Usando a estratégia de prova direta, temos:
\begin{flushleft}
\begin{tabular}{ll}
Hipóteses & Provar \\
$A,B,C$ são conjuntos & $x\in B$\\
 $A - B \subseteq C$ & \\
$x\in A - C$ & \\
\end{tabular}
\end{flushleft}
Note que se $x\in A- C$, temos que $x \in A$ e $x\not\in C$\footnote{O
leitor atento deve ter notado que isto é uma consequência da
representação de pertinência a conjuntos definidos por set
comprehension. Isto é, representamos $y\in\{x\in A\,|\, P(x)\}$ como
$y \in A \land P(y)$.}:
\begin{flushleft}
\begin{tabular}{ll}
Hipóteses & Provar \\
$A,B,C$ são conjuntos & $x\in B$\\
 $A - B \subseteq C$ & \\
$x\in A - C$ & \\
$x\in A$ & \\
$x\not\in C$ & \\
\end{tabular}
\end{flushleft}
Aparentemente não há como deduzir que $x\in B$ a partir das
hipóteses. Neste caso, podemos tentar uma prova por contradição.
\begin{flushleft}
\begin{tabular}{ll}
Hipóteses & Provar \\
$A,B,C$ são conjuntos & $\bot$\\
 $A - B \subseteq C$ & \\
$x\in A - C$ & \\
$x\in A$ & \\
$x\not\in C$ & \\
$x\not\in B$ & \\
\end{tabular}
\end{flushleft}
Uma vez que $x\in A$ e $x\not\in B$, temos que $x\in A - B$.
\begin{flushleft}
\begin{tabular}{ll}
Hipóteses & Provar \\
$A,B,C$ são conjuntos & $\bot$\\
 $A - B \subseteq C$ & \\
$x\in A - C$ & \\
$x\in A$ & \\
$x\not\in C$ & \\
$x\not\in B$ & \\
$x\in A - B$ & \\
\end{tabular}
\end{flushleft}
Como $x\in A - B$ e $A - B \subseteq C$, temos que $x\in C$, o que
contradiz a suposição de que $x\not\in C$, concluindo a demonstração.
Apresentamos o texto desta demonstração a seguir.

\begin{flushleft}
Suponha $A,B$ e $C$ são conjuntos e que $A - B \subseteq C$.\\
\verb|  |Suponha que $x\in A - C$.\\
\verb|    |Suponha que $x\not\in B$.\\
\verb|       |Como $x\in A - C$, temos que $x\in A$ e $x\not\in C$.\\
\verb|       |Como $x\in A$ e $x\not\in B$, temos que $x \in A - B$.\\
\verb|       |Como $x\in A - B$ e $A- B \subseteq C$, temos que $x\in
C$.\\
\verb|       |Como $x\in C$ e $x\not\in C$, temos uma contradição.\\
\verb|    |Assim, temos que $x\in B$.\\
\verb|  |Logo, se $x\in A - C$ então $x\in B$.\\
Portanto, $A,B$ e $C$ são conjuntos e que $A - B \subseteq C$ então se
$x \in A - C$, temos que  $x \in B$
\end{flushleft}
\end{Example}