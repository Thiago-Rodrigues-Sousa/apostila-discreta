\chapter{Indução Estrutural}\label{cap11}


\epigraph{Correctness is clearly the prime quality. If a system does
  not do what it is supposed to do, then everything else about it
  matters little.}{Berthrand Meyer, Criador da linguagem Eiffel, a
  primeira linguagem orientada a objetos.}

\section{Motivação}

Como vimos nos dois capítulos anteriores (capítulos \ref{cap9} e
\ref{cap10}), a indução matemática é uma técnica de demonstração
aplicável em diversas situações. Nestes capítulos, apresentamos um
enfoque sobre a indução matemática que essencialmente abordou
problemas matemáticos, porém, esta técnica é aplicável a números mas
também a provas de propriedades sobre estruturas de dados recursivas e
algoritmos sobre estas. A este tipo de demonstração de indução, damos
o nome de indução estrutural.

O objetivo deste capítulo é o estudo da indução estrutural para
demonstração de correção sobre alguns algoritmos sobre
estruturas de dados simples como listas. Para
evitar problemas relativos à utilização de atribuição de variáveis, e
aspectos específicos de linguagens de programação,
representaremos estruturas de dados, como definições sintáticas e
algoritmos como funções, de maneira similar ao que fizemos no capítulo
\ref{cap1}.

\section{Indução Estrutural}

Conforme apresentado no capítulo \ref{cap1}, definições sintáticas de
conjuntos de termos devem possuir elementos iniciais (casos base) e,
opcionalmente, formas de se construir termos mais complexos a
partir de termos existentes (passo(s) indutivos).
De maneira simples, a técnica de indução estrutural pode ser resumida
da seguinte maneira: Seja $P$ a propriedade a ser demonstrada para
todo termo $t$ pertencente a um conjunto $\mathcal{T}$. Para constatar
que $P(t)$ é verdade basta mostrar que esta propriedade é verdadeira
para cada um dos casos base e passos indutivos da definição do
conjunto $\mathcal{T}$.

As próximas seções apresentarão a indução estrutural em exemplos
concretos: números naturais na notação de Peano ($\Nat$), listas e
árvores binárias.

\subsection{Números Naturais na Notação de Peano}


Conforme apresentado no capítulo \ref{cap1}, o conjunto $\Nat$, dos
termos que representam números naturais na notação de Peano, pode ser definido pelas seguintes regras:
\[
   \begin{array}{l}
     \zero \in \Nat\\
     \text{se $n \in \Nat$ ent\~ao $\suc\, n\in\Nat$}
   \end{array}
\]
Nesta notação, o número natural $3$ é representado pelo termo
$\suc\,(\suc\,(\suc\,\zero))$, isto é todo número natural ou é
representado pelo termo $\zero$ ou por uma sequência de $n$ $\suc$'s
que terminam com a constante $\zero$.

Usando esta notação, podemos definir como funções recursivas operações
sobre números naturais, como por exemplo, a adição:
\[
\begin{array}{lclc}
plus(\zero,m) & = & m & (1)\\
plus(\suc\,n,m) & = & \suc (plus(n,m)) & (2)
\end{array}
\]
Numeramos as equações da definição de $plus$ para referenciar uma
equação específica quando necessário.
Como um exemplo da utilização da função $plus$, considere a soma: $2 + 3$, que é
representada como $plus(\suc(\suc\,\zero),\suc(\suc(\suc\,\zero)))$:
\[
\begin{array}{lcl}
plus(\suc(\suc\,\zero),\suc(\suc(\suc\,\zero))) & \equiv \\
\suc(plus(\suc\,\zero,\suc(\suc(\suc\,\zero)))) & \equiv &
\{\text{pela equação $2$ de $plus$}\}\\
\suc(\suc(plus(\zero,\suc(\suc(\suc\,\zero))))) & \equiv &
\{\text{pela equação $2$ de $plus$}\}\\
\suc(\suc(\suc(\suc(\suc\,\zero)))) & & \{\text{pela equação $1$ de $plus$}\}
\end{array}
\]
Note que o processo de execução da função $plus$ é completamente
determinado por sua definição: se o primeiro parâmetro desta função é
igual a $\zero$, o seu resultado será o segundo parâmetro ($m$, na
definição de $plus$). Porém, se o primeiro parâmetro não for igual a
$\zero$, necessariamente este deverá ser $\suc\,n$, para algum
$n\in\mathcal{N}$, e o resultado será o sucessor da chamada recursiva
$plus(n,m)$. É útil que você faça mais algumas execuções da função
$plus$ até que você tenha compreendido completamente seu
funcionamento.

De acordo com a definição da função $plus$, note que $\forall m. plus(\zero,m)
\equiv m$ (pela equação $1$ de $plus$), porém não é imediato que
$\forall n. plus(n,\zero)$. Isto se deve que o termo $plus(\zero,m)$ pode ser
reduzido imediatamente a $m$, de acordo com a equação $1$ de $plus$,
enquanto $plus(n,\zero)$ não, uma vez que não é possível determinar se $n$
é ou não igual a $\zero$.

Em lógica, dizemos que a expressão $plus(\zero,m)$ é \textit{igual por
  definição}\footnote{Tradução livre do termo: ``definitionally equal
  to''.} a $m$, uma vez que esta igualdade pode ser deduzida
diretamente pela definição de $plus$, executando-a. Note que apesar de
evidentemente verdadeira, a igualdade $plus(n,\zero)$ não pode ser
considerada igual por definição a $n$, visto que não existe uma única
possibilidade de execução para esta expressão pois, $n$ pode ser ou
não igual $\zero$. Neste caso, se desejamos demonstrar tal igualdade,
devemos prová-la usando indução. Antes disso, vamos apresentar a
definição do princípio de indução estrutural para o conjunto
$\mathcal{N}$.

\begin{Definition}[Indução sobre $\Nat$]
Seja $P$ uma propriedade qualquer sobre elementos de
$\mathcal{N}$. Podemos demonstrar que $\forall n. n\in\mathcal{N}\to
P(n)$ usando a seguinte fórmula:
\[
P(\zero)\land\forall n.n\in\mathcal{N}\land P(n) \to P(suc\,n)
\]
\end{Definition}
Note que esta definição é exatamente igual ao princípio de indução
matemática que vimos no capítulo \ref{cap9}, a menos do uso do
conjunto  $\Nat$ ao invés de $\mathbb{N}$ e das constantes $\zero$ e
$\suc$.

A seguir, apresentamos a prova da propriedade $\forall n. n\in\Nat\to
plus(n,\zero) \equiv n$, usando indução estrutural.

\begin{Theorem}
Para todo $n\in\Nat$, $plus(n,\zero) \equiv n$.
\end{Theorem}
\begin{proof}
Esta demonstração será por indução sobre $n$.
\begin{enumerate}
  \item Caso base ($n = \zero)$. Neste caso, temos que
  \[
\begin{array}{lcl}
plus(\zero,\zero) & \equiv \\
\zero                   & & \{\text{pela equação 1 de }plus\}
\end{array}
  \]
  conforme requerido.
  \item Passo indutivo $(n = \suc\, n')$. Suponha $n'\in\Nat$ arbitrário e que
    $plus(n',\zero) \equiv n'$. Temos que:
    \[
    \begin{array}{lcl}
plus(\suc\,n',\zero) & \equiv \\
\suc(plus(n',\zero))  & \equiv & \{\text{pela equação 2 de }plus\}\\
\suc\, n'                   & & \{\text{pela hipótese de indução}\}
    \end{array}
    \]
    conforme requerido.
\end{enumerate}
\end{proof}
Observe que no passo indutivo desta demonstração, consideramos que o
primeiro parâmetro $n$ é tal que $n = \suc\,n'$. A hipótese de indução
é obviamente definida para $n'$, o antecessor de $n$.

Usualmente, provas por indução estrutural sobre funções devem realizar
a indução sobre o parâmetro recursivo da definição da função. Como a
função $plus$ é definida recursivamente sobre seu $1^o$ parâmetro,
provas sobre esta devem ser feitas utilizando indução sobre este.
Como um segundo exemplo de demonstração por indução estrutural,
considere demonstrar que a adição é uma operação associativa, isto é:
\[
plus(n, plus(m,p)) \equiv plus (plus(n,m),p)
\]
Essa propriedade é demonstrada no teorema seguinte.
\begin{Theorem}[$plus$ é uma operação associativa]
Para todo $n,m,p\in\Nat$, temos que $plus(n, plus(m,p)) \equiv plus
(plus(n,m),p)$.
\end{Theorem}
\begin{proof}
Esta prova será por indução sobre $n$. Suponha $m,p\in\Nat$ arbitrários.
\begin{enumerate}
  \item Caso base ($n = \zero$). Temos que:
  \[
\begin{array}{lcl}
plus(\zero,plus(m,p)) & \equiv & \\
plus(m,p) & & \{\text{pela equação 1 de }plus\}
\end{array}
\]
conforme requerido.
\item Passo indutivo ($n = \suc\, n')$: Suponha $n'\in\Nat$ arbitrário
  e que $plus(n', plus(m,p)) \equiv plus (plus(n',m),p)$. Temos que:
\[
\begin{array}{lcl}
plus(\suc\,n', plus(m,p)) & \equiv & \\
\suc(plus(n',plus(m,p))) & \equiv & \{\text{pela equação 2 de }plus\}\\
\suc(plus(plus(n',m),p)) & \equiv & \{\text{pela hipótese de
  indução}\}\\
plus(\suc(plus(n',m)),p) & \equiv & \{\text{pela equação 2 de
}plus\}\\
plus(plus(\suc\,n',m),p) & &\{\text{pela equação 2 de }plus\}\\
\end{array}
\]
conforme requerido.
\end{enumerate}
\end{proof}

\subsection{Exercícios}

\begin{enumerate}
  \item Prove que a soma de números na notação de Peano é uma operação
    comutativa, isto é, prove que:
   \[
   \forall n. n\in\Nat\to \forall m. m\in\Nat\to plus(n,m) \equiv plus(m,n)
   \]
  \item Considere a seguinte definição alternativa da soma na notação
    de Peano.
\[
\begin{array}{lccc}
plus_{alt}(n,\zero) & = & n & (1)\\
plus_{alt}(n,\suc\,m) & = & \suc(plus_{alt}(n,m)) & (2)
\end{array}
\]
Prove que para quaisquer valores $n,m\in\Nat$, $plus_{alt}(n,m) \equiv
plus(n,m)$.
  \item Defina a função $mult(n,m)$ que realiza a multiplicação de
    números naturais na notação de Peano.
  \item Prove que a função de multiplicação definida por você é uma
    operação comutativa.
\end{enumerate}


\subsection{Listas}

Nesta seção, consideraremos algumas funções sobre listas e provas de
propriedades sobre estas utilizando indução estrutural. No capítulo
\ref{cap1}, apresentamos o conjunto de listas cujos elementos são de $\TType$,
 $\TList\, \TType$, como sendo os termos definidos recursivamente como:
  \[
  \begin{array}{l}
    [\,] \in \TList\,\TType \\
    \text{se $t \in \TType$ e $ts \in \TList\,\TType$ ent\~ao  $t :: ts \in \TList\,\TType$}
  \end{array}
  \]
Por questão de simplicidade, vamos considerar que os elementos de
listas são valores booleanos, cuja definição apresentamos a seguir:
  \[
      \begin{array}{l}
        T \in\Bool\\
        F \in\Bool
      \end{array}
  \]
É importante notar que esta simplificação será feita apenas para fins
de facilitar o entendimento e a escrita de exemplos. Todas as funções
e suas respectivas propriedades são válidas para listas cujos
elementos pertencem a um conjunto $\TType$ qualquer. Desta forma,
representaremos a lista que contém os elementos $T$ e $F$, nesta
ordem, como: $T :: F :: [\,]$. Note que o valor que representa uma
lista vazia ($[\,]$) possui funcionalidade similar ao um ponteiro
``nulo'' em implementações de listas encadeadas em linguagens de
programação como C/C++, a de indicar o final da lista em questão.

Como exemplos de funções sobre listas, considere, as funções para
determinar o número de elementos ($length$) e concatenação de duas
listas (\texttt{++}) apresentadas a seguir:

\[
\begin{array}{lclr}
  \length\,\,[\,] & = & 0 & (1)\\
  \length\,\,(t :: ts) & = & 1 + \length\,\, ts & (2)\\
\end{array}
\]

 \[
  \begin{array}{lclr}
    [\,] \text { ++ } ys & = & ys & (1)\\
    (x :: xs) \text{ ++ }  ys & = & x :: (xs\text{ ++ } ys) & (2)
  \end{array}
  \]

Novamente, numeramos as equações para futura referência. Antes de
apresentarmos um primeiro exemplo de propriedade a ser demonstrada
para listas, vamos definir o princípio de indução para listas.

\begin{Definition}[Indução Estrutural para $\TList\,\,\TType$]
Seja $\TType$ um conjunto qualquer. Seja $P$ uma propriedade sobre o
conjunto de listas finitas de elementos do conjunto $\TType$,
$\TList\,\,\TType$. Então, podemos provar que $\forall
t. t\in\TList\,\,\TType\to P(t)$ usando a seguinte fórmula:
\[
P([\,])\land\forall x. x\in \TType \to xs \in \TList\,\,\TType \land
P(xs) \to P(x :: xs)
\]
\end{Definition}
Intuitivamente, podemos provar que uma propriedade é verdadeira para
todas as listas finitas se formos capazes de provar que esta vale para
a lista vazia e, além disso,  provarmos que a propriedade continua
sendo verdadeira se inserirmos um novo elemento em uma lista qualquer
para a qual a propriedade em questão era válida.

Como um exemplo de demonstração por indução sobre listas,
considere a seguinte propriedade que pode ser usada para caracterizar
a correção de um algoritmo de concatenação de duas listas: o tamanho
da concatenação de duas listas $xs$ e $ys$ é igual a soma dos tamanhos
de cada uma destas listas. Mais formalmente, a propriedade em questão é:

\[
\forall xs. xs \in \TList\,\TType\to \forall
ys. ys\in\TList\,\TType\to \length (xs \texttt{ ++ }ys) = \length\: xs +
length\: ys
\]
Essa propriedade é facilmente demonstrada por indução sobre a primeira
lista ($xs$). A indução será feita sobre a primeira lista devido ao
fato de que a concatenação é definida recursivamente sobre a primeira
lista fornecida como parâmetro.
\begin{Theorem}
Seja $\TType$ um conjunto qualquer de termos. Então para todo
$xs,ys\in\TList\,\TType$, temos que $\length (xs \texttt{ ++ }ys) = \length\: xs +
length \:ys$.
\end{Theorem}
\begin{proof}
A prova será por indução sobre $xs$. Suponha $ys\in\TList\,\TType$
    arbitrário.
\begin{enumerate}
  \item Caso base ($xs = [\,]$). Temos que:
\[
\begin{array}{lcl}
\length([\,]\texttt{ ++ }ys) & \equiv &\\
\length\:ys  & \equiv & \{\text{pela equação 1 de \texttt{++}}\}\\
0 + length\: ys & \equiv & \\
length\:[\,] + length\: ys & & \{\text{pela equação 1 de \texttt{++}}\}
\end{array}
\]
conforme requerido.
\item Passo indutivo. Suponha $x\in\TType$ arbitrário e que $\length (xs \texttt{ ++ }ys) = \length\: xs +
length \:ys$. Temos que:
\[
\begin{array}{lcl}
\length((x :: xs) \texttt{ ++ } ys) & \equiv & \\
\length(x :: (xs \texttt{ ++ }ys)) & \equiv & \{\text{pela equação 2
  de \texttt{++}}\} \\
1 + \length(xs \texttt{ ++ }ys) & \equiv &\{\text{pela equação 2 de
  $\length$}\}\\
1 + \length\:xs + \length\: ys & \equiv &\{\text{pela hipótese de
  indução}\}\\
\length(x :: xs) + \length\:ys   & &\{\text{pela equação 2 de $\length$}\}
\end{array}
\]
conforme requerido.
\end{enumerate}
\end{proof}
Para o nosso próximo exemplo de uma propriedade sobre listas,
considere a função \textit{reverse}, que inverte uma lista fornecida
como parâmetro.
\[
\begin{array}{lclc}
reverse\,[\,] & = & [\,] & (1)\\
reverse\,(x :: xs) & = & reverse\: xs \texttt{ ++ } (x :: [\,]) & (2)
\end{array}
\]
De maneira simples, $reverse$ move o primeiro elemento da lista
fornecida como parâmetro para o final do resultado de se inverter o
restante desta lista. Note que só é possível inserir um elemento na
primeira posição de uma lista. Se desejamos inserir um elemento ao
final de uma lista, devemos concatená-lo ao final e não simplesmente
inserí-lo. Por isso, definimos a função $reverse$ em termos da
operação de concatenação de duas listas.

Como exemplo do funcionamento da função $reverse$, considere a
seguinte execução desta para a lista $T :: F :: T :: [\,]$, apresentada a
seguir:
\[
\begin{array}{lcl}
reverse(T :: F :: T :: [\,]) & \equiv \\
reverse(F :: T :: [\,]) \texttt{ ++ } (T :: [\,]) & \equiv &
\{\text{pela equação 2 de }reverse\} \\
(reverse(T :: [\,]) \texttt{ ++ } (F :: [\,])) \texttt{ ++ } (T :: [\,]) & \equiv &
\{\text{pela equação 2 de }reverse\} \\
((reverse\, [\,] \texttt{ ++ } (T :: [\,])) \texttt{ ++ } (F :: [\,])) \texttt{ ++ } (T :: [\,]) & \equiv &
\{\text{pela equação 2 de }reverse\} \\
(([\,] \texttt{ ++ } (T :: [\,])) \texttt{ ++ } (F :: [\,])) \texttt{ ++ } (T :: [\,]) & \equiv &
\{\text{pela equação 1 de }reverse\} \\
(((T :: [\,])) \texttt{ ++ } (F :: [\,])) \texttt{ ++ } (T :: [\,]) & \equiv &
\{\text{pela equação 1 de \texttt{++}}\} \\
(T :: ([\,] \texttt{ ++ } (F :: [\,]))) \texttt{ ++ } (T :: [\,]) & \equiv &
\{\text{pela equação 2 de \texttt{++}}\} \\
(T :: (F :: [\,])) \texttt{ ++ } (T :: [\,]) & \equiv &
\{\text{pela equação 1 de \texttt{++}}\} \\
T :: ((F :: [\,]) \texttt{ ++ } (T :: [\, ])) & \equiv & \{\text{pela
   equação 2 de \texttt{++}}\}\\
T :: (F :: ([\,] \texttt{ ++ } (T :: [\, ]))) & \equiv & \{\text{pela
  equação 2 de \texttt{++}}\}\\
T :: (F :: (T :: [\, ])) &  & \{\text{pela
  equação 1 de \texttt{++}}\}
\end{array}
\]
Como exemplo de propriedade sobre a função $reverse$, apresentaremos
como esta se relaciona com a operação de concatenação de listas.
\begin{Theorem}
Seja $\TType$ um conjunto qualquer de termos. Então para todo
$xs,ys\in\TList\,\TType$, temos que $reverse(xs \texttt{ ++ } ys)
\equiv reverse\: ys \texttt{ ++ } reverse\: xs$.
\end{Theorem}
\begin{proof}
A prova será por indução sobre $xs$. Suponha $ys \in \TList\TType$ arbitrário.
\begin{enumerate}
  \item Caso base ($xs = [\,]$). Temos que:
\[
\begin{array}{lcl}
reverse([\,] \texttt{++} ys) & \equiv & \\
reverse \: ys  & \equiv & \{pela equação 1 de \texttt{++}\} \\
reverse\: ys ++ [\,] &  & \{pela equação 1 de \texttt{++}\} \\
\end{array}
\]
conforme requerido.
\item Passo indutivo. Suponha $x\in\TType$ arbitrário e que
  $reverse(xs \texttt{++} ys) \equiv reverse\: ys \texttt{++}
  reverse\: xs$. Temos que:
\[
\begin{array}{lcl}
reverse((x :: xs) \texttt{++} ys) & \equiv & \\
reverse(x :: (xs \texttt{++} ys)) & \equiv & \{\text{pela equação 2 de
\texttt{++}}\}\\
reverse(xs \texttt{++} ys) \texttt{++} (x :: [\,]) & \equiv &
\{\text{pela equação 2 de }reverse\}\\
(reverse\: ys \texttt{++} reverse\: xs) \texttt{++} (x :: [\,]) &
\equiv & \{pela hipótese de indução\} \\
reverse\: ys \texttt{++} (reverse\: xs \texttt{++} (x :: [\,])) &
\equiv & \{\texttt{++} \text{ é associativo}\} \\
reverse\: ys \texttt{++} reverse(x ::xs)  &
 & \{\text{pela equação 2 de }reverse\} \\
\end{array}
\]
\end{enumerate}
\end{proof}
Note que nesta demonstração usamos, sem demonstrar, o fato de que a
operação de concatenação de listas é associativa, isto é:
\[
\forall xs.\forall ys. \forall zs. xs \in \TList\,\TType \land ys \in \TList\,\TType \land zs \in
\TList\,\TType \to xs \texttt{++} (ys \texttt{++} zs) \equiv (xs
\texttt{++} ys) \texttt{++} zs
\]
Esta demonstração simples é deixada como exercício para o leitor.

\subsection{Exercícios}

\begin{enumerate}
  \item Prove que a concatenação de listas é uma operação associativa.
  \item Prove o seguinte teorema envolvendo as funções $\length$ e
    $reverse$: Para toda lista $xs \in \TList\,\TType$, temos que
    $\length(reverse \:xs)\equiv \length\:xs$.
 \item Considere a seguinte definição alternativa de uma função que
   inverte uma dada lista:
\[
\begin{array}{lclc}
  reverse_{alt}\:xs & = & rev\: xs \: [\,] & (1)\\
   & \\
  rev \: [\,] \: ys & = & ys & (1) \\
  rev \: (x :: xs) \: ys & = & rev \: xs \:\:(x :: ys) & (2)\\
\end{array}
\]
 \begin{enumerate}
   \item Mostre, passo a passo, a execução de $reverse_{alt}(T :: F ::
     F :: [\,])$.
   \item Prove que para toda lista $xs \in \TList\,\TType$,
     $reverse_{alt}xs \equiv reverse\: xs$, em que $reverse$ é a
     primeira definição apresentada de reverse neste texto.
  \end{enumerate}
  \item Prove que para toda lista $xs\in\TList\,\TType$,
    $reverse(reverse\:xs) \equiv xs$.
\end{enumerate}

% \subsection{Árvores Binárias}

% Encerraremos este capítulo apresentando demonstrações simples de
% propriedades sobre árvores binárias, um tipo de estrutura de dados
% muito importante na ciência da computação.

% De maneira simples, uma árvore binária é vazia ou consiste de um nó
% que armazena um elemento e possui duas sub árvores (também binárias),
% denominadas sub árvore esquerda e direita, respectivamente. A seguir,
% apresentamos a definição formal do conjunto de árvores binárias.

%\begin{Definition}[Árvores binárias]
%Seja $\TType$ um tipo qualquer. O conjunto $\TTree\,\TType$ de árvores cujo
%\end{Definition}

\section{Notas Bibliográficas}
