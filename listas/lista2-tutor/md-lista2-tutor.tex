\documentclass[11pt,a4paper]{report}

\usepackage[brazil]{babel}
\usepackage[latin1]{inputenc}
\usepackage{amsmath}
\usepackage{amsfonts}
\usepackage{fullpage}
\usepackage{latexsym}
%\usepackage[linesnumbered, vlined]{algorithm2e}

\newcounter{conta}
 
\begin{document}
 
 \hfill DECEA - UFOP \\
{\it Matem\'atica Discreta I}
 \hfill $\mbox{2}^{\mbox{\underline{o}}}$ semestre de 2013 \\
Professor: \parbox[t]{14cm}{Rodrigo Geraldo Ribeiro \\
                     e-mail: rodrigogribeiro@decea.ufop.br}
 
\noindent {\bf Lista de Exerc\'icios} \hfill {\bf Tema: L\'ogica de Predicados, Conjuntos e Combinat\'oria}
 
\vspace*{3mm}
\begin{enumerate}
   \item Traduza as seguintes sente\c{c}as para f\'ormulas da l\'ogica de predicados utilizando os seguintes s\'imbolos
         para representar predicados:
   \[
      \begin{array}{ccl}
        h(x) & = & x \text{ \'e um homem.}\\
        l(x) & = & x \text{ \'e um livro.}\\
        p(x) & = & x \text{ pensa}\\
        g(x,y) & = & x \text{ gosta de }y\\
      \end{array}
   \]
   \begin{enumerate}
     \item Todos os homens pensam.
     \item Todos os homens que pensam, gostam de livros.
     \item Todos os homens gostam de livros, menos Di\'ogenes.
     \item Nenhum livro pensa.
     \item Pelo menos um homem n\~ao pensa.
     \item Alguns homens n\~ao pensam, mas gostam de todos os livros.
   \end{enumerate}
   \item Para cada uma das f\'ormulas a seguir, indique se ela \'e verdadeira ou falsa, 
         quando o universo de discurso \'e cada um dos seguintes conjuntos: $\mathbb{N}$: conjunto
         dos n\'umeros naturais, $\mathbb{Z}$: conjunto dos n\'umeros inteiros e $\mathbb{R}$ conjunto
         dos n\'umeros reais.
         \begin{center}
         \begin{tabular}{|l|c|c|c|}
             \hline
             F\'ormula & $\mathbb{N}$ & $\mathbb{Z}$ & $\mathbb{R}$ \\ \hline
             $\exists x. x^2 = 2$ & & &\\ \hline
             $\forall x. \exists y. x^2 = y$ & & & \\ \hline
             $\forall x. x\neq 0 \rightarrow \exists y. xy = 1$ & & & \\ \hline
             $\exists x. \exists y. (x + 2y^2 = 2) \land (2x + 4y = 5)$ & & & \\
             \hline
         \end{tabular}
         \end{center}
         \item Prove os seguintes sequentes utilizando dedu\c{c}\~ao natural.
         \begin{enumerate}
           \item $\forall x. p(x) \rightarrow q(x) \vdash \forall x. p(x) \rightarrow \forall x. q(x)$
           \item $\exists x. \neg p(x) \vdash \neg \forall x. p(x)$
           \item $\forall x. a(x)\rightarrow b(x) \lor c(x), \forall x. \neg b(x)\vdash \forall x. a(x)\rightarrow\forall x.b(x)\lor c(x)$
         \end{enumerate}
         \item Prove as seguintes equival\^encias utilizando regras alg\'ebricas para l\'ogica de predicados.
         \begin{enumerate}
            \item $\forall x. p(x) \rightarrow \neg q(x) \equiv \neg \exists x. p(x) \land q(x)$
            \item $\neg \forall x.\exists y. r(x,y)\land \neg p(x,y)\equiv\exists x.\forall y.r(x,y)\rightarrow p(x,y)$
         \end{enumerate}
         \item Sejam $A$ e $B$ dois conjuntos quaisquer. Define-se o operador de diferen\c{c}a sim\'etrica, $A\bigtriangleup B$, como
           \[
           A \bigtriangleup B = (A \cup B) - (A \cap B)
           \]
           Com base nesta equival\^encia e nas outras equival\^encias da teoria de conjuntos, prove:
          \begin{enumerate}
            \item $A \bigtriangleup A = \emptyset$
            \item $A \bigtriangleup B = B \bigtriangleup A$
            \item $A \bigtriangleup U = \overline{A}$
          \end{enumerate}
          \item Prove, usando equival\^encias alg\'ebricas da teoria de conjuntos, as seguintes equival\^encias: 
            \begin{enumerate}
              \item $(A - C) \cap (C - B) = \emptyset$.
              \item $(A\cup (B - A)) = A \cup B$
              \item $(A - (B - A)) = \emptyset$
            \end{enumerate}
          \item Prove o conhecido Paradoxo de Russell:
            \[
            \begin{array}{lcl}
              P(x) &= &x \not\in x\\
              S &= &\{x\,|\, P(x)\}
            \end{array}
            \]
             utilizando regras alg\'ebricas da teoria de conjuntos. Para isso, voc\^e dever\'a mostrar que tanto $S\in S$ e
             $S\not\in S$ s\~ao equivalentes a $F$.
             \item[\ ] \textbf{Aten\c{c}\~ao:} Nas quest\~oes envolvendo o conte\'udo de an\'alise combinat\'oria \'e \textbf{imprescind\'ivel}
                       a apresenta\c{c}\~ao de um texto descrevendo o racioc\'inio utilizado para a solu\c{c}\~ao.
             \item Suponha que, em Springfield, tenham entrado em cartaz 3 filmes, 2 pe\c{c}as de teatro e que Moe tenha dinheiro apenas para um
               desses eventos. Quantas possibilidades de programas Moe possui?
             \item Em uma lanchonete, existem 5 tipos de doces, 7 salgados e 3 bebidas. Supondo que Kyle possui dinheiro apenas para uma bebida 
                   e deve optar por um doce ou um salgado, quantas op\c{c}\~oes de lanche Kyle possui?
             \item Um amigo apresentou-me 10 livros de matem\'atica discreta, 5 de teoria da computa��o e 8 de compiladores. Al\'em disso, esse
                   amigo disse que eu poderia escolher dois livros desde que esses n\~ao fossem da mesma mat\'eria. Quantas op\c{c}\~oes de 
                   escolha eu possuo?
             \item De quantas maneiras duas pessoas podem estacionar seus carros em uma garagem com 6 vagas?
             \item Quantos n\'umeros de dois d\'igitos podemos formar com os algarismos do conjunto $A=\{1,2,3,4,5\}$?
             \item Quantos subconjuntos de dois elementos podemos formar a partir de $A=\{1,2,3,4,5\}$?
             \item De quantas maneiras podemos distribuir 8 brinquedos entre 3 garotos de maneira que os dois mais velhos recebam 3 briquedos e o
                   mais novo 2 brinquedos?
             \item Quantas s\~ao as permuta\c{c}\~oes da palavra BRASIL em que a letra B ocupa o primeiro lugar ou R ocupa o segundo ou 
                   L ocupa o sexto lugar?
             \item Se uma urna cont\'em 4 bolas vermelhas, 7 verdes, 9 azuis e 6 amarelas, qual \'e o menor n\'umero de bolas que devemos retirar
                   (sem olhar) de maneira a ter certeza de termos retirado pelo menos 3 da mesma cor?
             \item \'E verdade que em qualquer grupo de 20 pessoas, pelo menos 3 delas nasceram no mesmo dia da semana? Justifique sua resposta.
             \item De quantas maneiras podemos permutar tr\^es a's, tr\^es b's e tr\^es c's de modo que 3 letras iguais nunca sejam adjacentes?
\end{enumerate}
\end{document}
