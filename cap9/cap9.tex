\chapter{Indução Matemática}\label{cap9}

\epigraph{Induction makes you fell guilty for getting something out of
nothing, and it is artificial, but it is one of the greatest ideas of
civilization.}{Helbert S. Wilf}

\section{Motivação}

Na ciência e filosofia usamos, essencialmente, dois tipos de
raciocínio distintos: o dedutivo e indutivo. O raciocínio dedutivo é
governado por leis da lógica e foi tema de grande parte do curso de
matemática discreta. Se certo fato é deduzido usando lógica, este é
irrefutável, visto que, sistemas dedutivos para lógica são corretos e
completos. O raciocínio indutivo, por sua vez, é o que usamos quando
inferimos um padrão de comportamento futuro a partir de experiências
realizadas no passado.

Outro tipo de raciocínio indutivo é a chamada indução matemática, que
é usado para provar propriedades sobre números naturais. O objetivo
deste capítulo é apresentar esta técnica de prova e como esta é usada
para demonstração de diversos fatos em matemática e em ciência da
computação.


\section{Introdução à indução matemática}

A técnica de indução matemática pode ser utilizada para demonstrar
conclusões que possuam a seguinte estrutura:
\[
\forall n. n\in \mathbb{N}\to P(n)
\]
Evidentemente, para demonstrar a fórmula anterior devemos provar que
a propriedade $P$ é verdadeira para cada um dos valores do conjunto
\[\mathbb{N} = \{0,1,2,3...\}\]
Como existem infinitos valores em $\mathbb{N}$, não podemos
simplesmente verificar se a propriedade em questão é verdadeira para
cada um destes valores. Então, como demonstrar tais propriedades? A
chave da indução matemática está na própria estrutura do conjunto de
números naturais. Note que para listar todos os valores de
$\mathbb{N}$, tudo o que temos que fazer é iniciar com $0 \in
\mathbb{N}$ e repetidamente somar $1$, produzindo assim, um novo
número natural. Assim, para mostrar que todos os números naturais
possuem uma certa propriedade $P$, basta:
\begin{itemize}
    \item Mostrar que $0$ possui a propriedade $P$, isto é, $P(0)$ é
      verdadeiro.
    \item Mostrar que sempre que um número natural $n$ possuir a
      propriedade $P$ então o sucessor de $n$, $n + 1$, também
      possuirá essa propriedade. Isto é, devemos provar que $\forall
      n. n\in\mathbb{N} \land P(n) \to P(n+1)$.
\end{itemize}
Exatamente esta observação motiva a seguinte estratégia de prova.
\begin{ProofStrategy}[Para provar uma conclusão da forma $\forall
  n. n\in\mathbb{N} \to P(n)$.]\label{simpleinduction}
Prove que a seguinte fórmula é verdadeira.
\[
P(0)\land\forall n.n\in\mathbb{N}\land P(n)\to P(n+1)
\]
\begin{flushleft}
\textbf{Texto}:
Caso Base: [Prova de $P(0)$].\\
Passo Indutivo:[Prova de $\forall n. n\in\mathbb{N}\land P(n)\to P(n+1)$].
\end{flushleft}
\end{ProofStrategy}
Porém, como somente a demonstração destes dois passos pode comprovar
que $\forall n. n\in\mathbb{N} \to P(n)$? A idéia é bastante
simples. Note que ao usarmos indução matemática, provamos as seguintes
fórmulas:
\begin{itemize}
    \item $P(0)$
    \item $\forall n. n\in \mathbb{N} \land P(n) \to P(n+1)$
\end{itemize}
Observe que temos que para qualquer número natural n, $P(n) \to
P(n+1)$. Como esta fórmula é verdadeira para todo $n\in\mathbb{N}$,
temos que esta é verdadeira também para $n = 0$. Eliminando o
quantificador universal substituindo $n$ por $0$, obtemos
\[
0 \in \mathbb{N} \land P(0) \to P(1)
\]
Porém, é óbvio que $0 \in \mathbb{N}$ e como provamos que $P(0)$, por
eliminação da implicação podemos concluir que $P(1)$ também é
verdadeiro. Repetindo este processo para $n = 1$, temos que a seguinte
implicação é obtida a partir da eliminação do quantificador universal:
\[
1 \in\mathbb{N}\land P(1) \to P(2)
\]
o que nos permite deduzir que $P(2)$ é verdadeiro. Note que podemos
repetir esse processo para concluir que $P$ é verdadeira para qualquer
$n \in\mathbb{N}$. A seguir, apresentaremos um exemplo simples de uma
propriedade provável por indução matemática.

\begin{Theorem}
Para todo $n \in \mathbb{N}$, temos que $\sum_{k = 0}^n2^k =
2^{n + 1} - 1$
\end{Theorem}
\begin{Commentary}
Note que este teorema pode ser expresso pela seguinte fórmula:
\[
\forall n. n\in\mathbb{N} \land \sum_{k = 0}^n2^k = 2^{n+1} - 1
\]
em que a propriedade $P(n)$ é
\[
\sum_{k = 0}^n2^k = 2^{n+1} - 1
\]
logo, de acordo com a estratégia de prova \ref{simpleinduction},
devemos provar as seguintes fórmulas:
\begin{itemize}
    \item $P(0)$
    \item $\forall n. n\in\mathbb{N}\land P(n) \to P(n+1)$
\end{itemize}
em que $P(0)$ é dada por
\[
\sum_{k = 0}^02^k = 2^{0+1} - 1
\]
que é facilmente demonstrada como verdadeira pela seguinte equação:
\[
\begin{array}{lcl}
\sum_{k = 0}^02^k & = \\
2^0                       & = & \{\text{pela def. da noção }\Sigma\}\\
1                          & = &\\
2 - 1                    & = & \\
2^{0 + 1} - 1
\end{array}
\]
Logo, $P(0) = \sum_{k = 0}^02^k = 2^{0+1} - 1$ é verdadeiro.

No próximo passo, temos que demonstrar o passo indutivo, que é dado
pela seguinte fórmula:
\[
\forall n. n\in\mathbb{N}\land \left(\sum_{k = 0}^n 2^k = 2^{n + 1} -
  1\right) \to \left(\sum_{k = 0}^{n + 1} 2^k = 2^{n + 2} -
  1\right)
\]
Para demonstrar o passo indutivo, supomos $n\in\mathbb{N}$ arbitrário
e que $\sum_{k = 0}^n 2^k = 2^{n + 1}  - 1$. Usualmente, damos o nome
de hipótese de indução a suposição de que a propriedade que desejamos
provar é verdadeira para um número $n$ qualquer, isto é que $P(n)$ é
verdadeira.

Para concluir a prova, resta mostrar que a seguinte igualdade é
verdadeira:
\[
\sum_{k = 0}^{n+1} 2^k = 2^{n + 2} - 1
\]
que é facilmente demonstrável usando a hipótese de indução e
álgebra. A demonstração desta equação é apresentada a seguir.
\[
\begin{array}{lcl}
\sum_{k = 0}^{n + 1}2^k & = & \\
\sum_{k = 0}^{n}2^k + 2^{n + 1} & = &\{\text{pela definição de
}\Sigma\}\\
2^{n + 1} - 1 + 2^{n+1} & = &\{\text{pela hipótese de indução}\}\\
2^{n+2} -1
\end{array}
\]
A seguir, apresentamos a versão final do texto desta demonstração.
\end{Commentary}
\begin{proof}
\verb| |\\
\begin{enumerate}
  \item[\ ]Caso Base: Para $n = 0$, temos:
\[
\begin{array}{lcl}
\sum_{k = 0}^02^k & = \\
2^0                       & = & \{\text{pela def. da notação }\Sigma\}\\
1                          & = &\\
2 - 1                    & = & \\
2^{0 + 1} - 1
\end{array}
\]
  \item[\ ]Passo indutivo: Suponha $n$ arbitrário. Suponha $n
    \in\mathbb{N}$ e que $\sum_{k = 0}^n 2^k = 2^{n + 1} - 1$. Temos:
\[
\begin{array}{lcl}
\sum_{k = 0}^{n + 1}2^k & = & \\
\sum_{k = 0}^{n}2^k + 2^{n + 1} & = &\{\text{pela definição de
}\Sigma\}\\
2^{n + 1} - 1 + 2^{n+1} & = &\{\text{pela hipótese de indução}\}\\
2^{n+2} -1
\end{array}
\]
\end{enumerate}
\end{proof}
Apresentaremos mais alguns exemplos de demonstração por indução.
\begin{Theorem}
Para todo $n\in\mathbb{N}$, $3\,|\,(n^3 - n)$.
\end{Theorem}
\begin{Commentary}
Novamente, utilizaremos indução matemática. Note que o enunciado do
teorema é dado por:
\[
\forall n. n\in\mathbb{N} \to 3\,|\,(n^3 - n)
\]
em que a propriedade a ser demonstrada é
\[
3\,|\,(n^3 - n)
\]
Note que $3\,|\,(n^3 - n)$ é na verdade uma fórmula envolvendo um
quantificador existencial.
\[
3\,|\,(n^3 - n) \equiv \exists k. k\in\mathbb{N}\land n^3 - n = 3k
\]
Logo, temos que mostrar a veracidade das seguintes fórmulas, para
demonstrar o teorema usando indução:
\begin{itemize}
  \item $3\,|\,(0^3 - 0)$
  \item $\forall n. n\in\mathbb{N}\land 3\,|\,(n^3 - n)\to
    3\,|\,((n+1)^3 - (n + 1))$
\end{itemize}
A demonstração de $3\,|\,(0^3 - 0)$ envolve provar a seguinte fórmula
\[
\exists k. k\in\mathbb{N}\land 3k = 0
\]
que é evidentemente verdadeira, basta fazer que $k = 0$.

Para o passo indutivo, devemos demonstrar que
\[\forall n. n\in\mathbb{N}\land 3\,|\,(n^3 - n)\to
    3\,|\,((n+1)^3 - (n + 1))\]
Iniciamos a demonstração supondo $n\in\mathbb{N}$ arbitrário e que
$3\,|\,(n^3 - n)$ e devemos provar que
\[
\exists k. k\in\mathbb{N}\land (n + 1)^3 - (n+1) = 3k
\]
Logo, devemos encontrar um valor de $k\in\mathbb{N}$ que torne a
fórmula anterior verdadeira. Como $3\,|\,(n^3 - n)$, existe $a \in
\mathbb{N}$ tal que $n^3 - n = 3a$. Para encontrar uma ``dica'' de
qual seria este valor de $k$, vamos desenvolver o polinômio $(n+1)^3 -
(n + 1)$:
\[
\begin{array}{lc}
(n+1)^3 - (n + 1) & =\\
n^3 + 3n^2 + 3n + 1 - (n + 1) & = \\
n^3 + 3n^2 + 3n + 1 - n - 1 & = \\
n^3 -n + 3n^2 + 3n
\end{array}
\]
Porém, pela hipótese de indução, temos que $n^3 - n = 3a$. Com isso,
temos:
\[
\begin{array}{lc}
n^3 -n + 3n^2 + 3n & =\\
3a + 3n^3 + 3n & = \\
3 (a + n^2 + n)
\end{array}
\]
que obviamente é divisível por $3$. Logo, para concluir a
demonstração, basta escolher $k = a + n^2 + n$. O texto final da prova
é apresentado a seguir.
\end{Commentary}
\begin{proof}
\verb| |\\
\begin{enumerate}
  \item[\ ]Caso base: Seja $k = 0$. Temos que $0^3 - 0 = 3.0$, conforme requerido.
  \item[\ ]Passo indutivo: Suponha $n\in\mathbb{N}$ arbitrário e que
    $3\,|\,n^3 - n$. Como $3\,|\,n^3 - n$, existe $a\in\mathbb{N}$ tal
    que $n^3 - n = 3a$. Seja $k = a + n^2 + n$. Temos:
    \[
 \begin{array}{lcl}
   3k & = & \\
   3 (a + n^2 + n) & = & \{\text{por }k = a + n^2 + n\}\\
   3a + 3n^2 + 3n & = & \\
   (n^3 - n) + 3n^2 + 3n & = &\text{pela hipótese de indução}\\
   n^3 -n + 3n^2 + 3n + 1 - 1 & = & \\
   (n^3 + 3n^2 + 3n + 1) - n -1 & = & \\
   (n + 1)^3 - (n +1)
 \end{array}
    \]
    Logo, $3\,|\,(n+1)^3-(n+1)$, conforme requerido.
\end{enumerate}
\end{proof}

\subsection{Exercícios}

\begin{enumerate}
	\item Prove os seguintes teoremas.
	\begin{enumerate}
		\item Para todo $n\in\mathbb{N}$, $\sum_{i=0}^{n}3^i = \frac{3^{n + 1} - 1}{2}$
		\item Para todo $n\in\mathbb{N}$, $\sum_{i=0}^{n}i^{2} =
                  \frac{n(n+1)(2n + 1)}{6}$
                \item Para todo $n\in\mathbb{N}$, $\sum_{i=0}^{n}(8i - 5) =
                  4n^2 - n$
                \item Para todo $n\geq 1$, $5\,|\,(n^5-n)$
                \item Para todo $n\geq 1$, $6\,|\,(7^n-1)$
                \item Para todo $n\geq 1$, $6\,|\,(n^3+5n)$
                \item Seja $H_n = \sum_{i = 1}^n\frac{1}{i}$. Prove que
                  $\sum_{i=1}^nH_i = (n+1)H_n + n$.
		\item Para todo $n\in\mathbb{N}$, $2\,\mid\,(n^{2} + n)$
		\item Seja $A$ um conjunto qualquer tal que $|A| = n$, para algum $n\in\mathbb{N}$. Ent\~ao $|\mathcal{P}(A)| = 2^{n}$.
	\end{enumerate}
\end{enumerate}