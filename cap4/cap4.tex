\chapter{Demonstração de Teoremas}\label{cap4}

\epigraph{A matemática não é uma ciência dedutiva --- isto é um
  clichê. Quando você tenta provar um teorema, você não apenas lista
  as hipóteses, e começa a dedução. O que normalmente fazemos é fazer
  uso de experimentação e tentativa e erro.}{Paul Richard Halmos,
  Matemático.}

\section{Motivação}

Nos capítulos \ref{cap2} e \ref{cap3} foram apresentadas as lógicas
proposicional e de predicados. Para cada uma destas lógicas, estudamos
sua sintaxe, semântica e como verificar consequências lógicas
utilizando dedução natural. Neste capítulo, apresentaremos uma
importante aplicação de tudo que foi visto até o presente momento:
usar estas lógicas para demonstrar teoremas matemáticos.

Mas qual a importância do uso de demonstrações em computação? A única
tecnologia conhecida para garantir a ausência de erros em programas de
computador é provando que este não possui
erros. Evidentemente, isso requer a modelagem de programas em algum
formalismo matemático adequado para esta tarefa, o que está fora do
escopo deste texto. Porém, técnicas elementares de demonstração de
teoremas são a ``base'' para a formalização e verificação de sistemas
computacionais. Logo, é importante que todo estudante de computação
saiba construir e entender demonstrações formais.

\section{Introdução}

Damos o nome de \emph{teorema} a uma sentença matemática que é
verdadeira e pode ser verificada como tal. Teoremas são compostos por
um conjunto, possivelmente vazio, de sentenças, denominadas hipóteses
(ou premissas), que são assumidas como verdadeiras \emph{a priori} e
uma conclusão.
 Normalmente, teoremas
são expressos utilizando variáveis possivelmente livres. Damos o nome
de \emph{instância} de um teorema a uma particular atribuição de
valores às variáveis de um teorema.  A \emph{prova} ou
\emph{demonstração} de um teorema consiste de uma verificação que
mostra que o teorema em questão é verdadeiro, para todas as possíveis
instâncias deste. Note que um teorema só pode ser considerado como
válido se este o for para todas suas instâncias. Para mostrar que um
``teorema''\footnote{Note que uma sentença só pode ser
  considerada um teorema se esta for verdadeira. Afirmar que um
  teorema é falso é apenas um abuso de linguagem utilizado para
  facilitar a exposição deste conceito.} é inválido basta apresentar
uma instância que torna o enunciado deste falso. Damos o nome de
\emph{contra-exemplo} a uma instância que torna uma sentença falsa.

A seguir apresentamos um exemplo que ilustra os conceitos apresentados
no parágrafo anterior.

\begin{Example}
Considere a seguinte sentença matemática:
\begin{center}
\textit{Sejam $x,y$ dois números reais tais que $x > 3$ e $y < 2$. Então, $x^2
- 2y > 5$}.
\end{center}
Esta sentença é um teorema e sua prova será apresentada em um exemplo
posterior. Como esta é um teorema, ela deverá ser composta por um
conjunto de hipóteses e uma conclusão. Note que o enunciado deste
teorema assume que $x,y\in\mathbb{R}$ e que $x > 3$ e $y < 2$. Logo,
estas são as suas hipóteses. A conclusão deste teorema é que
a desigualdade $x^2 - 2y > 5$ deve ser verdadeira. Como um exemplo de
uma possível instância desse teorema são $x = 4$ e $y = 0$ que tornam a
desigualdade $16 - 2.0 > 5$ verdadeira. Evidentemente, caso $x = 3$ e
$y = 2$ (violando, assim, as hipóteses $x > 3$ e $y < 2$) temos que a
conclusão é falsa pois, $9 - 4 = 5 \not> 5$.

Agora, como exemplo de uma sentença inválida, considere:
\begin{center}
\textit{Sejam $x,y$ dois números reais tais que $x > 3$. Então, $x^2
- 2y > 5$.}
\end{center}
Esta sentença não pode ser considerada um teorema por possuir um
contra-exemplo. Seja $x= 4$ e $y = 6$. Temos que $x = 4 > 3$, mas
$16 - 12 \not > 5$, o que torna falsa a sentença em questão.

É importante ter em mente que para demonstrar um teorema devemos
construir uma prova (dedução) de que este é correto para todas as suas
instâncias. Se quisermos mostrar que uma sentença é falsa, basta
apresentar um contra-exemplo.
\end{Example}

Você deve ter percebido que
teoremas possuem a mesma estrutura de sequentes da dedução
natural. Na verdade, todos os sequentes que demonstramos em capítulos
anteriores, são teoremas! Neste capítulo, utilizaremos a dedução
natural para demonstrar a validade de sentenças quaisquer da
matemática. Porém, ao invés de utilizarmos uma notação hierárquica (em
forma de uma árvore), como fizemos com a dedução natural, utilizaremos
uma notação \textit{estruturada}, no sentido que organizaremos
demonstrações em blocos, similares à blocos de comandos presentes na
maioria das linguagens existentes (como C/C++, Java, Python,
etc.), visando facilitar a construção e o entendimento de provas.

\section{Técnicas de Demonstração de Teoremas}

Nesta seção apresentaremos as técnicas para demonstração de teoremas,
que essencialmente, são as regras já vistas em nosso estudo de dedução natural.
Visando facilitar a tarefa de construir demonstração corretas e similares
às encontradas em textos de matemática, dividiremos a tarefa de
provar um teorema em duas partes relacionadas: 1) construção de um
rascunho e 2) elaboração de um texto, a partir deste rascunho\footnote{A
  técnica que adotaremos neste texto para construção de demonstrações
  é a apresentada no livro de Daniel Velleman \cite{Velleman06}.}.

O rascunho é utilizado para realizar as deduções que formam a
demonstração de um teorema. Normalmente este é dividido em duas
colunas: a coluna de hipóteses e a de conclusão. Na coluna de hipóteses
encontram-se todas as hipóteses e suposições feitas durante a
demonstração e a coluna de conclusão registra qual fórmula devemos
deduzir a partir das hipóteses, para estabelecer que o teorema em
questão é realmente verdadeiro.

Mas, como construir um rascunho para um dado teorema? Como produzir um
texto a partir deste rascunho? Ambas estas perguntas são respondidas
considerando o que \cite{Velleman06} chama de \emph{estratégia de
  prova}. Uma estratégia de prova consiste de modelos para construção de
rascunho e de textos que são aplicáveis a um certo tipo de hipótese ou
conclusão. Veremos que a escolha de qual estratégia de prova será
utilizada depende de quais conectivos / quantificadores a fórmula em
questão possui.

Denominamos de \emph{estratégia para utilização de
hipóteses}, técnicas que permite-nos deduzir novas fórmulas a partir de
hipóteses. Estas estratégias correspondem às regras de eliminação de
quantificadores e conectivos da dedução natural. Por sua vez,
denominamos de \emph{estratégia para demonstrar uma conclusão}
técnicas que nos permitem deduzir uma fórmula com um certo conectivo /
quantificador. Usualmente, estas técnicas permitem transformar o
problema de demonstrar uma fórmula $\alpha$ em problemas mais
simples. Estratégias para demonstrar conclusões correspondem às regras
de introdução de conectivos / quantificadores da dedução natural.

Os modelos de rascunho presentes em uma estratégia de demonstração
dividem-se em duas partes, a primeira mostra um esquema de rascunho
antes de usar a estratégia e a segunda mostra o rascunho
resultante. Já o modelo de texto, usualmente apresenta ``buracos''  a
serem preenchidos com o texto de alguma sub-demonstração a ser
realizada. Partes a serem preenchidas com sub-demonstrações são
indicadas usando colchetes ('[' e ']').

A demonstração de um teorema deve seguir os seguintes passos:
\begin{enumerate}
  \item Identifique as hipóteses e conclusão de um teorema e
    expresse-os como fórmulas da lógica.
  \item A partir da representação do teorema como um conjunto de
    fórmulas da lógica, construa o rascunho que demonstra o teorema
    em questão.
  \item A partir do rascunho produzido, elabore o texto final da demonstração.
\end{enumerate}

As próximas seções apresentam
cada uma das técnicas para os conectivos e quantificadores e exemplos
que ilustram o uso destas estratégias.

\subsection{Estratégias para Implicação $(\to)$}

A primeira estratégia de demonstração que veremos é provavelmente a
que será mais utilizada. Esta permite demonstrar implicações lógicas
e é equivalente à regra de introdução da implicação da dedução
natural e é conhecida como prova direta.

\begin{ProofStrategy}[Para provar uma conclusão da forma
  $\alpha\to\beta$]\label{imp1}
Suponha que $\alpha$ é verdadeiro e então prove $\beta$.
\begin{flushleft}
 \textbf{Rascunho}.\\
\verb| |\\

\textit{Rascunho antes de usar a estratégia}.
\verb| |\\
\begin{tabular}{ll}
Hipóteses & Conclusão \\
$\gamma_1,\gamma_2,...,\gamma_n$ & $\alpha\to \beta$\\
\end{tabular}

\textit{Rascunho depois de usar a estratégia}.
\verb| |\\
\begin{tabular}{ll}
Hipóteses & Conclusão \\
$\gamma_1,\gamma_2,...,\gamma_n$ & $\beta$\\
$\alpha$                                              & \\
\end{tabular}
\end{flushleft}
\begin{flushleft}
\textbf{Texto:}\\
\textit{Suponha que $\alpha$}.\\
\verb|    |[\textit{Prova de $\beta$}]\\
\textit{Portanto, se $\alpha$ então $\beta$}.
\end{flushleft}
\end{ProofStrategy}
O exemplo a seguir utiliza esta estratégia para construção de um
teorema simples.
\begin{Example}
Considere a tarefa de demonstrar o seguinte teorema:
\begin{flushleft}
Suponha que  $a,b\in\mathbb{R}$. Se $0 < a < b$ então $a^2 < b^2$.
\end{flushleft}
Seguindo os passos descritos anteriormente, primeiro devemos
representar as hipóteses e a conclusão deste teorema como fórmulas da
lógica.
O teorema em questão possui como hipóteses que $a,b\in\mathbb{R}$ e a sua
conclusão é a fórmula:
\[
0 < a < b \to a^2 < b^2
\]
A partir da representação das hipóteses e da conclusão como fórmulas
da lógica, devemos proceder com a elaboração do rascunho.
Inicialmente, o rascunho possui a seguinte forma:
\begin{flushleft}
\begin{tabular}{ll}
Hipóteses & Conclusão \\
$a,b\in\mathbb{R}$ & $0 < a < b \to a^2 < b^2$\\
\end{tabular}
\end{flushleft}
Uma vez que a conclusão é uma implicação, podemos utilizar a
estratégia de provas \ref{imp1}. Abaixo é apresentado o rascunho após
a utilização desta estratégia:
\begin{flushleft}
\begin{tabular}{ll}
Hipóteses & Conclusão \\
$a,b\in\mathbb{R}$ & $ a^2 < b^2$\\
$0 < a < b$ & \\
\end{tabular}
\end{flushleft}
Evidentemente, se $0 < a < b$ então $a > 0$, $a < b$, $b > 0$. Como
tanto $a$ quanto $b$ são maiores que zero, podemos multiplicar ambos
os lados de $a < b$ por cada um destes valores. Multiplicando por $a$,
obtemos $a^2 < ab$, e ao multiplicarmos por $b$, obtemos $ab < b^2$.
\begin{flushleft}
\begin{tabular}{ll}
Hipóteses & Conclusão \\
$a,b\in\mathbb{R}$ & $ a^2 < b^2$\\
$0 < a < b$ & \\
$a^2 < ab$  & \\
$ab < b^2$ &\\
\end{tabular}
\end{flushleft}
Uma vez que $a^2 < ab$ e $ab < b^2$, temos que $a^2 < b^2$, como
queríamos demonstrar.

Como conseguimos deduzir a conclusão a partir das hipóteses,
utilizando o rascunho, devemos proceder para a elaboração do texto
final da prova. De acordo com o modelo de texto descrito na estratégia
de prova \ref{imp1}, o texto deve possuir a seguinte estrutura
inicial:
\begin{flushleft}
Suponha que $a,b\in\mathbb{R}$.
Suponha que $0< a < b$.\\
\verb|  |[Prova de $a^2 < b^2$]\\
Portanto, se $0 < a < b$ então $a^2 < b^2$.
\end{flushleft}
Para finalizar o texto, basta preencher o ``buraco'' com o texto da
dedução de $a^2 < b^2$.  O resultado final é apresentado a seguir.
\begin{flushleft}
Suponha que $a,b\in\mathbb{R}$.
Suponha que $0< a < b$.\\
\verb|  |Como $0< a < b$, temos que $a,b > 0$ e $a < b$.\\
\verb|  |Como $a > 0$ e $a < b$, temos que $a^2 < ab$.\\
\verb|  |Como $b > 0$ e $a < b$, temos que $ab < b^2$.\\
\verb|  |Como $a^2 < ab$ e $ab < b^2$, temos que $a^2 < b^2$.\\
Portanto, se $0 < a < b$ então $a^2 < b^2$.
\end{flushleft}
\end{Example}
Note que a estrutura da demonstração é indicada utilizando indentação,
de maneira similar a blocos de comandos em linguagens de
programação. Apesar de não ser uma padrão em textos sobre matemática,
há evidências que a utilização de indentação em provas ajuda no
entendimento\footnote{Uma argumentação detalhada a favor do uso de
  provas estruturadas é apresentada em \cite{Lamport12}.} e, por isso, este será o padrão adotado
neste texto.

Outra maneira de demonstrar uma implicação é a utilização da seguinte
equivalência lógica: $\alpha\to \beta \equiv \neg\beta\to\neg\alpha$,
que é facilmente demonstrável utilizando álgebra booleana.
Demonstrações de implicações baseadas nesta estratégia são comumente
denominadas de provas pela contrapositiva.
A próxima estratégia de prova é baseada nesta equivalência.

\begin{ProofStrategy}[Para provar uma conclusão da forma
  $\alpha\to\beta$]\label{imp2}
Suponha que $\beta$ é falso e prove  que $\alpha$ é falso.
\begin{flushleft}
 \textbf{Rascunho}.\\
\verb| |\\

\textit{Rascunho antes de usar a estratégia}.
\verb| |\\
\begin{tabular}{ll}
Hipóteses & Conclusão \\
$\gamma_1,\gamma_2,...,\gamma_n$ & $\alpha\to \beta$\\
\end{tabular}

\textit{Rascunho depois de usar a estratégia}.
\verb| |\\
\begin{tabular}{ll}
Hipóteses & Conclusão \\
$\gamma_1,\gamma_2,...,\gamma_n$ & $\neg\alpha$\\
$\neg\beta$                                              & \\
\end{tabular}
\end{flushleft}
\begin{flushleft}
\textbf{Texto:}\\
\textit{Suponha que $\beta$ é falso}.\\
\verb|    |[\textit{Prova de $\neg \alpha$}]\\
\textit{Portanto, se $\alpha$ então $\beta$}.
\end{flushleft}
\end{ProofStrategy}

O próximo exemplo ilustra o uso da estratégia de prova \ref{imp2} para
demonstrar um teorema simples.

\begin{Example}
Considere a tarefa de demonstrar o seguinte teorema:
\begin{flushleft}
Suponha que $a,b$ e $c$ são números reais tais que $a > b$. Se $ac
\leq bc$ então $c\leq 0$.
\end{flushleft}
Para demonstrar este teorema, primeiramente, devemos representar suas
hipóteses e conclusão como fórmulas da lógica. Evidentemente, as
hipóteses deste teorema são que $a,b,c\in\mathbb{R}$ e a sua conclusão
é representada pela seguinte fórmula:
\[
ac\leq bc \to c \leq 0
\]
A partir da representação das hipóteses e conclusão devemos proceder
para a construção do rascunho.
\begin{flushleft}
\begin{tabular}{ll}
Hipóteses & Conclusão \\
$a,b,c\in\mathbb{R}$ & $ ac\leq bc \to c \leq 0$\\
$a > b$ & \\
\end{tabular}
\end{flushleft}
Como a conclusão deste teorema é formada por uma implicação,
utilizaremos a estratégia de prova \ref{imp2} para demonstrá-la. O
resultado de se usar esta técnica de prova é apresentado no rascunho a
seguir.
\begin{flushleft}
\begin{tabular}{ll}
Hipóteses & Conclusão \\
$a,b,c\in\mathbb{R}$ & $ \neg (ac\leq bc)$\\
$a > b$ & \\
$\neg (c\leq 0)$ &
\end{tabular}
\end{flushleft}
É óbvio que $\neg (ac \leq bc) \equiv ac > bc$ e que $\neg (c \leq 0)
\equiv c > 0$.
\begin{flushleft}
\begin{tabular}{ll}
Hipóteses & Conclusão \\
$a,b,c\in\mathbb{R}$ & $ac > bc$\\
$a > b$ & \\
$c < 0$ &
\end{tabular}
\end{flushleft}
Mas como $c > 0$, podemos multiplicar ambos os lados de $a > b$
obtendo $ac > bc$, o que conclui a demonstração deste teorema.

Após a conclusão do rascunho, devemos proceder com a elaboração do
texto desta demonstração. De acordo com a estratégia \ref{imp2}, temos
a seguinte estrutura inicial:
\begin{flushleft}
Sejam $a,b,c \in \mathbb{R}$ tais que $a > b$.\\
Suponha que $c > 0$.\\
\verb|   |[Prova de $ac > bc$].\\
Portanto, se $ac \leq bc$ então $c \leq 0$.
\end{flushleft}
Que é imediatamente encerrada com a dedução de que $ac > bc$, a partir
das hipóteses $a > b$ e $c > 0$, conforme
apresentado a seguir.
\begin{flushleft}
Sejam $a,b,c \in \mathbb{R}$ tais que $a > b$.\\
Suponha que $c > 0$.\\
\verb|   |Como  $a > b$ e $c > 0$ temos que $ac > bc$.\\
Portanto, se $ac \leq bc$ então $c \leq 0$.
\end{flushleft}
\end{Example}

Note que este teorema pode ser provado usando a estratégia de prova
\ref{imp1}. Isso mostra que, muitas vezes, há mais de uma possível
estratégia aplicável a demonstração de um certo teorema. Porém,
certamente, uma das possibilidades resultará em uma prova mais
simples, como o caso do exemplo anterior. Apesar de demonstrável
usando uma prova direta, o uso de contrapositiva permitiu uma prova
quase que imediata.

As próximas seções deste capítulo apresentarão técnicas e exemplos de
utilização destas para demonstração de teoremas envolvendo outros
conectivos / quantificadores.

\subsection{Estratégias para Negação ($\neg$) e Implicação ($\to$)}

Para provar que uma dada conclusão é falsa (isto é, provar $\neg
\alpha$), devemos proceder de maneira similar ao que era feito na
dedução natural: tratar a negação como uma implicação (lembre-se $\neg
\alpha \equiv \alpha\to\bot$) e demonstrar uma contradição. Esta idéia
é formalizada pela próxima estratégia de prova.

\begin{ProofStrategy}[Para provar uma conclusão da forma $\neg\alpha$]\label{neg1}
Suponha que $\alpha$ é verdadeiro e obtenha uma contradição.
\begin{flushleft}
 \textbf{Rascunho}.\\
\verb| |\\

\textit{Rascunho antes de usar a estratégia}.
\verb| |\\
\begin{tabular}{ll}
Hipóteses & Conclusão \\
$\gamma_1,\gamma_2,...,\gamma_n$ & $\neg \alpha$\\
\end{tabular}

\textit{Rascunho depois de usar a estratégia}.
\verb| |\\
\begin{tabular}{ll}
Hipóteses & Conclusão \\
$\gamma_1,\gamma_2,...,\gamma_n$ & $\bot$\\
$\alpha$                                              & \\
\end{tabular}
\end{flushleft}
\begin{flushleft}
\textbf{Texto:}\\
\textit{Suponha que $\alpha$ é verdadeiro}.\\
\verb|    |[\textit{Prova de $\bot$}]\\
\textit{Portanto, $\alpha$ é falso.}
\end{flushleft}
\end{ProofStrategy}
\begin{Example}\label{exneg}
Considere o seguinte teorema:
\begin{flushleft}
Se $x^2+y = 13$ e $y \neq 4$ então $x \neq 3$.
\end{flushleft}
Note que este teorema não possui hipóteses e sua conclusão é uma
fórmula que possui o conectivo de implicação, conforme apresentado a
seguir:
\[
x^2 + y = 13 \land y \neq 4 \to x \neq 3
\]
\begin{flushleft}
\begin{tabular}{ll}
Hipóteses & Conclusão \\
     & $x^2 + y = 13 \land y \neq 4 \to x \neq 3$\\
\end{tabular}
\end{flushleft}
Utilizando a estratégia de prova direta, temos:
\begin{flushleft}
\begin{tabular}{ll}
Hipóteses & Conclusão \\
    $x^2 + y = 13$ & $x \neq 3$\\
   $y \neq 4$ & \\
\end{tabular}
\end{flushleft}
Como a conclusão é uma negação ($x \neq 3 \equiv \neg (x = 3)$),
podemos utilizar a estratégia de prova \ref{neg1} obtendo:
\begin{flushleft}
\begin{tabular}{ll}
Hipóteses & Conclusão \\
    $x^2 + y = 13$ & $\bot$\\
   $y \neq 4$ & \\
   $x = 3$ &
\end{tabular}
\end{flushleft}
Porém, ao substituir $x = 3$ em $x^2 + y = 13$ obtemos $y = 4$, o que
contradiz a suposição de que $y \neg 4$, o que conclui a demonstração
do teorema.

Agora, a partir do rascunho, basta construir o texto utilizando os
modelos para as estratégias de prova utilizadas. Primeiramente, usando
o modelo de texto para prova direta, obtemos:
\begin{flushleft}
Suponha que $x^2 + y = 13$ e que $y \neq 4$.\\
\verb|  |[Prova de $x\neq 3$]\\
Portanto, se $x^2 + y = 13$ e $y \neq 4$ então $x \neq 3$.
\end{flushleft}
Na sequência, utilizamos o modelo para negação:
\begin{flushleft}
Suponha que $x^2 + y = 13$ e que $y \neq 4$.\\
\verb|  |Suponha que $x = 3$.\\
\verb|    |[Prova de $\bot$]\\
\verb|  |Logo, $x\neq 3$.\\
Portanto, se $x^2 + y = 13$ e $y \neq 4$ então $x \neq 3$.
\end{flushleft}
Finalmente, encerramos a demonstração apresentando a contradição
obtida a partir das hipóteses.
\begin{flushleft}
Suponha que $x^2 + y = 13$ e que $y \neq 4$.\\
\verb|  |Suponha que $x = 3$.\\
\verb|    |Como $x^2 + y = 13$ e $y\neq 4$, temos que $y = 4$.\\
\verb|    |Como $y \neq 4$ e $y = 4$, temos uma contradição.\\
\verb|  |Logo, $x\neq 3$.\\
Portanto, se $x^2 + y = 13$ e $y \neq 4$ então $x \neq 3$.
\end{flushleft}
\end{Example}
Sentenças negativas são, usualmente, de mais difícil que
positivas. Isto motiva uma técnica para demonstração que é bem útil e
vale-se de equivalências algébricas da lógica.
\begin{ProofStrategy}[Para provar uma conclusão da forma $\neg
  \alpha$]\label{neg2}
Tente reexpressá-la como uma fórmula sem negação utilizando
equivalências da álgebra booleana e então utilize outras estratégias de prova.
\end{ProofStrategy}
Como um exemplo desta estratégia, considere a seguinte variação do
exemplo \ref{exneg}.
\begin{Example}
\begin{flushleft}
Sejam $x,y\in\mathbb{N}$. Se $x^2 + y = 13$ então  não é verdade que $x \neq 3$ e $y = 4$.
\end{flushleft}
É fácil perceber que este teorema é composto apenas por uma conclusão
e que esta é representada pela seguinte fórmula:
\[
x^2 + y = 13 \to \neg (x \neq 3 \land y = 4)
\]
O rascunho para este teorema possui a seguinte configuração inicial:
\begin{flushleft}
\begin{tabular}{ll}
Hipóteses & Conclusão \\
 $x,y\in\mathbb{N}$ & $x^2 + y = 13 \to \neg (x \neq 3 \land y = 4)$
\end{tabular}
\end{flushleft}
Como a conclusão é uma implicação, podemos iniciar a demonstração
usando a técnica de prova direta.
\begin{flushleft}
\begin{tabular}{ll}
Hipóteses & Conclusão \\
$x,y\in\mathbb{N}$ & $\neg (x \neq 3 \land y = 4)$\\
 $x^2 + y = 13$ & \\
\end{tabular}
\end{flushleft}
Note que a conclusão possui uma negação. Logo, podemos então tentar a
estratégia de prova \ref{neg2} e usar álgebra booleana para mudar a
forma da conclusão. Note que $\neg (x \neq 3 \land y = 4)$ é
equivalente a $y = 4 \to x = 3$, conforme a dedução seguinte:
\[
\begin{array}{lcl}
\neg(x \neq 3 \land y = 4) & \equiv &\\
x = 3 \lor y \neq 4 & \equiv & \\
y \neq 4 \lor x = 3 & \equiv & \\
y = 4 \to x = 3 & &
\end{array}
\]
Usando esta equivalência, temos que o rascunho pode ser alterado para:
\begin{flushleft}
\begin{tabular}{ll}
Hipóteses & Conclusão \\
$x,y\in\mathbb{N}$ & $y = 4 \to x = 3$\\
 $x^2 + y = 13$ & \\
\end{tabular}
\end{flushleft}
Logo, podemos utilizar novamente uma estratégia de prova direta,
obtendo:
\begin{flushleft}
\begin{tabular}{ll}
Hipóteses & Conclusão \\
$x,y\in\mathbb{N}$ & $x = 3$\\
 $x^2 + y = 13$ & \\
$y = 4$ & \\
\end{tabular}
\end{flushleft}

O que é evidentemente verdadeiro. A seguir, apresentamos a construção
passo-a-passo do texto desta demonstração.
\begin{flushleft}
Suponha que $x^2 + y = 13$ e que $y = 4$.\\
\verb|  |Como $y = 4$ e $x^2 + y = 13$, temos que $x = 3$.\\
Portanto, se $x^2 + y = 13$ então não é verdade que $x \neq 3$ e $y = 4$.
\end{flushleft}
Note que, a manipulação algébrica que transformou a negação em uma
implicação não é sequer citada no texto da demonstração. Manipulações
algébricas sobre fórmulas da lógica devem apenas fazer parte do
rascunho, nunca da demonstração final de um teorema.
\end{Example}

Até o presente momento foram apresentadas apenas estratégias para
demonstrar teoremas que possuem um certo conectivo. Um ponto chave na
demonstração de teoremas é a utilização adequada de hipóteses. As
próximas estratégias a serem apresentadas mostram como utilizar
hipóteses e por isso, são chamadas de estratégias para uso de
hipóteses.

\begin{HypothesisStrategy}[Para usar uma hipótese da forma
  $\neg\alpha$] Caso possível, reexpresse $\neg\alpha$ utilizando
  regras da álgebra booleana de maneira que a negação seja removida
  desta fórmula.
\end{HypothesisStrategy}

\begin{HypothesisStrategy}[Para usar uma hipótese da forma $\alpha \to
  \beta$] \label{imphyp1}
Caso seja possível deduzir  $\alpha$ ou $\neg \beta$, então
  podemos utilizar $\alpha\to\beta$ para deduzir $\beta$ ou
  $\neg\alpha$. Note que esta estratégia é equivalente a utilizar a
  regra $\impE$ ou o seguinte sequente da dedução natural $\{\alpha\to\beta,\neg\beta\}\vdash\alpha$.
\end{HypothesisStrategy}

\begin{Example}
\begin{flushleft}
Suponha que $P \to Q \to R$. Então, $\neg R \to (P \to \neg Q)$.
\end{flushleft}
Como este teorema envolve fórmulas da lógica diretamente, podemos
proceder para a construção do rascunho.
\begin{flushleft}
\begin{tabular}{ll}
Hipóteses & Conclusão \\
$P \to Q \to R$ & $\neg R \to (P \to \neg Q)$\\
\end{tabular}
\end{flushleft}
Como desejamos concluir uma implicação, vamos iniciar esta
demonstração usando a técnica de prova direta.
\begin{flushleft}
\begin{tabular}{ll}
Hipóteses & Conclusão \\
$P \to Q \to R$ & $ P \to \neg Q$\\
$\neg R$ & \\
\end{tabular}
\end{flushleft}
Novamente, usando prova direta temos:
\begin{flushleft}
\begin{tabular}{ll}
Hipóteses & Conclusão \\
$P \to Q \to R$ & $ \neg Q$\\
$\neg R$ & \\
$P$ & \\
\end{tabular}
\end{flushleft}
Como possuímos $P$ e $P\to Q\to R$, podemos utilizar a estratégia de
uso de hipóteses \ref{imphyp1} para concluir $Q\to R$.
\begin{flushleft}
\begin{tabular}{ll}
Hipóteses & Conclusão \\
$P \to Q \to R$ & $ \neg Q$\\
$\neg R$ & \\
$P$ & \\
$Q\to R$ &
\end{tabular}
\end{flushleft}
Como $Q\to R$ e  $\neg R$ são verdadeiras, temos que $\neg Q$ também o
é, terminando assim, a dedução.
A seguir apresentamos o texto desta demonstração.
\begin{flushleft}
Suponha que $P\to Q \to R$.\\
\verb|  |Suponha que $\neg R$.\\
\verb|    |Suponha que $P$.\\
\verb|      |Como $P \to Q \to R$ e $P$, temos que $Q\to R$.\\
\verb|      |Como $Q\to R$ e $\neg R$, temos que $\neg Q$.\\
\verb|    |Logo, $P\to \neg Q$.\\
\verb|  |Assim, $\neg R\to (P \to \neg Q)$.\\
Portanto, $\neg R \to (P \to \neg Q)$.
\end{flushleft}
\end{Example}

\subsection{Exercícios}

\begin{enumerate}
  \item Prove os seguintes teoremas.
  \begin{enumerate}
      \item Suponha $a,b\in\mathbb{R}$. Se $a < b < 0$ então $a^2 >
        b^2$.
      \item Suponha $a,b\in\mathbb{R}$. Se $0 < a < b$ então
        $\frac{1}{b}<\frac{1}{a}$.
     \item Suponha $a,b,c,d\in\mathbb{R}$, $0 < a < b$ e $d > 0$. Se
       $ac \geq bd$ então $c > d$.
     \item Suponha que $a,b\in\mathbb{R}$. Se $a^2b = 2a + b$, então
       se $b\neq 0$ então $a \neq 0$.
  \end{enumerate}
\end{enumerate}

\subsection{Estratégias para Quantificadores ($\forall$), ($\exists$)}

Estratégias de prova para conclusões envolvendo quantificadores são
análogas às regras de introdução destes apresentadas nos capítulos
\ref{cap2} e \ref{cap3}.

\begin{ProofStrategy}[Para provar uma conclusão da forma $\forall
  x. P(x)$]\label{quant1}
Suponha que $x$ é um valor arbitrário\footnote{Lembre-se: um valor $x$
é arbitrário se este não pertence ao conjunto de variáveis livres das
hipóteses do teorema em questão.} e prove $P(x)$.
\begin{flushleft}
 \textbf{Rascunho}.\\
\verb| |\\
\textit{Rascunho antes de usar a estratégia}.
\verb| |\\
\begin{tabular}{ll}
Hipóteses & Conclusão \\
$\gamma_1,\gamma_2,...,\gamma_n$ & $\forall x. P(x)$\\
\end{tabular}

\textit{Rascunho depois de usar a estratégia}.
\verb| |\\
\begin{tabular}{ll}
Hipóteses & Conclusão \\
$\gamma_1,\gamma_2,...,\gamma_n$ & $P(x)$\\
$x$\text{ arbitrário}                           & \\
\end{tabular}
\end{flushleft}
\begin{flushleft}
\textbf{Texto:}\\
\textit{Suponha que $x$ é arbitrário}.\\
\verb|    |[\textit{Prova de $P(x)$}]\\
\textit{Portanto, $\forall x.P(x)$}.
\end{flushleft}
\end{ProofStrategy}
\begin{ProofStrategy}[Para provar uma conclusão da forma $\exists
  x. P(x)$]\label{quant2}
Tente encontrar o valor de $x$ que torna $P(x)$ verdadeiro e então
prove esta fórmula.
\begin{flushleft}
 \textbf{Rascunho}.\\
\verb| |\\

\textit{Rascunho antes de usar a estratégia}.
\verb| |\\
\begin{tabular}{ll}
Hipóteses & Conclusão \\
$\gamma_1,\gamma_2,...,\gamma_n$ & $\exists x. P(x)$\\
\end{tabular}

\textit{Rascunho depois de usar a estratégia}.
\verb| |\\
\begin{tabular}{ll}
Hipóteses & Conclusão \\
$\gamma_1,\gamma_2,...,\gamma_n$ & $P(x)$\\
$x =$[valor escolhido por você.]           & \\
\end{tabular}
\end{flushleft}
\begin{flushleft}
\textbf{Texto:}\\
\textit{Seja $x =$}[valor escolhido por você].\\
\verb|    |[\textit{Prova de $P(x)$}]\\
\textit{Portanto, $\exists x.P(x)$}.
\end{flushleft}
\end{ProofStrategy}
A seguir, apresentamos um exemplo que ilustra a utilização destas duas
estratégias de prova.
\begin{Example}
Considere o seguinte teorema:
\begin{flushleft}
   Para todo $x\in\mathbb{R}$, se $x > 0$ então existe um $y\in\mathbb{R}$ tal que
   $y(y + 1) = x$.
\end{flushleft}
Este teorema é formado apenas pela conclusão, expressa simbolicamente
a seguir:
\[
\forall x. x\in\mathbb{R} \to x > 0 \to \exists y . y\in\mathbb{R}
\land y (y + 1) = x
\]
A partir da representação deste teorema, podemos iniciar a construção
de seu rascunho:
\begin{flushleft}
\begin{tabular}{ll}
Hipóteses & Conclusão \\
 & $\forall x. x\in\mathbb{R} \to x > 0 \to \exists y . y\in\mathbb{R}
\land y (y + 1) = x$\\
\end{tabular}
\end{flushleft}
Como a conclusão é uma fórmula envolvendo o quantificador universal,
podemos utilizar a estratégia de prova \ref{quant1}, obtendo:
\begin{flushleft}
\begin{tabular}{ll}
Hipóteses & Conclusão \\
 $x$ arbitrário & $ x\in\mathbb{R} \to x > 0 \to \exists y . y\in\mathbb{R}
\land y (y + 1) = x$\\
\end{tabular}
\end{flushleft}
Utilizando a estratégia de prova direta (duas vezes) obtemos:
\begin{flushleft}
\begin{tabular}{ll}
Hipóteses & Conclusão \\
 $x$ arbitrário & $\exists y . y\in\mathbb{R}
\land y (y + 1) = x$\\
$x\in\mathbb{R}$ & \\
$x > 0$ & \\
\end{tabular}
\end{flushleft}
Agora, temos que mostrar que existe um valor $y \in\mathbb{R}$ tal que
$y(y+1) = x$. Mas qual seria este valor? Olhando com um pouco de
atenção a equação $y(y+1) = x$, podemos perceber que esta é uma
equação de $2^o$ grau sobre a variável $y$. Resolvendo-a obtemos:
\[
\begin{array}{lcl}
\Delta & = & 1 -4.1.x \\
y' & = & \frac{-1 + \sqrt{1 + 4x}}{2}\\
y'' & = & \frac{-1 - \sqrt{1 + 4x}}{2}
\end{array}
\]
Desta forma, temos que tanto $\frac{-1 + \sqrt{1 + 4x}}{2}$ quanto
$\frac{-1 - \sqrt{1 - 4x}}{2}$ são possíveis valores para $y$ que
tornam a equação $y(y+1) = x$ verdadeira, conforme demonstrado a
seguir:
\[
\begin{array}{lcl}
y(y+1) & = &\{\text{por }y = \frac{-1 + \sqrt{1 + 4x}}{2}\}\\
\frac{-1 + \sqrt{1 + 4x}}{2}\left(\frac{-1 + \sqrt{1 + 4x}}{2} +
  1\right) & = &\\
\frac{-1 + \sqrt{1 + 4x}}{2}\left(\frac{-1 + \sqrt{1 + 4x} + 2}{2}
\right) & = \\
\frac{(-1 + \sqrt{1 + 4x})}{2}\frac{(1 + \sqrt{1 + 4x})}{2} & =\\
\frac{1 + 4x - 1}{4} & =\\
\frac{4x}{4} & = \\
x
\end{array}
\]
Logo, para $y = \frac{-1 + \sqrt{1 + 4x}}{2}$, temos que $y(y+1) = x$,
o que conclui a demonstração deste teorema. Abaixo, apresentamos
passo-a-passo a construção do texto a partir do
rascunho. Primeiramente, o texto para o uso da técnica de provas para
o quantificador universal:
\begin{flushleft}
Suponha $x$ arbitrário.\\
\verb|  |[Prova de $x\in\mathbb{R}\to x > 0 \to \exists
y.y\in\mathbb{R}\land y(y + 1) = x$]\\
Como $x$ é arbitrário temos que se $x > 0$ então existe y tal que
$y(y+1) = x$.
\end{flushleft}
Na sequência, o texto é alterado para refletir o uso da técnica de
prova direta.
\begin{flushleft}
Suponha $x$ arbitrário.\\
\verb|  |Suponha que $x \in \mathbb{R}$ e $x> 0$.\\
\verb|     |[Prova de $\exists y.y\in\mathbb{R}\land y(y + 1) = x$]\\
\verb|  |Logo, se $x \in \mathbb{R}$ e $x> 0$ então existe y tal que
$y(y+1) = x$\\
Como $x$ é arbitrário temos que se $x > 0$ então existe y tal que
$y(y+1) = x$.
\end{flushleft}
Agora, resta demonstrar o quantificador existencial da conclusão:
\begin{flushleft}
Suponha $x$ arbitrário.\\
\verb|  |Suponha que $x \in \mathbb{R}$ e $x> 0$.\\
\verb|     |Seja $y = \frac{-1 + \sqrt{1 + 4x}}{2}$
\verb|        |[Prova de que $y(y+1) = x$]
\verb|     |Logo, existe $y$ tal que $y(y+1) = x$.
\verb|  |Logo, se $x \in \mathbb{R}$ e $x> 0$ então existe y tal que
$y(y+1) = x$\\
Como $x$ é arbitrário temos que se $x > 0$ então existe y tal que
$y(y+1) = x$.
\end{flushleft}
Para encerrarmos a demonstração, basta utilizar o desenvolvimento
algébrico apresentado anteriormente.
\begin{flushleft}
Suponha $x$ arbitrário.\\
\verb|  |Suponha que $x \in \mathbb{R}$ e $x> 0$.\\
\verb|     |Seja $y = \frac{-1 + \sqrt{1 + 4x}}{2}$\\
\[
\begin{array}{llcl}
\,\,\,\,\,& y(y+1) & = &\{\text{por }y = \frac{-1 + \sqrt{1 + 4x}}{2}\}\\
 &\frac{-1 + \sqrt{1 + 4x}}{2}\left(\frac{-1 + \sqrt{1 + 4x}}{2} +
  1\right) & = &\\
& \frac{-1 + \sqrt{1 + 4x}}{2}\left(\frac{-1 + \sqrt{1 + 4x} + 2}{2}
\right) & = \\
& \frac{(-1 + \sqrt{1 + 4x})}{2}\frac{(1 + \sqrt{1 + 4x})}{2} & =\\
& \frac{1 + 4x - 1}{4} & =\\
& \frac{4x}{4} & = \\
& x
\end{array}
\]

\verb|     |Logo, existe $y$ tal que $y(y+1) = x$.\\
\verb|  |Logo, se $x \in \mathbb{R}$ e $x> 0$ então existe y tal que
$y(y+1) = x$\\
Como $x$ é arbitrário temos que se $x > 0$ então existe y tal que
$y(y+1) = x$.
\end{flushleft}
\end{Example}

Note que no texto final da demonstração de um teorema envolvendo o
quantificador existencial não há explicação sobre como o valor
utilizado para provar $\exists y. y (y + 1) = x$ foi encontrado. Isto
é uma prática padrão em matemática, já que a única coisa que estamos
interessados é em mostrar que um certo valor existe e não em como
obtê-lo.

As próximas estratégias de utilização de hipóteses mostram como
hipóteses envolvendo os quantificadores existencial e universal podem
ser utilizadas. O leitor verá que estas são exatamente as regras para
eliminação para estes quantificadores.

\begin{HypothesisStrategy}[Para utilizar uma hipótese da forma
  $\forall x. P(x)$]
Basta adicionar como hipótese $[x\mapsto a]P(x)$, em que $a$ é um
valor qualquer do universo de discurso. Note que esta estratégia é exatamente a regra de
eliminação do quantificador universal.
\begin{flushleft}
 \textbf{Rascunho}.\\
\verb| |\\

\textit{Rascunho antes de usar a estratégia}.
\verb| |\\
\begin{tabular}{ll}
Hipóteses & Conclusão \\
$\gamma_1,\gamma_2,...,\gamma_n$ & $\beta$\\
$\forall x. P(x)$ & \\
\end{tabular}

\textit{Rascunho depois de usar a estratégia}.
\verb| |\\
\begin{tabular}{ll}
Hipóteses & Conclusão \\
$\gamma_1,\gamma_2,...,\gamma_n$ & $\beta$\\
$\forall x. P(x)$ & \\
$[x\mapsto a]P(x)$ & \\
\end{tabular}
\end{flushleft}
\end{HypothesisStrategy}

\begin{HypothesisStrategy}[Para utilizar uma hipótese da forma
  $\exists x. P(x)$]
Basta adicionar como hipótese $[x\mapsto x_0]P(x)$, em que $x_0$ é um
valor arbitrário. Note que esta estratégia é exatamente a regra de
eliminação do quantificador existencial.
\begin{flushleft}
 \textbf{Rascunho}.\\
\verb| |\\

\textit{Rascunho antes de usar a estratégia}.
\verb| |\\
\begin{tabular}{ll}
Hipóteses & Conclusão \\
$\gamma_1,\gamma_2,...,\gamma_n$ & $\beta$\\
$\exists x. P(x)$ & \\
\end{tabular}

\textit{Rascunho depois de usar a estratégia}.
\verb| |\\
\begin{tabular}{ll}
Hipóteses & Conclusão \\
$\gamma_1,\gamma_2,...,\gamma_n$ & $\beta$\\
$\forall x. P(x)$ & \\
$[x\mapsto x_0]P(x)$ & \\
$x_0$ é arbitrário & \\
\end{tabular}
\end{flushleft}
\end{HypothesisStrategy}

Antes de apresentarmos um exemplo destas estratégias, daremos uma definição formal do
conceito de divisibilidade de dois números inteiros.

\begin{Definition}[Divisibilidade]\label{divdef}
Sejam $a,b\in\mathbb{Z}$. Dizemos que $a$ divide $b$, $a\,|\,b$, se
$\exists k. k\in\mathbb{Z}\land ka = b$.
\end{Definition}
\begin{Example}
Considere o seguinte teorema:
\begin{flushleft}
Para todo $a, b, c\in\mathbb{Z}$, se $a\,|\,b$ e $b\,|\,c$ então $a\,|\,c$.
\end{flushleft}
que pode ser representado pela seguinte fórmula da lógica:
\[
\forall a b c. a,b,c\in\mathbb{Z} \to a\,|\,b \land b\,|\,c \to a
\,|\, c
\]
A configuração inicial do rascunho deste teorema é:
\begin{flushleft}
\begin{tabular}{ll}
Hipóteses & Conclusão \\
& $\forall a\, b \,c. a,b,c\in\mathbb{Z} \to a\,|\,b \land b\,|\,c \to a
\,|\, c$
\end{tabular}
\end{flushleft}
Como a conclusão envolve um quantificador universal, utilizaremos a
estratégia de prova para este quantificador.
\begin{flushleft}
\begin{tabular}{ll}
Hipóteses & Conclusão \\
$a,b,c$ são arbitrários& $a,b,c\in\mathbb{Z} \to a\,|\,b \land b\,|\,c \to a
\,|\, c$
\end{tabular}
\end{flushleft}
Utilizando a estratégia de prova direta (duas vezes), temos:
\begin{flushleft}
\begin{tabular}{ll}
Hipóteses & Conclusão \\
$a,b,c$ são arbitrários& $ a\,|\, c$\\
$a,b,c\in\mathbb{Z}$ & \\
$a\,|\,b$ & \\
$b\,|\,c$ & \\
\end{tabular}
\end{flushleft}
Para continuar esta demonstração, devemos utilizar a definição
\ref{divdef}, obtendo a seguinte configuração do rascunho:
\begin{flushleft}
\begin{tabular}{ll}
Hipóteses & Conclusão \\
$a,b,c$ são arbitrários& $\exists k. k\in\mathbb{Z}\land ka = c$\\
$a,b,c\in\mathbb{Z}$ & \\
$\exists k_1. k_1\in\mathbb{Z}\land k_1a = b$ & \\
$\exists k_2. k_2\in\mathbb{Z}\land k_2b = c$ & \\
\end{tabular}
\end{flushleft}
Utilizando a estratégia para utilização de hipóteses envolvendo o
quantificadores existenciais, temos:
\begin{flushleft}
\begin{tabular}{ll}
Hipóteses & Conclusão \\
$a,b,c$ são arbitrários& $\exists k. k\in\mathbb{Z}\land ka = c$\\
$a,b,c\in\mathbb{Z}$ & \\
$\exists k_1. k_1\in\mathbb{Z}\land k_1a = b$ & \\
$\exists k_2. k_2\in\mathbb{Z}\land k_2b = c$ & \\
$k_1a = b$ & \\
$k_2b = c$ & \\
\end{tabular}
\end{flushleft}
A partir das hipóteses $k_1a = b$ e $k_2b = c$, temos que $c =
k_1k_2a$. Logo, temos que o valor de $k$ que torna a igualdade $ka =
c$ é $k = k_1k_2$.

Novamente, apresentaremos passo-a-passo a construção do texto para o
rascunho apresentado.
\begin{flushleft}
Suponha $a,b$ e $c$ arbitrários.\\
\verb|  |[Prova de $a,b,c\in\mathbb{Z} \to a\,|\,b \land b\,|\,c \to a
\,|\, c$]\\
Como $a,b$ e $c$ são arbitrários, temos que para todo $a,b$ e $c$ se
$a\,|\,b$ e $b\,|\,c$ então $a\,|\,c$.
\end{flushleft}
Agora, utilizando a estratégia de prova direta, temos a seguinte
versão parcial do texto:
\begin{flushleft}
Suponha $a,b$ e $c$ arbitrários.\\
\verb|  |Suponha que $a,b,c\in\mathbb{Z}$, $ a\,|\,b$ e $b\,|\,c$.\\
\verb|     |[Prova de $ a\,|\, c$]\\
\verb|  |Logo, se $a,b,c\in\mathbb{Z}$, $ a\,|\,b$ e $b\,|\,c$ então $a\,|\,c$.\\
Como $a,b$ e $c$ são arbitrários, temos que para todo $a,b$ e $c$ se
$a\,|\,b$ e $b\,|\,c$ então $a\,|\,c$.
\end{flushleft}
Utilizando a definição de divisibilidade, temos que a demonstração de
$a\,|\,c$ envolve o uso da estratégia do quantificador existencial:
\begin{flushleft}
Suponha $a,b$ e $c$ arbitrários.\\
\verb|  |Suponha que $a,b,c\in\mathbb{Z}$, $ a\,|\,b$ e $b\,|\,c$.\\
\verb|     |Seja $k=k_1k_2$.\\
\verb|       |Como $a\,|\,b$, temos que existe $k_1$ tal que $k_1a =
b$.\\
\verb|       |Como $b\,|\,c$, temos que existe $k_2$ tal que $k_2b =
c$.\\
\verb|       |Assim, temos que $k_1k_2a = c$.\\
\verb|     |Logo, $a\,|\,c$.\\
\verb|  |Logo, se $a,b,c\in\mathbb{Z}$, $ a\,|\,b$ e $b\,|\,c$ então $a\,|\,c$.\\
Como $a,b$ e $c$ são arbitrários, temos que para todo $a,b$ e $c$ se
$a\,|\,b$ e $b\,|\,c$ então $a\,|\,c$.
\end{flushleft}
\end{Example}

\subsection{Exercícios}

\begin{enumerate}
  \item Prove os seguintes teoremas:
   \begin{enumerate}
     \item Suponha que $x\in\mathbb{R}$. Se $x\neq 1$ então existe $y$
       tal que $\frac{y+1}{y-2}= x$.
     \item Suponha que $x\in\mathbb{R}$. Se $\frac{y+1}{y-2}= x$ então
       $x\neq 1$.
     \item Suponha que $x\in\mathbb{R}$. Se $x > 2$ então existe $y$
       tal que $y + \frac{1}{y} = x$.
     \item Suponha que $a,b,c\in\mathbb{Z}$. Se $a\,|\,b$ e $a\,|\,c$ e
       $a\,|\,(b+c)$.
     \item Suponha que $a,b,c\in\mathbb{Z}$. Se $ac\,|\,bc$ e $c\neq
       0$ então $a\,|\,b$.
   \end{enumerate}
\end{enumerate}

\subsection{Estratégias para Conjunção ($\land$) e Bicondicional
  ($\leftrightarrow$)}

As estratégias de prova para a conjunção e o bicondicional refletem
diretamente o significado destes conectivos.

\begin{ProofStrategy}[Para provar uma conclusão da forma
  $\alpha\land\beta$] Prove $\alpha$ e $\beta$ separadamente.
\begin{flushleft}
 \textbf{Rascunho}.\\
\verb| |\\

\textit{Rascunho antes de usar a estratégia}.
\verb| |\\
\begin{tabular}{ll}
Hipóteses & Conclusão \\
$\gamma_1,\gamma_2,...,\gamma_n$ & $\alpha\land\beta$\\
\end{tabular}

\textit{Rascunho depois de usar a estratégia}.
\verb| |\\
\begin{tabular}{ll}
Hipóteses & Conclusão \\
$\gamma_1,\gamma_2,...,\gamma_n$ & $\alpha$\\
                                                            & $\beta$\\
\end{tabular}
\end{flushleft}
\end{ProofStrategy}
Uma conclusão da forma $\alpha\land \beta$ deve ser considerada como
``duas'' conclusões\footnote{Note que isso é um abuso de linguagem, já
  que a conclusão de um teorema é única.}:
$\alpha$ e $\beta$. De maneira similar, tratamos hipóteses envolvendo
conjunções
\begin{HypothesisStrategy}[Para usar uma hipótese da forma
  $\alpha\land\beta$]
Considere-a como duas hipóteses separadas: $\alpha$ e $\beta$.
\end{HypothesisStrategy}
Agora que vimos como manipular conjunções em provas, você já deve ser
capaz de deduzir como serão as estratégias para manipulação de
bicondicionais, uma vez que $\alpha\leftrightarrow\beta\equiv(\alpha\to\beta)\land(\beta\to\alpha)$.
\begin{ProofStrategy}[Para provar uma conclusão da forma
  $\alpha\leftrightarrow\beta$]
Prove $\alpha\to\beta$ e $\beta\to\alpha$ separadamente.
\begin{flushleft}
\textbf{Texto:}\\
$(\to)$: [Prova de $\alpha\to\beta$].\\
$(\leftarrow)$: [Prova de $\beta\to\alpha$].
\end{flushleft}
\end{ProofStrategy}
Note que ao contrário da estratégia de provas para a conjunção,
apresentamos um modelo de texto para o bicondicional. Isto se deve ao
fato de que provas envolvendo este conectivo usualmente ``sinalizam''
qual lado da implicação está sendo demonstrado utilizando setas
apropriadas.

A manipulação de hipóteses envolvendo bicondicionais é imediata.
\begin{HypothesisStrategy}[Para usar uma hipótese da forma
  $\alpha\leftrightarrow\beta$]
Trate-a como duas hipóteses distintas: $\alpha\to\beta$ e
$\beta\to\alpha$.
\end{HypothesisStrategy}

\begin{Example}
Considere o seguinte teorema:
\begin{flushleft}
Suponha $n\in\mathbb{Z}$. Então, $n$ é par se e somente se $n^2$ é par.
\end{flushleft}
Este teorema possui como hipóteses o fato de que $n\in\mathbb{Z}$ e
conclusão $n$ é par se e somente se $n^2$ é par, que é representada
pela seguinte fórmula:
\begin{center}
$n$ é par $\leftrightarrow$ $n^2$ é par.
\end{center}
o que nos conduz a seguinte configuração inicial do rascunho:
\begin{flushleft}
\begin{tabular}{ll}
Hipóteses & Conclusão \\
$n\in\mathbb{Z}$ & $n$ é par $\leftrightarrow$ $n^2$ é par
\end{tabular}
\end{flushleft}
Utilizando a estratégia de provas para o conectivo bicondicional,
obtemos o seguinte rascunho:
\begin{flushleft}
\begin{tabular}{ll}
Hipóteses & Conclusão \\
$n\in\mathbb{Z}$ & $n$ é par $\to$ $n^2$ é par\\
 & $n^2$ é par $\to$ $n$ é par
\end{tabular}
\end{flushleft}
 Para facilitar a construção da demonstração, vamos dividir o rascunho
 em duas provas distintas, uma para cada uma das
 implicações. Primeiramente, temos:
\begin{flushleft}
\begin{tabular}{ll}
Hipóteses & Conclusão \\
$n\in\mathbb{Z}$ & $n$ é par $\to$ $n^2$ é par\\
\end{tabular}
\end{flushleft}
Utilizando a estratégia de prova direta, obtemos:
\begin{flushleft}
\begin{tabular}{ll}
Hipóteses & Conclusão \\
$n\in\mathbb{Z}$ & $n^2$ é par\\
$n$ é par  & \\
\end{tabular}
\end{flushleft}
Evidentemente, podemos representar o fato de que um número $x$ é par
por $\exists y. x = 2y$. Usando esta representação:
\begin{flushleft}
\begin{tabular}{ll}
Hipóteses & Conclusão \\
$n\in\mathbb{Z}$ & $\exists k . n^2 = 2k$ \\
$\exists m. n = 2m$  & \\
\end{tabular}
\end{flushleft}
Usando a estratégia de hipóteses para o quantificador existencial,
obtemos:
\begin{flushleft}
\begin{tabular}{ll}
Hipóteses & Conclusão \\
$n\in\mathbb{Z}$ & $\exists k . n^2 = 2k$ \\
$\exists m. n = 2m$  & \\
$n = 2m$ &
\end{tabular}
\end{flushleft}
Agora, resta encontrar um valor de $k$ que torne a igualdade $n^2 =
2k$ verdadeira. Note que possuímos como hipótese que $n = 2m$. Logo,
temos que $n^2 = 4m^2$ e desta forma, temos que $k = 2m^2$, uma vez
que $n^2 = 2k$ e $n^2 = 4m^2$.

Para a segunda implicação, ao invés de utilizarmos a técnica de prova
direta, usaremos demonstração pela contrapositiva. Inicialmente, temos
a seguinte configuração do rascunho:
\begin{flushleft}
\begin{tabular}{ll}
Hipóteses & Conclusão \\
$n\in\mathbb{Z}$ & $n^2$ é par $\to$ $n$ é par\\
\end{tabular}
\end{flushleft}
Ao usarmos a técnica de prova pela contrapositiva, temos:
\begin{flushleft}
\begin{tabular}{ll}
Hipóteses & Conclusão \\
$n\in\mathbb{Z}$ & $\neg(n^2\text{ é par })$\\
$\neg(n\text{ é par})$ & \\
\end{tabular}
\end{flushleft}
Evidentemente, como $n\in\mathbb{Z}$, se $n$ não é par, temos que este
deve ser ímpar. Logo:
\begin{flushleft}
\begin{tabular}{ll}
Hipóteses & Conclusão \\
$n\in\mathbb{Z}$ & $n^2\text{ é ímpar }$\\
$n\text{ é ímpar}$ & \\
\end{tabular}
\end{flushleft}
Representamos o fato de que $x$ é um número ímpar por $\exists y. x = 2y + 1$:
\begin{flushleft}
\begin{tabular}{ll}
Hipóteses & Conclusão \\
$n\in\mathbb{Z}$ & $\exists k. n^2 = 2k + 1$\\
$\exists m. n = 2m + 1$ & \\
\end{tabular}
\end{flushleft}
Usando a hipótese existencial, obtemos que $n = 2m + 1$.
\begin{flushleft}
\begin{tabular}{ll}
Hipóteses & Conclusão \\
$n\in\mathbb{Z}$ & $\exists k. n^2 = 2k + 1$\\
$\exists m. n = 2m + 1$ & \\
$n = 2m + 1$ &
\end{tabular}
\end{flushleft}
Se $n = 2m + 1$, temos que $n^2 = 4m^2 + 4m + 1$. Para terminar a
demonstração, devemos encontrar um valor $k$ tal que $n^2 = 2k +
1$. Usando o fato de que $n^2 = 4m^2 + 4m + 1$ chega-se a conclusão de
que $k = 2m^2 + 2m$\footnote{Note que não apresentamos a demonstração
  da igualdade anterior e desta propositalmente, visto que estas serão
apresentadas no texto final desta prova.}.

Agora resta construirmos o texto a partir deste rascunho. Como já
feito em outros exemplos, faremos a construção deste passo a
passo. Primeiramente, utilizamos a estratégia de prova para o
conectivo bicondicional.
\begin{flushleft}
Suponha $n\in\mathbb{Z}$.\\
\verb|  |($\to$): [Prova de $n$ é par $\to$ $n^2$ é par]\\
\verb|  |($\to$): [Prova de $n^2$ é par $\to$ $n$ é par]\\
Portanto, se $n\in\mathbb{Z}$ então $n$ é par sse $n^2$ é par.
\end{flushleft}
Provando a primeira implicação por prova direta, obtemos a seguinte
versão parcial do texto:
\begin{flushleft}
Suponha $n\in\mathbb{Z}$.\\
\verb|  |($\to$): Suponha $n$ é par.\\
\verb|    |[Prova de $n^2$ é par].\\
\verb|  |Logo, se $n$ é par, $n^2$ é par.\\
\verb|  |($\to$): [Prova de $n^2$ é par $\to$ $n$ é par]\\
Portanto, se $n\in\mathbb{Z}$ então $n$ é par sse $n^2$ é par.
\end{flushleft}
Para completar a primeira parte da demonstração, basta provar que
$n^2$ é par usando a estratégia de prova para quantificadores
existenciais.
\begin{flushleft}
Suponha $n\in\mathbb{Z}$.\\
\verb|  |($\to$): Suponha $n$ é par.\\
\verb|    |Como $n$ é par, temos que $n = 2m$.\\
\verb|    |Seja $k = 2m^2$. Temos:\\
\[
\begin{array}{lc}
2k & = \\
2.2m^2 & = \\
4m^2 & = \\
(2m)^2 & = \\
n^2
\end{array}
\]\verb|    |Logo, $n^2$ é par.\\
\verb|  |Logo, se $n$ é par, $n^2$ é par.\\
\verb|  |($\to$): [Prova de $n^2$ é par $\to$ $n$ é par]\\
Portanto, se $n\in\mathbb{Z}$ então $n$ é par sse $n^2$ é par.
\end{flushleft}
Para a segunda implicação, utilizaremos a estratégia da
contrapositiva, seguida da estratégia para o quantificador existencial.
\begin{flushleft}
Suponha $n\in\mathbb{Z}$.\\
\verb|  |($\to$): Suponha $n$ é par.\\
\verb|    |Como $n$ é par, temos que $n = 2m$.\\
\verb|    |Seja $k = 2m^2$. Temos:\\
\[
\begin{array}{lc}
2k & = \\
2.2m^2 & = \\
4m^2 & = \\
(2m)^2 & = \\
n^2
\end{array}
\]
\verb|    |Logo, $n^2$ é par.\\
\verb|  |Logo, se $n$ é par, $n^2$ é par.\\
\verb|  |($\to$): Suponha que $n$ é ímpar.\\
\verb|    |Como $n$ é ímpar, temos que $n = 2m + 1$.\\
\verb|    |Seja $k = 2m^2 + 2m$. Temos:\\
\[
\begin{array}{lc}
2k+1 & = \\
2(2m^2 + 2m) + 1 & = \\
4m^2 + 4m + 1 & = \\
(2m + 1)^2 & = \\
n^2
\end{array}
\]
\verb|    |Logo, $n^2$ é ímpar.\\
\verb|  |Logo, se $n^2$ é par, $n$ é par.\\
Portanto, se $n\in\mathbb{Z}$ então $n$ é par se e somente se $n^2$ é par.
\end{flushleft}
\end{Example}

\subsection{Exercícios}

\begin{enumerate}
  \item Prove os seguintes teoremas:
    \begin{enumerate}
      \item Suponha $x,y\in\mathbb{Z}$ ímpares. Então $xy$ é um número
        ímpar.
      \item Suponha $n\in\mathbb{Z}$. Então $n^3$ é par se e somente
        se $n$ é par.
      \item Suponha $a,b\in\mathbb{Z}$ arbitrários. Então, existe
        $c\in\mathbb{Z}$ tal que $a\,|\,c$ e $b\,|\,c$.
      \item Para todo $n\in\mathbb{Z}$, $15\,|\,n$ se e somente se
        $3\,|\,n$ e $5\,|\,n$.
    \end{enumerate}
\end{enumerate}

\subsection{Estratégias para Disjunção ($\lor$)}

Suponha que você possua uma hipótese da forma $\alpha\lor\beta$. Como
utilizar esta hipótese para deduzir uma conclusão $\gamma$? Na dedução
natural, a regra de eliminação da disjunção fornece uma forma de como
utilizar $\alpha\lor\beta$ para deduzir $\gamma$: primeiramente,
supomos que $\alpha$ é verdade e deduzimos $\gamma$ e na sequência
deduzimos $\gamma$ a partir de $\beta$.

A utilização da eliminação da disjunção é comumente denominada de
``prova por análise de casos'' pois, considera-se cada uma das
possibilidades de $\alpha\lor\beta$ ser verdadeiro para construção da
demonstração. A seguir, apresentamos a estratégia de uso de hipóteses
que resume esta idéia.

\begin{HypothesisStrategy}[Para usar uma hipótese da forma
  $\alpha\lor\beta$]
Divida sua prova em casos. No primeiro caso, assuma que $\alpha$ é
verdadeiro e deduza a conclusão $\gamma$. No segundo caso, assuma que
$\beta$ é verdadeiro e deduza a conclusão $\gamma$.
\begin{flushleft}
\textbf{Rascunho}.\\
\verb| |\\

\textit{Rascunho antes de usar a estratégia}.
\verb| |\\
\begin{tabular}{ll}
Hipóteses & Conclusão \\
$\alpha_1,\alpha_2,...,\alpha_n$ & $\gamma$\\
$\alpha\lor\beta$ & \\
\end{tabular}

\textit{Rascunho depois de usar a estratégia}.
\verb| |\\
\begin{tabular}{ll}
Hipóteses & Conclusão \\
Caso 1: & \\
$\alpha_1,\alpha_2,...,\alpha_n$ & $\gamma$\\
$\alpha$ & \\
Caso 2: & \\
$\alpha_1,\alpha_2,...,\alpha_n$ & $\gamma$\\
$\beta$ & \\
\end{tabular}
\end{flushleft}
\begin{flushleft}
\textbf{Texto:}\\
\textit{Caso 1: $\alpha$ é verdadeiro.}.\\
\verb|    |[\textit{Prova de $\gamma$}]\\
\textit{Caso 2: $\beta$ é verdadeiro.}.\\
\verb|    |[\textit{Prova de $\gamma$}]\\
Como os casos cobrem todas as possibilidades, temos que $\gamma$.
\end{flushleft}
\end{HypothesisStrategy}

A seguir apresentamos um exemplo de uso desta estratégia de prova.
\begin{Example}
Considere a tarefa de demonstrar o seguinte teorema:
\begin{flushleft}
Suponha que $x\in\mathbb{R}$ é arbitrário. Então se $|x - 3| > 3$
então $x^2 > 6x$.
\end{flushleft}
Neste exemplo, possuímos como hipótese que $x\in\mathbb{R}$ e a
conclusão é representada pela seguinte fórmula:
\[
|x - 3| > 3 \to x^2 > 6x
\]
Logo, a versão inicial do rascunho possui a seguinte forma:
\begin{flushleft}
\begin{tabular}{ll}
Hipóteses & Conclusão \\
$x\in\mathbb{R}$ & $|x - 3| > 3 \to x^2 > 6x$\\
\end{tabular}
\end{flushleft}
Como a conclusão possui uma implicação, iniciamos a dedução utilizando
a técnica de prova direta, o que nos leva a:
\begin{flushleft}
\begin{tabular}{ll}
Hipóteses & Conclusão \\
$x\in\mathbb{R}$ & $x^2 > 6x$\\
$|x - 3| > 3$ &
\end{tabular}
\end{flushleft}
Note que para usarmos a hipótese $|x- 3| > 3$ devemos saber se $x - 3
\geq 0$ ou se $x - 3 < 0$. Logo, devemos considerar uma divisão da
prova em casos, usando a estratégia apresentada anteriormente.
\begin{flushleft}
\begin{tabular}{ll}
Hipóteses & Conclusão \\
Caso 1: & \\
$x\in\mathbb{R}$ & $x^2 > 6x$\\
$|x - 3| > 3$ & \\
$x - 3 \geq 0$ & \\
Caso 2: & \\
$x\in\mathbb{R}$ & $x^2 > 6x$\\
$|x - 3| > 3$ & \\
$x - 3 < 0$ & \\
\end{tabular}
\end{flushleft}
Se $x- 3 \geq 0$, temos que $|x - 3| = x - 3$. Assim, temos que $|x -
3| > 3 \equiv x - 3 > 3 \equiv x > 6$. Logo, $x^2 > 6x$.

Por sua vez, se $x - 3 < 0$, temos que $|x - 3| = 3 - x$. Assim, temos
que $|x - 3| > 3 \equiv 3 - x > 3 \equiv -x > 3 - 3 \equiv - x > 0
\equiv x < 0$. Logo,  $x^2 > 6x$.

Como terminamos a dedução da conclusão a partir das hipóteses, vamos
proceder para a construção passo a passo do texto. Inicialmente,
utilizamos a estratégia de prova direta.
\begin{flushleft}
Suponha $x\in\mathbb{R}$.\\
\verb| |Suponha que $|x-3| > 3$.\\
\verb|    |[Prova $x^2 > 6x$.]\\
\verb| |Logo, se $|x-3| > 3$ então $x^2 > 6x$.\\
Logo, se $x\in\mathbb{R}$ então se $|x - 3| > 3$ então $x^2 > 6x$.
\end{flushleft}
Dividindo a prova em casos, temos que:
\begin{flushleft}
Suponha $x\in\mathbb{R}$.\\
\verb| |Suponha que $|x-3| > 3$.\\
\verb|    |Caso 1: $x - 3 \geq 0$.\\
\verb|       | [Prova de $x^2 > 6x$]\\
\verb|    |Caso 2: $x - 3 < 0$.\\
\verb|       | [Prova de $x^2 > 6x$]\\
\verb|    |Como os casos cobrem todas as possibilidades, temos que
$x^2 > 6x$.\\
\verb| |Logo, se $|x-3| > 3$ então $x^2 > 6x$.
Logo, se $x\in\mathbb{R}$ então se $|x - 3| > 3$ então $x^2 > 6x$.
\end{flushleft}
Finalmente, concluímos o texto apresentando a dedução de $x^2 > 6x$ em
cada caso.
\begin{flushleft}
Suponha $x\in\mathbb{R}$.\\
\verb| |Suponha que $|x-3| > 3$.\\
\verb|    |Caso 1: $x - 3 \geq 0$.\\
\verb|       |Como $x - 3 \geq 0$, temos que $|x - 3| = x - 3$.\\
\verb|       |Como $|x - 3| = x - 3$ e $|x - 3| > 3$, temos que $x >
6$.\\
\verb|       |Como $x > 6$, temos que $x^2 > 6x$.\\
\verb|    |Caso 2: $x - 3 < 0$.\\
\verb|       |Como $x - 3 < 0$, temos que $|x - 3| = 3 -x$.\\
\verb|       |Como $|x - 3| > 3$ e $|x - 3| = 3 - x$, temos que $x <
0$.\\
\verb|       |Como $x < 0$, temos que $x^2 > 6x$.\\
\verb|    |Como os casos cobrem todas as possibilidades, temos que
$x^2 > 6x$.\\
\verb| |Logo, se $|x-3| > 3$ então $x^2 > 6x$.\\
Logo, se $x\in\mathbb{R}$ então se $|x - 3| > 3$ então $x^2 > 6x$.
\end{flushleft}
\end{Example}

Para demonstrar uma conclusão que é uma disjunção, devemos proceder de
maneira similar às regras de introdução deste conectivo, conforme
apresentado na estratégia seguinte.

\begin{ProofStrategy}[Para provar uma conclusão da forma
  $\alpha\lor\beta$]
Prove $\alpha$ ou prove $\beta$.
\end{ProofStrategy}

O próximo exemplo ilustra esta estratégia.

\begin{Example}
Considere a tarefa de provar o seguinte teorema:
\begin{flushleft}
Para todo $x\in\mathbb{Z}$, o resto da divisão de $x^2$ por $4$ é $0$
ou $1$.
\end{flushleft}
O teorema considerado pode ser representado pela seguinte fórmula, que
corresponde a sua conclusão:
\[
\forall x. x \in\mathbb{Z} \to x^2 \text{ mod } 4 = 0 \lor x^2 \text{
  mod } 4 = 1
\]
Assim, temos a seguinte versão inicial do rascunho:
\begin{flushleft}
\begin{tabular}{ll}
Hipóteses & Conclusão \\
  & $\forall x. x \in\mathbb{Z} \to x^2 \text{ mod } 4 = 0 \lor x^2 \text{
  mod } 4 = 1$\\
\end{tabular}
\end{flushleft}
Como a conclusão envolve uma fórmula que utiliza o quantificador
universal e uma implicação, iniciamos a introdução utilizando as
estratégias para estes símbolos da lógica.
\begin{flushleft}
\begin{tabular}{ll}
Hipóteses & Conclusão \\
 $x$ arbitrário & $x^2 \text{ mod } 4 = 0 \lor x^2 \text{
  mod } 4 = 1$\\
$x\in\mathbb{Z}$ & \\
\end{tabular}
\end{flushleft}
Neste ponto, surge a seguinte questão: Como continuar com esta prova?
Visto que as hipóteses não acrescentam nenhuma idéia de como
concluí-la, vamos montar uma tabela com alguns valores simples para
tentar perceber se existe alguma estrutura ``oculta''.
\begin{table}
\begin{tabular}{cccc}
$x$ & $x^2$ & $x^2 \div 4$ & $x^2 \text{ mod } 4$ \\
\hline
1 & 1 & 0 & 1 \\
2 & 4 & 1 & 0 \\
3 & 9 & 2 & 1 \\
4 & 16 & 4 & 0 \\
5 & 25 & 6 & 1 \\
\end{tabular}
\centering
\end{table}
Aparentemente, temos que o resto é zero sempre que $x$ é par e um caso
$x$ é ímpar. Logo, dividiremos a prova em casos. No primeiro caso,
consideraremos que $x$ é par e no segundo que $x$ é ímpar.
\begin{flushleft}
\begin{tabular}{ll}
Hipóteses & Conclusão \\
Caso 1: & \\
$x$ é par & $x^2 \text{ mod } 4 = 0 \lor x^2 \text{
  mod } 4 = 1$\\
 $x$ arbitrário & \\
$x\in\mathbb{Z}$ & \\
Caso 2: & \\
$x$ é ímpar & $x^2 \text{ mod } 4 = 0 \lor x^2 \text{
  mod } 4 = 1$\\
 $x$ arbitrário & \\
$x\in\mathbb{Z}$ & \\
\end{tabular}
\end{flushleft}
Utilizando as definições de $x$ é par e $x$ é ímpar, temos que:
\begin{flushleft}
\begin{tabular}{ll}
Hipóteses & Conclusão \\
Caso 1: & \\
$x$ é par & $x^2 \text{ mod } 4 = 0 \lor x^2 \text{
  mod } 4 = 1$\\
 $x$ arbitrário & \\
$x\in\mathbb{Z}$ & \\
$\exists k_1. x= 2k_1$ & \\
Caso 2: & \\
$x$ é ímpar & $x^2 \text{ mod } 4 = 0 \lor x^2 \text{
  mod } 4 = 1$\\
 $x$ arbitrário & \\
$x\in\mathbb{Z}$ & \\
$\exists k_2. x= 2k_2 + 1$ & \\
\end{tabular}
\end{flushleft}
Se $x$ é par, existe $k_1$ tal que $x = 2k_1$. Assim, temos que $x^2 =
(2k_1)^2 = 4k_1^2$, que é divisível por 4 (resto igual a zero). Então,
neste caso, optamos por demonstrar o lado esquerdo de $x^2 \text{ mod } 4 = 0 \lor x^2 \text{
  mod } 4 = 1$, o que corresponde a regra de introdução da disjunção à
esquerda.

Caso $x$ seja ímpar, existe $k_2$ tal que $x = 2k_2 + 1$. Assim, temos
que $x^2 = (2k_2 + 1)^2 = (2k_2)^2 + 2(2k_2) + 1 = 4k_2^2 + 4k_2 + 1 =
4(k^2 + k) + 1$, que dividido por 4 deixa um resto igual a um.
Então,
neste caso, optamos por demonstrar o lado direito de $x^2 \text{ mod } 4 = 0 \lor x^2 \text{
  mod } 4 = 1$, o que corresponde a regra de introdução da disjunção à
direita.

A construção do texto desta demonstração é apresentado a seguir.
\begin{flushleft}
Suponha $x$ arbitrário.\\
\verb|   |[Prova de $x\in\mathbb{Z}\to x^2\text{ mod }4 = 0 \lor
x^2\text{ mod }4 = 1$].\\
Como $x$ é arbitrário, temos que para todo $x\in\mathbb{Z}$ o resto da divisão de $x^2$ por $4$ é $0$
ou $1$.
\end{flushleft}
Utilizando o texto para demonstrações de implicações, temos:
\begin{flushleft}
Suponha $x$ arbitrário.\\
\verb|   |Suponha que $x\in\mathbb{Z}$.\\
\verb|      |[Prova de $x^2\text{ mod }4 = 0 \lor
x^2\text{ mod }4 = 1$].\\
\verb|   |Logo, se $x\in\mathbb{Z}$ temos que  $x^2\text{ mod }4 = 0 \lor
x^2\text{ mod }4 = 1$.\\
Como $x$ é arbitrário, temos que para todo $x\in\mathbb{Z}$ o resto da divisão de $x^2$ por $4$ é $0$
ou $1$.
\end{flushleft}
Neste ponto, consideramos os casos de que todo $x\in\mathbb{Z}$ é par
ou ímpar.
\begin{flushleft}
Suponha $x$ arbitrário.\\
\verb|   |Suponha que $x\in\mathbb{Z}$.\\
\verb|      |Caso 1: $x$ é par.\\
\verb|         |[Prova de $x^2\text{ mod }4 = 0 \lor
x^2\text{ mod }4 = 1$].\\
\verb|      |Caso 2: $x$ é ímpar.\\
\verb|         |[Prova de $x^2\text{ mod }4 = 0 \lor
x^2\text{ mod }4 = 1$].\\
\verb|   |Logo, se $x\in\mathbb{Z}$ temos que  $x^2\text{ mod }4 = 0 \lor
x^2\text{ mod }4 = 1$.\\
Como $x$ é arbitrário, temos que para todo $x\in\mathbb{Z}$ o resto da divisão de $x^2$ por $4$ é $0$
ou $1$.
\end{flushleft}

Agora, para o caso de $x$ ser par, provamos que $x^2\text{ mod }4 = 0$
e para o caso de ser ímpar, provamos que $x^2\text{ mod }4 = 1$.
\begin{flushleft}
Suponha $x$ arbitrário.\\
\verb|   |Suponha que $x\in\mathbb{Z}$.\\
\verb|      |Caso 1: $x$ é par.\\
\verb|         |[Prova de $x^2\text{ mod }4 = 0$].\\
\verb|      |Caso 2: $x$ é ímpar.\\
\verb|         |[Prova de $x^2\text{ mod }4 = 1$].\\
\verb|   |Logo, se $x\in\mathbb{Z}$ temos que  $x^2\text{ mod }4 = 0 \lor
x^2\text{ mod }4 = 1$.\\
Como $x$ é arbitrário, temos que para todo $x\in\mathbb{Z}$ o resto da divisão de $x^2$ por $4$ é $0$
ou $1$.
\end{flushleft}
Finalmente, concluímos a prova utilizando a hipótese existencial de
que $x$ é par ou ímpar em cada caso.
\begin{flushleft}
Suponha $x$ arbitrário.\\
\verb|   |Suponha que $x\in\mathbb{Z}$.\\
\verb|      |Caso 1: $x$ é par.\\
\verb|         |Como $x$ é par, existe $k_1$ tal que $x = 2k_1$.\\
\verb|         |Como $x = 2k_1$, temos que $x^2 = (2k_1)^2 = 4k_1^2$.\\
\verb|         |Como $x^2 = 4k_1^2$, temos que $x^2\text{ mod }4 =
0$.\\
\verb|      |Logo, $x^2\text{ mod }4 = 0$ ou $x^2\text{ mod }4 = 1$\\
\verb|      |Caso 2: $x$ é ímpar.\\
\verb|         |Como $x$ é ímpar, existe $k_2$ tal que $x = 2k_2+1$.\\
\verb|         |Como $x = 2k_2 + 1$, temos que $x^2 = (2k_2 + 1)^2 =
4k_2^2 + 4k_2 + 1$.\\
\verb|         |Como $x^2 = 4k_2^2 + 4k_2 + 1$, temos que $x^2\text{ mod }4 =
1$.\\
\verb|      |Logo, $x^2\text{ mod }4 = 0$ ou $x^2\text{ mod }4 = 1$\\
\verb|   |Logo, se $x\in\mathbb{Z}$ temos que  $x^2\text{ mod }4 = 0 \lor
x^2\text{ mod }4 = 1$.\\
Como $x$ é arbitrário, temos que para todo $x\in\mathbb{Z}$ o resto da divisão de $x^2$ por $4$ é $0$
ou $1$.
\end{flushleft}
\end{Example}
Ainda resta uma última técnica pode ser utilizada para manipular hipóteses ou
conclusões da forma $\alpha\lor\beta$. Esta técnica é baseada nas
seguintes equivalências: $\alpha\lor\beta \equiv \neg \alpha \to \beta
\equiv \neg \beta \to \alpha$\footnote{Demonstre essas equivalências!}.
\begin{ProofStrategy}[Para provar uma conclusão da forma
  $\alpha\lor\beta$]
Se $\alpha$ é verdadeiro, é evidente que $\alpha\lor\beta$ é
verdadeiro. Logo, podemos supor que $\alpha$ é falso e demonstrar que
$\beta$ é verdadeiro para concluir $\alpha\lor\beta$.
\begin{flushleft}
\textbf{Rascunho}.\\
\verb| |\\

\textit{Rascunho antes de usar a estratégia}.
\verb| |\\
\begin{tabular}{ll}
Hipóteses & Conclusão \\
$\alpha_1,\alpha_2,...,\alpha_n$ & $\alpha\lor\beta$\\
\end{tabular}

\textit{Rascunho depois de usar a estratégia}.
\verb| |\\
\begin{tabular}{ll}
Hipóteses & Conclusão \\
$\alpha_1,\alpha_2,...,\alpha_n$ & $\beta$\\
$\neg \alpha$ & \\
\end{tabular}
\end{flushleft}
\begin{flushleft}
\textbf{Texto:}\\
\textit{Se $\alpha$ é verdadeiro, então $\alpha\lor\beta$ é
  verdadeiro. Então, suponha que $\neg\alpha$.}\\
\verb|   |[Prova de $\beta$].\\
\textit{Logo, temos que $\alpha\lor\beta$.}
\end{flushleft}
\end{ProofStrategy}
\begin{Example}
Considere demonstrar o seguinte teorema simples:
\begin{flushleft}
  Para todo $x\in\mathbb{R}$, se $x^2 \geq x$ então $x \leq 0$ ou
  $x\geq 1$.
\end{flushleft}
Seguindo os passos já apresentados para demonstração de teoremas,
temos que o teorema acima é representado pela seguinte fórmula:
\[
\forall x. x \in \mathbb{R} \to x^2 \geq x \to x \leq 0 \lor x \geq 1
\]
A configuração inicial do rascunho é:
\begin{flushleft}
\begin{tabular}{ll}
Hipóteses & Conclusão \\
 & $\forall x. x \in \mathbb{R} \to x^2 \geq x \to x \leq 0 \lor x \geq 1$
\end{tabular}
\end{flushleft}
Devido a composição desta fórmula, iniciaremos a demonstração deste
teorema utilizando as técnicas para o quantificador universal e
implicação (prova direta). Com isso, obtemos:
\begin{flushleft}
\begin{tabular}{ll}
Hipóteses & Conclusão \\
 $x$ arbitrário & \\ $x \leq 0 \lor x \geq 1$\\
$x \in \mathbb{R} $ & \\
$x^2 \geq x$ &
\end{tabular}
\end{flushleft}
Agora, utilizaremos a estratégia de considerar que $\alpha\lor\beta$ é
equivalente a $\neg \alpha \to \beta$:
\begin{flushleft}
\begin{tabular}{ll}
Hipóteses & Conclusão \\
 $x$ arbitrário & $x \geq 1$\\
$x \in \mathbb{R} $ & \\
$x^2 \geq x$ &\\
$x > 0$
\end{tabular}
\end{flushleft}
É óbvio que $\neg (x\leq 0)\equiv x > 0$. Como $x > 0$ e $x^2 \geq x$,
dividindo ambos os lados da última desigualdade por $x$, obtemos
$x\geq 1$, conforme requerido. O texto é construído passo a passo
utilizando os modelos para cada uma das estratégias
utilizadas. Inicialmente, o texto para o quantificador universal e
provas diretas.
\begin{flushleft}
Suponha $x$ arbitrário.\\
\verb|  |Suponha $x\in\mathbb{R}$.\\
\verb|     |Suponha $x^2 \geq x$.\\
\verb|       |[Prova de $x \leq 0 \lor x \geq 1$].\\
\verb|     |Logo, se $x^2\geq x$ então $x \leq 0 \lor x \geq 1$.\\
\verb|  |Logo, se $x\in\mathbb{R}$, então se $x^2\geq x$ então $x \leq 0 \lor x \geq 1$.\\
Como $x$ é arbitrário, temos que para todo $x\in\mathbb{R}$, se $x^2 \geq x$ então $x \leq 0$ ou
  $x\geq 1$.
\end{flushleft}
Agora, utilizando o modelo de texto para disjunção, temos:
\begin{flushleft}
Suponha $x$ arbitrário.\\
\verb|  |Suponha $x\in\mathbb{R}$.\\
\verb|     |Suponha $x^2 \geq x$.\\
\verb|       |Se $x\leq 0$, temos que $x \leq 0$ ou $x \geq 1$. Então,
suponha $x >0$.
\verb|          |[Prova de $x \geq 1$].\\
\verb|       |Logo, $x \leq 0$ ou $x \geq 1$.\\
\verb|     |Logo, se $x^2\geq x$ então $x \leq 0 \lor x \geq 1$.\\
\verb|  |Logo, se $x\in\mathbb{R}$, então se $x^2\geq x$ então $x \leq 0 \lor x \geq 1$.\\
Como $x$ é arbitrário, temos que para todo $x\in\mathbb{R}$, se $x^2 \geq x$ então $x \leq 0$ ou
  $x\geq 1$.
\end{flushleft}
Finalmente, encerramos o texto desta demonstração utilizando a dedução
de $x\geq 1$ a partir de $x^2 \geq x$ e $x > 0$.
\begin{flushleft}
Suponha $x$ arbitrário.\\
\verb|  |Suponha $x\in\mathbb{R}$.\\
\verb|     |Suponha $x^2 \geq x$.\\
\verb|       |Se $x\leq 0$, temos que $x \leq 0$ ou $x \geq 1$. Então,
suponha $x >0$.
\verb|          |Como $x^2 \geq x$ e $x > 0$, temos que $x \geq 1$.\\
\verb|       |Logo, $x \leq 0$ ou $x \geq 1$.\\
\verb|     |Logo, se $x^2 \geq x$ então $x \leq 0 \lor x \geq 1$.\\
\verb|  |Logo, se $x\in\mathbb{R}$, então se $x^2\geq x$ então $x \leq 0 \lor x \geq 1$.\\
Como $x$ é arbitrário, temos que para todo $x\in\mathbb{R}$, se $x^2 \geq x$ então $x \leq 0$ ou
  $x\geq 1$.
\end{flushleft}
\end{Example}

A próxima estratégia de uso de hipóteses mostra como podemos usar uma
disjunção como uma implicação.

\begin{HypothesisStrategy}[Para utilizar uma hipótese da forma
  $\alpha\lor\beta$]
Considere-a equivalente a $\neg\alpha\to \beta$ ou a $\neg\beta\to\alpha$.
\end{HypothesisStrategy}


\subsection{Exercícios}

\begin{enumerate}
\item Prove os seguintes teoremas:
\begin{enumerate}
    \item Suponha $x,y\in\mathbb{R}$ e que $x\neq 0$. Então, $y +
      \frac{1}{x} = 1 +\frac{y}{x}$ se e somente se $x = 1$ ou $y =
      1$.
    \item Para todo $x\in\mathbb{Z}$, $x^2 + x$ é par.
    \item Para todo $a,b\in\mathbb{R}$, $|a|\leq b$ se e somente se $-b
      \leq a \leq b$.
    \item Para todo $x\in\mathbb{R}$, $|2x - 6| > x$ se e somente se
      $|x - 4| > 2$.
\end{enumerate}
\end{enumerate}


\subsection{Existência e Unicidade}

Em matemática é comum a especificação de propriedades similares a
``existe um único elemento $x$ que possui uma propriedade $P$''.
Considerando um certo universo de discurso $U$, dizemos que existe um
único elemento de $U$ que satisfaz uma propriedade $P$ usando a seguinte
fórmula:
\[
\exists x. P(x) \land \neg \exists y. P(y) \land y \neq x.
\]
Que essencialmente especifica que não existe um elemento diferente de
$x$ que satisfaça $P$. Normalmente, matemáticos expressam esta fórmula
como um novo quantificador (representado por $\exists !$). Utilizando
este quantificador, a fórmula anterior pode ser representada de
maneira mais concisa como $\exists ! x.P(x)$.

Porém, a fórmula

\[\exists x. P(x) \land \neg \exists y. P(y) \land y \neq x.\]

não é a única maneira de representarmos $\exists ! x. P(x)$. Se
utilizarmos um pouco de álgebra booleana podemos eliminar a negação da
fórmula anterior, conforme apresentado a seguir:

\[
\begin{array}{lc}
\exists x. P(x) \land \neg \exists y. P(y) \land y \neq x & \equiv \\
\exists x. P(x) \land \forall y. \neg (P(y)\land y \neq x) & \equiv \\
\exists x. P(x) \land \forall y. \neg P(y) \lor \neg y \neq x &
\equiv\\
\exists x .P(x) \land \forall y. P(y) \to x = y &
\end{array}
\]
Note que esta versão possui a vantagem de não envolver negação, o que
usualmente facilita as demonstrações. Outra fórmula equivalente a
$\exists ! x. P(x)$ é:

\[
\exists x. P(x) \land \forall y.\forall z. P(y)\land P(z) \to y = z
\]

Note que a última fórmula apresentada é bastante similar a $\exists x
.P(x) \land \forall y. P(y) \to x = y$. A diferença é a introdução da
nova variável quantificada $z$. Acredito que o leitor deva estar se
perguntando, ``mas porquê introduzir uma nova variável?''. O ponto é
que na fórmula
\[
\exists x. P(x) \land \forall y.\forall z. P(y)\land P(z) \to y = z
\]
a variável $x$ não aparece livre em $\forall y.\forall z. P(y)\land
P(z) \to y = z$, o que nos permite dividir a tarefa de demonstrar
$\exists x. P(x) \land \forall y.\forall z. P(y)\land P(z) \to y = z$
nas demonstrações:
\begin{itemize}
  \item $\exists x. P(x)$
  \item $\forall y.\forall z. P(y)\land P(z) \to y = z$
\end{itemize}
o que não pode ser feito com a fórmula
\[
\exists x .P(x) \land \forall y. P(y) \to x = y
\]
já que o $x$ aparece livre em $\forall y. P(y) \to x = y$.

A utilização destas equivalências é o que determinará as estratégias
de prova para este quantificador.

\subsection{Estratégias para Existências e Unicidade}

Conforme discutido na seção anterior, existem diversas maneiras de se
representar o quantificador $\exists ! x. P(x)$ e estas determinam as
estratégias de demonstração e uso de hipóteses para este tipo de
fórmula. Estas estratégias são apresentadas a seguir.

\begin{ProofStrategy}[Para provar uma conclusão da forma $\exists
  !x. P(x)$.]
Prove $\exists x. P(x)$ e $\forall y.\forall z. P(x) \land P(z) \to y
= z$. A primeira parte da prova mostra que existe um valor $x$ tal que
$P(x)$ e a segunda mostra que este valor é único.

A construção do rascunho será omitida, visto que este utilizará
estratégias de prova adequadas para cada uma das partes desta
demonstração. Normalmente,  adicionamos um
rótulo no texto correspondente a cada uma das partes da prova. O rótulo ``Existência'' é utilizado para
a demonstração de $\exists x. P(x)$ e ``Unicidade'' é utilizado para
$\forall y.\forall z. P(x) \land P(z) \to y = z$. Isto é formalizado
pelo seguinte modelo de texto.
\begin{flushleft}
\textbf{Texto}:\\
\verb| |\\
Existência: [Prova de $\exists x.P(x)$]\\
Unicidade: [Prova de $\forall y.\forall z. P(x) \land P(z) \to y = z$]\\
\end{flushleft}
\end{ProofStrategy}

Outra possível estratégia de prova é baseada em outra equivalência
para $\exists ! x. P(x)$, conforme apresentado a seguir.

\begin{ProofStrategy}[Para provar uma conclusão da forma $\exists
  !x. P(x)$.]
Prove $\exists x.P(x) \land \forall y. P(y) \to x = y$ utilizando
outras estratégias de demonstração.
\end{ProofStrategy}

\begin{Example}
Considere a tarefa de demonstrar o seguinte teorema
\begin{flushleft}
Para todo $x\in\mathbb{R}$ se $x \neq 2$ então existe um único $y$ tal
que $\frac{2y}{y+ 1} = x$.
\end{flushleft}
Este teorema é representado pela seguinte conclusão:
\[
\forall x. x\in\mathbb{R} \to x \neq 2 \to \exists ! y . \frac{2y}{y +
1} = x
\]
o que nos leva ao seguinte rascunho inicial
\begin{flushleft}
\begin{tabular}{ll}
Hipóteses & Conclusão \\
 & $\forall x. x\in\mathbb{R} \to x \neq 2 \to \exists ! y . \frac{2y}{y +
1} = x
$
\end{tabular}
\end{flushleft}
que utilizando estratégias de prova já conhecidas nos leva a seguinte
situação do rascunho:
\begin{flushleft}
\begin{tabular}{ll}
Hipóteses & Conclusão \\
$x$ arbitrário &  $\exists ! y . \frac{2y}{y +
1} = x$ \\
$x\in\mathbb{R}$ & \\
$x\neq 2$ & \\
\end{tabular}
\end{flushleft}
Para concluir a demonstração, utilizaremos a seguinte equivalência:
\[
\exists ! x. P(x) \equiv \exists x.P(x) \land \forall y. P(y) \to x = y
\]
o que nos leva ao seguinte rascunho:
\begin{flushleft}
\begin{tabular}{ll}
Hipóteses & Conclusão \\
$x$ arbitrário &  $\exists y . \frac{2y}{y +
1} = x \land \forall z. \frac{2z}{z+1} = x \to z = y$ \\
$x\in\mathbb{R}$ & \\
$x\neq 2$ & \\
\end{tabular}
\end{flushleft}
Agora, temos que encontrar um valor de $y$ que permita provar que
$\exists y. \frac{2y}{y+1} = x$. Encontraremos o valor apropriado para
$y$ resolvendo a equação $\frac{2y}{y+1} = x$ para $y$, conforme
apresentado a seguir:
\[
\begin{array}{lclc}
\frac{2y}{y + 1} & = & x & \Rightarrow\\
x(y + 1) & = & 2y & \Rightarrow\\
xy + x - 2y & = & 0 & \Rightarrow \\
y(x - 2) & = & -x & \Rightarrow \\
y = \frac{x}{2 - x}
\end{array}
\]
Utilizando o valor de $y = \frac{x}{2 - x}$ concluímos a demonstração
sem maiores problemas. Abaixo apresentamos a versão final do texto
desta prova.
\begin{flushleft}
Suponha $x$ arbitrário.\\
\verb|  |Suponha $x\in\mathbb{R}$.\\
\verb|    | Suponha $x\neq 2$.\\
\verb|        |Seja $y = \frac{x}{2 - x}$. Temos:\\
\[
\begin{array}{lc}
\frac{2y}{y+1} & =\\
\frac{2\frac{x}{2 - x}}{\frac{x}{2 - x} + 1} & = \\
\frac{\frac{2x}{2- x}}{\frac{x + 2 - x}{2 - x}} & = \\
\frac{2x}{2 - x}\times\frac{2 - x}{2} & = \\
\frac{2x}{2} & = \\
x
\end{array}
\]
\verb|        |Logo, existe $y$ tal que $\frac{2y}{2- y} = x$.\\
\verb|        |Suponha $z$ arbitrário.\\
\verb|           |Suponha que $\frac{2z}{z + 1} = x$.\\
\verb|              |Como $\frac{2z}{z + 1} = x$, temos:\\
\[
\begin{array}{lclc}
\frac{2z}{z + 1} & = & x & \Rightarrow\\
x(z + 1) & = & 2z & \Rightarrow\\
xz + x - 2z & = & 0 & \Rightarrow \\
z(x - 2) & = & -x & \Rightarrow \\
z = \frac{x}{2 - x}
\end{array}
\]
\verb|              |Logo, $z = y$.\\
\verb|           |Logo, se $\frac{2z}{z + 1} = x$ então $z = y$.\\
\verb|        |Como $z$ é arbitrário, temos que para todo $z$, se $\frac{2z}{z + 1} = x$ então $z = y$.\\
Como $x$ é arbitrário, temos que se $x\neq 2$ então existe um único
$y$ tal que $\frac{2y}{y+1} = x$.
\end{flushleft}
\end{Example}

\subsection{Estratégia de prova por absurdo}

Agora, apresentaremos uma estratégia de prova que é aplicável a
qualquer conclusão. Esta é equivalente a regra \emph{reductio ad
  absurdum} da dedução natural.