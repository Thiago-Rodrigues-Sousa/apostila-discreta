\chapter{L\'ogica de Predicados}\label{cap3}

\epigraph{``Observe a estrada e diga-me quem você vê'', disse o Rei.\\
                 ``Eu vejo ninguém'', disse Alice.\\
                 ``Mas que excelente visão você possui!'', exclamou o
                 Rei. ``Ver Ninguém a tal distância! Eu nunca o vi!''}{Lewis Carroll, Alice no País dos Espelhos.}

\section{Motivação}

No capítulo anterior, estudamos a lógica proposicional de um ponto de
vista sintático e semântico e, além disso, utilizamos a dedução
natural e álgebra Booleana para verificar consequências e
equivalências lógicas.

Apesar de possuir uma série de aplicações, a lógica proposicional possui
limitações. A seguir apresentamos um exemplo que ilustra este problema.
\begin{Example}
Considere o seguinte argumento dedutivo:
\begin{quote}
Todo homem é mortal.\\ Sócrates é um homem. \\Logo, Sócrates é mortal.
\end{quote}
De acordo com nossa noção informal de dedução, este parece ser um
argumento válido. Sendo assim, este pode ser representado como um
sequente demonstrável utilizando dedução natural. Porém, quando tentamos
representar estas sentenças como fórmulas da lógica, podemos
perceber que nenhuma delas possui conectivos lógicos. Logo, todas
podem ser consideradas proposições simples, conforme mostramos
na tabela a seguir:

\begin{table}[h]
  \begin{tabular}{|l|c|}
    \hline
    Sentença & Fórmula \\ \hline
    Todo homem é mortal & $A$ \\
    Sócrates é um homem & $B$ \\
    Sócrates é mortal & $C$ \\\hline
  \end{tabular}
  \centering
\end{table}

Utilizando a modelagem apresentada na tabela acima, o sequente
\[
\{A,B\}\,\vdash\,C
\]
representa o argumento dedutivo em questão. Mas, como o leitor já deve
ter percebido, este não é provável utilizando o sistema de dedução
natural apresentado neste texto.
\end{Example}

Na seção \ref{soundcompleteprop} apresentamos que o sistema de dedução
natural é completo para a lógica proposicional, desta forma, toda
consequência lógica deve possuir um sequente provável
correspondente. De acordo com uma noção intuitiva de dedução lógica, o
argumento anterior é correto e portanto, deveríamos conseguir
representá-lo como um sequente demonstrável, o que, conforme
apresentado, não é possível.

O problema na modelagem formal deste argumento é que a lógica proposicional não possui expressividade para
representar sentenças que possuam alguma das seguintes formas:
\begin{itemize}
  \item Todo $x$ possui a propriedade $p$.
  \item Algum $x$ possui a propriedade $p$.
\end{itemize}
Tais sentenças possuem, implicitamente, um conjunto sobre o qual a
frase em questão deve ser interpretada como verdadeira ou falsa. No
caso do exemplo anterior, temos que a frase:
\begin{center}
Todo homem é mortal.
\end{center}
implicitamente se refere ao conjunto de todos os seres humanos. Esta
mesma frase poderia ser re-escrita de maneira a tornar o conjunto  de
seres humanos (que está ``implícito'') explícito como:
\begin{center}
Todo elemento do conjunto de seres humanos possui a propriedade ``mortal''.
\end{center}
Para representar sentenças como ``Todo homem é mortal'' precisamos de
estender a lógica proposicional de forma que sejamos capazes de
expressar propriedades sobre elementos de um certo conjunto. O
objetivo deste capítulo é estudarmos esta lógica, conhecida como
lógica de predicados ou lógica de primeira ordem.

\section{Introdução à lógica de predicados}\label{intropredlogic}

Para representar argumentos dedutivos como o apresentado na seção
anterior, devemos estender a lógica proposicional de maneira a sermos
capazes de nos referir a elementos de um certo conjunto, denominado
universo de discurso, e suas
propriedades. Para isso, a linguagem da lógica proposicional será
extendida com termos, que denotam elementos do universo de discurso;
predicados, que representam propriedades destes elementos
e quantificadores, que permite a especificação de que ``todos'' ou
``algum'' elemento do conjunto possui uma certa propriedade
especificada.

As próximas subseções apresentam uma descrição informal dos conceitos
de universo de discurso, predicados e quantificadores.

\subsection{Universo de discurso}

De maneira simples, qualquer conjunto não vazio pode ser considerado
como um universo de discurso para interpretação de uma fórmula.
O conjunto $\{\text{Sócrates}\}$ é um universo de discurso válido para o
argumento dedutivo apresentado no início deste capítulo, assim como o
conjunto $\{a\}$ ou o conjunto de todos os seres humanos,
já que todos são conjuntos não vazios de elementos.

Denominamos por \textit{constante}, um elemento qualquer do
universo de discurso. A lógica de predicados também permite a
definição de símbolos funcionais (funções), que podem ser utilizados
para representar elementos do universo de discurso sem a necessidade
de nomeá-lo. O próximo exemplo ilustra a utilização de símbolos
funcionais e constantes.

\begin{Example}
Considere como universo de discurso o conjunto $H$ de todos os seres
humanos, as constantes \textit{Hermengarda} e \textit{Eudésio} e a
função \textit{mãe}, que a partir de uma constante $h$ que representa um
ser humano retorna a constante que denota a mãe de $h$. Se
considerarmos que \textit{Hermengarda} é mãe de \textit{Eudésio}, temos que ao
aplicarmos a função \textit{mãe} a constante \textit{Eudésio} o
resultado será \textit{Hermengarda}.

Note que podemos nos referir ao mesmo elemento usando uma constante
(como, por exemplo, \textit{Hermengarda}) ou utilizando símbolos
funcionais (como, por exemplo, \textit{mãe(Eudésio)}.
\end{Example}

\subsection{Predicados}

Predicados descrevem propriedades que elementos do universo de
discurso podem ou não possuir portanto, possuem valor verdadeiro ou
falso. A seguir apresentamos alguns exemplos de predicados.

\begin{Example}
Vamos considerar o argumento dedutivo apresentado no início deste
capítulo, repetido abaixo:

\begin{table}[h]
  \begin{tabular}{|l|c|}
    \hline
    Sentença & Fórmula \\ \hline
    Todo homem é mortal & $A$ \\
    Sócrates é um homem & $B$ \\
    Sócrates é mortal & $C$ \\\hline
  \end{tabular}
  \centering
\end{table}
Note que nestas sentenças existe uma propriedade:
\textit{mortal}. Logo, ao formalizarmos este argumento,
\textit{mortal} será um predicado que descreverá a propriedade de
``ser mortal'' dos elementos do universo de discurso sobre o qual esta
fórmula está sendo interpretada.

Como outros exemplos de predicados, considere  $x > 10$ que é um exemplo de um predicado e seu
valor lógico depende do valor da variável $x$. Note que, além de
possuir a variável $x$, o predicado $x > 10$ também envolve uma
constante: $10$. Evidentemente,
predicados podem envolver diversas variáveis como $x > y$ ou
apenas constantes, como em $10 < 4$.
\end{Example}

Usualmente representamos predicados de maneira concisa como em $F(x)$,
em que $F$ é o símbolo que representa uma certa
propriedade. Considerando a propriedade \textit{mortal}, esta poderia
ser representada pelo predicado $M(x)$, que pode ser lido como ``$x$ é
mortal''. Predicados podem ter uma quantidade $n \geq 0$ de
parâmetros. Quando um predicado possui nenhum parâmetro, dizemos que
este é uma variável proposicional.

\subsection{Quantificadores}

Existem dois quantificadores na lógica de predicados: o quantificador universal,
representado pelo símbolo $\forall$, e o quantificador existencial,
representado pelo símbolo $\exists$.

Na lógica de predicados utilizamos variáveis para representar objetos
arbitrários do universo de discurso em questão. Por exemplo, se
desejamos especificar uma propriedade da álgebra, uma variável (por
exemplo, $x$) representa um número qualquer. Se a propriedade se
refere a geometria, variáveis podem representar objetos geométricos,
como pontos, triângulos, etc.

Se $P(x)$ é uma fórmula qualquer da lógica de predicados,
representamos a sentença ``todo $x$ possui a propriedade $P$'', por
$\forall x.P(x)$. De maneira similar, representamos a sentença ``algum
$x$ possui a propriedade $P$'' por $\exists x.P(x)$.

Dizemos que a fórmula $\forall x.P(x)$ é considerada verdadeira se para todo
elemento do universo de discurso em questão a propriedade $P$ é
verdadeira. Por sua vez, a fórmula $\exists x.P(x)$ é considerada
verdadeira se pelo menos um elemento do universo de discurso torna a
propriedade $P$ verdadeira. A seguir apresentamos alguns exemplos.

\begin{Example}
Considere o seguinte universo de discurso $U = \{\text{Zeus,Sócrates}\}$,
que a constante Zeus representa um deus da mitologia grega e Sócrates
o conhecido filósofo. Além disso, considere o predicado $M(x)$ que é
verdadeiro se ``$x$ é um mortal''. Desta forma, temos que a fórmula
$\forall x. M(x)$ é falsa em $U$, já que nem todo elemento deste
conjunto torna o predicado $M$ verdadeiro\footnote{O elemento Zeus torna
este predicado falso, já que, este representa o deus grego, que é
imortal.}.

Porém, se considerarmos o universo de discurso $F = \{\text{Sócrates, Platão}\}$, temos que
a fórmula $\forall x. M(x)$ é verdadeira, visto que todos os elementos
de $F$ satisfazem a propriedade $M$. Em ambos os conjuntos $U$ e $F$ a
fórmula $\exists x. M(x)$ é verdadeira, visto que há pelo menos um
elemento nestes conjuntos que representa um mortal.
\end{Example}

\begin{Example}
Neste exemplo vamos considerar a tarefa de interpretar a validade de
algumas fórmulas envolvendo o predicado $>$ sobre
números. Estas fórmulas são: $\forall x. x > 0$ e $\forall x. \exists
y. x > y$.

Inicialmente, vamos considerar como universo de discurso o
conjunto dos números naturais. A primeira fórmula é \textit{falsa}
pois, temos que o número $0 \in \mathbb{N}$ não é maior que $0$.

A segunda fórmula também é falsa pois esta especifica que para qualquer número
natural $x$, existe $y$ tal que $x > y$, o que não é verdadeiro para
$x = 0$. Porém, ao considerarmos o conjunto dos números inteiros,
$\mathbb{Z}$, temos que a primeira fórmula é falsa (porquê?) e a
segunda verdadeira, visto que no conjunto dos números inteiros, para
qualquer $x$ existe um número menor que $x$.

Considerando qualquer conjunto numérico a seguinte propriedade é
falsa: $\exists y. \forall x. y > x$, já que esta especifica que
existe algum valor $y$ que é maior que qualquer outro valor $x$.
\end{Example}

\subsection{Formalizando sentenças}

Para a formalização de sentenças utilizando a lógica de predicados
devemos especificar o universo de discurso, a interpretação de
predicados e dos símbolos funcionais que podem ser utilizados. Os
próximos exemplos ilustram a utilização destes conceitos na
formalização de sentenças na língua portuguesa.

\begin{Example}
Nos próximos
exemplos, vamos considerar sentenças envolvendo o predicado $C(x,y)$,
que denota ``$x$ conhece $y$'', o predicado $G(x,y)$ que representa
``$x$ gosta de $y$'',  a função \textit{mãe} que possui significado
óbvio. O universo de discurso considerado será, novamente, o conjunto
de todos os seres humanos.
\begin{itemize}
  \item A sentença ``Todo mundo gosta de alguém'' pode ser
    representada como: $\forall x. \exists y. G(x,y)$.
  \item A sentença ``Astobaldo não gosta de sua mãe'' pode ser
    representada como $\neg
    G(\text{Astobaldo},\textit{mãe}(\text{Astobaldo}))$.
  \item A sentença ``Ninguém gosta de todo mundo'' pode ser
    formalizada como $\neg \exists x. \forall y. G(x,y)$. Note que
    esta sentença é equivalente a ``Não existe alguém que goste de
    todo  mundo''.
  \item A sentença ``Todos gostam da mãe de Carlos'' pode ser
    representada como $\forall x. G(x, \text{\textit{mãe}(Carlos))}$.
  \item A sentença ``Todos que conhecem Clementino, não gostam da mãe
    dele'' pode ser representada como $\forall
    x. C(x,\text{Clementino}) \to \neg G(x, \textit{mãe}(\text{Clementino}))$.
\end{itemize}
\end{Example}

\section{Exercícios}

\begin{enumerate}
  \item Considere como universo de discurso o conjunto de todos os
    seres humanos, e que \textit{Holmes} e \textit{Moriarty} são
    constantes. Além disso, considere o predicado $C(x,y)$ que denota
    ``$x$ pode capturar $y$''. Com base no apresentado, represente as
    seguintes sentenças como fórmulas da lógica de predicados.
    \begin{enumerate}
      \item Holmes pode capturar qualquer um que pode capturar Moriarty.
      \item Holmes pode capturar alguém que Moriarty pode capturar.
      \item Se alguém pode capturar Moriarty, então Holmes também
        pode.
      \item Ninguém pode capturar Holmes, a menos que possa capturar
        Moriarty.
      \item Qualquer um que pode capturar Holmes pode capturar todos
        que Holmes pode capturar.
    \end{enumerate}
    \item Expresse as seguintes frases utilizando l\'ogica de predicados.
	      Para isso, crie predicados, fun\c{c}\~oes e constantes do dom\'inio
	      de interpreta\c{c}\~ao que julgar adequados.
	\begin{enumerate}
		\item Quem faz exerc\'icios tem melhor qualidade de vida.
		\item Alunos n\~ao gostam de fazer provas.
		\item Nem tudo que reluz \'e ouro.
		\item Quem conhece Godofredo o adora.
		\item N\~ao conhe\c{c}o quem n\~ao odeie as brincadeiras de Eud\'esio.
		\item Ningu\'em visita Hermengarda, a menos que ela esteja af\^onica.
	\end{enumerate}
    \item Considerando como universo de discurso o conjunto de alunos
      e professores de uma universidade e os seguintes predicados:

      \begin{table}[h]
           \begin{tabular}{|c|l|}
             \hline
             $A(x,y)$ & $x$ admira $y$\\
             $S(x,y)$ & $x$ estava presente em $y$\\
             $P(x)$    & $x$ é um professor\\
             $E(x)$    & $x$ é um estudante \\
             $L(x)$    &  $x$ é uma aula \\ \hline
           \end{tabular}
           \centering
      \end{table}

      e a constante \textit{Maria}, represente as seguintes sentenças
      como fórmulas da lógica de predicados.
      \begin{enumerate}
          \item Maria admira todo professor.
           \item Algum professor admira Maria.
           \item Maria admira a si própria.
           \item Nenhum estudante estava presente em todas as aulas.
           \item Nenhuma aula teve a presença de todos os estudantes.
           \item Nenhuma aula teve a presença de qualquer estudante.
      \end{enumerate}
\end{enumerate}

\section{Sintaxe da lógica de predicados}

A seção anterior teve como objetivo mostrar como codificar sentenças
como fórmulas da lógica de predicados e introduziu, de maneira
informal, a sintaxe e como fórmulas
são interpretadas em um determinado universo de
discurso. Nesta seção vamos definir de maneira precisa a sintaxe da
lógica de predicados, para na próxima seção definirmos a semântica de
fórmulas bem formadas nesta lógica.

Ao observarmos com atenção os exemplos de fórmulas, podemos perceber
que estas são compostas de componentes de dois tipos: valores que
representam elementos do universo de discurso e componentes
lógicos. Damos o nome de \textit{termos} aos componentes da sintaxe da
lógica de predicados que representam elementos do universo de
discurso.

\subsection{Termos}

O conjunto $\mathcal{T}$ de termos da lógica de predicados é formado
por variáveis, constantes e funções aplicadas a ambos. A seguir
apresentamos a definição formal do conjunto $\mathcal{T}$.

\begin{Definition}[Conjunto de Termos da Lógica de Predicados]\label{termdef}
O conjunto $\mathcal{T}$ de termos da lógica de predicados é definido
recursivamente como:
\begin{itemize}
  \item Seja $\mathcal{V}$ o conjunto de todas as variáveis da lógica
    de predicados. Então $\mathcal{V} \subseteq \mathcal{T}$, isto é,
    toda variável é um termo.
  \item Seja $\mathcal{C}$ o conjunto de todas as constantes da lógica
    de predicados. Então, $\mathcal{C}\subseteq\mathcal{T}$, isto é,
    toda constante é um termo.
  \item Seja $\mathcal{F}$ o conjunto de todos os símbolos funcionais
    da lógica de predicados. Considere que $f\in\mathcal{F}$ é uma
    função de aridade\footnote{Denomina-se por aridade o número de
      parâmetros de uma função.} $n$, $n\geq 1$, e que $t_1,...,t_n \in
    \mathcal{T}$. Então, $f(t_1,...,t_n)\in\mathcal{T}$, isto é, toda
    função de aridade $n$ aplicada a $n$ termos é também um termo.
\end{itemize}
Todos os elementos de $\mathcal{T}$ podem ser construídos pelas regras anteriores.
\end{Definition}
A seguir apresentamos alguns exemplos de termos e como estes são
construídos utilizando a definição \ref{termdef}.
\begin{Example}
Suponha que $a,b$ e $c$ sejam constantes de algum universo de
discurso, $f$ e $g$ duas funções de aridade 1 e 2, respectivamente. As
expressões seguintes são termos da lógica de predicados:
\begin{enumerate}
  \item $g(a,b)$
  \item $f(g(f(a),c))$
\end{enumerate}

A fórmula 1) pode ser construída da seguinte maneira: primeiramente,
$a$ e $b$, por serem constantes, são termos. Finalmente, $g(a,b)$ é um
termo pois a função $g$, de aridade $2$, está aplicada a dois termos.
Por sua vez, a fórmula 2) é bem formada, pois tanto $a$ quanto $c$ são
termos (já que ambos são constantes). Sendo assim, $f(a)$ é um termo,
já que a função $f$, de aridade 1, está aplicada a um termo. De
maneira similar, temos que $g(f(a),c)$ é um termo pois, a função $g$
(de aridade 2) está aplicada a $f(a)$ e $c$. Finalmente,
$f(g(f(a),c))$ é um termo pois, a função $f$, de aridade 1, está
aplicada a $g(f(a),c)$.

As seguintes expressões não podem ser consideradas termos já que não
respeitam a aridade das funções $f$ e $g$: $f(a,c)$, $g(f(a))$.
\end{Example}

\subsection{Fórmulas}

A partir da definição de termos, podemos definir o conjunto de
fórmulas bem formadas da lógica de predicados,
$\mathbb{F}$.

\begin{Definition}[Fórmulas bem formadas]\label{predsyntaxdef}
O conjunto de fórmulas bem formadas da lógica de predicados,
$\mathbb{F}$, pode ser definido recursivamente da seguinte maneira:
\begin{enumerate}
  \item Seja $p$ um predicado de aridade $n \geq 0$ e
    $t_1,...,t_n\in\mathcal{T}$ termos. Então,
    $p(t_1,...,t_n)\in\mathbb{F}$, isto é, $p(t_1,...,t_n)$ é uma
    fórmula (tais fórmulas são usualmente denominadas de fórmulas atômicas).
  \item $\bot,\top \in \mathbb{F}$.
  \item Sejam $\alpha, \beta \in \mathbb{F}$ fórmulas
    quaisquer. Então:
    \begin{enumerate}
      \item $\neg \alpha \in \mathbb{F}$.
      \item $\alpha \circ \beta \in \mathbb{F}$, em que
        $\circ\in\{\lor,\land,\to,\leftrightarrow\}$.
      \item Se $x \in \mathcal{V}$ (isto é, $x$ é uma variável),
        então: $\forall x. \alpha \in \mathbb{F}$ e $\exists x. \alpha
        \in \mathbb{F}$.
      \item $(\alpha)\in\mathbb{F}$.
    \end{enumerate}
    Toda fórmula bem formada da lógica de predicados pode ser
    construída utilizando as regras anteriores.
\end{enumerate}
\end{Definition}

A seguir apresentamos alguns exemplos de fórmulas bem formadas.
\begin{Example}
Primeiramente, considere os seguintes exemplos de fórmulas atômicas:
\begin{enumerate}
  \item $f$ --- um predicado de aridade 0 (isto é, uma variável
    proposicional).
  \item pai(Adão,Abel) --- um predicado de aridade 2 (pai) e duas
    constantes: Adão e Abel. Esta fórmula poderia representar a
    sentença ``Adão é pai de Abel''.
  \item casados(João,irmã(Maria)) --- um predicado de aridade 2
    (casados), aplicado a constante João e ao termo irmã(Maria), em
    que irmã é uma função. Esta fórmula poderia representar a sentença
    ``João é casado com a irmã de Maria''.
\end{enumerate}
A seguir apresentamos alguns exemplos de fórmulas não atômicas.
\begin{enumerate}
  \item pai(Adão,Abel) $\land$ pai(Adão,Caim). Esta fórmula representa
    a sentença ``Adão é pai de Abel e Caim''.
  \item $\exists x. $tia($x$,Joaquim). Esta fórmula representa a
    sentença ``Joaquim tem uma tia''.
  \item $\forall x. $gosta($x$,mãe($x$)) Esta fórmula representa a
    sentença ``Todos gostam de sua respectiva mãe''.
\end{enumerate}
\end{Example}

Assim como na lógica proposicional, utilizaremos precedências entre
conectivos e quantificadores na lógica de predicados para evitar
o uso excessivo de parênteses. Para os conectivos, utilizaremos as
mesmas regras de precedência da lógica proposicional e consideraremos
que quantificadores possuem a mesma precedência que o conectivo
$\neg$.

\subsection{Variáveis Livres e Ligadas}

Antes de apresentarmos os conceitos de variável livre e ligada,
devemos definir de maneira precisa o escopo de um quantificador em uma
fórmula.
\begin{Definition}[Escopo de quantificadores]
Seja $x\in\mathcal{V}$ uma variável e $\alpha \in\mathbb{F}$ uma
fórmula. Dizemos que o escopo da variável $x$ em $\forall x. F$ é a
fórmula $F$. De maneira similar, o escopo de $x$ em $\exists x. F$ é
também a fórmula $F$.
\end{Definition}
Dizemos que uma variável é livre em uma certa fórmula se esta não está
no escopo de nenhum quantificador. Uma variável que não é livre é dita
ser ligada. O próximo exemplo ilustra estes conceitos.
\begin{Example}\label{freeboundexample}
  Considere a seguinte equação envolvendo símbolos da aritmética sobre
  números naturais:
  \[
  x = 5y
  \]
  Esta equação não pode ser considerada verdadeira ou falsa, uma vez
  que o seu valor lógico depende dos valores atribuídos às variáveis
  $x$ e $y$. Note que, tanto $x$ quanto $y$, são variáveis que ocorrem
  livres na fórmula anterior. Considere a seguinte variação da fórmula
  anterior:
 \[
 \exists y. x = 5y
 \]
 Note que o valor lógico da fórmula acima depende apenas do valor de
 $x$ e não de $y$. Esta fórmula pode ser descrita na língua portuguesa
 sem mencionarmos a variável $y$ como ``$x$ é um múltiplo de 5''. O
 fato do valor lógico desta fórmula não depender da variável $y$ é uma
 consequência desta ser uma variável ligada.
\end{Example}
De maneira simples, podemos determinar se uma variável $x$ é livre ou
ligada em uma fórmula utilizando a função $fv$, que calcula o
conjunto de variáveis livres de uma dada fórmula. Esta
é definida a seguir.
\begin{Definition}[Variáveis livres]\label{freevars}
Seja $t\in\mathcal{T}$ um termo qualquer. O conjunto de variáveis
livres em $t$, $fv_{\mathcal{T}}(t)$, é definido recursivamente como:
\[
fv_{\mathcal{T}}(t)=\left\{
                             \begin{array}{ll}
                               \{x\} & \text{se }t = x, \text{ para
                                 algum }x \in \mathcal{V},\text{ isto é,
                               se }t\text{ é uma variável}. \\
                               \emptyset & \text{se }t = c, \text{
                                 para algum } c\in\mathcal{C}, \text{
                                 isto é, se }t\text{ é uma constante}.\\
                               \bigcup_{i = 1}^n fv_{\mathcal{T}}(t_i)
                               & \text{se }t = f(t_1,...,t_n), \text{
                                 em que }t_1,...,t_n\in\mathcal{T},
                               \text{ e }f\text{ possui aridade }n.
                             \end{array}
                          \right .
\]
Dada uma fórmula $\alpha\in\mathbb{F}$, o conjunto de variáveis livres
de $\alpha$, $fv(\alpha)$, é definido recursivamente como:
\[
fv(\alpha)=\left\{
                     \begin{array}{ll}
                       \bigcup_{i = 1}^n fv_{\mathcal{T}}(t_i) & \text{se
                       }t = p(t_1,...,t_n) \text{ em que }p \text{
                         possui aridade }n\geq 0.\\
                       \emptyset &  \text{se }\alpha = \top \text{ ou
                       }\alpha = \bot\\
                       fv(\beta) & \text{se }\alpha = \neg \beta\\
                       fv(\beta)\cup fv(\gamma) & \text{se }\alpha =
                       \beta\circ\gamma,
                       \circ\in\{\lor,\land,\to,\leftrightarrow\}.\\
                       fv(\beta) - \{x\} & \text{se }\alpha = \forall
                       x.\beta \text{ ou } \alpha = \exists x.\beta.
                     \end{array}
                  \right .
\]
Uma fórmula $\alpha\in\mathbb{F}$ em que $fv(\alpha) = \emptyset$ é
dita ser \emph{fechada}. Caso contrário, \emph{aberta}.
\end{Definition}
O leitor deve ter notado que não apresentamos uma função para o
cálculo de variáveis ligadas de uma fórmula. Esta é deixada como
exercício (veja o exercício \ref{boundex} da seção \ref{exsyntaxpred}).

\subsection{Substituição}

Conforme discutido no exemplo \ref{freeboundexample}, fórmulas que
possuem variáveis livres só podem possuir significado se estas
forem substituídas por ``valores concretos'', isto é, termos
cujo significado não depende de nenhum valor externo à definição da
fórmula em questão.

De maneira mais formal, para atribuir significado a fórmulas com
variáveis livres, estas precisam ser substituídas por termos que não
possuem este tipo de variável. A operação de substituir uma variável
por um termo qualquer é denominada \textit{substituição}.

\begin{Definition}[Substituição]
Sejam $x\in\mathcal{V}$ e $t,s\in\mathcal{T}$ uma variável e termos,
respectivamente. Denotamos por $[x\mapsto s]\,t$ o termo obtido pela
substituição de toda ocorrência livre de $x$ em $t$ por $s$. Mais
formalmente (considere que $y\in\mathcal{V}$, $c\in\mathcal{C}$ e que
$x \equiv y$ é verdadeiro se $x$ for igual a variável $y$):
\begin{equation*}
\begin{array}{lclc}
\lbrack x\mapsto s\rbrack \, y & = & \left\{
                                         \begin{array}{ll}
                                           s & \text{se } x \equiv y\\
                                           y & \text{caso contrário}.
                                         \end{array}
                                     \right . & (1)\\
\lbrack x\mapsto s\rbrack\,c & = & c & (2)\\
\lbrack x\mapsto s\rbrack\,f(t_1,...,t_n) & = & f(\lbrack x\mapsto s\rbrack\,t_1,...,
\lbrack x\mapsto s\rbrack\,t_n)  & (3) \\
\end{array}
\end{equation*}
Utilizando a definição de substituição para termos, podemos definir a
substituição para fórmulas quaisquer (em que $\alpha,\beta
\in\mathbb{F}, \circ\in\{\lor,\land,\to,\leftrightarrow\}$):
\[
\begin{array}{lclc}
\lbrack x \mapsto s\rbrack p(t_1,...,t_n) & = & p(\lbrack x \mapsto
s\rbrack\,t_1,...,\lbrack x \mapsto s\rbrack\,t_n)  & (4)\\
\lbrack x \mapsto s\rbrack \top & = & \top  & (5)\\
\lbrack x \mapsto s\rbrack \bot & = & \bot & (6) \\
\lbrack x \mapsto s\rbrack (\neg \alpha) & = & \neg (\lbrack x \mapsto
s\rbrack\,\alpha) & (7) \\
\lbrack x \mapsto s\rbrack (\alpha \circ \beta) & = & (\lbrack x
\mapsto s\rbrack\alpha) \circ (\lbrack x \mapsto s\rbrack \beta) & (8)\\
\lbrack x \mapsto s\rbrack (\forall y. \alpha) & = & \left\{
                                                                                       \begin{array}{ll}
                                                                                         \forall
                                                                                         y. \lbrack
                                                                                         x
                                                                                         \mapsto
                                                                                         s\rbrack
                                                                                         \alpha
                                                                                         &
                                                                                         \text{se
                                                                                         }x\not\equiv
                                                                                         y
                                                                                         \\
                                                                                         \forall
                                                                                         y. \alpha
                                                                                         &
                                                                                         \text{se
                                                                                         }x
                                                                                         \equiv y
                                                                                        \end{array}
                                                                                   \right. &
                                                                                   (9)\\
\lbrack x \mapsto s\rbrack (\exists y. \alpha) & = & \left\{
                                                                                       \begin{array}{ll}
                                                                                         \exists
                                                                                         y. \lbrack
                                                                                         x
                                                                                         \mapsto
                                                                                         s\rbrack
                                                                                         \alpha
                                                                                         &
                                                                                         \text{se
                                                                                         }x\not\equiv
                                                                                         y
                                                                                         \\
                                                                                         \exists
                                                                                         y. \alpha
                                                                                         &
                                                                                         \text{se
                                                                                         }x
                                                                                         \equiv y
                                                                                        \end{array}
                                                                                   \right. & (10)
\end{array}
\]
Note que as equações que definem a substituição para fórmulas com
quantificadores proíbem que a substituição seja realizada sobre
variáveis ligadas.
\end{Definition}
O seguinte ilustra a operação de substituição sobre termos e fórmulas.
\begin{Example}
Seja $x \in \mathcal{V}$ uma variável e $f(a)$ um termo ($a$ é uma
constante). Temos que o resultado de aplicar a substituição $\lbrack x \mapsto
f(a)\rbrack$ ao termo $g(a,h(x,a))$ é $g(a,h(f(a),a))$. O resultado de
aplicar esta mesma substituição à fórmula $\forall y. g(y,x)$ é
$\forall y. g(y,f(a))$. De maneira similar, a aplicação da
substituição $\lbrack x \mapsto g(a,a) \rbrack$ à fórmula $\exists
x. f(x)$ produz o termo $\exists x. f(x)$, já que este não possui
variáveis livres.
\end{Example}

\section{Exercícios} \label{exsyntaxpred}

\begin{enumerate}
  \item Apresente a definição recursiva para uma função que calcula o
    conjunto de variáveis ligadas de uma fórmula da lógica de
    predicados. \label{boundex}
  \item Para cada um dos termos da lógica de predicados a seguir, use
    a definição de fórmulas bem formadas (definição
    \ref{predsyntaxdef}) para justificar o porquê estes podem ser
    considerados fórmulas bem formadas. Considere que os símbolos $f,
    g$ são funções de aridade 1 e 2, respectivamente, que $a,b$ são
    constantes e $p, q$ são predicados de aridade 1 e 2
    respectivamente.
    \begin{enumerate}
      \item $f(a)$
      \item $\forall x. q(f(a),x)$
      \item $\exists y. p(y) \land \forall x. q(f(a),g(x,b))$.
    \end{enumerate}
    \item Obtenha o conjunto de variáveis livres para cada uma das
      fórmulas seguintes utilizando a definição
      \ref{freevars}. Considere que $a$ e $b$ são constantes.
    \begin{enumerate}
		\item $\forall x. (p(x,z)\rightarrow q(y)) \land
                  s(a,x)$
                \item $\exists y. p(y,z) \land \forall
                  x. q(f(a),g(x,b))$.
                \item $\exists y. p(y) \land \forall x. q(f(a),g(x,b))$.
    \end{enumerate}
\end{enumerate}

\section{Semântica da lógica de predicados}

Na lógica proposicional, para ser possível determinar o valor lógico
de uma fórmula basta uma interpretação para as variáveis nela
contidas. Isso é feito atribuindo, às variável da fórmula, todas as
possíveis combinações de verdadeiro ($T$) ou falso ($F$). Porém, como
apresentado informalmente na seção \ref{intropredlogic}, para
interpretarmos fórmulas da lógica de predicados, devemos possuir um
universo de discurso (que dá significado às constantes) e conjuntos de
relações e funções que atribuem significado aos símbolos predicativos
e funcionais, respectivamente.Isto é, para definirmos o significado de
fórmulas da lógica de predicados, necessitamos de uma
\textit{estrutura}, conceito este apresentado na definição seguinte.

\begin{Definition}[Estrutura]
Uma estrutura é uma tripla $I = (U,R,F)$, em que:
\begin{itemize}
  \item $U$ é um conjunto não vazio tal que para cada constante $a \in
    \mathcal{T}$, temos que $a^I\in U$, em que $a^I$ é a denotação de
    $a$ em $U$.
  \item $R$ é um conjunto de relações, em que, para cada símbolo
    predicativo $p$ de aridade $n \geq 1$, existe uma relação $n$-ária
    $p^I \subseteq U^n$.
  \item $F$ é um conjunto de funções, em que, para cada símbolo
    funcional $f$, de aridade $n\geq 1$, existe uma função $f^I :
    U^n\to U \in F$.
\end{itemize}
\end{Definition}

A definição de uma função para atribuição significado a fórmulas da
lógica de predicados é apresentada a seguir. Primeiramente,
apresentamos a definição da semântica de um termo. Como termos denotam
elementos do universo de discurso, a função semântica para termos
deverá produzir como resultado um elemento do conjunto $U$,
considerando uma estrutura $I=(U,R,F)$.

\begin{Definition}[Semântica de termos]
Seja $I=(U,R,F)$ uma estrutura. Definimos a função $\varepsilon : \mathcal{T}
\to U$, que define a semântica de um termo $t$ como um elemento $u\in
U$, recursivamente como:
\[
\begin{array}{lcllc}
\varepsilon(a) & = & a^I, & \text{em que }a^I\in U & (1)\\
\varepsilon(f(t_1,...,t_n)) & = &
f^I(\varepsilon(t_1),...,\varepsilon(t_n)) & \text{em que }f^I\in
F\text{ e } t_1,...,t_n \in \mathcal{T} & (2)
\end{array}
\]
\end{Definition}
Note que apesar de variáveis serem consideradas termos, não
apresentamos a semântica destas, uma vez que termos possuem somente
variáveis livres e, como já citado anteriormente, apresentaremos a
semântica apenas de fórmulas fechadas. Isto não constitui uma
limitação, uma vez que, substituições podem ser utilizadas para
eliminar variáveis livres de fórmulas.

Antes de apresentarmos a função semântica de fórmulas da lógica de
predicados, vamos considerar um exemplo para ilustrar a semântica de
termos desta lógica.

\begin{Example}
Suponha um universo de discurso em que objetos sejam países e cidades,
dentre as quais citamos \textit{Rio de Janeiro, Berlim, Nova York,
  Tóquio,} entre outras. Desejamos formalizar as seguintes funções e
constantes envolvendo países e cidades:
\begin{itemize}
   \item \textit{capital}: função de aridade 1 que associa a cada país
     sua respectiva capital. Representaremos, na lógica de predicados,
     a  função \textit{capital} pelo símbolo funcional $cap$. Logo,
     $cap^I = \textit{capital}$.
   \item Representarmos as constantes \textit{Rio de Janeiro, Berlim, Nova York,
  Tóquio} por $RJ, BL, NY$ e $TK$, respectivamente. Além disso,
   consideraremos que a constante Brasil é representada pelo termo
   $BR$ e Alemanha por $GE$.
\end{itemize}
Considere, agora, a tarefa de interpretrar o significado do termo
$cap(GE)$. Utilizando a definição da semântica de termos, temos:
\[
\begin{array}{lcl}
\varepsilon(cap(GE)) & = &\\
cap^I(\varepsilon(GE)) & = & \{\text{pela eq. (2) de }\varepsilon.\}\\
cap^I(GE^I) & = & \{\text{pela eq. (1) de }\varepsilon.\}\\
cap^I(\textit{Alemanha}) & = & \{\text{pela semântica da constante }GE\}\\
\textit{capital}(\textit{Alemanha}) & = & \{\text{pela semântica do
  símbolo funcional }cap.\}\\
\textit{Berlim} & & \{\text{pela semântica da função da função \textit{capital}}\}.
\end{array}
\]
Logo, o termo $cap(GE)$ denota o mesmo elemento (a cidade de Berlim)
que a constante $BL$.
\end{Example}
A seguir, é apresentada a semântica para fórmulas da lógica de
predicados.
\begin{Definition}[Semântica de Fórmulas]\label{predlogicformulasem}
Seja $I=(U,R,F)$ uma estrutura. Definimos a função $\llbracket\_\rrbracket :
\mathbb{F}\to \{T,F\}$, que associa a cada fórmula fechada da lógica de
predicados o seu respetivo valor lógico, recursivamente como (em que
$t_1,...,t_n$ representam termos, $u$ uma constante qualquer, $\alpha$,
$\beta$ fórmulas quaisquer e $\circ\in\{\lor,\land,\to,\leftrightarrow\}$):
\[
\begin{array}{lclc}
\llbracket \bot \rrbracket & = & F & (1)\\
\llbracket \top \rrbracket & = & T & (2)\\
\llbracket p(t_1,...,t_n) \rrbracket & = & (\varepsilon(t_1),...,\varepsilon(t_n))
\in p^I & (3)\\
\llbracket \neg \alpha \rrbracket & = & \neg \llbracket \alpha
\rrbracket & (4) \\
\llbracket \alpha \circ \beta \rrbracket & = & \llbracket \alpha
\rrbracket \circ \llbracket \beta \rrbracket & (5)\\
\llbracket \forall x. \alpha \rrbracket & = & \bigwedge\limits_{u\in
  U}\llbracket [x \mapsto u] \alpha \rrbracket & (6)\\
\llbracket \exists x. \alpha \rrbracket & = & \bigvee\limits_{u\in
  U}\llbracket [x \mapsto u] \alpha \rrbracket & (7)\\
\end{array}
\]
A notação $[x\mapsto u]\alpha$ denota a fórmula $\alpha$ em que toda
ocorrência da variável livre $x$ é substituída pela constante $u$.
\end{Definition}
O próximo exemplo ilustra a utilização das funções semânticas para
lógica de predicados.
\begin{Example}
Considere a seguinte estrutura $I=(U,R,F)$, em que:
\begin{itemize}
  \item $U = \{2,3,4\}$, em que cada um dos números $x \in U$, será
    representado pela constante $x$.
  \item O conjunto $R$ é formado pelos seguintes conjuntos:
  \begin{itemize}
    \item $par = \{2,4\}$, que será representado pelo símbolo
      predicativo $p$, de aridade 1.
    \item $\textit{ímpar} = \{3\}$, que será representado pelo símbolo
      predicativo $i$, de aridade 1.
    \item $M = \{(2,3),(2,4),(3,4)\}$, que será representado pelo
      símbolo predicativo $m$, de aridade 2.
  \end{itemize}
  \item O conjunto $F$ é vazio, isto é, não existem funções nesta estrutura.
\end{itemize}
Tendo apresentado o significado dos símbolos não funcionais em termos
da estrutura $I$, considere a tarefa de calcular o valor lógico da
seguinte fórmula $\forall x. p(x) \lor i(x)$. O cálculo de cada uma destas, utilizando a definição
\ref{predlogicformulasem}, é mostrada passo-a-passo a seguir.
\[
\begin{array}{lcl}
\forall x . p(x) \lor i(x) & = &\\
(\llbracket p(2)\lor i(2) \rrbracket) \land (\llbracket p(3)\lor i(3)
\rrbracket) \land (\llbracket p(4)\lor i(4) \rrbracket) & = &
\{\text{pela eq.} (6)\}\\
(\llbracket p(2) \rrbracket \lor \llbracket i(2) \rrbracket) \land
(\llbracket p(3) \rrbracket \lor \llbracket i(3)
\rrbracket) \land (\llbracket p(4) \rrbracket \lor \llbracket i(4) \rrbracket) & = &
\{\text{pela eq.} (5)\}\\
\begin{array}{lc}
(\varepsilon(2) \in \textit{par} \lor \varepsilon(2) \in
\textit{ímpar}) & \land \\
(\varepsilon(3) \in \textit{par} \lor \varepsilon(3) \in
\textit{ímpar}) & \land \\
(\varepsilon(4) \in \textit{par} \lor \varepsilon(4) \in
\textit{ímpar}) \end{array} & = & \{\text{pela eq. }(3)\} \\
T \land T \land T & = & \{\text{pela def. de }\varepsilon \text{ e }
\textit{par, ímpar}.\} \\
T
\end{array}
\]
Logo, de acordo com a definição \ref{predlogicformulasem}, a fórmula
$\forall x. p(x) \lor i(x)$ é verdadeira para a estrutura $I$.
\end{Example}
Com base na semântica de fórmulas, podemos classificá-las de maneira
similar ao que fazemos com a lógica proposicional. A próxima definição
formaliza estes conceitos.
\begin{Definition}[Classificação de fórmulas]
Seja $\alpha$ uma fórmula bem formada da lógica de predicados. Dizemos
que $\alpha$ é satisfazível se existe uma estrutura $I$ tal que
$\llbracket \alpha \rrbracket = T$. De maneira similar, dizemos que
$\alpha$ é falseável se existe uma estrutura tal que $\llbracket \alpha
\rrbracket = F$. Uma fórmula $\alpha$ é dita ser uma tautologia\footnote{Também
chamada por alguns autores de fórmulas válidas} se esta é verdadeira
para toda estrutura $I$. Uma fórmula $\alpha$ é uma contradição se não
existe uma estrutura que a satisfaça. Finalmente, $\alpha$ é uma
contingência se esta for satisfazível e falseável.
\end{Definition}

\section{Exercícios}

\begin{enumerate}

   \item Para cada uma das f\'ormulas a seguir, indique se ela \'e verdadeira ou falsa,
         quando o universo de discurso \'e cada um dos seguintes conjuntos: $\mathbb{N}$: conjunto
         dos n\'umeros naturais, $\mathbb{Z}$: conjunto dos n\'umeros inteiros e $\mathbb{R}$ conjunto
         dos n\'umeros reais. Considere que os símbolos matemáticos
         possuem o significado usual.
         \begin{center}
         \begin{tabular}{|l|c|c|c|}
             \hline
             F\'ormula & $\mathbb{N}$ & $\mathbb{Z}$ & $\mathbb{R}$ \\ \hline
             $\exists x. x^2 = 2$ & & &\\ \hline
             $\forall x. \exists y. x^2 = y$ & & & \\ \hline
             $\forall x. x\neq 0 \rightarrow \exists y. xy = 1$ & & & \\ \hline
             $\exists x. \exists y. (x + 2y^2 = 2) \land (2x + 4y = 5)$ & & & \\
             \hline
         \end{tabular}
         \end{center}
\end{enumerate}

\section{Dedução natural para lógica de predicados}

As regras para dedução natural para lógica proposicional podem ser
estendidas para lidar com os quantificadores da lógica de predicados.
Apenas quatro regras adicionais são necessárias para lidar com a
lógica de predicados, a saber: regras para introdução e eliminação dos
quantificadores universal e existencial.

Uma maneira de compreender as regras para ambos os quantificadores é
vê-los como generalizações dos conectivos de conjunção (para o
quantificador universal) e disjunção (para o quantificador
existencial). As próximas subseções utilizarão essa analogia para
apresentar as regras de introdução e eliminação destes
quantificadores.

\subsection{Regras para o quantificador universal}

Conforme apresentado na definição \ref{predlogicformulasem}, o
quantificador universal pode ser entendido como uma conjunção de
fórmulas, em que a variável ligada a este quantificador é substituída
por cada um dos elementos do universo de discurso em questão. Mais
formalmente:

\[\forall x. P(x) \equiv \bigwedge\limits_{u\in U}[x\mapsto u]P(x)\]

Nas subseções seguintes, utilizaremos esta analogia para
apresentarmos, informalmente, as regras de introdução e eliminação
deste quantificador.


\subsubsection{Introdução do quantificador universal $\forallI$}

Se pensarmos que o quantificador universal é uma generalização da
conjunção, podemos conjecturar que a regra de introdução deste
quantificador deve ser similar a

\[
\infer[\forall_{I1}] {\forall x.P(x)}
                           {\bigwedge\limits_{u\in U}[x\mapsto u] P(x)}
\]
Isto é, para concluirmos que $\forall x.P(x)$ é provável basta
demonstrar $[x\mapsto u]P(x)$, em que $u$ representa cada um dos
elementos pertencentes ao universo de discurso sobre o qual esta
fórmula deve ser interpretada. Esta analogia é válida (e útil) se o
universo de discurso é finito e possui poucos elementos. Caso
contrário, essa abordagem para provar $\forall x. P(x)$ é
impraticável.

Propriedades que envolvem ``todos'' os possíveis valores de um
universo de discurso são comuns na matemática e, portanto, deve haver
uma maneira mais simples de demonstrar afirmativas da forma $\forall
x.P(x)$. A idéia utilizada para provar fórmulas que utilizam o
quantificador universal pode ser expressa de maneira intuitiva da
seguinte forma: Se uma certa propriedade $P$ é verdadeira para um
objeto \emph{arbitrário} do universo de discurso, então esta deve ser
verdadeira para todo elemento deste conjunto. Porém, esta explicação
deixa a seguinte pergunta: Quando podemos considerar que um certo
objeto é ou não ``arbitrário''? Há uma resposta simples para isso,
baseada em um critério sintático sobre sequentes. Dizemos que um valor
$x$ é arbitrário se este não pertence ao conjunto de variáveis livres
do sequente a ser demonstrado. Logo, podemos concluir que
$\Gamma \vdash \forall x. P(x)$ se conseguirmos demonstrar $P(x)$, em
que $x$ é um valor arbitrário, isto é $x\not\in\,fv(\Gamma)$ e $fv(\Gamma)$ é definido como:
\[fv(\Gamma) = \bigcup_{\alpha\in\Gamma}fv(\alpha)\]

Abaixo apresentamos a regra de introdução do quantificador universal.

\[
\infer[\forallI]{\forall x. P(x)}
                      {P(x) & x \not\in fv(\Gamma)}
\]
Assim como fizemos para lógica proposicional, vamos omitir
completamente o conjunto de hipóteses $\Gamma$ e também a demonstração
de que $x\not\in fv(\Gamma)$, pois, normalmente a última demonstração
é imediata a partir das hipóteses de um dado sequente. Visando
ilustrar essa convenção, o seguinte exemplo ilustra a utilização da
regra $\forallI$.
\begin{Example}
Considere a tarefa de demonstrar o sequente:
$\vdash \forall x. E(x) \to E(x) \lor \neg E(x)$. Como este utiliza o
quantificador universal, iniciaremos a demonstração utilizando a regra
$\forallI$. Ao aplicarmos esta regra, devemos demonstrar que $E(x) \to
E(x) \lor \neg E(x)$, para um valor $x$ arbitrário. Uma vez que o
conjunto de hipóteses deste sequente é vazio, a variável livre $x$ em
$E(x) \to E(x) \lor \neg E(x)$ pode ser considerada arbitrária, já que
esta não ocorre livre nas hipóteses.

A prova deste sequente é apresentada abaixo:

\[
\infer[\forallI]{\forall
x. E(x) \to E(x) \lor \neg E(x)}
        {
          \infer[\impI^1]{E(x) \to E(x) \lor \neg E(x)}
                             {
                               \infer[\orIE]{E(x) \lor \neg E(x)}
                                         {
                                           \infer[\Id]{E(x)^1}{}}
                             }}
\]
\end{Example}

\subsubsection{Eliminação do quantificador universal $\forallE$}

A regra de eliminação do quantificador universal permite-nos concluir,
a partir de $\forall x. P(x)$, que $[x\mapsto a]P(x)$, em que $a$ é uma
constante ou uma variável livre na conclusão da regra $\forallE$.

\[
\infer[\forallE]{[x\mapsto a]P(x)}
                      {\forall x. P(x)}
\]
A regra $\forallE$ é uma generalização das regras para eliminação da
conjunção, uma vez que, ao utilizarmos esta regra estamos concluindo
um dos possivelmente infinitos componentes da conjunção
$\bigwedge_{u\in U}[x\mapsto u]\,P(x)$.
\begin{Example}
Consider a tarefa de demonstrar o seguinte sequente $\{F(a),\forall
x. F(x) \to G(x)\} \vdash G(a)$. Para demonstrar esse sequente,
utilizaremos a eliminação da implicação, para concluir $G(a)$ a partir
de $F(a)$ e $F(a) \to G(a)$. Esta última pode ser deduzida utilizando
a regra $\forallE$ sobre a hipótese $\forall x. F(x) \to G(x)$,
conforme apresentado abaixo:
\[
\infer[\impE]{G(a)}
        {\infer[\Id]{F(a)}{} &
         \infer[\forallE]{F(a) \to G(a)} {
           \infer[\Id]{\forall x. F(x)\to G(x)}{}
                  }}
\]
\end{Example}

\subsubsection{Restrições sobre as regras do quantificador universal}

O objetivo desta seção é apresentar exemplos que mostrem a necessidade
das restrições sobre a aplicabilidade das regras de introdução e
eliminação do quantificador universal.

Primeiramente, vamos considerar a restrição $x \not\in fv(\Gamma)$
sobre a regra $\forallI$. Esta é realmente necessária? Ao
invés de tentarmos apresentar um argumento formal (o que foge ao
escopo deste texto), apresentaremos um exemplo que, ao não utilizarmos
essa restrição, produziremos um argumento incorreto.
\begin{Example}
Considere a seguinte fórmula da lógica de predicados:
$0= 0 \to \forall x. (x = 0)$
que evidentemente não é uma tautologia\footnote{A menos que o universo de
discurso em questão possua apenas a constante $0$.} e a seguinte
``demonstração'' (incorreta):
\[
\infer[\forallE]{0 = 0 \to \forall x . x = 0}
         {
           \infer[\forallI]{\forall x. x = 0 \to \forall x. x = 0}
                     {
                       \infer[\impI^1]{x = 0 \to \forall x. x = 0}
                                {
                                  \infer[\forallI]{\forall x. x = 0}
                                           {
                                             \infer[\Id]{x = 0^1}{}
                                           }
                                }
                     }
         }
\]
Note que a aplicação da regra $\forallI$ sobre a hipótese $x= 0$ é
ilegal, uma vez que a variável $x$ ocorre livre nas hipóteses.
\end{Example}

A restrição imposta sobre a regra $\forallE$ é que o valor que
substitui a variável ligada ao quantificador eliminado deve ocorrer
livre na conclusão. O próximo exemplo ilustra que, ao ignorar essa
restrição, podemos deduzir fórmulas que não são consequências lógicas
das hipóteses do sequente em questão.
\begin{Example}
Considere a tarefa de demonstrar o seguinte sequente: \[\vdash \forall
x.\neg\forall y. x = y \to \neg \forall y. y = y\]. A ``demonstração'' (incorreta)
deste é apresentada abaixo:
\[
\infer[\impI^1]{(\forall x.\neg\forall y. x = y) \to \neg \forall y . y
= y}
        {
          \infer[\forallE]{\neg \forall y. y = y}
                   {
                     \infer[\Id]{\forall x.\neg\forall y. x = y^1}{}
                   }
        }
\]
Note que a aplicação da regra $\forallE$ logo no início da dedução
está incorreto, já que a variável eliminada ($x$) foi substituída por
$y$, que ocorre ligada na conclusão, alterando a semântica da
fórmula em questão.
\end{Example}

\subsection{Regras para o quantificador existencial}

Conforme apresentado na definição \ref{predlogicformulasem}, o
quantificador existencial pode ser entendido como uma disjunção de
fórmulas, em que a variável ligada a este quantificador é substituída
por cada um dos elementos do universo de discurso em questão. Mais
formalmente:

\[\exists x. P(x) \equiv \bigvee\limits_{u\in U}[x\mapsto u]P(x)\]

Nas subseções seguintes, utilizaremos esta analogia para
apresentarmos, informalmente, as regras de introdução e eliminação
deste quantificador.

\subsubsection{Introdução do quantificador existencial $\existsI$}

De maneira intuitiva, podemos concluir que $\exists x. P(x)$ se for
possível provar que a propriedade $P$ é verdadeira para algum valor
$a$. Mais formalmente:
\[
\infer[\existsI]{\exists x. P(x)}
         {
           [x\mapsto a]P(x)
         }
\]
Note que, para demonstrar que $\exists x. P(x)$ basta mostrar que
\emph{existe} um valor $a$ que torna a propriedade $P$
verdadeira. Desta forma, podemos entender a regra $\existsI$ como uma
generalização das regras de introdução da disjunção, já que para
provar $\bigvee_{u\in U}  [x\mapsto u]P(x)$

Caso o valor $a$ seja uma variável, esta deve ocorrer livre em
$[x\mapsto a]P(x)$. A seguir apresentamos um exemplo que ilustra a
utilização desta regra.
\begin{Example}
Considere a tarefa de demonstrar o sequente $\{\forall x. P(x)\}
\vdash \exists x. P(x)$. Iniciamos a demonstração utilizando a regra
$\existsI$, logo, devemos mostrar que $[x\mapsto b]P(x)$, para algum
valor $b$. A partir da hipótese $\forall x .P(x)$, podemos concluir
$[x\mapsto b]P(x)$ utilizando $\forallE$. Esta dedução é apresentada a
seguir.
\[
\infer[\existsI]{\exists x. P(x)}
        {
          \infer[\forallE]{P(b)}
                   {
                     \infer[\Id]{\forall x. P(x)}{}
                   }
        }
\]
\end{Example}

\subsubsection{Eliminação do quantificador existencial $\existsE$}

A regra para eliminação do quantificador existencial $\existsE$
generaliza para um universo possivelmente infinito a regra de
eliminação da disjunção. Intuitivamente,  a regra $\orE$ especifica
que se $A\lor B$ é provável e que $C$ pode ser deduzido a partir de
$A$ e que $C$ pode ser deduzido a partir de $B$, então podemos
concluir $C$ a partir destes fatos. Mais formalmente:

\[
\infer[\orE]{\Gamma \vdash C}
         {\Gamma \vdash A \lor B & \Gamma \cup \{A\}\vdash C & \Gamma
           \cup \{B\} \vdash C}
\]

Evidentemente, podemos estender essa regra para disjunções envolvendo
$3$ termos de maneira quase que imediata:
\[
\infer[\orE_3]{\Gamma \vdash D}
         {\Gamma \vdash A \lor B \lor C & \Gamma\cup\{A\}\vdash D &
           \Gamma \cup \{B\}\vdash D & \Gamma\cup\{C\}\vdash D}
\]
Note que ao generalizarmos a regra de eliminação para 3 elementos,
adicionamos uma nova premissa: $\Gamma \cup \{C\}\vdash D$ para que
seja possível deduzir $D$. Desta forma, para concluir uma fórmula
$\alpha$ a partir de uma disjunção de $n$ fórmulas, devemos provar
$\alpha$ a partir da suposição de cada uma da subfórmulas que formam a
disjunção em questão. Como $\exists x. P(x)$ pode ser considerada uma
disjunção envolvendo um número possivelmente infinito de componentes,
isso nos leva a seguinte questão: como provar uma conclusão $\alpha$ a
partir de um número possivelmente infinito de componentes que devem
ser supostos para concluir esta fórmula $\alpha$?

A solução para este problema é adotar uma estratégia similar ao que
foi feito para a regra $\forallI$: utilizar um valor arbitrário. Para
deduzir uma fórmula $\alpha$ a partir de $\exists x. P(x)$, utilizando
a regra $\existsE$, devemos supor $[x\mapsto y]P(x)$, em que $y$ é um
valor arbitrário ($y\not\in fv(\Gamma)$). Esta regra é apresentada a
seguir.
\[
\infer[\existsE]{\Gamma\vdash\alpha}
        {\exists x. P(x) & \Gamma \cup \{[x\mapsto y] P (x)\}\vdash
          \alpha & y \not\in fv(\Gamma)}
\]
A seguir apresentamos um exemplo desta regra.
\begin{Example}
Considere a tarefa de demonstrar o sequente
\[
\{\exists x. P(x), \forall x. P(x) \to Q(x) \} \vdash \exists y. Q(y)
\]
A dedução é iniciada utilizando a regra $\existsE$. Ao utilizar esta
regra, podemos introduzir a hipótese $P(k)$\footnote{Note que $P(k)$ é
equivalente a $[x\mapsto k]P(x)$.} que possibilita utilizar a
introdução da implicação para finalizar a demonstração. Esta é
apresentada abaixo:
\[
                     \infer[(\exists\,E)^{1}]
                           {\exists y. Q(y)}
                           {\infer[\Id]{\exists x. P(x)}{} &
                             \infer[(\exists\,I)]
                                   {\exists y. Q(y)}
                                   {
                                     \infer[(\rightarrow\,E)]
                                           {Q(k)}
                                           {
                                             \infer[(\forall\,E)]
                                                   {P(k)\rightarrow Q(k)}
                                                   {\forall x. P(x)\rightarrow Q(x)}
                                             &
                                                   \infer[\Id]{P(k)^1}{}
                                           }
                                   }
                           }
\]
\end{Example}
Note que a única restrição aplicável à regra $\existsE$ é a mesma que
se aplica a regra $\forallI$: a variável ``arbitrária'' não deve
ocorrer livre no conjunto de hipóteses.

\section{Exercícios}

\begin{enumerate}
	\item Prove os seguintes sequentes usando dedu\c{c}\~ao natural:
	\begin{enumerate}
		\item $\{\forall x. (P(x)\rightarrow Q(x))\}\vdash(\forall x.\neg Q(x))\rightarrow(\forall x.\neg P(x))$
		\item $\{\forall x. (P(x)\rightarrow \neg Q(x))\}\vdash\neg (\exists x. (P(x)\land Q(x)))$
		\item $\{\forall x.(A(x)\rightarrow (B(x)\lor C(x))),\forall x.\neg B(x)\}\vdash(\forall x. A(x))\rightarrow
		       (\forall x. C(x))$
		\item $\{\exists x. (P(x)\land Q(x)), \forall x. (P(x)\rightarrow R(x))\}\vdash\exists x.(R(x)\land Q(x))$
		\item $\{\forall x. P(a,x,x), \forall x.\forall
                  y.\forall z. P(x,y,z)\rightarrow
                  P(f(x),y,f(z))\}\vdash P(f(a),a,f(a))$
           \item $\{\forall x. P(x) \rightarrow Q(x)\} \vdash \forall x. P(x) \rightarrow \forall x. Q(x)$
           \item $\{\exists x. \neg P(x) \}\vdash \neg \forall x. P(x)$
	\end{enumerate}
\end{enumerate}

\section{Equivalências algébricas para lógica de predicados}

Assim como na dedução natural, todas as leis algébricas já vistas para
lógica proposicional continuam válidas para lógica de predicados. O
que faremos é apenas incluir novas leis para a manipulação adequada
dos quantificadores universal e existencial.

As leis algébricas para manipulação dos quantificadores são
apresentadas abaixo:

\[
\begin{array}{|cccl|}
\hline
\neg \forall x. P(x) & \equiv & \exists x. \neg P(x) &
\{\neg-\forall\}\\
\neg \exists x. P(x) & \equiv & \forall x. \neg P(x) &
\{\neg-\exists\}\\
\forall x. P(x) \land Q(x) & \equiv & \forall x. P(x) \land \forall
x. Q(x) & \{\land-\forall\} \\
\exists x. P(x) \lor Q(x) & \equiv & \exists x. P(x) \lor \exists
x. Q(x) & \{\lor-\exists\} \\ \hline
\end{array}
\]
As primeiras duas leis expressam a relação dos quantificadores com a
negação lógica. O leitor atento deve ter percebido que estas regras
são uma generalização das leis de DeMorgan para lógica proposicional.
O segundo grupo de regras expressa como os quantificadores universal e
existencial distribuem sobre a conjunção e disjunção, respectivamente.

\begin{Example}
As fórmulas $\forall x. F(x) \land \neg G(x)$ e $\forall x. F(x) \land
\neg \exists x. G(x)$ são equivalentes, conforme a dedução a seguir:
\[
      \begin{array}{lcl}
          \forall x.(F(x)\land\neg G(x)) & = & \\
	 \forall x.F(x)\land\forall x.\neg G(x) & = &\{\land-\forall\}\\
         \forall x.F(x)\land\neg\exists x.G(x) &  & \{\neg-\forall\}
      \end{array}
\]
\end{Example}

\section{Exercícios}

\begin{enumerate}
         \item Prove as seguintes equival\^encias utilizando regras alg\'ebricas para l\'ogica de predicados.
         \begin{enumerate}
            \item $\forall x. P(x) \rightarrow \neg Q(x) \equiv \neg \exists x. P(x) \land Q(x)$
            \item $\neg \forall x.\exists y. R(x,y)\land \neg P(x,y)\equiv\exists x.\forall y.R(x,y)\rightarrow P(x,y)$
         \end{enumerate}
\end{enumerate}

\section{Considerações meta-matemáticas}

Nesta seção consideraremos, sem demonstração, algumas propriedades
meta-matemáticas da lógica de predicados, a saber: corretude,
completude e decidibilidade. Assim como na lógica proposicional, a
dedução natural para lógica de predicados é um sistema formal correto
e completo. Porém, a teoria associada a noção de satisfazibilidade da
lógica de predicados não é decidível.

\subsection{Correção e Completude}

Nesta seção enunciaremos teoremas que afirmam que o sistema de dedução
natural para lógica de predicados é correto e completo com respeito a
noção de consequência lógica.

\begin{Theorem}[Correção da dedução natural]
Seja $\alpha$ uma fórmula bem formada qualquer da lógica
de predicados. Se $\vdash\,\alpha$, então $\models\,\alpha$.
\end{Theorem}

\begin{Theorem}[Completude da dedução natural]
Seja $\alpha$ uma fórmula bem formada qualquer da lógica
de predicados. Se $\models\,\alpha$, então $\vdash\,\alpha$.
\end{Theorem}

A prova da correção da dedução natural para lógica de predicados
possui uma estrutura similar à demonstração para lógica
proposicional. Basta utilizar indução sobre a estrutura das derivações
de provas. Porém, a demonstração da propriedade de completude exige
técnicas que vão além do objetivo deste texto.

\subsection{Decidibilidade}

Conforme apresentamos no capítulo \ref{cap2}, a teoria associada a
linguagem da lógica proposicional é decidível, isto é, existe um
algoritmo que responde ``sim'' sempre que a fórmula em questão for
válida (tautologia) e ``não'', caso contrário.

Na seção anterior, apresentamos que a lógica de predicados possui as
propriedades de correção e completude, como a lógica
proposicional. Desta forma, podemos perguntar se a lógica de predicados  também
possui uma teoria decidível associada. Normalmente, para a lógica de
predicados, considera-se a teoria que envolve o conjunto de fórmulas
bem formadas desta lógica e a noção de satisfazibilidade como conceito
de validade. Infelizmente, o problema de determinar se uma fórmula
\emph{arbitrária} da lógica de predicados é satisfazível é
indecidível, isto é, não existe um algoritmo capaz de apresentar uma
resposta correta para toda fórmula bem formada desta lógica. A
demonstração deste resultado pode ser encontrada em livros que abordam
teoria de computabilidade e está fora do escopo deste texto.

\section{Notas Bibliográficas}