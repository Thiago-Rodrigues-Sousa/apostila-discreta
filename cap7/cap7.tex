\chapter{Relações}\label{cap7}

\epigraph{O cliente pode ter um carro pintado com a cor que desejar,
  contanto que seja preto.}{Henry Ford}


\section{Motivação}

Existem diversos tipos de relações em nosso cotidiano. Algumas destas
descrevem como membros de uma família estão relacionados entre si:
pais, filhos, irmãos, irmãs, sobrinhos, etc. Uma outra possível
relação especifica que certas cidades pertecem a um determinado país:
por exemplo, Londres está na Inglaterra, e Paris na França. Ou podemos
ter uma relação que descreve quais automóveis são montados por um
certo fabricante. Relações são utilizadas na matemática para descrever
como dois números se relacionam: por exemplo, dados dois números $x$ e
$y$ temos que $x \geq y$, $x < y$, em que $\geq$ e $<$ são relações
entre números.

Relações estão presentes em diversos ramos da computação, pois a
terminologia da teoria de relações permite descrever conceitos de
maneira precisa. Talvez, a aplicação mais famosa de relações em
ciência da computação são os bancos de dados relacionais. Porém,
relações formam a base teórica de muitas outras áreas como a semântica
de linguagens de programação, demonstração de terminação de
algoritmos, representação de informação armazenada em máquinas de
busca, teoria de grafos, etc. Uma vez que relações são ubíquas e
importantes, é útil definí-las como objetos matemáticos e descrever
suas propriedades. O objetivo deste capítulo é apresentar a teoria de
relações e a demonstração de alguns resultados importantes desta.

\begin{Remark}
Neste capítulo assumimos que o leitor já possui a maturidade para
compreender e demonstrar teoremas. Portanto, na maioria das
demonstrações o rascunho será completamente omitido. Porém,
recomenda-se que este seja ``reconstruído'' pelo leitor para um maior
entendimento do conteúdo.
\end{Remark}