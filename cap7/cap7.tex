\chapter{Relações}\label{cap7}

\epigraph{O cliente pode ter um carro pintado com a cor que desejar,
  contanto que esta seja preto.}{Henry Ford, Pioneiro da indústria automobilística.}


\section{Motivação}

Existem diversos tipos de relações em nosso cotidiano. Algumas destas
descrevem como membros de uma família estão relacionados entre si:
pais, filhos, irmãos, irmãs, sobrinhos, etc. Outras especificam, por
exemplo, que certas cidades pertecem a um determinado país:
por exemplo, Londres está na Inglaterra, e Paris na França. Ou podemos
ter uma relação que descreve quais automóveis são montados por um
certo fabricante. Relações são utilizadas na matemática para descrever
como dois números se relacionam: por exemplo, dados dois números $x$ e
$y$ temos que $x \geq y$, $x < y$, em que $\geq$ e $<$ são relações
entre números.

Relações estão presentes em diversos ramos da computação, pois a
terminologia da teoria de relações permite descrever conceitos de
maneira precisa. Talvez, a aplicação mais famosa de relações em
ciência da computação são os bancos de dados relacionais. Porém,
relações formam a base teórica de muitas outras áreas como a semântica
de linguagens de programação, demonstração de terminação de
algoritmos, representação de informação armazenada em máquinas de
busca, teoria de grafos, etc. Uma vez que relações são ubíquas e
importantes, é útil definí-las como objetos matemáticos e descrever
suas propriedades. O objetivo deste capítulo é apresentar a teoria de
relações e a demonstração de alguns resultados importantes desta.

\begin{Remark}
Neste capítulo assumimos que o leitor já possui a maturidade para
compreender e demonstrar teoremas. Portanto, na maioria das
demonstrações o rascunho será completamente omitido. Porém,
recomenda-se que este seja ``reconstruído'' pelo leitor para um maior
entendimento do conteúdo.
\end{Remark}

\section{Pares Ordenados e Produto Cartesiano}

Em capítulos anteriores, lidamos com conjuntos em que cada elemento é
um ``componente'' deste. Porém, como você aprendeu em outros cursos,
existem conjuntos formados por pares de números que representam pontos
em um plano. Nesta seção, vamos introduzir formalmente o conceito de
par ordenado e como podemos construir conjuntos de pares utilizando
uma operação conhecida como produto cartesiano. As definições
seguintes apresentam estes conceitos.

\begin{Definition}[Par ordenado]
Sejam $A$ e $B$ conjuntos quaisquer em que $a \in A$ e $b \in
B$. Dizemos que $(a,b)$ é um par ordenado em que o primeiro elemento é
$a \in A$ e o segundo $b\in B$.
\end{Definition}

A operação sobre conjuntos que permite a criação de pares ordenados é
o chamado produto cartesiano, que é definido a seguir.

\begin{Definition}[Produto Cartesiano]
Sejam  $A$ e $B$ dois conjuntos quaisquer. O produto cartesiano de $A$
por $B$, $A\times B$, é definido como:
\[
A\times B = \{(a,b)\,|\,a \in A \land b\in B\}
\]
\end{Definition}

\begin{Example}
Sejam $A =\{1,2,3\}$ e $B = \{4,5,6\}$. Temos:
\[
A \times B = \{(1,4),(2,5),(3,6)\}
\]
Evidentemente, temos que $(1,4) \in A \times B$. Além disso, para o
par ordenado $(1,4)$ temos que $1$ é o primeiro elemento (um elemento
do conjunto $A$) e $4$ é o segundo (um elemento de $B$).
\end{Example}
Uma boa maneira de atestarmos a compreensão de um novo conceito
matemático é demonstrando teoremas sobre este.
\begin{Theorem}
Sejam $A, B$ e $C$ conjuntos arbitrários. Então $A \times (B\cap C) =
(A \times B) \cap (A \times C)$.
\end{Theorem}
\begin{proof}
Suponha que $A,B$ e $C$ são conjuntos arbitrários. Suponha $p$
arbitrário.
\begin{itemize}
   \item[$(\to)$] Suponha $p \in A \times (B \cap C)$. Pela definição
     de produto cartesiano, temos que $p = (a,b)$ em que $a \in A$ e
     $b \in B \cap C$. Já que $b\in B\cap C$, temos que $b \in B$ e $b
     \in C$. Como $a \in A$ e $b \in B$, temos que $(a,b)\in A \times
     B$. De maneira similar, já que $a \in A$ e $b \in C$, temos que
     $(a,b) \in A \times C$. Logo, $(a,b) \in (A\times B) \cap
     (B\times C)$. Portanto, se $p \in A\times (B\cap C)$ então
     $(A\times B)\cap (A \times C)$.
   \item[$(\leftarrow)$] Suponha $p\in (A\times B) \cap (A \times
     C)$. Assim, temos que $p \in A \times B$ e $p \in A \times C$.
     Pela definição de produto cartesiano, temos que $p = (a,b)$ em
     que $a \in A$, $b \in B$ e $b\in C$. Já que $b\in B$ e $b\in C$,
     temos que $b\in B\cap C$. Mas, como $a \in A$ e $b\in B\cap C$,
     temos que $(a,b)\in A\times (B\cap C)$. Logo, se $p \in (A\times
     B) \cap (A \times C)$ então $p \in A \times (B\cap C)$.
\end{itemize}
Como $p$ é arbitrário, temos que $A \times (B\cap C) = (A \times B)
\cap (A \times C)$. Portanto, para todos conjuntos $A,B$ e $C$ temos
que $A \times (B\cap C) = (A \times B)
\cap (A \times C)$.
\end{proof}

\begin{Commentary}
Um ponto crucial desta demonstração é a utilização das hipóteses que
um elemento $p$ que pertence ao produto cartesiano de dois conjuntos
deve ser um par em que o primeiro elemento pertence ao primeiro
conjunto e o segundo elemento, ao segundo conjunto.

No caso do teorema anterior, em um momento temos que $p \in A \times
(B\cap C)$, então $p = (a,b)$ em que $a \in A$ e $b \in B\cap C$.

Além deste
detalhe, toda a demonstração consiste apenas de uso de técnicas de
provas que já vimos no capítulos \ref{cap4} e \ref{cap5}.

Evidentemente, como o produto cartesiano de conjuntos é apenas um
conjunto de pares ordenados, todas as notações da teoria de conjuntos
(apresentadas no capítulo \ref{cap5}) são aplicáveis.
\end{Commentary}

Abaixo apresentamos outra demonstração similar.

\begin{Theorem}
Seja $A$ um conjunto qualquer. Então, $A \times \emptyset =
\emptyset$.
\end{Theorem}
\begin{proof}
Suponha $p$ arbitrário.
\begin{itemize}
    \item[$(\to)$] Suponha que $p \in A \times \emptyset$. Como $p \in
      A \times \emptyset$, existem $a \in A$ e $b\in \emptyset$ tais
      que $p = (a,b)$. Mas, não existe $b\in\emptyset$. Logo, o
      resultado desejado é provado por contradição.
    \item[$(\leftarrow)$] Suponha que $p \in \emptyset$. Como não
      existe $p\in\emptyset$, por contradição, o resultado é provado.
\end{itemize}
Como $p$ é arbitrário, temos que $A \times \emptyset = \emptyset$
\end{proof}

\begin{Commentary}
A chave da demonstração anterior é o uso do fato de que não existe
elemento $x \in \emptyset$, o que nos permite concluir a demonstração
usando contradição.
\end{Commentary}
\subsection{Exercícios}

\begin{enumerate}
  \item Prove os seguintes teoremas:
  \begin{enumerate}
    \item Seja $A$ um conjunto qualquer. Então $A \times \emptyset =
      \emptyset$.
    \item Sejam $A$ e $B$ conjuntos quaisquer. Se $A \times B =
      B\times A$ se e somente se $A = \emptyset$ ou $B = \emptyset$ ou
      $A = B$.
  \end{enumerate}
\end{enumerate}

\section{Introdução às Relações}

Matematicamente, especificamos que dois objetos $a$ e $b$ estão
relacionados dizendo que o par $(a,b)$ pertence ao conjunto de pares
que descreve uma propriedade de interesse sobre estes objetos. Usamos
relações (conceito matemático) para expressar relacionamentos entre
objetos modelados matematicamente como elementos de conjuntos.

\begin{Definition}[Relação]
Suponha que $A$ e $B$ são conjuntos quaisquer. Denominamos o conjunto
$R \subseteq A \times B$ uma relação de $A$ em $B$.
\end{Definition}

A seguir apresentamos alguns exemplos de relações.

\begin{Example}
Considere os seguintes conjuntos $A = \{1,2,3\}$ e $B =
\{4,5,6\}$. Temos que $R = \{(1,5),(3,4)\}$ é uma relação de $A$ em
$B$, já que $R\subseteq A \times B$.

Outro exemplo de relação, agora envolvendo um conjunto infinito de
pares, é:
\[ G = \{(x,y)\in\mathbb{R}\times\mathbb{R}\,|\,x < y\}\]
esta relação representa, utilizando pares, o conceito de ``menor''
sobre números reais.
\end{Example}

Relações não necessariamente são formadas apenas por conjuntos
numéricos. O próximo exemplo mostra relações sobre conjuntos não
numéricos.

\begin{Example}
Considere os seguintes conjuntos que poderiam ser utilizados para
modelar um sistema de informação em uma universidade:
\begin{itemize}
  \item $S$ : conjunto de todos os estudantes da universidade.
  \item $C$: conjunto de todos os cursos de graduação da universidade.
  \item $D$: conjunto de todas as disciplinas oferecidas em cursos da
    universidade.
  \item $P$: conjunto de todos os professores que lecionam na universidade.
\end{itemize}
Utilizando estes conjuntos, temos as seguintes relações:
\begin{itemize}
  \item $R = \{(e,c)\in S \times C\,|\,\text{O estudante } e\text{
      está matriculado no curso }c\}$.
  \item $R_1 =\{(p,d)\in P \times D\,|\,\text{O professor }p\text{
      leciona a disciplina }d\}$.
\end{itemize}
Evidentemente, estas relações estariam representadas por mecanismos
apropriados de bancos de dados relacionais em um sistema de informação
de uma universidade. A primeira relação modela as informações sobre
qual é o curso em que um aluno está matriculado e a segunda, qual
disciplina um professor leciona.
\end{Example}

A seguir, apresentamos alguns conceitos provavelmente já conhecidos
pelo leitor, mas em um contexto de funções e não de
relações\footnote{Veremos, posteriormente, que funções são apenas um
  tipo especial de relações.}.

\begin{Definition}[Domínio, Imagem, Inversa]
Suponha que $R$ é uma relação de $A$ em $B$. Então o domínio de $R$ é
o conjunto:
\[
dom(R) = \{a \in A \,|\, \exists b. b\in B \land (a,b) \in R\}
\]
A imagem\footnote{Normalmente, livros denotam o conjunto imagem usando
a abreviação $ran$ de \textit{range} de imagem em inglês.} de $R$ é definida pelo seguinte conjunto:
\[
ran(R) = \{b\in B \,|\, \exists a. a\in A \land (a,b) \in R\}
\]
Finalmente, a relação inversa de $R$, $R^{-1} \subseteq B\times A$, é:
\[
R^{-1} =\{(b,a)\,|\,\exists a. \exists b. a \in A \land b\in B \land
(a,b) \in R\}
\]
\end{Definition}

\begin{Example}
Considere:
\begin{itemize}
   \item $A = \{1,2,3,4\}$ e $B = \{6,7,8,9,0\}$.
   \item $R = \{(1,6),(3,0),(2,9)\}$.
\end{itemize}
Temos:
\begin{itemize}
   \item $dom(R) =\{1,2,3\}$
   \item $ran(R) =\{0,6,9\}$
   \item $R^{-1} = \{(6,1),(0,3),(9,2)\}$
\end{itemize}
\end{Example}

Encerraremos esta seção com um conceito importante: o de composição de
relações. Este conceito permite a construção de uma nova relação a
partir de duas relações existentes. A seguir definimos formalmente
este conceito e apresentamos alguns exemplos na sequência.

\begin{Definition}[Composição de Relações]
Sejam $R \subseteq A \times B$ e $S \subseteq B \times C$ duas
relações sobre conjuntos $A,B,C$ e $D$. A relação composta
de $S$ e $R$, $S \circ R$, é uma relação de $A$ em $C$ definida
como:
\[
S \circ R =\{(a,c)\,|\,\exists b. b \in B \land (a,b) \in R \land
(b,c) \in S\}
\]
\end{Definition}

\begin{Example}
Considere os seguintes conjuntos que poderiam ser utilizados para
modelar um sistema de informação em uma universidade:
\begin{itemize}
  \item $S$ : conjunto de todos os estudantes da universidade.
  \item $C$: conjunto de todos os cursos de graduação da universidade.
  \item $D$: conjunto de todas as disciplinas oferecidas em cursos da
    universidade.
  \item $P$: conjunto de todos os professores que lecionam na universidade.
\end{itemize}
e as seguintes relações
\begin{itemize}
	\item $R_1 = \{(p,d)\in P \times D \,\mid\,\text{$p$ leciona a disc. } d\}$.
	\item $R_2 = \{(d,c)\in D \times C\,\mid\,\text{$d$ est\'a no curso $c$}\}$.
	\item $R_3 = \{(e,d)\in E\times D\,\mid\,\text{$e$ est\'a matr. em $d$}\}$.
\end{itemize}
Temos que a relação $R_1 \circ R_2 \subseteq P \times C$ é definida
como:
\[
R_1 \circ R_2 =\{(p,c) \in P \times C \,|\,\text{o professor $p$
  leciona alguma disciplina do curso $c$}\}
\]
De maneira similar, podemos definir uma relação que especifica que um
certo aluno está matriculado em um curso, usando composição e as
relações $R_2$ e $R_3$:
\[
R_3 \circ R_2 =\{(e,c)\,|\,\text{o aluno $e$ está matriculado em
  alguma disciplina do curso $c$}\}
\]
\end{Example}

Finalizaremos esta seção com alguns teoremas envolvendo as definições
apresentadas. Novamente vale ressaltar que cabe ao leitor a tarefa de
reconstruir o rascunho para um melhor entendimento do conteúdo
apresentado.

\begin{Theorem}
Suponha que $R \subseteq A \times B$. Então, $(R^{-1})^{-1} = R$.
\end{Theorem}
\begin{proof}
Suponha $p$ arbitrário.
\begin{itemize}
  \item[$(\to)$]: Suponha que $p\in (R^{-1})^{-1}$. Se $R\subseteq A
    \times B$, então $R^{-1}\subseteq B\times A$. Já que
    $R^{-1}\subseteq B\times A$ então $(R^{-1})^{-1}\subseteq A\times
    B$.Como $p\in (R^{-1})^{-1}$, temos que existem $a\in A$ e $b\in
    B$ e $p = (a,b)$. Se $(a,b) \in (R^{-1})^{-1}$, então $(b,a)\in
    R^{-1}$ e, portanto, pela definição de relação inversa, temos que
    $(a,b)\in R$. Logo, se $(a,b)\in
    (R^{-1})^{1}$ então $(a,b)\in R$.
  \item[$(\leftarrow)$]: Suponha que $p \in R$. Como $R\subseteq A$,
    temos que existem $a\in A$ e $b\in B$ tais que $p = (a,b)$. Pela
    definição de inversa, temos que se $(a,b)\in R$ temos que $(b,a)
    \in R^{-1}$ e $(a,b) \in (R^{-1})^{-1}$. Logo, se $(a,b) \in R$
    temos que $(a,b) \in (R^{-1})^{-1}$.
\end{itemize}
   Como $p$ é arbitrário, temos que $(R^{-1})^{-1} = R$.
\end{proof}

\begin{Commentary}
A demonstração do teorema anterior utiliza a representação lógica da
igualdade de dois conjuntos de pares ordenados
(relações). Formalmente, definimos a igualdade de dois conjuntos $A$ e
$B$ da seguinte maneira:
\[ A = B \equiv \forall x. x \in A \leftrightarrow x \in B \]
Além disso, utilizamos a definição de inversa de uma relação, que
consiste em ``trocar'' a ordem dos elementos de um par ordenado. Se
par $(x,y)\in R$ então temos que $(y,x)\in R^{-1}$. O restante da
demonstração consiste em uso das estratégias de prova para o
quantificador universal e o conectivo bicondicional.
\end{Commentary}

\begin{Theorem}
Suponha que $R\subseteq A \times B$, $S \subseteq B \times C$ e $T
\subseteq C \times D$. Então, $T \circ (S \circ R) = (T \circ S) \circ
R$.
\end{Theorem}
\begin{proof}
Suponha que $R\subseteq A \times B$, $S \subseteq B \times C$ e $T
\subseteq C \times D$.
 Suponha $p$ arbitrário.
\begin{itemize}
  \item[$(\to)$]: Suponha que $p \in T \circ (S \circ R)$. Como
    $R\subseteq A \times B$, $S \subseteq B \times C$ e $T
\subseteq C \times D$, temos que $S \circ R \subseteq A \times C$ e
   $T \circ (S \circ R)\subseteq A \times D$. Assim, como  $T \circ (S
   \circ R)\subseteq A \times D$, temos que existem $a \in A$ e $d\in
   D$ tais que $p = (a,d)$. Pela definição de composição, temos que
   para $(a,d) \in T \circ (S \circ R)$, deve existir $c \in C$ tal
   que $(a,c) \in S \circ R$ e $(c,d) \in T$. Mas, para $(a,c) \in S
   \circ R$ deve existir $b\in B$ tal que $(a,b) \in R$ e $(b,c) \in
   S$. Logo, pela definição de composição, temos que $(b,d) \in T
   \circ S$. Novamente, por composição, podemos concluir que
   $(a,d) \in (T \circ S) \circ R$. Logo, se $p \in T \circ (S \circ
   R)$ então $p \in (T \circ S) \circ R$.
  \item[$(\leftarrow)$]: Suponha que $p \in (T \circ S) \circ R$. Como
    $R\subseteq A \times B$, $S \subseteq B \times C$ e $T
\subseteq C \times D$, temos que $T \circ S\subseteq B \times D$ e
 $(T\circ S) \circ R \subseteq A \times D$. Assim, como $(T\circ S)
 \circ R \subseteq A \times D$, temos que existem $a \in A$ e $d\in D$
 tais que $p = (a,d)$. Pela definição de composição, temos que para
 $(a,d) \in (T\circ S) \circ R$ deve existir  $b\in B$ tal que $(b,d)
 \in T \circ S$ e $(a,b) \in R$. Por sua vez, para $(b,d)
 \in T \circ S$, deve existir $c\in C$ tal que $(b,c) \in S$ e $(c,d)
 \in T$. Novamente, por composição, temos que $(a,c)\in S\circ R$ e
 que $(a,d) \in T\circ (S \circ R)$. Logo, se $p \in (T\circ S) \circ
 R$ então $p \in T \circ (S \circ R)$.
\end{itemize}
Como $p$ é arbitrário, temos que $T \circ (S \circ R) = (T \circ S)
\circ R$.
\end{proof}

\begin{Commentary}
Neste teorema utilizou-se extensivamente a definição de composição de
relações. Se $R \subseteq A \times  B$ e $S \subseteq B \times C$,
então $S\circ R \subseteq A \times C$ é definido como:
\[
S \circ R = \{(a,c) \,|\, \exists b. b\in B \land (a,b) \in R \land
(b,c) \in S\}
\]
A partir desta definição, utilizamos a hipótese envolvendo o
quantificador existencial para deduzir cada um dos pares que pertencem
as relações $R,S$ e $T$ que foram utilizados para construir o par $p =
(a,d)$ utilizado na demonstração.
\end{Commentary}

\subsection{Exercícios}

\begin{enumerate}
	\item Sejam $A=\{1,2,3\}$, $B=\{4,5,6\}$, $R=\{(1,4), (1,5), (2,5), (3,6)\}$ e $S=\{(4,5), (4,6), (5,4), (6,6)\}$.
	      Note que $R\subseteq A\times B$ e $S\subseteq B\times B$. Encontre as seguintes rela\c{c}\~oes:
	\begin{enumerate}
		\item $S\circ R$
		\item $S\circ S$
		\item $S^{-1}\circ R$
		\item $R^{-1}\circ S$
	\end{enumerate}
	\item Seja $R$ uma rela\c{c}\~ao sobre um conjunto $A$. Prove que
          $R\circ R^{-1}\subseteq i_{A}$, em que $i_{A}=\{(x,x)\,|\,x\in A\}$.
	\item Sejam $A$ e $B$ dois conjuntos quaisquer.
	\begin{enumerate}
		\item Prove que para toda rela\c{c}\~ao $R\subseteq A\times B$, $R\circ i_{A} = R$, em que $i_{A}=\{(x,x)\,|\,x\in A\}$.
		\item Prove que para toda rela\c{c}\~ao $R\subseteq A\times B$, $i_{B}\circ R = R$, em que $i_{B}=\{(x,x)\,|\,x\in B\}$.
	\end{enumerate}
\end{enumerate}

\section{Relações Binárias}

Nesta seção apresentaremos propriedades de um tipo especial de
relação: as relações binárias, cuja definição apresentamos a seguir.

\begin{Definition}[Relação Binária]
Seja $A$ um conjunto qualquer. Dizemos que $R$ é uma relação binária
sobre $A$ se $R \subseteq A\times A$.
\end{Definition}

\begin{Example}
As seguintes relações são relações binárias sobre os seguintes
conjuntos $A = \{1,2\}$, $\mathbb{N}$, $P$ (conjunto de todas as
pessoas) e subconjuntos de um conjunto $B$ ($\mathcal{P}(B)$).
\begin{itemize}
  \item $R = \{(1,2),(1,1)\}$.
  \item $G = \{(x,y)\in \mathbb{N}\times\mathbb{N}\,|\,x > y\}$.
  \item $I =\{(x,y)\in P \times P\,|\,x\text{ é irmão de }y\}$.
  \item $S = \{(x,y)\in \mathcal{P}(B)\times\mathcal{P}(B)\,|\,x
    \subseteq y\}$.
\end{itemize}
\end{Example}

Relações binárias são interessantes por possuírem diversas
propriedades que permitem que possamos classificá-las e usar diversos
resultados sobre estas propriedades. Antes de apresentarmos estas propriedades,
vamos introduzir uma notação para representar o fato que um certo par
pertence a uma relação $R$.
\begin{Notation}
Seja $R\subseteq A \times A$ uma relação binária qualquer sobre um
conjunto $A$. Representaremos o fato de que $(x,y) \in R$ como $xRy$.
\end{Notation}
A seguir definimos estas propriedades.

\begin{Definition}[Relação Reflexiva]
Seja $R\subseteq A \times A$ uma relação binária qualquer. Dizemos que
$R$ é uma relação reflexiva se
\[
\forall x. x\in A \to  xRx.
\]
\end{Definition}

\begin{Example}
Abaixo apresentamos diversos exemplos de relações reflexivas.
\begin{itemize}
  \item $R = \{(x,y) \in \mathbb{N} \times \mathbb{N}\,|\, x \leq
    y\}$ é uma relação reflexiva pois todo número $n\in\mathbb{N}$ é
    menor ou igual a si próprio.
  \item $R_1 = \{(p,q)\,|\,\text{as palavras $p$ e $q$ iniciam com a
      mesma letra do alfabeto.}\}$ é uma relação reflexiva pois toda
    palavra $p$ inicia com a mesma letra que ela própria.
  \item $R_2=\{(x,y)\in\mathcal{P}(A)\times\mathcal{P}(A)\,|\,x
    \subseteq y\}$ é uma relação reflexiva pois todo conjunto $x$ é
    subconjunto de si próprio.
\end{itemize}
\end{Example}

\begin{Definition}[Relação Irreflexiva]
Seja $R\subseteq A \times A$ uma relação binária qualquer. Dizemos que
$R$ é uma relação irreflexiva se
\[
\forall x. x\in A \to  \neg xRx.
\]
\end{Definition}

\begin{Example}
São exemplos de relações irreflexivas.
\begin{itemize}
  \item $R = \{(x,y)\in\mathbb{R}\times \mathbb{R}\,|\, x < y\}$, é
    uma relação irreflexiva pois, para qualquer número $x \in
    \mathbb{R}$, temos que não é verdade que $x < x$.
  \item $R_1 =\{(a,b) \,|\, \text{A pessoa $a$ é pai da pessoa
      $b$}\}$, é uma relação irreflexiva pois, não é possível uma
    pessoa ser pai dela própria.
\end{itemize}
\end{Example}

\begin{Definition}[Relação Simétrica]
Seja $R\subseteq A \times A$ uma relação binária qualquer. Dizemos que
$R$ é uma relação simétrica se
\[\forall x.\forall y. x \in A \land y\in A \land x R y \to y R x. \]
\end{Definition}
\begin{Example}
São exemplos de relações simétricas.
\begin{itemize}
   \item $R = \{(p,q)\,|\,\text{a pessoa $p$ é irmã(o) da pessoa
     }q.\}$ é uma relação simétrica já que, para quaisquer $p$ e $q$,
    se $p$ é irmão de $q$ então $q$ também é irmão de $p$.
  \item $R_1 = \{(p,q)\,|\,\text{as palavras $p$ e $q$ iniciam com a
      mesma letra do alfabeto.}\}$ é uma relação simétrica já que,
    para quaisquer palavras $p$ e $q$, se $p$ inicia com a mesma letra
    que $q$, então $q$ inicia com a mesma letra que $p$.
  \item $R_2=\{(x,y)\in\mathcal{P}(A)\times\mathcal{P}(A)\,|\,\exists
    z. z \in x \land z \in y\}$ é uma relação simétrica, pois se um
    elemento $z$ pertence a um conjunto $x$ e a um conjunto $y$ então
    este mesmo elemento pertence ao conjunto $y$ e $x$.
\end{itemize}
\end{Example}

\begin{Definition}[Relação Transitiva]
Seja $R\subseteq A \times A$ uma relação binária qualquer. Dizemos que
$R$ é uma relação transitiva se
\[
\forall x.\forall y. \forall z. x \in A \land y \in A \land z \in A
\land xRy \land yRz \to xRz
\]
\end{Definition}


\begin{Example}
São exemplos de relações transitivas.
\begin{itemize}
  \item $R = \{(x,y) \in \mathbb{N} \times \mathbb{N}\,|\, x \leq
    y\}$ é uma relação transitiva, pois se $x \leq y$ e $y \leq z$
    então $x\leq z$.
  \item $R_1 = \{(p,q)\,|\,\text{as palavras $p$ e $q$ iniciam com a
      mesma letra do alfabeto.}\}$ é uma relação transitiva já que,
    para quaisquer palavras $p$, $q$ e $r$, se $p$ inicia com a mesma letra
    que $q$ e $q$ inicia com a mesma letra que $r$ então $p$ inicia
    com a mesma letra que $r$.
  \item
    $R_2=\{(x,y)\in\mathcal{P}(A)\times\mathcal{P}(A)\,|\,x\subseteq y\}$ é uma relação transitiva, pois se um
    conjunto $x$ está contido em um conjunto $y$ e $y$ está contido em
    um conjunto $z$, então $x$ está contido em $z$.
\end{itemize}
\end{Example}


\begin{Definition}[Relação Anti-simétrica]
Seja $R\subseteq A \times A$ uma relação binária qualquer. Dizemos que
$R$ é uma relação anti-simétrica se
\[
\forall x.\forall y.  x \in A \land y \in A
\land xRy \land yRx \to x = y
\]
\end{Definition}

\begin{Example}
São exemplos de relações anti-simétricas.
\begin{itemize}
  \item $R = \{(x,y) \in \mathbb{N} \times \mathbb{N}\,|\, x \leq
    y\}$ é uma relação anti-simétrica, pois se $x \leq y$ e $y \leq x$
    então $x = y$.
  \item
    $R_2=\{(x,y)\in\mathcal{P}(A)\times\mathcal{P}(A)\,|\,x\subseteq
    y\}$ é uma relação anti-simétrica, pois se um
    conjunto $x$ está contido em um conjunto $y$ e $y$ está contido em
    $x$, então os conjuntos $x$ e $y$ são iguais (pela definição de
    igualdade de conjuntos).
\end{itemize}
\end{Example}
Agora que estas definições foram apresentadas juntamente com alguns
exemplos, as utilizaremos para demonstrar alguns teoremas. Vamos
demonstrar um dos teoremas em detalhes (apresentando o rascunho e a
construção passa-a-passo do texto) e os outros dois mostraremos apenas
o texto final.
\begin{Theorem}
Suponha que $R$ é uma relação binária sobre um conjunto $A$. Então,
Se $R$ é reflexiva então $i_A \subseteq R$, em que  $i_A=\{(x,x)\,|\,x \in A\}$.
\end{Theorem}
\begin{Example}
Demonstraremos o primeiro item em detalhes.
A partir do enunciado do primeiro item, temos a seguinte configuração
inicial do rascunho.
\begin{flushleft}
\begin{tabular}{ll}
Hipóteses & Conclusão \\
$R\subseteq A \times A$ &  $R$ é reflexiva $\to i_{A}\subseteq R$
\end{tabular}
\end{flushleft}
Utilizando a estratégia de prova direta, temos a seguinte configuração
do rascunho.
\begin{flushleft}
\begin{tabular}{ll}
Hipóteses & Conclusão \\
$R\subseteq A \times A$ &  $i_{A}\subseteq R$\\
$R$ é reflexiva & \\
\end{tabular}
\end{flushleft}
Utilizando a definição de subconjunto, temos
\begin{flushleft}
\begin{tabular}{ll}
Hipóteses & Conclusão \\
$R\subseteq A \times A$ &  $\forall p. p \in i_{A} \to p \in  R$\\
$R$ é reflexiva & \\
\end{tabular}
\end{flushleft}
Agora, aplicando as estratégias de prova para o quantificador
universal e implicação (nesta ordem) temos
\begin{flushleft}
\begin{tabular}{ll}
Hipóteses & Conclusão \\
$R\subseteq A \times A$ &  $p \in  R$\\
$R$ é reflexiva & \\
$p$ arbitrário & \\
$p \in i_{A}$ & \\
\end{tabular}
\end{flushleft}
Se $p \in i_{A}$ então existe $y$ tal que $p = (y,y)$.
\begin{flushleft}
\begin{tabular}{ll}
Hipóteses & Conclusão \\
$R\subseteq A \times A$ &  $p \in  R$\\
$R$ é reflexiva & \\
$p$ arbitrário & \\
$p \in i_{A}$ & \\
$\exists y. y\in A \land p = (y,y)$ & \\
\end{tabular}
\end{flushleft}
Usando a estratégia de hipóteses para o quantificador existencial,
temos
\begin{flushleft}
\begin{tabular}{ll}
Hipóteses & Conclusão \\
$R\subseteq A \times A$ &  $p \in  R$\\
$R$ é reflexiva & \\
$p$ arbitrário & \\
$p \in i_{A}$ & \\
$\exists y. y\in A \land p = (y,y)$ & \\
$y\in A$ & \\
$p = (y,y)$ & \\
\end{tabular}
\end{flushleft}
Utilizando a definição de relação reflexiva, temos:
\begin{flushleft}
\begin{tabular}{ll}
Hipóteses & Conclusão \\
$R\subseteq A \times A$ &  $p \in  R$\\
$R$ é reflexiva & \\
$p$ arbitrário & \\
$p \in i_{A}$ & \\
$\exists y. y\in A \land p = (y,y)$ & \\
$y\in A$ & \\
$p = (y,y)$ & \\
$\forall x. x \in A \to xRx$ & \\
\end{tabular}
\end{flushleft}
Agora, basta usar a eliminação do quantificador universal
(substituindo $x$ por $y$), temos:
\begin{flushleft}
\begin{tabular}{ll}
Hipóteses & Conclusão \\
$R\subseteq A \times A$ &  $p \in  R$\\
$R$ é reflexiva & \\
$p$ arbitrário & \\
$p \in i_{A}$ & \\
$\exists y. y\in A \land p = (y,y)$ & \\
$y\in A$ & \\
$p = (y,y)$ & \\
$\forall x. x \in A \to xRx$ & \\
$y \in A \to y R y$ & \\
\end{tabular}
\end{flushleft}
Usando as hipóteses $y \in A$ e $y\in A \to y R y$, concluímos a
demonstração do teorema.

Agora, vamos construir o texto deste teorema
passo-a-passo. Primeiramente, mostramos a parte do texto
correspondente a primeira implicação deste teorema.
\begin{flushleft}
Suponha que $R$ seja uma relação reflexiva.\\
\verb|   |[Prova de $i_{A}\subseteq R$].\\
Portanto, se $R$ é uma relação reflexiva então $i_{A} \subseteq R$.
\end{flushleft}
Agora, utilizando a definição de $\subseteq$ em termos do
quantificador universal, temos:
\begin{flushleft}
Suponha que $R$ seja uma relação reflexiva.\\
\verb|   |Suponha $p$ arbitrário.\\
\verb|      |[Prova de $p \in i_{A} \to p \in R$].\\
\verb|   |Como $p$ é arbitrário, temos que $i_{A} \subseteq R$.\\
Portanto, se $R$ é uma relação reflexiva então $i_{A} \subseteq R$.
\end{flushleft}
Usando prova direta, temos
\begin{flushleft}
Suponha que $R$ seja uma relação reflexiva.\\
\verb|   |Suponha $p$ arbitrário.\\
\verb|      |Suponha que $p \in i_A$.\\
\verb|         |[Prova de $p \in R$].\\
\verb|      |Logo, se $p\in i_A$ então $p \in R$.\\
\verb|   |Como $p$ é arbitrário, temos que $i_{A} \subseteq R$.\\
Portanto, se $R$ é uma relação reflexiva então $i_{A} \subseteq R$.
\end{flushleft}
Agora, concluímos o texto utilizando as hipóteses.
\begin{flushleft}
Suponha que $R$ seja uma relação reflexiva.\\
\verb|   |Suponha $p$ arbitrário.\\
\verb|      |Suponha que $p \in i_A$.\\
\verb|         |Como $p\in i_A$, existe $y \in A$ tal que $p = (y,y)$.\\
\verb|         |Como $R$ é reflexiva e $y \in A$, temos que $yRy$.\\
\verb|      |Logo, se $p\in i_A$ então $p \in R$.\\
\verb|   |Como $p$ é arbitrário, temos que $i_{A} \subseteq R$.\\
Portanto, se $R$ é uma relação reflexiva então $i_{A} \subseteq R$.
\end{flushleft}
\end{Example}
Agora, mais dois teoremas sobre relações. Estes serão apresentados sem
detalhes\footnote{\textit{Dica do professor amigo: \textbf{Entenda} todas essas demonstrações!}}.
\begin{Theorem}
Seja $R$ uma relação binária sobre um conjunto $A$. Então,
Se $R$ é simétrica então $R = R^{-1}$.
\end{Theorem}
\begin{proof}
Suponha que $R$ seja uma relação simétrica. Suponha $p$ arbitrário.
\begin{itemize}
  \item[$(\to)$]: Suponha que $p \in R$. Como $R\subseteq A \times A$,
    então existem $x,y \in A$ tais que $p = (x,y)$. Uma vez que $
    xRy$ e $R$ é simétrica, temos que $yRx$. Já que $yRx$, pela
    definição de inversa, temos que $xR^{-1}y$. Logo, se $p \in R$
    então $p\in R^{-1}$.
  \item[$(\leftarrow)$]: Suponha que $p \in R^{-1}$. Como $R\subseteq
    A \times A$, então existem $x,y \in A$ tais que $p = (x,y)$. Uma
    vez que $xR^{-1}y$, pela definição de inversa, temos que
    $yRx$. Como $yRx$ e $R$ é simétrica, temos que $xRy$. Logo, se $p
    \in R^{-1}$ então $p \in R$.
\end{itemize}
Como $p$ é arbitrário, temos que $R = R^{-1}$. Portanto, se $R$ é uma
relação simétrica, temos que $R = R^{-1}$.
\end{proof}
\begin{Theorem}
Seja $R$ uma relação binária sobre um conjunto $A$. Então,
Se $R$ é transitiva então $R \circ R \subseteq R$.
\end{Theorem}
\begin{proof}
Suponha que $R$ é uma relação transitiva. Suponha $p$ arbitrário.
Suponha que $p \in R \circ R$. Como $R\subseteq A
    \times A$, temos que existem $a,c \in A$ tais que $p =
    (a,c)$. Como $(a,c) \in R\circ R$, pela definição de composição de
    relações, temos que existe $b\in A$ tal que $aRb$ e $bRc$. Como
    $R$ é transitiva, $aRb$ e $bRc$, temos que $aRc$. Logo, se $p \in
    R \circ R$ então $p\in R$. Como $p$ é arbitrário, temos que $R\circ R
    \subseteq R$. Portanto, se $R$ é uma relação transitiva então $R
    \circ R \subseteq R$.
\end{proof}


\subsection{Exercícios}

\begin{enumerate}
  \item Seja $A = \{\text{banana, abacate, melancia, ovo, ócio}\}$ e $R =
    \{(x,y)\in A \times A\,|\, \text{a palavra $x$ tem alguma letra em
    comum com a palavra $y$}\}$.
  \begin{enumerate}
    \item Liste os pares que formam a relação $R$.
    \item Quais propriedades (reflexiva, irreflexiva, simétrica,
      transitiva, anti-simétrica) possui a relação $R$?
  \end{enumerate}
  \item Demonstre os seguintes teoremas.
  \begin{enumerate}
      \item Suponha que $R$ é uma relação binária sobre um conjunto
        $A$. Se $R$ é reflexiva então $R \subseteq R \circ R$.
      \item Suponha que $R$ é uma relação binária sobre um conjunto
        $A$. Se $R$ é reflexiva então $R^{-1}$ também é reflexiva.
      \item Sejam $R_1$ e $R_2$ duas relações binárias sobre um
        conjunto $A$. Então se $R_1$ e $R_2$ são relações simétricas,
        então $R_1\cup R_2$ e $R_1\cap R_2$ também são simétricas.
  \end{enumerate}
\end{enumerate}

\section{Relações de Ordem}

\subsection{Introdução}

Usando as definições de propriedades de relações, apresentadas na seção
anterior, podemos notar que diversas relações possuem características
similares. Note as relações seguintes:

\begin{enumerate}
  \item $R =\{(x,y)\in\mathbb{R}\times\mathbb{R}\,|\,x \leq y\}$
  \item $S = \{(x,y)\in\mathcal{P}(A)\times\mathcal{P}(A)\,|\,x
    \subseteq y\}$
\end{enumerate}
são ambas relações reflexivas, transitivas e anti-simétricas. Relações
com estas propriedades são denominadas ordens parciais e possuem
características que permitem entendê-las como um critério de ordenação
entre elementos de um conjunto. A próxima definição apresenta estes
conceitos.

\begin{Definition}[Pré-Ordem e Ordens Parciais]
Seja $R \subseteq A \times A$ uma relação. Dizemos que $R$ é uma
pré-ordem se $R$ for reflexiva e transitiva. Uma ordem parcial é uma
relação reflexiva, transitiva (pré-ordem) e anti-simétrica.
\end{Definition}

\begin{Example}
São exemplos de ordens parciais (e, evidentemente, de pré-ordens) as
seguintes relações.
\begin{enumerate}
  \item $R =\{(x,y)\in\mathbb{R}\times\mathbb{R}\,|\,x \leq y\}$.
  \item $S = \{(x,y)\in\mathcal{P}(A)\times\mathcal{P}(A)\,|\,x
    \subseteq y\}$.
  \item $T = \{(x,y)\in\{1,2\}\times\{1,2\}\,|\,|x| \leq |y|\}$.
\end{enumerate}
\end{Example}

Relações de ordem especificam uma maneira de compararmos elementos de
um conjunto. A próxima definição torna este conceito preciso.
\begin{Definition}[Elementos Comparáveis]
Seja $R \subseteq A \times A$ uma relação de ordem qualquer
(pré-ordem, ordem parcial, ordem total, lexicográfica) e $a,b \in
A$. Dizemos que $a$ e $b$ são comparáveis em $R$ se $aRb$ ou $bRa$.
\end{Definition}

O nome ordem ``parcial'' deve-se ao fato de que nem sempre todos
elementos de um conjunto $A$ são comparáveis de acordo com uma ordem
parcial $R$. Como exemplo, o par $(\{1,2\},\emptyset) \not\in S$, pois
$\{1,2\}\not\subseteq \emptyset$. Por sua vez, para quaisquer números
reais $x,y$ temos que $x \leq y$ e $x \not\leq y$. Quando uma ordem
parcial permite a comparação entre quaisquer elementos de um conjunto,
dizemos que esta é uma ordem total.
\begin{Definition}[Ordem Total]
Seja $R \subseteq A \times A$ uma relação. Dizemos que $R$ é uma ordem
total se $R$ for uma ordem parcial e adicionalmente a seguinte
condição é verdadeira:
\[
\forall x. \forall y. x\in A \land y \in A \land (xRy \lor yRx)
\]
\end{Definition}

Evidentemente, relações como $\leq$ e $\geq$ definidas sobre conjuntos
numéricos são ordens totais pois, sempre é possível comparar dois
números para determinarmos qual destes é o maior (ou menor).

\begin{Example}
Para ilustrar estes conceitos, vamos considerar as seguintes relações
definidas sobre $A = \{1,2\}$:
\begin{itemize}
  \item $S = \{(x,y)\in\mathcal{P}(A)\times\mathcal{P}(A)\,|\,x
    \subseteq y\}$.
  \item $T = \{(x,y)\in A \times A\,|\,|x| \leq |y|\}$.
\end{itemize}
É fácil constatar que ambas estas relações são ordens parciais. Porém,
note que apenas $T$ é total, pois todo par de conjuntos pode ser
comparado com respeito ao tamanho destes, mas o mesmo não acontece com
a noção de subconjunto, pois temos que o seguinte par não pertence a
relação $S$: $(\{1,2\},\{1\})$. Logo, temos que $S$ não é uma ordem
total, pois não atende a condição
\[
\forall x. \forall y. x\in A \land y \in A \land (xSy \lor ySx)
\]
\end{Example}

Outros tipos importantes de relações são as chamadas ordens estritas
e ordens lexicográficas. A primeira é um tipo de relação de ordem em
que possui características similares a relação de $<$ e $>$ para
números e ordens lexicográficas constituem ordens parciais para
$n$-uplas de valores.

\begin{Definition}[Ordem Estrita]
Seja $R \subseteq A \times A$ uma relação. Dizemos que $R$ é uma ordem
estrita se $R$ é irreflexiva, transitiva e anti-simétrica.
\end{Definition}

Note que toda ordem parcial $R$ pode ser ``transformada'' em uma ordem
estrita eliminando os pares $i_{A}=\{(x,x)\,|\,x\in A\}$, isto é, se
$R$ é uma ordem parcial, então $R - i_A$ é uma ordem estrita.

\begin{Definition}[Ordem Lexicográfica]\label{lexorder}
Seja $\sqsubset \subseteq A \times A$ uma relação de ordem parcial
sobre $A$. Definimos a ordem lexicográfica induzida por $\sqsubset$,
$R \subseteq (A \times A) \times (A \times A)$, entre pares de
elementos de $A$, como:
\[
R = \{((x,y),(x',y')\,|\,x\sqsubset x' \land [x = x' \lor y \sqsubset y']\}
\]
\end{Definition}
\subsubsection{Exercícios}
\begin{enumerate}
  \item Considere as relações apresentadas no exemplo
    125.
  \begin{enumerate}
    \item Liste os pares que formam as relações $S$ e $T$.
    \item Denomine por $X$ um conjunto de pares que se presentes na
      relação $T$ a tornariam uma ordem total. Determine o menor
      conjunto $X$ tal que $X \cup T$ é uma ordem total.
  \end{enumerate}
  \item Seja $R$ uma ordem parcial sobre um conjunto $A$
    qualquer. Prove que $R^{-1}$ também é uma ordem parcial.
  \item Seja $R$ uma ordem parcial sobre um conjunto $A$ qualquer.
   Prove que $R - i_A$ é uma ordem estrita.
  \item A definição \ref{lexorder} apresentou como definir uma ordem
   lexicográfica para pares de valores de um certo conjunto
   $A$. Apresente uma definição similar de uma ordem para triplas de
   valores de um conjunto $A$.
\end{enumerate}

\subsection{Elementos Máximos e Mínimos}

Considere o seguinte conjunto $A =\{\text{me, tame, men, mental,
  mentalist}\}$ e a seguinte relação sobre este:
\[
R = \{(x,y)\in A \times A\,|\,x \text{ é uma subpalavra de }y\}
\]
É fácil mostrar que a relação $R$ é uma ordem parcial sobre este
conjunto $A$ (prove isto!). Conforme já mencionado em diversos
momentos, relações de ordem especificam critérios de comparação
(ordem) entre os elementos do conjunto sobre o qual a relação está definida.

Desta forma, temos que, como a palavra me é subpalavra de tame, men,
mental e mentalist. Se considerarmos que um par $xRy$ denota que $x$ é
``menor'' que $y$ de acordo com a ordem parcial $R$, temos que o
elemento me é o menor de todos os elementos do conjunto $A$. Elementos
com esta propriedade são ditos elementos mínimos de um conjunto. A
próxima definição formaliza este conceito.

\begin{Definition}[Elemento Mínimo e Máximo]
Seja $R$ uma relação de ordem parcial sobre um conjunto $A$,
$B\subseteq A$ e $b \in B$. Dizemos que $b$ é um elemento mínimo de
$B$, com respeito a relação $R$, se
\[
\forall x. x\in B \to b R x
\]
De maneira similar, dizemos que $b$ é um elemento máximo de $B$ se
\[
\forall x. x \in B \to x R b
\]
\end{Definition}

\begin{Definition}[Elementos Minimal e Maximal]
Seja $R$ uma relação de ordem parcial sobre um conjunto $A$,
$B\subseteq A$ e $b \in B$. Dizemos que $b$ é um elemento minimal de
$B$, com respeito a relação $R$, se
\[
\neg \exists x. x\in B \land x R b \land x \neq b
\]
De maneira similar, dizemos que $b$ é um elemento maximal de $B$ se
\[
\neg \exists x. x\in B \land b R x \land x \neq b
\]
\end{Definition}
A seguir apresentamos alguns exemplos que ilustram estas definições.
\begin{Example}
Considere os seguintes problemas.
\begin{itemize}
  \item Seja $L =\{(x,y)\in\mathbb{R}\times\mathbb{R}\,|\, x \leq
    y\}$, que evidentemente é uma ordem parcial.  Seja $B = \{x \in
    \mathbb{R}\,|\,x \geq 7\}$. O conjunto $B$ possui elementos
    mínimos / minimais? E o conjunto $C =\{x\in\mathbb{R}\,|\,x >
    7\}$?
    \item Seja $S
      =\{(X,Y)\in\mathcal{P}(\mathbb{N})\times\mathcal{P}(\mathbb{N})\,|\,X
      \subseteq Y\}$, que evidentemente é uma ordem parcial. Seja
      $\mathcal{F}=\{X\in\mathcal{P}(\mathbb{N})\,|\,2\in X \land 3
      \in X\}$. O conjunto $\mathcal{F}$ possui elementos mínimos e minimais?
\end{itemize}
\textbf{Solução}: É evidente que $7 \leq x$, para todo $x \in
B$. Logo, $7$ é um elemento mínimo de $B$. Como não existe $x \in B$,
$x\neq 7$ e $x R 7$, temos que $7$ também é um elemento minimal de
$B$.

Note que para o segundo item, temos que o conjunto $\{2,3\}$ é
subconjunto de todo $X\in\mathcal{F}$. Como $\{2,3\}\in\mathcal{F}$,
temos que este é o elemento mínimo e minimal deste conjunto (porquê?).
\end{Example}

Vale a pena ressaltar que todo elemento mínimo (máximo) é minimal
(maximal), mas a recíproca não é verdadeira, além disso, se um
conjunto possui um elemento mínimo (máximo), este é o único minimal
(maximal) do conjunto em questão. Estes fatos serão demonstrados
formalmente pelos teoremas a seguir \footnote{\textbf{Entenda} esses
  teoremas e suas demonstrações. --- Dica do seu professor camarada.}.


\begin{Theorem}
	Suponha $R$ uma ordem parcial sobre $A$ e $B\subseteq A$. Se $B$ possui um elemento m\'inimo, ent\~ao este \'e \'unico.
\end{Theorem}

\begin{tabular}{lcl}
 Hip\'oteses & \hspace{1cm} & Conclusão\\
 		     & & $B$ possui m\'inimo $\rightarrow$ este m\'inimo \'e \'unico\\
 $B$ possui m\'inimo & & este m\'inimo \'e \'unico\\
\end{tabular}
\vspace{1cm}

Evidentemente, a senten\c{c}a ``$B$ possui m\'inimo'', pode ser representada por um quantificador existencial:

\vspace{1cm}
\begin{tabular}{lcl}
 Hip\'oteses & \hspace{1cm} & Conclusão\\
 		     & & $B$ possui m\'inimo $\rightarrow$ este m\'inimo \'e \'unico\\
 $B$ possui m\'inimo & & este m\'inimo \'e \'unico\\
 $\exists b. b \in B \land b$ \'e o m\'inimo de $B$ & & este m\'inimo \'e \'unico\\
\end{tabular}
\vspace{1cm}

Como temos uma hip\'otese envolvendo um quantificador existencial, podemos introduzir uma nova vari\'avel para representar
o elemento m\'inimo do conjunto $B$:

\vspace{1cm}

\begin{tabular}{lcl}
 Hip\'oteses & \hspace{1cm} & Conclusão\\
 		     & & $B$ possui m\'inimo $\rightarrow$ este m\'inimo \'e \'unico\\
 $B$ possui m\'inimo & & este m\'inimo \'e \'unico\\
 $\exists b. b \in B \land b$ \'e o m\'inimo de $B$ & &\\
 $b\in B$    & & $b$ \'e o \'unico m\'inimo de $B$\\
 $b$ \'e m\'inimo de $B$ & & \\
\end{tabular}

\vspace{1cm}

A senten\c{c}a ``$b$ \'e o \'unico m\'inimo de $B$'' pode ser representada como ``todo m\'inimo de $B$ \'e igual a $b$''. Com isso temos:

\vspace{1cm}

\begin{tabular}{lcl}
 Hip\'oteses & \hspace{1cm} & Conclusão\\
 		     & & $B$ possui m\'inimo $\rightarrow$ este m\'inimo \'e \'unico\\
 $B$ possui m\'inimo & & este m\'inimo \'e \'unico\\
 $\exists b. b \in B \land b$ \'e o m\'inimo de $B$ & &\\
 $b\in B$    & & $b$ \'e o \'unico m\'inimo de $B$\\
 $b$ \'e m\'inimo de $B$ & & \\
  & & $\forall c. c$ \'e m\'inimo de $B\rightarrow b = c$
\end{tabular}

\vspace{1cm}

Utilizando as t\'ecnicas de prova para quantificadores universais e implica\c{c}\~oes:

\vspace{1cm}

\begin{tabular}{lcl}
 Hip\'oteses & \hspace{1cm} & Conclusão\\
 		     & & $B$ possui m\'inimo $\rightarrow$ este m\'inimo \'e \'unico\\
 $B$ possui m\'inimo & & este m\'inimo \'e \'unico\\
 $\exists b. b \in B \land b$ \'e o m\'inimo de $B$ & &\\
 $b\in B$    & & $b$ \'e o \'unico m\'inimo de $B$\\
 $b$ \'e m\'inimo de $B$ & & \\
  & & $\forall c. c$ \'e m\'inimo de $B\rightarrow b = c$ \\
 $c$ arbitr\'ario & & $b = c$\\
 $c$ \'e m\'inimo de $B$ & & \\
\end{tabular}

\vspace{1cm}

Expandindo as defini\c{c}\~oes de m\'inimo:

\vspace{1cm}

\begin{tabular}{lcl}
 Hip\'oteses & \hspace{1cm} & Conclusão\\
 		     & & $B$ possui m\'inimo $\rightarrow$ este m\'inimo \'e \'unico\\
 $B$ possui m\'inimo & & este m\'inimo \'e \'unico\\
 $\exists b. b \in B \land b$ \'e o m\'inimo de $B$ & &\\
 $b\in B$    & & $b$ \'e o \'unico m\'inimo de $B$\\
 $\forall x. x\in B \rightarrow b R x$ & & \\
  & & $\forall c. c$ \'e m\'inimo de $B\rightarrow b = c$ \\
 $c$ arbitr\'ario & & $b = c$\\
 $\forall y. y\in B \rightarrow c R y$ & & \\
\end{tabular}

\vspace{1cm}

Como $c \in B$ e $\forall x. x\in B \rightarrow b R x$, podemos concluir que $b R c$. De maneira similar, como $b \in B$ e
$\forall y. y\in B \rightarrow c R y$, temos que $c R b$. Uma vez que $R$ \'e uma ordem parcial, temos que $R$ \'e anti-sim\'etrica,
sendo assim, como $bRc$ e $cRb$, temos que $b = c$, como requerido. O texto desta demonstra\c{c}\~ao \'e apresentado a seguir.

\begin{proof}
	Suponha que $b$ \'e o elemento m\'inimo de $B$ e que $c\in B$ arbitr\'ario tamb\'em  \'e m\'inimo de $B$. Como $b$ \'e m\'inimo e
	$c \in B$, temos que $bRc$. Da mesma forma, como $c$ \'e m\'inimo de $B$ e $b\in B$, temos que $cRb$. Como $R$ \'e uma ordem parcial,
	temos que $b = c$. J\'a que $c$ \'e arbitr\'ario, podemos concluir que $b\in B$ \'e o \'unico elemento m\'inimo de $B$. Portanto,
	se $B$ possui um elemento m\'inimo, este \'e \'unico.
\end{proof}

\begin{Theorem}
Suponha $R$ uma ordem parcial sobre $A$, $B\subseteq A$ e que $b \in B$ \'e o m\'inimo de $B$. Ent\~ao $b$ \'e minimal e \'e o \'unico
minimal de $B$.
\end{Theorem}

\vspace{1cm}

\begin{tabular}{lcl}
 Hip\'oteses & \hspace{1cm} & Conclusão\\
 $b$ \'e m\'inimo de $B$ & & $b$ \'e minimal de $B \land $ $b$ \'e o \'unico minimal de $B$
\end{tabular}

Primeiro, vamos provar que $b$ \'e minimal de $B$. Usando as defini\c{c}\~oes de m\'inimo e minimal:

\vspace{1cm}

\begin{tabular}{lcl}
 Hip\'oteses & \hspace{1cm} & Conclusão\\
 $\forall x. x \in B \rightarrow bRx$ & & $\neg \exists y. y \in B \land y R b \land y \neq b$.
\end{tabular}
\vspace{1cm}

Utilizando \'algebra booleana, temos que $\neg \exists y. y \in B \land y R b \land y \neq b = \forall y. y R b \rightarrow y = b$.

\vspace{1cm}

\begin{tabular}{lcl}
 Hip\'oteses & \hspace{1cm} & Conclusão\\
 $\forall x. x \in B \rightarrow bRx$ & & $\forall y. y R b \rightarrow y = b$.
\end{tabular}
\vspace{1cm}

Utilizando as t\'ecnicas de prova para quantificadores universais e implica\c{c}\~oes:

\vspace{1cm}

\begin{tabular}{lcl}
 Hip\'oteses & \hspace{1cm} & Conclusão\\
 $\forall x. x \in B \rightarrow bRx$ & & $\forall y. y R b \rightarrow y = b$\\
 $y$ arbit\'ario & & $y = b$\\
 $yRb$ & & \\
\end{tabular}
\vspace{1cm}

Mas, como $\forall x. x \in B \rightarrow bRx$ e $y \in B$ temos que $bRy$. Uma vez que $R$ \'e anti-sim\'etrica, $yRb$ e $bRy$, temos
que $b = y$, conforme requerido. Com isso, provamos a primeira parte, que $b \in B$ \'e minimal. Agora resta provar que $b$ \'e o \'unico
minimal de $B$.

\vspace{1cm}

\begin{tabular}{lcl}
 Hip\'oteses & \hspace{1cm} & Conclusão\\
 $\forall x. x \in B \rightarrow bRx$ & & $\forall c. c$ \'e minimal de $B \rightarrow b = c$\\
 \end{tabular}
\vspace{1cm}

Utilizando as t\'ecnicas de prova para quantificadores universais, implica\c{c}\~oes e a defini\c{c}\~ao de minimal, temos:

\vspace{1cm}

\begin{tabular}{lcl}
 Hip\'oteses & \hspace{1cm} & Conclusão\\
 $\forall x. x \in B \rightarrow bRx$ & & $b = c$\\
 $c$ arbit\'ario & & \\
 $\neg \exists y. yRc \land y \neq c$ & &
 \end{tabular}
\vspace{1cm}

Por \'algebra booleana, temos que:

\vspace{1cm}

\begin{tabular}{lcl}
 Hip\'oteses & \hspace{1cm} & Conclusão\\
 $\forall x. x \in B \rightarrow bRx$ & & $b = c$\\
 $c$ arbit\'ario & & \\
 $\forall y. yRc \rightarrow y = c$ & &
 \end{tabular}
\vspace{1cm}

Como $c \in B$ e $\forall x. x \in B \rightarrow bRx$, temos que $bRc$. Finalmente, como $bRc$ e  $\forall y. yRc \rightarrow y = c$,
podemos concluir que $b = c$, conforme requerido. O texto desta demonstra\c{c}\~ao \'e apresentado a seguir.

\begin{proof}
	Suponha que $b \in B$ \'e m\'inimo e $y\in B$ arbitr\'ario tal que $yRb$. Como $b$ \'e m\'inimo, temos que $bRy$. Uma vez que
	$yRb$, $bRy$ e $R$ \'e anti-sim\'etrica, temos que $y = b$. Portanto, $b$ \'e um elemento minimal de $B$.
	Suponha $c$ arbitr\'ario tal que $c$ \'e minimal de $B$. Como $b$ \'e m\'inimo, temos que $bRc$. Uma vez que $c$ \'e minimal, temos
	que $cRb$ e j\'a que $R$ \'e uma ordem parcial, temos que $c = b$. Como $c$  \'e arbit\'ario, temos que $b$ \'e o \'unico minimal de $B$.
\end{proof}

\begin{Theorem}
Seja $R$ uma ordem total sobre $A$, $B\subseteq A$, $b\in B$. Se $b$ é
um elemento minimal de $B$, então $b$ é o mínimo de $B$.
\end{Theorem}
Para este teorema, temos a seguinte configuração inicial do rascunho.
\begin{flushleft}
\begin{tabular}{lcl}
 Hip\'oteses & \hspace{1cm} & Conclusão\\
$R$ é uma ordem total & & $b$ é o mínimo de $B$\\
$B \subseteq A$ & & \\
$b$ é minimal de $B$ \\
 \end{tabular}
\end{flushleft}
Expandindo as definições de mínimo e minimal, temos:
\begin{flushleft}
\begin{tabular}{lcl}
 Hip\'oteses & \hspace{1cm} & Conclusão\\
$R$ é uma ordem total & & $\forall x. x \in B \to b R x$\\
$B \subseteq A$ & & \\
$\neg \exists y. y\in B \land yRb \land y \neq b$\\
 \end{tabular}
\end{flushleft}
Utilizando as estratégias de prova para o quantificador universal e
implicação (nesta ordem) temos:
\begin{flushleft}
\begin{tabular}{lcl}
 Hip\'oteses & \hspace{1cm} & Conclusão\\
$R$ é uma ordem total & & $b R x$\\
$B \subseteq A$ & & \\
$\neg \exists y. y\in B \land yRb \land y \neq b$\\
$x$ arbitrário & &\\
$x \in B$ & & \\
 \end{tabular}
\end{flushleft}
É óbvio que  $x = b \lor x \neq b$. Usando este fato, temos:
\begin{flushleft}
\begin{tabular}{lcl}
 Hip\'oteses & \hspace{1cm} & Conclusão\\
$R$ é uma ordem total & & $b R x$\\
$B \subseteq A$ & & \\
$\neg \exists y. y\in B \land yRb \land y \neq b$\\
$x$ arbitrário & &\\
$x \in B$ & & \\
$x = b \lor x \neq b$ & & \\
 \end{tabular}
\end{flushleft}
Como $R$ é reflexiva, para $x = b$, o resultado é imediato. Logo,
vamos considerar que $x \neq b$.
\begin{flushleft}
\begin{tabular}{lcl}
 Hip\'oteses & \hspace{1cm} & Conclusão\\
$R$ é uma ordem total & & $b R x$\\
$B \subseteq A$ & & \\
$\neg \exists y. y\in B \land yRb \land y \neq b$\\
$x$ arbitrário & &\\
$x \in B$ & & \\
$x \neq b$ & & \\
 \end{tabular}
\end{flushleft}
Como $R$ é uma ordem total temos que $xRb \lor bRx$.
\begin{flushleft}
\begin{tabular}{lcl}
 Hip\'oteses & \hspace{1cm} & Conclusão\\
$R$ é uma ordem total & & $b R x$\\
$B \subseteq A$ & & \\
$\neg \exists y. y\in B \land yRb \land y \neq b$\\
$x$ arbitrário & &\\
$x \in B$ & & \\
$x \neq b$ & & \\
$xRb \lor bRx$ & &\\
 \end{tabular}
\end{flushleft}
Agora, dividindo esta prova em casos, temos:
\begin{flushleft}
\begin{tabular}{lcl}
 Hip\'oteses & \hspace{1cm} & Conclusão\\
$R$ é uma ordem total & & $b R x$\\
$B \subseteq A$ & & \\
$\neg \exists y. y\in B \land yRb \land y \neq b$\\
$x$ arbitrário & &\\
$x \in B$ & & \\
$x \neq b$ & & \\
$xRb \lor bRx$ & &\\
Caso $xRb$:   & & $bRx$ \\
Caso $bRx$:  & & $bRx$
 \end{tabular}
\end{flushleft}
O segundo caso é trivial. Para o primeiro, como $xRb$ e $x \neq b$,
temos que $\exists y. y\in B \land y R b \land y \neq b$, o que
contradiz a hipótese $\neg \exists y. y\in B \land yRb \land y \neq
b$, concluindo a demonstração deste teorema.

A seguir apresentamos o texto deste teorema.

\begin{proof}
Suponha que $R$ é uma ordem total sobre $A$, $B\subseteq A$, $b \in
B$. Suponha que $b$ é um minimal de $B$. Suponha $x$
arbitrário. Suponha que $x\in B$. Se $x = b$, como $R$ é reflexiva,
temos que $bRx$. Suponha que $x\neq b$. Uma vez que $R$ é uma ordem
total, temos que $bRx$ ou $xRb$. Considere os casos:
\begin{itemize}
  \item Caso $bRx$: imediato.
  \item Caso $xRb$: Como $xRb$ e $x\neq B$, existe um valor $y \in B$
    tal que $yRb$ e $y\neq B$, o que contradiz a suposição de que $b$
    é minimal de $B$. Logo, $bRx$.
\end{itemize}
Como $x$ é arbitrário, podemos concluir que $b$ é o mínimo de $B$.
\end{proof}

\subsection{Limites Inferiores e Superiores}

Nesta seção estenderemos os conceitos de elementos mínimos e máximos
para o que chamamos de limites inferiores e superiores, conceitos
amplamente utilizados em diversos ramos da computação.

\begin{Definition}[Limite Inferior e Limite Superior]
Seja $R$ uma ordem parcial sobre $A$, $B \subseteq A$ e $a \in
A$. Dizemos que $a$ é um limite inferior de $B$ se
\[
\forall x. x \in B \to a R x
\]
De maneira similar, dizemos que $a$ é um limite superior de $B$ se
\[
\forall x. x \in B \to x R a
\]
\end{Definition}
Note que a única diferença entre limites inferiores (superiores) e
elementos mínimos (máximos) de um conjunto é que os primeiros não
necessariamente devem ser elementos do conjunto em questão.
Além disso, Limites inferiores (superiores) para um conjunto não são únicos,
conforme mostraremos no exemplo seguinte.
\begin{Example}
Considere a seguinte relação $R
=\{(x,y)\in\mathbb{R}\times\mathbb{R}\,|\,x \leq y\}$ e os conjuntos
 $B = \{x \in \mathbb{R}\,|\,x \geq 7\}$ e  $C = \{x \in
 \mathbb{R}\,|\, x \leq 7\}$. Temos que os elementos do intervalo
 $(-\infty,7]$ são todos limites inferiores do conjunto $B$ e o
 intervalo $[7,+\infty)$ é o conjunto de limites superiores de $C$.
\end{Example}
Note que o número de limites inferiores (superiores) para um certo
conjunto pode ser infinito. Logo, faz sentido em falarmos do conjunto
de limites inferiores (superiores) de um conjunto qualquer. No exemplo
anterior, temos que os intervalores $(-\infty,7]$ e $[7,+\infty)$ são
os conjuntos de limites inferiores de $B$ e superiores de $C$,
respectivamente. Em ambos os conjuntos, $(-\infty,7]$ e $[7,+\infty)$,
temos que $7$ é o elemento máximo do conjunto de limites inferiores de
$B$ e, além disso, $7$ também é o elemento mínimo do conjunto de
limites superiores de $C$. A definição seguinte apresenta uma
caracterização formal destes elementos máximos (mínimos) de conjuntos
de limites inferiores (superiores).

\begin{Definition}[Maior Limite Inferior e Menor Limite Superior]
Seja $R$ uma ordem parcial sobre $A$, $B \subseteq A$. Seja $I$ o
conjunto de limites inferiores de $B$ e $S$ o conjunto de limites
superiores de $B$. Se $I$ possui um elemento máximo, dizemos que este
é o maior limite inferior de $B$. Caso $S$ possua um elemento mínimo,
dizemos que este é o menor limite superior de $B$.
\end{Definition}

\begin{Example}
Considere a seguinte relação $R
=\{(x,y)\in\mathbb{R}\times\mathbb{R}\,|\,x \leq y\}$ e os conjuntos
 $B = \{x \in \mathbb{R}\,|\,x \geq 7\}$ e  $C = \{x \in
 \mathbb{R}\,|\, x \leq 7\}$. Temos que os elementos do intervalo
 $(-\infty,7]$ são todos limites inferiores do conjunto $B$ e o
 intervalo $[7,+\infty)$ é o conjunto de limites superiores de
 $C$. Logo, o maior limite inferior de $B$ é $7$ e o menor limite
 superior de $C$ é também $7$.
\end{Example}


\subsection{Exercícios}

\begin{enumerate}
	\item Seja $R$ uma ordem parcial sobre $A$, $B\subseteq A$ e $b\in B$.
	\begin{enumerate}
		\item Prove que se $b$ \'e o elemento m\'inimo de $B$, ent\~ao $b$ \'e o maior limite inferior de $B$.
		\item Prove que se $b$ \'e o elemento m\'aximo de $B$, ent\~ao $b$ \'e o menor limite superior de $B$.
	\end{enumerate}
        \item Suponha que $R$ é uma ordem parcial sobre um conjunto
          $A$ e $B \subseteq A$. Prove que $R \cap (B \times B)$ é uma
          ordem parcial.
        \item Suponha que $R$ é uma ordem parcial sobre um conjunto
          parcial. Prove que $R^{-1}$ é uma ordem parcial sobre $A$.
        \item Suponha que $A$ é um conjunto, $\mathcal{F}\subseteq
          \mathcal{P}(A)$, $\mathcal{F}\neq \emptyset$ e \[R =
          \{(x,y)\times \mathcal{P}(A) \times
          \mathcal{P}(A)\,|\,x\subseteq y\}\]
        \begin{enumerate}
          \item Prove que $\bigcup \mathcal{F}$ é o menor limite
            superior de $\mathcal{F}$ para a ordem parcial $R$.
          \item Prove que $\bigcap \mathcal{F}$ é o maior limite
            inferior de $\mathcal{F}$ para a ordem parcial $R$.
        \end{enumerate}
\end{enumerate}


\section{Relações de Equivalência}

\subsection{Introdução}

Se relações de ordem podem ser consideradas como generalizações dos
conceitos de ``maior'' e ``menor'' para números, relações de
equivalência são consideradas generalizações do conceito de igualdade
em matemática. A definição de relação de equivalência é apresentada a
seguir.

\begin{Definition}[Relação de Equivalência]
Seja $R \subseteq A \times A$ uma relação. Dizemos que $R$ é uma
relação de equivalência se $R$ for reflexiva, transitiva e simétrica.
\end{Definition}
A seguir apresentamos alguns exemplos de relações de equivalência
\footnote{Recomendo \textbf{fortemente} que você tente provar o porquê
cada uma destas relações é uma relação de equivalência.}.
\begin{Example}
São relações de equivalências:
\begin{itemize}
  \item Seja $P$ o conjunto de todas as pessoas do planeta. As
    seguintes relações sobre $P$ são relações de equivalência.
  \begin{itemize}
    \item $R=\{(x,y)\in P \times P\,|\,\text{as pessoas $x$ e $y$
        possuem a mesma idade}\}$.
    \item $R_1=\{(x,y)\in P \times P\,|\,\text{as pessoas $x$ e $y$
        possuem a mesma profissão}\}$.
    \item $R_2=\{(x,y)\in P \times P\,|\,\text{as pessoas $x$ e $y$
        possuem o mesmo modelo de carro}\}$.
  \end{itemize}
  \item Seja $A$ um conjunto qualquer. A seguinte relação sobre
    $\mathcal{P}(A)$ é uma relação de equivalência:
  \begin{itemize}
    \item $R = \{(x,y)\in \mathcal{P}(A)\times \mathcal{P}(A)\,|\, |x|
      = |y|\}$.
  \end{itemize}
\end{itemize}
\end{Example}
 De maneira intuitiva, se $R$ é uma relação de equivalência sobre um
 conjunto $A$ e $xRy$, podemos dizer que $x$ e $y$ são iguais de
 acordo com o critério definido por $R$. Por exemplo, considere a
 relação
    \[R_1=\{(x,y)\in P \times P\,|\,\text{as pessoas $x$ e $y$
        possuem a mesma profissão}\}\]
definida sobre o conjunto de pessoas, $P$. Se Asdrúbal e Criosvaldo
são ambos advogados, então (Asdrúbal, Criosvaldo) $\in R_1$, pois
ambos possuem a mesma profissão e, portanto, Asdrúbal e Criosvaldo
podem ser considerados ``iguais'' de acordo com a relação $R_1$.
Desta forma, podemos considerar que esta relação $R_1$ divide o
conjunto de todas as pessoas em uma família de conjuntos em que cada
um destes conjuntos possui todas as pessoas que exercem uma certa
profissão. Estes conceitos são formalizados pelas definições
seguintes.

\begin{Definition}[Classes de Equivalência]
Seja $R$ uma relação de equivalência sobre um conjunto $A$ e $x \in
A$. A classe de equivalência de $x$ com respeito a $R$, $[x]$, é
definida como:
\[
[x] =\{y \in A\,|\, yRx\}
\]

O conjunto de todas as classes de equivalência de elementos de um
conjunto $A$, $A / R$, é definida como:
\[
A / R =\{[x]\,|\,x \in A\}
\]
\end{Definition}
Como exemplo destes conceitos, considere:
\begin{Example}
Seja $A =\{1,2\}$ e $R
=\{(x,y)\in\mathcal{P}(A)\times\mathcal{P}(A)\,|\,|x| = |y|\}$. Temos
que a relação $R$ é composta pelos seguintes pares:
\[
\begin{array}{lcl}
R &=&\left\{
\begin{array}{lc}
   (\emptyset,\emptyset)&,\\
   (\{1\},\{1\}),(\{1\},\{2\}) &,\\
   (\{2\},\{1\}),(\{2\},\{2\}) &,\\
   (\{1,2\},\{1,2\}) &
\end{array}
            \right\}
\end{array}
\]
Com isso, temos que $[\{1\}]=\{\{1\},\{2\}\} = [\{2\}]$, isto é, a
classe de equivalência de $1$, $[1]$, é igual a classe de equivalência
de $2$, $[2]$.
\end{Example}
Note que, no exemplo anterior, as classes de equivalência de dois
elementos distintos ($\{1\}$ e $\{2\}$) é o mesmo conjunto,
$\{\{1\},\{2\}\}$. Este fato é válido para qualquer relação de
equivalência e é demonstrado pelos teoremas seguintes\footnote{É
  altamente recomendável que você faça os rascunhos destas demonstrações!}.
\begin{Theorem}\label{eqrel1}
Seja $R$ uma relação de equivalência sobre um conjunto $A$ e $x \in
A$. Então $x \in [x]$.
\end{Theorem}
\begin{proof}
Suponha que $R$ é uma relação de equivalência sobre um conjunto $A$ e
que $x \in A$. Como $R$ é uma relação reflexiva e $x \in A$, temos que
$xRx$. Já que $xRx$, pela definição de $[x]$, temos que $x\in
[x]$. Portanto, se $R$ é uma relação de equivalência sobre um conjunto $A$ e
que $x \in A$ então $x \in [x]$.
\end{proof}
\begin{Theorem}\label{eqrel2}
Seja $R$ uma relação de equivalência sobre um conjunto $A$. Então,
para todo $x,y \in A$ temos que $y \in [x]$ se e somente se $[x] = [y]$.
\end{Theorem}
\begin{proof}
Suponha $x,y$ arbitrários. Suponha $x,y \in A$. Suponha que $y \in
[x]$. Suponha $z$ arbitrário.
\begin{enumerate}
  \item[$(\to)$]: Suponha que $z \in [x]$. Pela definição de classe de
    equivalência, temos que $zRx$. Uma vez que $y \in [x]$, temos que
    $yRx$. Como $R$ é simétrica e $yRx$, temos que $xRy$. Como $R$ é
    transitiva, $zRx$ e $xRy$ temos que $zRy$ e, assim, $z\in
    [y]$. Logo, se $z\in [x]$ então $z \in [y]$.
  \item[$(\leftarrow)$]: Suponha que $z\in [y]$. Como $y \in [x]$,
    temos que $yRx$. Já que $z\in [y]$, temos que $zRy$. Uma vez que
    $zRy$ e $yRx$, já que $R$ é transitiva, temos que $zRx$. Logo, se
    $z \in [y]$ então $z \in [x]$.
\end{enumerate}
Como $z$ é arbitrário, temos que $[x] = [y]$. Logo, se $y \in [x]$
então $[x] = [y]$. Assim, se $x,y \in A$ então se $y \in [x]$ então
$[x] = [y]$. Como $x,y$ são arbitrários temos que para todo $x,y \in A$ temos que $y \in [x]$ se e somente se $[x] = [y]$.
\end{proof}

Na próxima seção apresentaremos o conceito de partição de um
conjunto e como este é relacionado ao conceito de classes de
equivalência.

\subsection{Partições e Classes de Equivalência}

Partições são um tipo especial de famílias de conjuntos que possuem
certas propriedades definidas a seguir.

\begin{Definition}[Partição]\label{partdef}
Seja $A$ um conjunto qualquer e $\mathcal{F}\subseteq \mathcal{P}(A)$
uma família de conjuntos. Dizemos que $\mathcal{F}$ é uma partição de
$A$ se as seguintes condições são verdadeiras.
\begin{enumerate}
  \item $\bigcup\mathcal{F} = A$
  \item $\mathcal{F}$ é disjunta par a par. Dizemos que uma família é
    par a par disjunta se $\forall X. \forall Y. X \neq Y \to X \cap Y
    = \emptyset$.
  \item $\forall X. X\in F \to X \neq \emptyset$
\end{enumerate}
\end{Definition}
\begin{Example}
Seja $A = \{1,2,3,4\}$, $\mathcal{F} =\{\{2\},\{1,3\},\{4\}\}$ e
$\mathcal{G} =\{\{2\},\{1,3\},\{1,4\}\}$. Temos que a família
$\mathcal{F}$ é  partição de $A$, pois:
\begin{itemize}
  \item $\bigcup \mathcal{F} = \{2\} \cup \{1,3\} \cup \{4\} =
    \{1,2,3,4\} = A$.
  \item Todos os conjuntos de $\mathcal{F}$ são disjuntos par a par.
  \item Nenhum conjunto de $\mathcal{F}$ é vazio.
\end{itemize}
Por sua vez, a família $\mathcal{G}$ não é uma partição de $A$ pois,
esta não é disjunta par a par.
\end{Example}

De maneira intuitiva, toda relação de equivalência sobre um conjunto
produz uma partição que corresponde aos conjuntos de elementos
``iguais entre si'' de acordo com o critério da relação. Mais ainda,
toda partição de um conjunto gera uma relação de equivalência sobre
este. Estes resultados são apresentados (e demonstrados) nos teoremas
seguintes.

\begin{Theorem}
Seja $R$ uma relação de equivalência sobre $A$. Então, $A / R$ é uma
partição de $A$.
\end{Theorem}
\begin{proof}
\begin{enumerate}
  Para demonstrar que $A / R$ devemos provar as 3 partes da definição
  de partição (definição \ref{partdef}).
  \item Suponha $x$ arbitrário.
    \begin{enumerate}
         \item[$(\to)$] : Suponha que $x \in \bigcup A /
    R$. Como $x\in \bigcup A / R$, temos que existe $y$ tal que $x \in
    [y]$ e $[y] \in A /R$. Se $x \in [y]$, temos que $xRy$ e, logo, $x
    \in A$. Logo, se $x\in \bigcup A / R$ então $x \in A$.
         \item[$(\leftarrow)$]: Suponha que $x \in A$. Como $x \in A$,
           pelo teorema \ref{eqrel1}, temos que $x \in [x]$ e,
           portanto, $x \in \bigcup A / R$. Logo, se $x \in A$ então
           $x \in \bigcup A / R$.
    \end{enumerate}
    Portanto, como $x$ é arbitrário temos que $\bigcup A / R = A$.
  \item Suponha $X$ e $Y$ arbitrários. Suponha que $X,Y \in A / R$. Suponha que $X \cap Y \neq
    \emptyset$. Como $X \cap Y \neq \emptyset$, existe $z$ tal que $z
    \in X$ e $z \in Y$. Como $X \in A / R$, existe $x \in A$ tal que
    $X = [x]$. Como $Y \in A / R$, existe $y \in A$ tal que $Y =
    [y]$. Como $z \in [x]$ e $z \in [y]$, pelo teorema \ref{eqrel2},
    temos que $[x] = [y]$. Logo, se $X \neq Y$ então $X \cap Y =
    \emptyset$. Assim, se $X ,Y\in A / R$ temos que se $X \neq Y$ então $X \cap Y =
    \emptyset$. Como $X,Y$ são arbitrários, temos que $A / R$ é
    disjunta par a par.
  \item Suponha $X$ arbitrário. Suponha que $X \in A /R$. Se $X \in A
    / R$, existe $x \in A$ tal que $X = [x]$. Pelo teorema
    \ref{eqrel1}, temos que se $x \in A$ então $x \in [x]$. Logo, $X
    \neq \emptyset$. Assim, se $X \in A / R$ então $X \neq \emptyset$.
    Como $X$ é arbitrário, temos que o conjunto $A / R$ não possui
    como elemento o conjunto vazio.
\end{enumerate}
Portanto, temos que $A / R$ é uma partição do conjunto $A$.
\end{proof}

\begin{Commentary}
Ao invés de apresentar um rascunho da demonstração anterior, iremos
comentar alguns pontos chaves desta que permitirão o leitor
interessado a construir o rascunho.

Para mostrar que o conjunto \[A / R = \{ [x] \,|\, x\in A\} \] é uma
partição de um conjunto $A$, devemos provar as 3 partes da definição
de partição. Isto é, devemos demonstrar que:
\begin{enumerate}
  \item $\bigcup A / R = A$
  \item $A / R$ é uma família disjunta par a par.
  \item $\forall X. X \in A / R \to X \neq \emptyset$
\end{enumerate}
Para o primeiro item, utilizamos a definição da igualdade de
conjuntos. Logo, devemos provar que
\[
\forall x. x \in \bigcup A / R \leftrightarrow x \in A.
\]
Para mostrar que $x \in \bigcup A / R \to x \in A$, usamos a definição de $x
\in \bigcup A / R$: deve existir um conjunto $X \in A / R$ tal que $x
\in X$. Porém, como $A / R$ é o conjunto de classes de equivalências
de $A$, temos que existe $y \in A$ tal que $X = [y]$ e, assim, temos
que $x \in [y]$ e, portanto, temos que $xRy$ o que nos permite
concluir que $x \in A$, conforme requerido.

A implicação
\[
x\in A \to x \in \bigcup A / R
\]
é uma consequência do teorema \ref{eqrel1}.

No segundo item, devemos demonstrar que $A / R$ é uma família disjunta
par a par, isto é:
\[\forall X . \forall Y. X \in A /R \land Y \in A / R \to X \neq Y
\to X \cap Y = \emptyset \]
Para demonstrar esta fórmula procedemos como usual: supomos $X$ e $Y$
arbitrários e que $X,Y \in A /R$, nos deixando com a seguinte
implicação:
\[
X \neq Y \to X \cap Y \neq \emptyset
\]
Note que a implicação anterior é equivalente a
\[
\neg (X  = Y) \to \neg \exists x. x \in X \cap Y
\]
e, portanto, como ambos os lados desta são negações, isto sugere que
usemos uma prova pela contrapositiva.
Logo, supomos que $X \cap Y \neq \emptyset$ e mostramos que $X =
Y$. O restante da dedução envolve o uso do teorema \ref{eqrel2} e das
definições de classes de equivalência e $A / R$.

Finalmente, a terceira parte da demonstração é uma consequência
imediata do teorema \ref{eqrel1}, que pode ser usado para concluir que
nenhum conjunto de $A / R$ é vazio.
\end{Commentary}
Para demonstrar que toda partição produz uma relação de equivalência,
primeiro
devemos provar alguns
resultados ``auxiliares''. Estes serão enunciados sem demonstração
(prová-los fica como exercício) e os utilizaremos na prova deste
teorema.

\begin{Theorem}\label{eqrel3}
Suponha que $A$ é um conjunto e $\mathcal{F}$ é uma partição de
$A$. Então $R = \bigcup_{X \in \mathcal{F}}(X \times X)$ é uma relação
de equivalência sobre $A$.
\end{Theorem}

\begin{Theorem}\label{eqrel4}
Suponha que $A$ é um conjunto, $\mathcal{F}$ é uma partição de
$A$ e $R = \bigcup_{X \in \mathcal{F}}(X \times X)$. Se $X
\in\mathcal{F}$ e $x \in X$ então $X = [x]$.
\end{Theorem}

Finalmente, mostramos como uma partição produz uma relação de equivalência.

\begin{Theorem}
Suponha que $A$ é um conjunto e que $\mathcal{F}$ é uma partição de
$A$. Então, existe uma relação de equivalência sobre $A$ tal que $A /
R = \mathcal{F}$.
\end{Theorem}
\begin{proof}
Seja $R = \bigcup_{X \in \mathcal{F}}(X \times X)$. Pelo teorema
\ref{eqrel3}, temos que $R$ é uma relação de equivalência.
Para mostrar que $A / R = \mathcal{F}$, suponha $X$ arbitrário.
\begin{enumerate}
  \item[$(\to)$]: Suponha que  $X \in A / R$. Como $X \in A /R$, temos que existe $x \in A$ tal que
$X = [x]$. Como $x \in A$ e $\bigcup \mathcal{F} = A$ (pois
$\mathcal{F}$ é uma partição de $A$), temos que $x \in A$. Logo, $x
\in \bigcup\mathcal{F}$ e, portanto, deve existir $Y \in \mathcal{F}$
tal que $x \in Y$. Mas, pelo teorema \ref{eqrel4}, temos que $[x] =
Y$. Assim, $X \in \mathcal{F}$. Como $X$ é arbitrário, temos que $A /
R \subseteq \mathcal{F}$.
   \item[$(\leftarrow)$]: Suponha $X \in \mathcal{F}$. Como $F$ é uma
     partição, temos que $X \neq \emptyset$ e, portanto, existe $x \in
     X$. Pelo teorema \ref{eqrel4}, temos que $X = [x] \in A /
     R$. Logo, $\mathcal{F}\subseteq A / R$.
\end{enumerate}
Assim $\mathcal{F} = A / R$, conforme requerido.
\end{proof}


\subsection{Exercícios}

\begin{enumerate}
  \item Seja $A = \{1,2,3,4\}$.
  \begin{enumerate}
    \item Apresente duas partições de $A$.
    \item Para cada uma as partições apresentadas por você, apresente
      a relação de equivalência por elas  gerada.
   \end{enumerate}
  \item Suponha que $R$ é uma relação reflexiva e transitiva sobre um
    conjunto $A$. Prove que $R \cap R^{-1}$ é uma relação de equivalência.
  \item Prove os teoremas \ref{eqrel3} e \ref{eqrel4}.
\end{enumerate}


\section{Fechos de Relações}

Um problema comum em se manipular relações é o de que nem sempre as
relações modeladas possuem certas propriedades de interesse
(reflexivas, simétricas e transitivas). Nestes contextos é útil
considerar o que chamamos de fechos de uma relação, que consiste na
menor relação (com respeito a $\subseteq$) de forma a possuir certa
propriedade. A próxima definição apresenta o conceito de fecho de uma
relação

\begin{Definition}[Fecho de uma relação]
Seja $R$ uma relação binária sobre um conjunto $A$. A relação $S$ é o
fecho de $R$ com respeito a propriedade $\rho$ se:
\begin{enumerate}
  \item $S$ possui a propriedade $\rho$.
  \item $R \subseteq S$.
  \item $\forall T. R \subseteq T \land  T$ possui a propriedade $\rho
    \to S \subseteq T$, isto é $S$ é a ``menor'' relação que contém
    $S$ e possui a propriedade $\rho$.
\end{enumerate}
\end{Definition}
As próximas seções apresentam os fechos para três propriedades de relações.

\subsection{Fecho Reflexivo}

Seja $R$ uma relação sobre um conjunto $A$. Se $R$ não é reflexiva é
porque existe $a \in A$ tal que $\neg aRa$. Neste caso, para garantir
que $R$ é reflexiva basta fazer a união desta relação com
\[
i_A =\{(x,x)\,|\,x\in A\}
\]
O fato de que o conjunto $R \cup i_A$ é o fecho reflexivo de uma
relação $R$ é expresso pelo teorema seguinte.

\begin{Theorem}\label{fecho1}
Seja $R$ uma relação sobre um conjunto $A$. Então $R \cup i_{A}$, em
que $i_A=\{(x,x)\,|\,x\in A\}$, é o fecho reflexivo de $R$.
\end{Theorem}

\subsection{Fecho Simétrico}

Seja $R$ uma relação sobre um conjunto $A$. Se $R$ não é simétrica é
porque existem $a,b\in A$ tais que $aRb$ e $\neg bRa$. Para tornar
$R$ uma relação simétrica, basta garantir que todo par presente em $R$
possua o seu respectivo inverso. Isto é, dada uma relação $R$, o seu
fecho simétrico é dado por $R \cup R^{-1}$.

\begin{Theorem}\label{fecho2}
Seja $R$ uma relação sobre um conjunto $A$. Então $R \cup R^{-1}$
é o fecho simétrico de $R$.
\end{Theorem}

\subsection{Fecho Transitivo}

Seja $R$ uma relação sobre um conjunto $A$. Note que para $R$ não ser
uma relação transitiva, temos que devem existir $a,b,c\in A$ tais que
$aRb$, $bRc$ e $\neg aRc$. Para entender melhor o fecho transitivo de
uma relação, é conveniente considerarmos um exemplo simples.

\begin{Example}
Seja $A = \{1,2,3,4\}$ e $R =\{(1,2),(2,3),(3,4)\}$. Temos que $R$ não
é uma relação transitiva, pois:
\begin{itemize}
  \item $1R2$ e $2R3$ mas $\neg 1R3$;
  \item $2R3$ e $3R4$ mas $\neg 2R4$.
\end{itemize}
Assim, podemos tentar tornar $R$ transitiva incluindo os pares $(1,3)$
e $(2,4)$, obtendo a relação $R'$ abaixo:
\[
R' = \{(1,2),(2,3),(3,4),(1,3),(2,4)\}
\]
que ainda não é transitiva, já que $1R3$ e $3R4$ e $\neg
1R4$. Incluindo o par $1R4$, temos
\[
R'' = \{(1,2),(2,3),(3,4),(1,3),(2,4),(1,4)\}
\]
que é transitiva.
Note
que a necessidade do par $1R4$ surgiu quando da inclusão do par
$1R3$. Os pares que faltam a relação $R$ para que esta seja transitiva
são os pares  da forma $aRc$ em que $aRb$ e $bRc$, isto é, pares
$(a,c)\in R \circ R = R^2$. Assim, ao acrescentarmos estes pares,
estamos construindo a relação $R'= R \cup R^2$. De maneira similar, os
pares que faltam a $R'$ devem estar na relação $R' \circ R'$ que é
igual a:
\[
(R\cup R^2)^2 = R^2 \cup R^3 \cup R^3
\]
\end{Example}
Simplificando, temos que os pares que faltam para uma certa relação
$R$ se tornar transitiva são formados pela composição desta
relação. O fecho transitivo de uma relação $R$, $R^*$, é definido
como:
\[
R^* = \bigcup_{i = 1^\infty}R^i
\]
Este resultado é formalizado pelo seguinte teorema.
\begin{Theorem}
Seja $R$ uma relação sobre um conjunto $A$. O fecho transitivo de $R$
é dado por:
\[
R^* = \bigcup_{i = 1^\infty}R^i
\]
\end{Theorem}

Não discutiremos este teorema por este ser demonstrável utilizando uma
técnica que veremos posteriormente: indução matemática.

\subsection{Exercícios}

\begin{enumerate}
  \item Prove os teoremas \ref{fecho1} e \ref{fecho2}.
\end{enumerate}


\section{Notas Bibliográficas}

Relações são a base de diversos conceitos e áreas da ciência da
computação. A fundamentação matemática de bancos de dados relacionais
é feita sobre diversas operações sobre relações. Além disso,
algoritmos para a solução de diversos problemas de inteligência
artificial podem ser modelados como algoritmos sobre grafos, que
essencialmente são relações.

Grande parte deste capítulo é baseado em \cite{Velleman06}.