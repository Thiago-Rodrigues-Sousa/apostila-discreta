\chapter{Recursão}\label{cap10}

\epigraph{Recursive. adj. See RECURSIVE.}{Stan Kelly-Bootie --- The
  Devil's DP Dictionary}

\section{Motivação}

Tanto em matemática, quanto na ciência da computação, diversas
operações são definidas recursivamente, isto é, alguns valores
iniciais para esta operação são dados e os demais são obtidos
aplicando-se uma ou mais regras sucessivamente. Um exemplo de função
recursiva é a definição do fatorial, apresentada abaixo:

\[
\begin{array}{lcl}
0! & = & 1\\
n! & = & n \times (n - 1)!
\end{array}
\]

Como valor inicial, temos que o fatorial de $0$ é $1$ e, demais
valores são obtidos pela segunda equação da definição.

De certa forma, provas por indução possuem uma estrutura similar a
definições recursivas: apresenta-se provas de fatos elementares (casos
base) e usa-se uma regra (passo indutivo) para mostrar que o fato em
questão é válido para elementos diferentes dos considerados nos casos
base. Neste capítulo, veremos como a indução é utilizada para
demonstrar propriedades sobre definições recursivas.

\section{Funções Recursivas}

Existem diversas maneiras de se definir funções. Podemos definir uma
função usando uma expressão que caracteriza a relação entre o domíno e
sua imagem (método usualmente utilizado na matemática). Outra maneira
de se definir uma função é através do uso de composição, que permite a
definição de funções utilizando definições prévias. Esta forma de
definir funções é o mais próximo do que idealmente deve ser feito em
computação. Existe, ainda uma terceira forma de se definir uma função:
utilizando recursão. Como exemplo,
considere a seguinte função $f : \mathbb{N} \to \mathbb{N}$ definida
como
\[
\left\{
\begin{array}{lcl}
  f(0) & = & 1 \\
  f(n) & = & 2n + f(n - 1)\\
\end{array}
\right .
\]
Note que esta definição especifica um valor inicial para $f$, $f(0) =
1$ e os demais valores são obtidos a partir de valores ``anteriores''
desta função. Como exemplo, considere o cálculo de $f(5)$, apresentado
abaixo:
\[
\begin{array}{lc}
f(5) & = \\
2.5 + f(4) & = \\
10 + (2.4 + f(3)) & = \\
10 + (8 + (2.3 + f(2))) & = \\
10 + (8 + (6 + 2.2 + f(1))) & = \\
10 + (8 + (6 + (4 + (2.1 + f(0))))) & = \\
10 + (8 + (6 + (4 + (2 + 1)))) & = \\
31
\end{array}
\]
Apesar de simples compreensão, a função $f$ poderia ser expressa pela
seguinte fórmula:
\[
f(n) = n(n + 1) + 1
\]
Ao montarmos uma pequena tabela de valores para $f$, podemos constatar
que esta função realmente é $n(n + 1) + 1$:
\[
\begin{array}{|c|c|}
  \hline
  n & f(n) \\ \hline
  0 &  1 \\
  1 &  3 \\
  2 &  7 \\
  3 & 13 \\
  4 & 21 \\
  5 & 31 \\
  6 & 43 \\ \hline
\end{array}
\]
Porém, somente construir e verificar esta tabela para alguns valores
não é suficiente para mostrar que $f(n) = n(n+1) + 1$. Para isso,
devemos provar que:
\[
\forall n. n\in\mathbb{N} \to f(n) = n(n+1) + 1
\]
que pode ser provado por indução matemática.

\begin{Theorem}
Seja $f(n)$ uma função definida como:
\[
\left\{
\begin{array}{lcl}
  f(0) & = & 1 \\
  f(n) & = & 2n + f(n - 1)\\
\end{array}
\right .
\]
então $f(n) = n(n+1) + 1$.
\end{Theorem}
\begin{proof}
\verb| |\\
\begin{enumerate}
  \item[\ ]Caso base ($n = 0$): Temos que $f(0) = 1 = 0(0 +1) + 1$,
    conforme requerido.
  \item[\ ]Passo indutivo: Suponha $n\in\mathbb{N}$ arbitrário e que
    $f(n) = n(n+1) + 1$. Temos:
   \[
      \begin{array}{lcl}
      f(n+1) & = \\
      2 (n+1) + f(n) & = & \text{pela definição de }f(n)\\
      2(n+1) + n(n+1) + 1 & = & \text{pela hipótese de indução}\\
      (n+1)[(n+1) + 1] +1
      \end{array}
   \]
   Logo, $f(n+1) = (n+1)[(n+1) + 1] + 1$ conforme requerido.
\end{enumerate}
\end{proof}