\chapter{Combinatória Elementar}\label{cap6}

\epigraph{Counting pairs is the oldest trick in combinatorics... Every
  time we count pairs, we learn something from it.}{Gil Kalai,
  Matemático Israelense.}

\section{Motivação}

A combinatória é o ramo da matemática que estabelece métodos para
determinar o número de elementos de conjuntos finitos. Mas, qual o
interesse de estudantes de computação em se determinar o número de
elementos de um certo conjunto?

Uma das várias aplicações da combinatória em computação é o projeto e análise
de algoritmos. Diversos problemas podem ser caracterizados por
possuírem um conjunto de soluções e, em tais problemas,
estamos interessados na ``melhor'' solução deste conjunto. Algoritmos que resolvem este tipo de problemas
possuem um custo computacional proporcional ao tamanho do conjunto de
soluções para o problema em questão. Logo, para sermos capazes de
entender o comportamento de algoritmos para certos problemas, devemos
ser capazes de determinar, de maneira precisa, o número de elementos
de conjuntos finitos.

Desta forma, o objetivo deste capítulo é o estudo de técnicas para
determinar o número de elementos de conjuntos finitos.

\section{Noções Básicas de Combinatória}

\subsection{Princípio Multiplicativo}

O princípio multiplicativo permite-nos determinar o número de
elementos de um conjunto construído por tarefas\footnote{Alguns
  autores chamam tarefas de ``eventos''.} separadas.

\begin{Definition}[Princípio Multiplicativo]
Suponha que um procedimento possa ser dividido em uma sequência de
duas tarefas. Se houver $n_1$ formas de se fazer a primeira tarefa e
para cada uma destas houver $n_2$ formas de se fazer a segunda tarefa,
então há $n_1\times n_2$ formas de se concluir este procedimento.
\end{Definition}

A seguir apresentamos alguns exemplos que ilustram o uso do princípio
multiplicativo.

\begin{Example}
Uma empresa que possui apenas dois empregados, Asdrúbal e Credirceu,
alugou um andar de um prédio com 12 salas. De quantas maneiras podemos
distribuir as salas deste andar para estes dois funcionários?

\textit{Solução:} O procedimento para atribuir salas deve funcionar da
seguinte maneira: Primeiramente, devemos selecionar uma das 12 salas para
Asdrúbal e, na sequência, escolher uma das 11 salas restantes para
Credirceu. Logo, temos um total de $12 \times 11$ possibilidades para
distribuição de salas entre estes funcionários.
\end{Example}

\begin{Example}
As cadeiras de um auditório devem ser etiquetadas com uma letra e um
número inteiro positivo que não exceda a 100. Qual é o maior número
de cadeiras que podem ser etiquetadas de maneira diferente?

\textit{Solução:} Para etiquetarmos uma cadeira devemos selecionar uma
das 26 letras do alfabeto e, para cada uma destas, devemos selecionar
um número inteiro entre $1$ e $100$. Logo, temos um total de $26
\times 100$ possibilidades.
\end{Example}

\begin{Example}
Há 32 computadores em uma sala de aula. Cada computador tem 24
portas para conexões de dispositivos. Quantas possibilidades existem
nesta sala para a conexão de um certo dispositivo?

\textit{Solução}: Note que o procedimento para conectar o dispositivo
envolve escolher um dos 32 computadores e uma das 24 portas possíveis
do computador escolhido. Logo, temos um total de $32 \times 24$
possibilidades de conexão.
\end{Example}

\begin{Example}
Qual o valor da variável teste após a execução do seguinte trecho de
código?
\begin{algorithm}
  \begin{algorithmic}[0]
      \State{$\text{teste}\leftarrow 0$}
      \For{$i_1 \leftarrow 1$ to $n_1$}
           \For{$i_2 \leftarrow 1$ to $n_2$}
                \State{$\vdots$}
                \For{$i_m \leftarrow 1$ to $n_m$}
                    \State{$\text{teste}\leftarrow \text{teste} + 1$}
               \EndFor
           \EndFor
      \EndFor
  \end{algorithmic}
\end{algorithm}

\textit{Solução}: O valor inicial da variável teste é zero. Cada vez
que uma repetição é feita, o valor de teste é acrescido de uma
unidade. Denomine por $T_i$ a tarefa de executar o $i$-ésimo laço.
Logo, a tarefa de determinar o número de vezes que a variável teste é
incrementada é equivalente a tarefa de contar $T_i$, $1\leq i \leq
m$. Como cada laço é executado $n_i$ ($1\leq i \leq
m$) vezes, temos que o número de repetições deste algoritmo é
\[
\begin{array}{lc}
T_1\times T_2 \times ... \times T_m &  = \\
n_1 \times n_2 \times ... n_m
\end{array}
\]
Portanto, temos que o valor final da variável teste é $n_1\times n_2
\times ... \times n_m$.
\end{Example}

\subsection{Princípio Aditivo}

O princípio aditivo permite-nos determinar o número de
elementos de um conjunto construído por tarefas independentes. Dizemos
que duas tarefas $t_1$ e $t_2$ são independentes se a execução de
$t_1$ não interfere no número de possibilidades de $t_2$, isto é, os
conjuntos relativos a estas tarefas são disjuntos.

\begin{Definition}[Princípio Aditivo]
Se uma tarefa puder ser realizada em uma das $n_1$ formas ou em uma
das $n_2$ formas, em que nenhum dos elementos do conjunto das $n_1$ formas
pertence ao conjunto das $n_2$ formas (isto é, estes conjuntos são
disjuntos), então há $n_1 + n_2$ formas de se realizar a tarefa.
\end{Definition}

A seguir apresentamos exemplos que ilustram a utilização deste
princípio.

\begin{Example}
Suponha que um aluno de mestrado ou um calouro deve ser escolhido para
participar de uma comissão em uma certa universidade. Sabendo-se que
há 48 alunos de mestrado e 60 calouros, de quantas maneiras podemos
escolher o representante desta comissão\footnote{supondo que, evidentemente,
alunos de mestrado não são considerados calouros.}?

\textit{Solução}: Como há 48 estudantes de mestrado e 60 calouros e
estes conjuntos de alunos são disjuntos, pelo princípio aditivo
podemos concluir que o número de maneiras de escolher um representante
é de $60 + 48$.
\end{Example}

\begin{Example}
Diógenes, um estudante de Computação, tem interesse em participar de
um projeto de iniciação científica. Sabendo-se que há 17 projetos de
computação, 5 de engenharia, 2 de ciências básicas e 3 de
administração, quantas possibilidades de escolha de projetos Diógenes
possui? Considere que nenhum projeto pode ser classificado em duas
áreas distintas.

\textit{Solução}: Como nenhum dos grupos de projetos possui
interseção, podemos utilizar o princípio aditivo para concluir que o
número total de possibilidades é de $17 + 5 + 2 + 3$.
\end{Example}

\begin{Example}
Qual o valor da variável teste após a execução do seguinte trecho de
código?
\begin{algorithm}
  \begin{algorithmic}[0]
      \State{$\text{teste}\leftarrow 0$}
      \For{$i_1 \leftarrow 1$ to $n_1$}
      \State{$\text{teste}\leftarrow \text{teste} + 1$}
      \EndFor
      \For{$i_2 \leftarrow 1$ to $n_2$}
      \State{$\text{teste}\leftarrow \text{teste} + 1$}
      \EndFor
      \State{$\vdots$}
      \For{$i_m \leftarrow 1$ to $n_m$}
      \State{$\text{teste}\leftarrow \text{teste} + 1$}
      \EndFor
  \end{algorithmic}
\end{algorithm}

\textit{Solução}: O valor inicial da variável teste é zero. Note que a
variável teste é incrementada a cada iteração de cada um destes
laços. Como o $i$-ésimo laço executa $n_i$ vezes, temos que a variável
teste possuíra o valor final igual a:
\[
\sum_{i=1}^mn_i = n_1 + n_2 + ... +n_m
\]
\end{Example}

\subsection{Utilizando ambos os princípios}

Diversos problemas de combinatória utilizam não apenas um dos dois
princípios apresentados nesta seção. Muitos problemas podem ser
resolvidos utilizando-se ambos os princípios. Nesta seção
apresentaremos alguns exemplos que usam o princípio multiplicativo e
aditivo simultaneamente.

\begin{Example}
Crébison deseja desenvolver um compilador para uma linguagem
experimental em que todos os nomes de variáveis devem possuir três
caracteres, sendo o primeiro uma letra e outros dois quaisquer
caracteres alfanuméricos. Quantos nomes diferentes de variáveis são
possíveis nesta linguagem projetada por Crébison?

\textit{Solução}: Note que para o primeiro caractere, temos um total
de 26 possibilidades (número de letras do alfabeto). Porém, para cada
uma das outras posições temos 36 possibilidades, pois temos 26 letras do alfabeto
e 10 dígitos, usando o princípio aditivo. Mas, a tarefa de se escolher
um nome de variável pode ser dividida nas seguintes tarefas (em
sequência): escolher uma letra, escolher um símbolo alfanumérico e
escolher um símbolo alfanumérico. Logo, pelo princípio multiplicativo,
temos que o número total de possíveis nomes de variáveis nesta
linguagem é $26 \times 36 \times 36$.
\end{Example}

\begin{Example}
Diosbaldo usa o sistema operacional X, que exige que todas as senhas
definidas tenham tamanho 6, 7 ou 8 e que estas possuam pelo menos um
dígito. Os demais caracteres podem ser qualquer símbolo alfa numérico. Quantas senhas
diferentes podem ser construídas para este sistema operacional?

\textit{Solução}: Evidentemente, o número total de senhas, $S$, pode
ser expresso como $S = S_6 + S_7 + S_8$, em que $S_i$ é o conjunto de
senhas contendo $i$ caracteres. Mas, isso nos deixa com a seguinte
questão: como descobrir o valor de cada um dos $S_i$? Note que $S_6$
pode ser obtido a partir de todas as sequências de letras e números de
tamanho 6 removendo aquelas que não possuem dígitos. Isto é:
\[S_6 = 36^6 - 26^6\]
pois, temos $36^6$ possibilidades de sequências de letras e dígitos de
tamanho 6 (note que utilizamos os princípios aditivo e multiplicativo
para isto) e $26^6$ são as sequências formadas apenas por
letras. Logo,  o número de sequências com pelo menos um dígito é
$S_6 = 36^6 - 26^6$. Utilizando um raciocínio similar para $S_7$ e
$S_8$, podemos concluir que $S$ é igual a:
\[
\begin{array}{lc}
S_6 + S_7 + S_8 & =\\
(36^6 - 26^6) + (36^7 - 26^7) + (36^8 - 26^8)
\end{array}
\]
\end{Example}

\subsection{Exercícios}

\begin{enumerate}
   \item Em uma universidade há 30 graduandos em Sistemas de
     Informação e 20 de Engenharia de Computação.
   \begin{enumerate}
       \item De quantas maneiras podemos escolher dois representantes
         de maneira que um seja de Sistemas de Informação e outro de
         Engenharia de Computação?
      \item De quantas maneiras podemos escolher um representante que
        seja de Sistemas de Informação ou de Engenharia de Computação?
   \end{enumerate}
   \item Uma avaliação de múltipla escolha contém 10 questões. Há
     quatro possíveis respostas para cada questão.
     \begin{enumerate}
         \item De quantas maneiras um estudante pode responder às
           questões do exame se este responder a todas elas?
        \item De quantas maneiras o estudante pode responder às
          questões se ele pode deixar questões em branco?
     \end{enumerate}
     \item Quantas sequências de 8 bits são possíveis?
     \item Para os itens a seguir, considere os números inteiros
       positivos entre 100 e 999.
       \begin{enumerate}
         \item Quantos são pares?
         \item Quantos são ímpares?
         \item Quantos são divisíveis por 5?
       \end{enumerate}
     \item Quantas sequências de oito letras do alfabeto da língua
       portuguesa são possíveis
     \begin{enumerate}
         \item se as letras puderem ser repetidas?
         \item se as letras não puderem ser repetidas?
         \item que comecem e terminem com a letra X?
     \end{enumerate}
     \item Um palíndromo é uma sequência que pode ser lida da esquerda
       para direita da mesma forma que da direita para esquerda.
     \begin{enumerate}
       \item Quantos palíndromos existem para sequências de 8 bits?
       \item Quantos palíndromos de tamanho 4 ou 5 podem ser formados
         utilizando-se as 26 letras do alfabeto?
     \end{enumerate}
     \item Utilize o princípio multiplicativo para demonstrar que toda
       fórmula da lógica proposicional que possui $n$ variáveis terá
       uma tabela verdade com $2^n$ linhas.
     \item Utilize o princípio multiplicativo para demonstrar que
       existem $2^{2^n}$ tabelas verdades distintas para proposições
       envolvendo $n$ variáveis.
\end{enumerate}

\section{Princípio da Inclusão-Exclusão}

Um problema muito comum em combinatória é o de determinar o número de
elementos de um conjunto que pode ser expresso como a união de outros.

Suponha que desejamos determinar o número de elementos de um conjunto
$A$ em que $A = B \cup C$. Podemos dizer que o número de elementos de
$A$ é simplesmente a soma dos números de elementos de  $B$
 e $C$, isto é, $|A| = |B| + |C|$?

De modo geral, esta idéia não é válida, conforme apresentado no
exemplo seguinte:

\begin{Example}
Considere os seguintes conjuntos: $B = \{1,2,3\}$, $C=\{3,4,5\}$ e $A
= B\cup C$. Note que $|A| = 5$ e não simplesmente a soma da
cardinalidade de $B$ e $C$, que resultaria em $6$.
\end{Example}

Se $A = B \cup C$, não podemos dizer que $|A| = |B| + |C|$, uma vez
que estaremos ``contando'' duas vezes os elementos de $B\cap C$, já
que estes aparecem tanto em $B$ quanto em $C$. Desta forma, para obter
a cardinalidade correta de $A$, temos que ``somar apenas uma vez'' os
elementos de $B\cap C$. Isso nos leva a seguinte equação:

\[|A| = |B\cup C| = |B| + |C| - |B\cap C|\]

Esta equação é apenas um caso específico do chamado princípio da
inclusão exclusão, que é apresentado na próxima definição.

\begin{Definition}[Princípio da Inclusão-Exclusão]
Sejam $A_1,A_2,...,A_n$ $n$ conjuntos finitos. O número de elementos
de $\bigcup_{i=1}^{n}A_i$ é dado pela seguinte equação:
\[
\begin{array}{lclc}
|\bigcup_{i=1}^nA_i| & = & \sum_{1\leq i \leq ni}|A_i| & - \\
                                &    & \sum_{1\leq i < j \leq
                                  n}|A_i\cap A_j| & + \\
                                &   & \sum_{1\leq i < j < k \leq n}|A_i
                                \cap A_j \cap A_k| & - \\
                                &   & \sum_{1\leq i < j < k < p \leq
                                  n}|A_i \cap A_j \cap A_k \cap A_p| &
                                + \\
                                &   & ... \\
                                &   & + (-1)^{n - 1}|\bigcap_{1\leq
                                  i\leq n} A_i|
\end{array}
\]
\end{Definition}
\begin{Example}
Utilizando o princípio da inclusão-exclusão podemos definir uma
fórmula para o número de elementos da união de 3 conjuntos:
\[
|A\cup B \cup C| = |A| + |B| + |C| - |A\cap B| - |B \cap C| - |A \cap
C| + |A \cap B \cap C|
\]
De maneira similar, podemos definir uma equação para a união de 4
conjuntos:
\[
\begin{array}{lcl}
|A\cup B\cup C \cup D| & = & |A| + |B| + |C| + |D|\\
                                       &    & - |A \cap B| - |A \cap
                                       C| - |A \cap D| - |B \cap C| -
                                       |B \cap D| - |C \cap D| \\
                                       &    & + |A \cap B \cap C| +
                                                    |B \cap C \cap D|
                                                    +
                                                    |A \cap C \cap
                                                    D| + |A \cap B
                                                    \cap D|\\
                                       &    & - |A\cap B \cap C \cap D|

\end{array}
\]
\end{Example}
A seguir apresentamos exemplos que utilizam o princípio da
inclusão-exclusão na prática.
\begin{Example}
Em uma classe de matemática discreta, todo estudante é graduando em
engenharia de computação ou em sistemas de informação ou em ambos. O
número de estudantes matriculados em engenharia de computação é 25, em
sistemas, 13 e em ambos os cursos estão matriculados 8 alunos. Quantos
estudantes há nesta classe?

\textit{Solução}: Considere como $A$ o conjunto de estudantes de
engenharia de computação e $B$ o conjunto de alunos de sistemas de
informação. Logo, pelo princípio de inclusão-exclusão temos:
\[
\begin{array}{lcl}
|A\cup B| & = & |A| + |B| - |A\cap B| \\
|A\cup B| & = & 25 + 13 - 8 \\
|A\cup B| & = & 30\\
\end{array}
\]
\end{Example}
\begin{Example}
Um conjunto de 1232 alunos cursam espanhol, 879 francês e 114
russo. Além disso, 103 cursam espanhol e francês, 23 cursam espanhol e
russo e 14 cursam francês e russo. Se 2092 estudantes cursam pelo
menos uma das três línguas, quantos estudantes cursam as três?

\textit{Solução:}Chame de $S$ o conjunto de estudantes que cursam
espanhol, $F$ os que estudam francês e $R$ os que estudam
russo. Evidentemente, o conjunto de estudantes que cursa as 3 línguas
é $S \cap R \cap F$. Logo, pelo princípio de inclusão-exclusão, temos:
\[
\begin{array}{lcl}
|S\cup R \cup F| & = & |S| + |R| + |F| - |S \cap R| - |R \cap F| - |S
\cap F| +|S\cap R \cap F| \\
       2092            & = & 1232 + 114 + 879 - 23 - 103 - 14 + |S\cap
       R \cap F|\\
|S\cap R\cap F| & = & 7
\end{array}
\]
Assim, temos que o número total de estudantes que cursam as três
línguas é de 7 alunos.
\end{Example}

\subsection{Exercícios}

\begin{enumerate}
  \item Quantos números entre 1 e 3.600 são divisíveis por 3, 5 ou 7?
  \item Quantos números entre 1 e 42.000 não são divisíveis por 2, por
    3 e nem por 7?
  \item Quantos elementos há na união de cinco conjuntos se os
    conjuntos contêm 10.000 elementos cada, cada par de conjuntos tem
    1.000 elementos, cada trio possui 100 elementos, cada quatro
    conjuntos possui 10 elementos em comum e todos os 5 conjuntos possui 1
    elemento em comum?
\end{enumerate}

\section{Princípio da Casa dos Pombos}

Nesta seção apresentaremos um princípio que, apesar de bastante
simples, permite a solução elegante de diversos problemas de
contagem. De maneira simples, este princípio especifica que se mais
de $k$ objetos forem colocados em $k$ caixas, então pelo menos uma
caixa possuirá mais de um objeto. A definição formal deste princípio é
apresentada a seguir.

\begin{Definition}[O Princípio da Casa dos Pombos]
Se $n$ objetos são colocados em $k$ caixas, então há pelo menos uma
caixa com $\lceil \frac{n}{k} \rceil$\footnote{A função teto de $a$,
  $\lceil a \rceil$, retorna o menor inteiro maior que $a$.}.
\end{Definition}

A seguir apresentamos exemplos que aplicam o princípio da casa dos
pombos para a solução de alguns problemas.

\begin{Example}
Considere as seguintes afirmativas:
\begin{enumerate}
  \item Em um grupo de 367 pessoas haverá pelo menos 2 que fazem
    aniversário no mesmo dia do ano.
  \item Em um grupo de 27 palavras haverá pelo menos 2 que iniciam com
    a mesma letra.
\end{enumerate}

\textit{Solução}: Note que como um ano pode possuir 366 dias, temos
que em um conjunto com 367 pessoas, pelo menos duas fazem aniversário
no mesmo dia. Neste caso, consideramos como ``caixas'' o número de
dias do ano e como ``objetos'' cada uma das 367 pessoas, colocando
cada pessoa na caixa correspondente ao dia de seu aniversário. Para a
segunda sentença, temos que o número de caixas é exatamente o número
de letras do alfabeto, 26. Logo, se colocarmos cada palavra na caixa
correspondente à sua letra inicial, temos que pelo princípio da casa
dos pombos, pelo menos duas palavras do conjunto de 27 iniciam com a
mesma letra.
\end{Example}

\begin{Example}
Qual o número mínimo de estudantes em uma classe para
que tenhamos certeza de que pelo menos 6 receberão a mesma nota,
sabendo que apenas 5 notas diferentes são possíveis?

\textit{Solução}: De acordo com o princípio da casa dos pombos, o
número mínimo de estudantes pode ser obtido da seguinte maneira:
Primeiro, consideramos que o número de caixas é o número de notas,
5. Logo, o número de estudantes é o menor valor $n$ que satisfaz a
equação $\lceil \frac{n}{5} \rceil = 6$. Para determinar $n$, podemos
usar o seguinte raciocínio: Se temos 5 caixas, 25 alunos são
suficientes para que todas as caixas possuam 5 alunos. Se somarmos 1,
temos que uma caixa com certeza terá 6 elementos. Logo, o número
mínimo de alunos para garantir que pelo menos 6 tirem a mesma nota
será de 26 alunos.
\end{Example}

\subsection{Exercícios}

\begin{enumerate}
  \item Em um baralho padrão existem 52 cartas distribuídas entre 4
    naipes. Cada naipe possui 13 cartas distintas.
  \begin{enumerate}
    \item Quantas cartas devem ser removidas do baralho para garantir
      que pelo menos 3 cartas do mesmo naipe sejam retiradas?
    \item Quantas cartas devem ser removidas para se garantir que 2
      cartas de mesmo tipo sejam retiradas?
    \item Mostre que em um grupo de 5 números inteiros não
      necessariamente consecutivos, há pelo menos dois com o mesmo
      resto quando divididos por 4.
  \end{enumerate}
  \item Há 38 horários diferentes de aulas em uma universidade. Se há
    677 classes diferentes, quantas salas de aula diferentes são
    necessárias para acomodar esta demanda de ocupação?
  \item Qual o menor número de cabos necessários para conectar oito
    computadores a quatro impressoras de maneira que quatro
    computadores diferentes estejam conectados a quatro impressoras diferentes.
\end{enumerate}

\section{Permutações, Combinações e Arranjos}

Muitos problemas combinatórios podem ser resolvidos encontrando-se o
número de maneiras de se organizar um número específico de elementos
distintos de um conjunto em que a ordem destes elementos pode ser ou
não relevante.

\subsection{Arranjos e Permutações}

Iniciamos esta seção apresentando os conceitos de arranjo e permutação.

\begin{Definition}[Arranjo e Permutação]
Um arranjo de $r$ objetos de um conjunto de $n$ objetos distintos,
$A(n,r)$, é uma coleção ordenada de objetos. Se $n = r$ denominamos o
esta coleção de permutação e representamos este conjunto por $P(n)$.
\end{Definition}

Pela definição anterior, podemos concluir que utilizamos arranjos e
permutações sempre que estamos interessados em determinar o número de
maneiras de se organizar objetos de maneira em que a ordem destes é
relevante. O teorema seguinte deduz uma fórmula para $A(n,r)$
utilizando o princípio multiplicativo.

\begin{Theorem}\label{thmperm}
Seja $n$ um inteiro positivo qualquer e $r$ um inteiro tal que $1 \leq
r \leq n$, então há $A(n,r) = \Pi_{1 \leq i \leq (r - 1)}(n - i)$
permutações de $r$ objetos obtidas a partir de um conjunto contendo
$n$ destes objetos.
\end{Theorem}
\begin{proof}
Suponha $n$ e $r$ arbitrários. Suponha que $1 \leq r \leq n$. Como
desejamos formar coleções de $r$ objetos em que a ordem interessa,
devemos, inicialmente, escolher o primeiro elemento, possuindo $n$
opções. Para o segundo elemento, teremos $n -1$ opções e para o
terceiro $n- 2$. De maneira similar, para o $r$-ésimo objeto teremos
$n - (r - 1) = n -r + 1$ possibilidades de escolha. Logo, pelo
princípio multiplicativo, temos que
\[
    A(n,r) = \displaystyle{\Pi_{1 \leq i \leq (r - 1)}(n - i)}
\]
\end{proof}

Um corolário útil deste teorema é apresentado a seguir:

\begin{Corollary}\label{corperm}
Sejam $n,r$ inteiros arbitrários tais que $1 \leq r \leq n$. Então,
$A(n,r) = \frac{n!}{(n - r)!}$.
\end{Corollary}
\begin{proof}
Suponha $n,r$ inteiros arbitrários e que $1 \leq r \leq n$. Logo, pelo
teorema \ref{thmperm}, temos que $A(n,r) = \Pi_{1 \leq i \leq (r -
  1)}(n - i)$. Mas:
\[
\begin{array}{lc}
\dfrac{n!}{(n - r)!} & = \\
\dfrac{n(n-1)(n-2) ... (n- r + 1) (n - r)!}{(n - r)!} & = \\
n(n - 1)(n-2) ... (n - r + 1) & = \\
\Pi_{1 \leq i \leq (r - 1)}(n - i) & = \\
A(n,r)
\end{array}
\]
conforme requerido.
\end{proof}
Evidentemente, pelo corolário \ref{corperm}, temos que o número de
permutações de $n$ objetos é igual a:
\[
\begin{array}{lcl}
P(n) & = &\\
A(n,n) & = &\{\text{pela def. de }P(n)\}\\
\dfrac{n!}{(n - n)!} & = & \{\text{pelo corolário \ref{corperm}.}\}\\
n!
\end{array}
\]

\begin{Example}
De quantas maneiras podemos selecionar o primeiro, o segundo e o
terceiro colocados em um concurso com 100 inscritos?

\textit{Solução}: Note que neste caso, estamos interessados em
construir coleções de 3 pessoas a partir de um conjunto de 100
inscritos em que a ordem destas 3 pessoas é importante (uma será o
primeiro, outra o segundo, etc). Logo, desejamos determinar o número
de arranjos de 100 elementos escolhidos 3 a 3, que de acordo com o
teorema \ref{thmperm} é:
\[
A(100,3) = \dfrac{100!}{97!}
\]
\end{Example}

\begin{Example}
Quantas permutações das letras ABCDEFGH contêm a sequência ABC?

\textit{Solução}: Visto que as letras ABC devem aparecer na mesma
ordem, estas devem ser consideradas como um ``único bloco'' nas
permutações. Logo, devemos considerar permutações das 5 letras (D, E
,F ,G e H) e o bloco ABC. Logo, o número total de permutações de 6
elementos é $P(6) = 6!$.
\end{Example}

\subsection{Combinações}

Nesta seção abordaremos a noção de combinação, que consiste de uma
coleção não ordenada de objetos. A seguir apresentamos uma definição
precisa deste conceito.

\begin{Definition}[Combinação]
Uma combinação de $r$ objetos obtidos a partir de um conjunto de $n$
elementos distintos é uma coleção não ordenada de $r$
objetos. Combinações de $n$ objetos escolhidos $r$ a $r$ são
usualmente representadas por $C(n,r)$ ou $\binom{n}{r}$.
\end{Definition}

O seguinte teorema demonstra a fórmula para determinar o número de
combinações.

\begin{Theorem}\label{thmcomb}
Suponha $n$ e $r$ inteiros arbitrários tais que $0 \leq r \leq
n$. Então, o número de combinações de $n$ elementos escolhidos $r$ a
$r$ é $C(n,r) = \binom{n}{r} = \dfrac{n!}{r!(n- r)!}$.
\end{Theorem}
\begin{proof}
Note que o número de arranjos de $n$ objetos escolhidos $r$ a $r$ pode
ser obtido a partir da formação das combinações $C(n,r)$ e, então,
pela ordenação de cada uma destas, o que pode ser feito de $P(r)$
maneiras. Logo, temos que:
\[
A(n,r) = C(n,r).P(r)
\]
que implica que
\[
C(n,r) = \dfrac{A(n,r)}{P(r)} = \dfrac{\dfrac{n!}{(n-r)!}}{r!} =
\dfrac{n!}{r!(n - r)!}
\]
\end{proof}

\begin{Example}
Quantas mãos de poker podem ser retiradas de um baralho de 5 cartas a
partir de um baralho com 52 cartas?

\textit{Solução}: Neste caso estamos interessados em conjuntos de 5
cartas, obtidos a partir de um conjunto contendo 52 cartas, em que a
ordem das cartas não é importante. Logo, o total de mãos é dado por $C(52,5)$.
\end{Example}

\begin{Example}
Quantas maneiras há para selecionar cinco jogadores de um time de
basquete com 10 membros para disputar uma partida em outra escola?

\textit{Solução}: A resposta é dada por $C(10,5)$.
\end{Example}

\begin{Example}
  Quantos são os anagramas formados por duas vogais e cinco consoantes
  escolhidas a partir de um conjunto de 18 consoantes e 5 vogais?

\textit{Solução}: A escolha das vogais se dá por $C(5,2)$ e das
consoantes em $C(18,5)$. Porém, como estas podem ocorrer em qualquer
posição, devemos permutar as 7 posições da coleção gerada. Logo, temos
que o número total de anagramas será de: $C(5,2)C(18,3)P(7)$.
\end{Example}

\subsection{Exercícios}

\begin{enumerate}
  \item Demonstre a seguinte equivalência: $C(n,r) = C(n,n - r)$.
   \item Quantos anagramas da palavra UNIFORMES iniciam com consoante
     e terminam em vogal?
  \item Considere a palavra NÚMERO:
  \begin{enumerate}
    \item Quantos são seus anagramas?
    \item Quantos anagramas começam com uma vogal?
    \item Quantos começam com uma vogal e terminam com uma consoante?
  \end{enumerate}
\end{enumerate}

\section{Arranjos e Combinações com Repetições}

Em muitos problemas de combinatória devemos formar coleções em que a
repetição de elementos é permitida. Nesta seção analisaremos
estratégias para lidar com tais problemas combinatórios.

\subsection{Arranjos com Repetições}

Arranjos com repetições são caracterizados por coleções ordenadas em
que repetições de elementos são permitidas. O número de arranjos de
$r$ elementos escolhidos a partir de um conjunto de $n$, em que
repetições são permitidas pode ser obtido facilmente usando a regra do
produto, conforme o teorema seguinte.

\begin{Theorem}
O número de arranjos de $r$ elementos escolhidos a partir
de um conjunto de $n$ elementos distintos é $n^r$.
\end{Theorem}
\begin{proof}
Para cada uma das $r$ posições do arranjo temos um total de $n$
possibilidades de escolha. Logo, pelo princípio multiplicativo, temos
um total de $n^r$ possíveis arranjos.
\end{proof}

\begin{Example}
Qual o total de placas de carros que podem ser construídas utilizando
7 símbolos, sendo os 3 primeiros letras e os 4 últimos dígitos?

\textit{Solução}: Neste caso, temos que o total de placas é dado por
$26^3 \times 10^4$.
\end{Example}

\subsection{Permutações com Repetições}

Para o caso de permutações em que elementos podem ser repetidos,
devemos ter cuidado para evitar a contagem de uma mesma coleção mais
de uma vez.

Por exemplo, se considerarmos quantas permutações possui a palavra ELE
como sendo $3!$ estaremos contando coleções mais de uma vez, pois, não
é possível diferenciar entre as diferentes ocorrências da letra E. O
próximo exemplo ilustra a intuição a ser utilizada para demonstrar a
fórmula para o cálculo do número de permutações com repetições.

\begin{Example}
Quantas permutações diferentes podem ser obtidas para a palavra
SUCCESS?

\textit{Solução}:Como a palavra em questão possui símbolos repetidos,
não podemos simplesmente calcular o número de permutações desta. Para
obtermos o número de sequências diferentes que podem ser construídas
utilizando-se estas letras, primeiro, devemos colocar os S's em 3 das
7 posições disponíveis, o que resulta em $C(7,3)$, o que deixa apenas
4 posições livres. Os dois C's podem ser distribuídos nas 4 posições
restantes de $C(4,2)$, deixando 2 posições disponíveis. Finalmente,
podemos distribuir o U de $C(2,1)$ e a letra $E$ de $C(1,1)$ maneiras.
Consequentemente, temos que pelo princípio multiplicativo temos que o
total de sequências é igual a $C(7,3)C(4,2)C(2,1)C(1,1)$.
\end{Example}
O padrão de raciocínio utilizado no exemplo anterior pode ser
abstraído, conforme apresentado no teorema a seguir.
\begin{Theorem}
O número de permutações diferentes de $n$ objetos, em que há $n_1$
objetos de tipo 1, $n_2$ objetos de tipo $2$, ... e $n_k$ objetos de tipo
$k$, é:
\[
\dfrac{n!}{n_1!n_2!...n_k!}
\]
\end{Theorem}
\begin{proof}
Para determinar o número de permutações, primeiro devemos colocar
$n_1$ objetos de tipo 1 de $C(n,n_1)$ maneiras, deixando $n - n_1$
posições livres. Então os objetos de tipo 2 podem ser colocados de
$C(n - n_1,n_2)$ maneiras, deixando $n - n_1 - n_2$ posições
disponíveis. Continuamos este processo até posicionarmos os objetos de
tipo $n_k$. Após esta etapa, pelo princípio multiplicativo, temos que
o total de permutações diferentes é:
\[
\begin{array}{lc}
C(n,n_1) C(n - n_1, n_2) ... C(n - n_1 - ... - n_{k - 1},n_k) & = \\
\dfrac{n!}{n_1!(n - n_1)!}\dfrac{(n- n_1)!}{n_2!(n - n_1 - n_2)!}
... \dfrac{(n - n_1 - ... -n_k)!}{n_k! 0!} & = \\
\dfrac{n!}{n_1!n_2!...n_k!}
\end{array}
\]
\end{proof}

\subsection{Combinações com Repetições}

Para determinar o número de combinações possíveis com repetições, como
a ordem de elementos é irrelevante em combinações, devemos ser capazes
de ``separar'' grupos de elementos diferentes em uma mesma combinação.
Assim como fizemos para arranjos, primeiramente resolveremos um
exemplo e na sequência, vamos abstrair o padrão de raciocínio
utilizado para determinar uma fórmula para este tipo de combinações.

\begin{Example}
Há quantas maneiras de escolher cinco cédulas em uma caixa que contém
cédulas de $1,2,5,10,20,50,100$ reais? Considere que a ordem de
escolha é irrelevante, que cada nota de um mesmo valor é
indistinguível de outras de mesmo valor e que há pelo menos cinco
cédulas de cada valor.

\textit{Solução}:
Como temos 7 tipos de cédulas e desejamos selecionar 5 cédulas com
repetições permitidas, temos que, de alguma maneira, agrupar os
elementos indistinguíveis que farão parte da combinação gerada. Para
isso, utilizaremos a seguinte analogia: considere a existência de uma
caixa com separações, onde cada compartimento desta caixa seja
destinado a um tipo de cédula. Abaixo, ilustramos esta caixa, como uma
tabela:

\begin{table}[h]
\begin{tabular}{|c|c|c|c|c|c|c|c|}
\hline
1 & 2 & 5 &10 & 20 & 50 & 100 \\ \hline
\end{tabular}
\centering
\end{table}

Note que a caixa apresentada possui 6 separações. Como cada
compartimento é destinado a um tipo de cédula, iremos colocar as
cédulas de um mesmo valor em compartimento correspondente a este tipo
de cédula. Abaixo, apresentamos a configuração da caixa ao
selecionarmos 2 cédulas de 2 reais e 3 de 100 reais.

\begin{table}[h]
\begin{tabular}{|c|c|c|c|c|c|c|c|}
\hline
0 & 2 & 0 & 0 & 0 & 0 & 3 \\ \hline
\end{tabular}
\centering
\end{table}

Veja que anotamos em cada compartimento a quantidade de cédulas que
este possui. Perceba que esta caixa pode ser representada de maneira
abstrata como sequências de ``*'' (para representar cédulas) e ``|''
para representar divisões da caixa. A configuração anterior pode ser
representada pela seguinte sequência de carateres:
``|**|||||***''. Note que usando este truque da caixa, determinamos
uma configuração de escolha das cédulas determinando as sequências não
ordenadas formadas por 5 objetos a partir de um conjunto de 11 objetos
(5 asteriscos e 6 barras). Isso é exatamente $C(11,5)$.
\end{Example}
O teorema seguinte generaliza as discussões anteriores.
\begin{Theorem}
O número de combinações de $r$ elementos, permitindo repetições, a
partir de um conjunto de $n$ elementos é de $C(n + r - 1, r)$.
\end{Theorem}
\begin{proof}
Cada uma das combinações de um conjunto com $n$ elementos, quando a
repetição é permitida, pode ser representada por uma lista de $n -1$
separadores (``| '') e $r$ asteriscos. Então, os $n - 1$ separadores
são utilizados para representar os $n$ elementos diferentes, em que a
$i$-ésima posição é igual a um asterisco toda vez que o $i$-ésimo
elemento do conjunto aparecer na combinação. Logo, o número de
combinações é dado por $C(n - 1 + r,r)$, pois cada lista corresponde a
uma escolha de $r$ posições para colocar os $r$ asteriscos a partir de
$n - 1 + r$ posições que contém $r$ asteriscos e $n - 1$ separadores.
\end{proof}

\subsection{Exercícios}

\begin{enumerate}
  \item Quantas sequências de 6 letras existem (com repetições permitidas)?
  \item Há quantas maneiras de se selecionar três elementos não
    ordenados a partir de um conjunto com cinco elementos, em que
    repetições são permitidas?
  \item Quantas permutações podem ser feitas com a palavra MISSISSIPI?
  \item Quantas permutações podem ser feitas com a palavra
    ABRACADABRA?
  \item Há quantas soluções possíveis para a equação
   \[x_1 + x_2 + x_3 + x_4 = 17\]
   em que $x_1,x_2,x_3$ e $x_4$ são inteiros não negativos?
\end{enumerate}

\section{Notas Bibliográficas}

Noções de combinatória estão presentes em quase todos os livros de
matemática discreta, variando um pouco a terminologia utilizada. Neste
capítulo utilizamos exemplos de \cite{Rosen02}.