\chapter{Combinatória Elementar}\label{cap6}

\epigraph{Counting pairs is the oldest trick in combinatorics... Every
  time we count pairs, we learn something from it.}{Gil Kalai,
  Matemático Israelense.}

\section{Motivação}

A combinatória é o ramo da matemática que estabelece métodos para
determinar o número de elementos de conjuntos finitos. Mas, qual o
interesse de estudantes de computação em se determinar o número de
elementos de um certo conjunto?

Uma das várias aplicações da combinatória em computação é o projeto e análise
de algoritmos. Diversos problemas podem ser caracterizados por
possuírem um conjunto de soluções e, em tais problemas,
estamos interessados na ``melhor'' solução deste conjunto. Algoritmos que resolvem este tipo de problemas
possuem um custo computacional proporcional ao tamanho do conjunto de
soluções para o problema em questão. Logo, para sermos capazes de
entender o comportamento de algoritmos para certos problemas, devemos
ser capazes de determinar, de maneira precisa, o número de elementos
de conjuntos finitos.

Desta forma, o objetivo deste capítulo é o estudo de técnicas para
determinar o número de elementos de conjuntos finitos.

\section{Noções Básicas de Combinatória}

\subsection{Princípio Multiplicativo}

O princípio multiplicativo permite-nos determinar o número de
elementos de um conjunto construído por tarefas\footnote{Alguns
  autores chamam tarefas de ``eventos''.} separadas.

\begin{Definition}[Princípio Multiplicativo]
Suponha que um procedimento possa ser dividido em uma sequência de
duas tarefas. Se houver $n_1$ formas de se fazer a primeira tarefa e
para cada uma destas houver $n_2$ formas de se fazer a segunda tarefa,
então há $n_1\times n_2$ formas de se concluir este procedimento.
\end{Definition}

A seguir apresentamos alguns exemplos que ilustram o uso do princípio
multiplicativo.

\begin{Example}
Uma empresa que possui apenas dois empregados, Asdrúbal e Credirceu,
alugou um andar de um prédio com 12 salas. De quantas maneiras podemos
distribuir as salas deste andar para estes dois funcionários?

\textit{Solução:} O procedimento para atribuir salas deve funcionar da
seguinte maneira: Primeiramente, devemos selecionar uma das 12 salas para
Asdrúbal e, na sequência, escolher uma das 11 salas restantes para
Credirceu. Logo, temos um total de $12 \times 11$ possibilidades para
distribuição de salas entre estes funcionários.
\end{Example}

\begin{Example}
As cadeiras de um auditório devem ser etiquetadas com uma letra e um
números inteiro positivo que não exceda a 100. Qual é o maior número
de cadeiras que podem ser etiquetadas de maneira diferente?

\textit{Solução:} Para etiquetarmos uma cadeira devemos selecionar uma
das 26 letras do alfabeto e, para cada uma destas, devemos selecionar
um número inteiro entre $1$ e $100$. Logo, temos um total de $26
\times 100$ possibilidades.
\end{Example}

\begin{Example}
Há 32 computadores em uma sala de aula. Cada computador tem 24
portas para conexões de dispositivos. Quantas possibilidades existem
nesta sala para a conexão de um certo dispositivo?

\textit{Solução}: Note que o procedimento para conectar o dispositivo
envolve escolher um dos 32 computadores e uma das 24 portas possíveis
do computador escolhido. Logo, temos um total de $32 \times 24$
possibilidades de conexão.
\end{Example}

\begin{Example}
Qual o valor da variável teste após a execução do seguinte trecho de
código?
\begin{algorithm}
  \begin{algorithmic}[0]
      \State{$\text{teste}\leftarrow 0$}
      \For{$i_1 \leftarrow 1$ to $n_1$}
           \For{$i_2 \leftarrow 1$ to $n_2$}
                \State{$\vdots$}
                \For{$i_m \leftarrow 1$ to $n_m$}
                    \State{$\text{teste}\leftarrow \text{teste} + 1$}
               \EndFor
           \EndFor
      \EndFor
  \end{algorithmic}
\end{algorithm}

\textit{Solução}: O valor inicial da variável teste é zero. Cada vez
que uma repetição é feita, o valor de teste é acrescido de uma
unidade. Denomine por $T_i$ a tarefa de executar o $i$-ésimo laço.
Logo, a tarefa de determinar o número de vezes que a variável teste é
incrementada é equivalente a tarefa de contar $T_i$, $1\leq i \leq
m$. Como cada laço é executado $n_i$ ($1\leq i \leq
m$) vezes, temos que o número de repetições deste algoritmo é
\[
\begin{array}{lc}
T_1\times T_2 \times ... \times T_m &  = \\
n_1 \times n_2 \times ... n_m
\end{array}
\]
Portanto, temos que o valor final da variável teste é $n_1\times n_2
\times ... \times n_m$.
\end{Example}

\subsection{Princípio Aditivo}

O princípio aditivo permite-nos determinar o número de
elementos de um conjunto construído por tarefas independentes. Dizemos
que duas tarefas $t_1$ e $t_2$ são independentes se a execução de
$t_1$ não interfere no número de possibilidades de $t_2$, isto é, os
conjuntos relativos a estas tarefas são disjuntos.

\begin{Definition}[Princípio Aditivo]
Se uma tarefa puder ser realizada em uma das $n_1$ formas ou em uma
das $n_2$ formas, em que nenhum dos elementos do conjunto $n_1$ formas
pertence ao conjunto $n_2$
\end{Definition}

A seguir apresentamos exemplos que ilustram a utilização deste princípio.

\section{Princípio da Inclusão e Exclusão}

\section{Princípio da Casa dos Pombos}

\section{Permutações e Combinações}