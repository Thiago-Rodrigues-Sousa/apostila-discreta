\chapter{Combinatória Elementar}\label{cap6}

\epigraph{Counting pairs is the oldest trick in combinatorics... Every
  time we count pairs, we learn something from it.}{Gil Kalai,
  Matemático Israelense.}

\section{Motivação}

A combinatória é o ramo da matemática que estabelece métodos para
determinar o número de elementos de conjuntos finitos. Mas, qual o
interesse de estudantes de computação em se determinar o número de
elementos de um certo conjunto?

Uma das várias aplicações da combinatória em computação é o projeto e análise
de algoritmos. Diversos problemas podem ser caracterizados por
possuírem um conjunto de soluções e, em tais problemas,
estamos interessados na ``melhor'' solução deste conjunto. Algoritmos que resolvem este tipo de problemas
possuem um custo computacional proporcional ao tamanho do conjunto de
soluções para o problema em questão. Logo, para sermos capazes de
entender o comportamento de algoritmos para certos problemas, devemos
ser capazes de determinar, de maneira precisa, o número de elementos
de conjuntos finitos.

Desta forma, o objetivo deste capítulo é o estudo de técnicas para
determinar o número de elementos de conjuntos finitos.

\section{Noções Básicas de Combinatória}

\subsection{Princípio Multiplicativo}

O princípio multiplicativo permite-nos determinar o número de
elementos de um conjunto construído por tarefas\footnote{Alguns
  autores chamam tarefas de ``eventos''.} separadas.

\begin{Definition}[Princípio Multiplicativo]
Suponha que um procedimento possa ser dividido em uma sequência de
duas tarefas. Se houver $n_1$ formas de se fazer a primeira tarefa e
para cada uma destas houver $n_2$ formas de se fazer a segunda tarefa,
então há $n_1\times n_2$ formas de se concluir este procedimento.
\end{Definition}

A seguir apresentamos alguns exemplos que ilustram o uso do princípio
multiplicativo.

\begin{Example}
Uma empresa que possui apenas dois empregados, Asdrúbal e Credirceu,
alugou um andar de um prédio com 12 salas. De quantas maneiras podemos
distribuir as salas deste andar para estes dois funcionários?

\textit{Solução:} O procedimento para atribuir salas deve funcionar da
seguinte maneira: Primeiramente, devemos selecionar uma das 12 salas para
Asdrúbal e, na sequência, escolher uma das 11 salas restantes para
Credirceu. Logo, temos um total de $12 \times 11$ possibilidades para
distribuição de salas entre estes funcionários.
\end{Example}

\begin{Example}
As cadeiras de um auditório devem ser etiquetadas com uma letra e um
números inteiro positivo que não exceda a 100. Qual é o maior número
de cadeiras que podem ser etiquetadas de maneira diferente?

\textit{Solução:} Para etiquetarmos uma cadeira devemos selecionar uma
das 26 letras do alfabeto e, para cada uma destas, devemos selecionar
um número inteiro entre $1$ e $100$. Logo, temos um total de $26
\times 100$ possibilidades.
\end{Example}

\begin{Example}
Há 32 computadores em uma sala de aula. Cada computador tem 24
portas para conexões de dispositivos. Quantas possibilidades existem
nesta sala para a conexão de um certo dispositivo?

\textit{Solução}: Note que o procedimento para conectar o dispositivo
envolve escolher um dos 32 computadores e uma das 24 portas possíveis
do computador escolhido. Logo, temos um total de $32 \times 24$
possibilidades de conexão.
\end{Example}

\begin{Example}
Qual o valor da variável teste após a execução do seguinte trecho de
código?
\begin{algorithm}
  \begin{algorithmic}[0]
      \State{$\text{teste}\leftarrow 0$}
      \For{$i_1 \leftarrow 1$ to $n_1$}
           \For{$i_2 \leftarrow 1$ to $n_2$}
                \State{$\vdots$}
                \For{$i_m \leftarrow 1$ to $n_m$}
                    \State{$\text{teste}\leftarrow \text{teste} + 1$}
               \EndFor
           \EndFor
      \EndFor
  \end{algorithmic}
\end{algorithm}

\textit{Solução}: O valor inicial da variável teste é zero. Cada vez
que uma repetição é feita, o valor de teste é acrescido de uma
unidade. Denomine por $T_i$ a tarefa de executar o $i$-ésimo laço.
Logo, a tarefa de determinar o número de vezes que a variável teste é
incrementada é equivalente a tarefa de contar $T_i$, $1\leq i \leq
m$. Como cada laço é executado $n_i$ ($1\leq i \leq
m$) vezes, temos que o número de repetições deste algoritmo é
\[
\begin{array}{lc}
T_1\times T_2 \times ... \times T_m &  = \\
n_1 \times n_2 \times ... n_m
\end{array}
\]
Portanto, temos que o valor final da variável teste é $n_1\times n_2
\times ... \times n_m$.
\end{Example}

\subsection{Princípio Aditivo}

O princípio aditivo permite-nos determinar o número de
elementos de um conjunto construído por tarefas independentes. Dizemos
que duas tarefas $t_1$ e $t_2$ são independentes se a execução de
$t_1$ não interfere no número de possibilidades de $t_2$, isto é, os
conjuntos relativos a estas tarefas são disjuntos.

\begin{Definition}[Princípio Aditivo]
Se uma tarefa puder ser realizada em uma das $n_1$ formas ou em uma
das $n_2$ formas, em que nenhum dos elementos do conjunto das $n_1$ formas
pertence ao conjunto das $n_2$ formas (isto é, estes conjuntos são
disjuntos), então há $n_1 + n_2$ formas de se realizar a tarefa.
\end{Definition}

A seguir apresentamos exemplos que ilustram a utilização deste
princípio.

\begin{Example}
Suponha que um aluno de mestrado ou um calouro deve ser escolhido para
participar de uma comissão em uma certa universidade. Sabendo-se que
há 48 alunos de mestrado e 60 calouros, de quantas maneiras podemos
escolher o representante desta comissão\footnote{supondo que, evidentemente,
alunos de mestrado não são considerados calouros.}?

\textit{Solução}: Como há 48 estudantes de mestrado e 60 calouros e
estes conjuntos de alunos são disjuntos, pelo princípio aditivo
podemos concluir que o número de maneira de escolher um representante
é de $60 + 48$.
\end{Example}

\begin{Example}
Diógenes, um estudante de Computação, tem interesse em participar de
um projeto de iniciação científica. Sabendo-se que há 17 projetos de
computação, 5 de engenharia, 2 de ciências básicas e 3 de
administração, quantas possibilidades de escolha de projetos Diógenes
possui? Considere que nenhum projeto pode ser classificado em duas
áreas distintas.

\textit{Solução}: Como nenhum dos grupos de projetos possui
interseção, podemos utilizar o princípio aditivo para concluir que o
número total de possibilidades é de $17 + 5 + 2 + 3$.
\end{Example}

\begin{Example}
Qual o valor da variável teste após a execução do seguinte trecho de
código?
\begin{algorithm}
  \begin{algorithmic}[0]
      \State{$\text{teste}\leftarrow 0$}
      \For{$i_1 \leftarrow 1$ to $n_1$}
      \State{$\text{teste}\leftarrow \text{teste} + 1$}
      \EndFor
      \For{$i_2 \leftarrow 1$ to $n_2$}
      \State{$\text{teste}\leftarrow \text{teste} + 1$}
      \EndFor
      \State{$\vdots$}
      \For{$i_m \leftarrow 1$ to $n_m$}
      \State{$\text{teste}\leftarrow \text{teste} + 1$}
      \EndFor
  \end{algorithmic}
\end{algorithm}

\textit{Solução}: O valor inicial da variável teste é zero. Note que a
variável teste é incrementada a cada iteração de cada um destes
laços. Como o $i$-ésimo laço executa $n_i$ vezes, temos que a variável
teste possuíra o valor final igual a:
\[
\sum_{i=1}^mn_i = n_1 + n_2 + ... +n_m
\]
\end{Example}

\subsection{Utilizando ambos os princípios}

Diversos problemas de combinatória utilizam não apenas um dos dois
princípios apresentados nesta seção. Muitos problemas podem ser
resolvidos utilizando-se ambos os princípios. Nesta seção
apresentaremos alguns exemplos que usam o princípio multiplicativo e
aditivo simultaneamente.

\begin{Example}
Crébison deseja desenvolver um compilador para uma linguagem
experimental em que todos os nomes de variáveis devem possuir três
caracteres, sendo o primeiro uma letra e outros dois quaisquer
caracteres alfanuméricos. Quantos nomes diferentes de variáveis são
possíveis nesta linguagem projetada por Crébison?

\textit{Solução}: Note que para o primeiro caractere, temos um total
de 26 possibilidades (número de letras do alfabeto). Porém, para cada
uma das outras posições temos 36 possibilidades, pois temos 26 letras do alfabeto
e 10 dígitos, usando o princípio aditivo. Mas, a tarefa de se escolher
um nome de variável pode ser dividida nas seguintes tarefas (em
sequência): escolher uma letra, escolher um símbolo alfanumérico e
escolher um símbolo alfanumérico. Logo, pelo princípio multiplicativo,
temos que o número total de possíveis nomes de variáveis nesta
linguagem é $26 \times 36 \times 36$.
\end{Example}

\begin{Example}
Diosbaldo usa o sistema operacional X, que exige que todas as senhas
definidas tenham tamanho 6, 7 ou 8 e que estas possuam pelo menos um
dígito (demais símbolos podem ser letras ou números). Quantas senhas
diferentes podem ser construídas para este sistema operacional?

\textit{Solução}: Evidentemente, o número total de senhas, $S$, pode
ser expresso como $S = S_6 + S_7 + S_8$, em que $S_i$ é o conjunto de
senhas contendo $i$ caracteres. Mas, isso nos deixa com a seguinte
questão: como descobrir o valor de cada um dos $S_i$? Note que $S_6$
pode ser obtido a partir de todas as sequências de letras e números de
tamanho 6 removendo aquelas que não possuem dígitos. Isto é:
\[S_6 = 36^6 - 26^6\]
pois, temos $36^6$ possibilidades de sequências de letras e dígitos de
tamanho 6 (note que utilizamos os princípios aditivo e multiplicativo
para isto) e $26^6$ são as sequências formadas apenas por
letras. Logo,  o número de sequências com pelo menos um dígito é
$S_6 = 36^6 - 26^6$. Utilizando um raciocínio similar para $S_7$ e
$S_8$, podemos concluir que $S$ é igual a:
\[
\begin{array}{lc}
S_6 + S_7 + S_8 & =\\
(36^6 - 26^6) + (36^7 - 26^7) + (36^8 - 26^8)
\end{array}
\]
\end{Example}

\subsection{Exercícios}

\begin{enumerate}
   \item Em uma universidade há 30 graduandos em Sistemas de
     Informação e 20 de Engenharia de Computação.
   \begin{enumerate}
       \item De quantas maneiras podemos escolher dois representantes
         de maneira que um seja de Sistemas de Informação e outro de
         Engenharia de Computação?
      \item De quantas maneiras podemos escolher um representante que
        seja de Sistemas de Informação ou de Engenharia de Computação?
   \end{enumerate}
   \item Uma avaliação de múltipla escolha contém 10 questões. Há
     quatro possíveis respostas para cada questão.
     \begin{enumerate}
         \item De quantas maneiras um estudante pode responder às
           questões do exame se este responder a todas elas?
        \item De quantas maneiras o estudante pode responder às
          questões se ele pode deixar questões em branco?
     \end{enumerate}
     \item Quantas sequências de 8 bits são possíveis?
     \item Para os itens a seguir, considere os números inteiros
       positivos entre 100 e 999.
       \begin{enumerate}
         \item Quantos são pares?
         \item Quantos são ímpares?
         \item Quantos são divisíveis por 5?
       \end{enumerate}
\end{enumerate}

\section{Princípio da Inclusão e Exclusão}

\section{Princípio da Casa dos Pombos}

\section{Permutações, Combinações e Arranjos}

\section{Notas Bibliográficas}