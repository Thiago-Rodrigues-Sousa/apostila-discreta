\chapter{L\'ogica Proposicional}

\begin{chapquote}
`Uma vez pessoa me disse: `Me conven\c{c}a de que l\'ogica \'e \'util.' --- `Voc\^e deseja que eu prove isso?', respondi.
--- `Sim.', ele respondeu. --- `Ent\~ao, eu devo produzir um argumento que comprove este fato?' --- Ele concordou. --- `Ent\~ao,
como voc\^e saber\'a que eu n\~ao produzi um argumento falacioso?' --- Ele nada disse. --- `Veja, voc\^e acaba de se convencer
de que a l\'ogica \'e necess\'aria, uma vez que sem ela voc\^e n\~ao \'e capaz de saber se esta \'e necess\'aria ou n\~ao.'
Epicteto 1, Discursos 3 II xxv.
\end{chapquote}

\section{Motiva\c{c}\~ao}

A l\'ogica prov\^e um ferramental para o racioc\'inio sobre matem\'atica, algoritmos
e circuitos digitais. A aplicabilidade da l\'ogica permeia diversas \'areas da computa\c{c}\~ao,
citaremos aqui apenas alguns exemplos:

\begin{itemize}
  \item \textbf{Engenharia de Software}: considera-se uma boa pr\'atica especificar um sistema antes 
        de iniciar a sua codifica\c{c}\~ao. Uma das v\'arias t\'ecnicas de especifica\c{c}\~ao de 
        software \'e o uso de l\'ogica.
  \item \textbf{Aplica\c{c}\~oes de Miss\~ao Cr\'itica}: dizemos que uma determinada aplica\c{c}\~ao 
        \'e de miss\~ao cr\'itica se essa est\'a relacionada a algum risco (de vida, elevados preju\'izos financeiros, etc.).
        Em tais aplica\c{c}\~oes, a utiliza\c{c}\~ao de testes para garantir o funcionamento adequado 
        n\~ao \'e suficiente. O que espera-se \'e uma prova de corretude do programa em quest\~ao, isso \'e uma
        demonstra\c{c}\~ao de que o programa comporta-se de acordo com sua especifica\c{c}\~ao em todas as
        situa\c{c}\~oes poss\'iveis. A l\'ogica \'e a fundamenta\c{c}\~ao matem\'atica de demonstra\c{c}\~oes de
        corre\c{c}\~ao de programas.
   \item \textbf{Recupera\c{c}\~ao de informa\c{c}\~ao}: em m\'aquinas de busca para Web, utiliza-se l\'ogica para
         especificar como propriedades que classificam uma determinada p\'agina como relevante ou n\~ao com base
         em seu conte\'udo.
   \item \textbf{Circuitos Digitais e Arquitetura de Computadores:} l\'ogica \'e a linguagem utilizada para descrever
         sinais produzidos e recebidos como entrada por componentes eletr\^onicos. Um problema comum no projeto de
         circuitos eletr\^onicos \'e determinar uma vers\~ao equivalente, por\'em mais eficiente, de um circuito l\'ogico.
         T\'ecnicas para solu\c{c}\~ao desse problema s\~ao baseadas em algoritmos eficientes para o processamento de
         f\'ormulas da l\'ogica.
   \item \textbf{Bancos de dados}: um recurso fundamental de qualquer sistema gerenciador de bancos de dados \'e uma linguagem
         simples e expressiva para recuperar informa\c{c}\~oes nele armazenadas. A utiliza\c{c}\~ao de recursos baseados em 
         l\'ogica \'e a chave para a expressividade de linguagens para consultas a bancos de dados.       
\end{itemize}

Al\'em das \'areas citadas anteriormente, a l\'ogica \'e fundamental no estudo e no projeto de linguagens de programa\c{c}\~ao
e da teoria de computabilidade.

Neste cap\'itulo, discutiremos as dificuldades presentes na utiliza\c{c}\~ao do Portugu\^es para expressar racioc\'inio 
l\'ogico e como contorn\'a-las utilizando l\'ogica formal. Existem diversos tipos de l\'ogicas formais, cada uma com 
uma aplica\c{c}\~ao espec\'ifica. Inicialmente consideraremos uma l\'ogica bem simples chamada de l\'ogica proposicional.
Primeiramente, iremos analisar a sintaxe da linguagem da l\'ogica proposicional e depois consideraremos tr\^es sistemas 
matem\'aticos para racioc\'inio sobre f\'ormulas da l\'ogica proposicional: tabelas verdade, dedu\c{c}\~ao natural e \'algebra 
Booleana.

\emph{Tabelas verdade} definem o significado dos conectivos l\'ogicos e como eles podem ser utilizados para calcular os 
valores de express\~oes e provar que duas proposi\c{c}\~oes s\~ao logicamente equivalentes. Como tabelas verdade
expressam diretamente o siginificado de prpoposi\c{c}\~oes, dizemos que essas s\~ao uma abordagem baseada em sem\^antica para l\'ogica.
Tabelas verdade s\~ao de simples entendimento, por\'em n\~ao possuem s\~ao \'uteis na solu\c{c}\~ao de problemas reais devido ao seu tamanho.

\emph{Dedu\c{c}\~ao Natural} \'e uma formaliza\c{c}\~ao de princ\'ipios b\'asicos de racioc\'inio l\'ogico utilizado no cotidiano.
A dedu\c{c}\~ao natural prov\^e um conjunto de regras de infer\^encia que especificam exatamente quais fatos podem ser deduzidos
a partir de um conjunto de fatos dados. Em dedu\c{c}\~ao natural n\~ao h\'a a no\c{c}\~ao de `valor l\'ogico` de proposi\c{c}\~oes;
j\'a que tudo no sistema est\'a encapsulado em suas regras de infer\^encia. Conforme veremos posteriormente, essas regras s\~ao baseadas
na estrutura das proposi\c{c}\~oes envolvidas, a dedu\c{c}\~ao natural \'e uma abordagem puramente sint\'atica para a l\'ogica. Diversas
t\'ecnicas utilizadas em pesquisas na \'areas de linguagens de programa\c{c}\~ao s\~ao baseadas em sistemas l\'ogicos que s\~ao de alguma
maneira relacionados \`a dedu\c{c}\~ao natural.

\emph{\'Algebra Booleana} \'e uma abordagem para formaliza\c{c}\~ao da l\'ogica baseada em um conjunto de equa\c{c}\~oes --- as leis
da \'algebra Booleana --- para especificar que certas proposi\c{c}\~oes s\~ao iguais a outras. A \'algebra Booleana \'e uma abordagem
axiom\'atica, similar a \'algebra elementar e geometria, pois prov\^e um conjunto de leis para manipular proposi\c{c}\~oes. T\'ecnicas
alg\'ebricas para a l\'ogica s\~ao fundamentais para o projeto de circuitos digitais.

\section{Introdu\c{c}\~ao \`a L\'ogica Formal}\label{cap1:sec1}

A l\'ogica formal foi inicialmente concebida na gr\'ecia antiga onde fil\'osofos desejavam ser capazes de analisar argumentos 
em linguagem natural. Os gregos eram fascinados pela id\'eia de que alguns argumentos eram sempre verdadeiros e outros sempre
falsos. Por\'em, eles rapidamente perceberam que o racioc\'inio l\'ogico \'e dif\'icil de ser analisado usando linguagens naturais
como o Grego (ou o Portugu\^es!). Isso se deve principalmente devido \`as \emph{ambiguidades} inerentes \`as linguagens naturais.
Uma das maneiras de se evitar essas dificuldades \'e o uso de vari\'aveis que denominaremos \emph{vari\'aveis proposicionais}. 

Suponha que um conhecido lhe diga `O dia est\'a ensolarado e estou feliz`. Aparentemente essa frase possui interpreta\c{c}\~ao
\'obvia, mas ao observ\'a-la com cuidado percebe-se que o significado dessa n\~ao \'e t\~ao evidente. Talvez essa pessoa goste
de dias ensolarados e fica contente quando esse fato ocorre. Note que existe uma conex\~ao entre as duas partes da senten\c{c}a,
dessa forma, a palavra `e` presente na frase `O dia est\'a ensolarado e estou feliz` significa `e, portanto`. Por\'em, essa an\'alise
depende de nossa experi\^encia em relacionar o clima com a felicidade das pessoas. Considere agora o seguinte exemplo: 
`Gatos s\~ao peludos e elefantes pesados`. Essa senten\c{c}a possui a mesma estrutura do exemplo anterior, mas ningu\'em ir\'a tentar
relacionar o peso de elefantes com a quantidade de pelos de gatos. Neste caso, a palavra `e` significa `e, tamb\'em`. Pode-se perceber que a
palavra `e` possui diversos significados sutis, e escolhemos o significado apropriado usando nosso conhecimento do mundo \`a nossa volta.
Perceba que as duas simples frases de exemplo consideradas ilustram as dificuldades de interpreta\c{c}\~ao que podem surgir ao se utilizar
uma linguagem natural. As dificuldades em se dar um significado preciso a frases em linguagem natural n\~ao se restringem a somente como 
intepretar a palavra `e`. O estudo preciso da sem\^antica de senten\c{c}as expressas em linguagem natural \'e objeto de estudo da lingu\'istica
e da filosofia.

Ao inv\'es de tentarmos o imposs\'ivel --- expressar, de maneira precisa,  racioc\'inio l\'ogico em linguagem natural  --- n\'os iremos separar
a estrutura l\'ogica de um argumento de todas as conota\c{c}\~oes da l\'ingua portuguesa. Faremos isso utilizando \textbf{proposi\c{c}\~oes}, que
s\~ao definidas a seguir.

\begin{Definition}[Proposi\c{c}\~ao]
  Definimos por proposi\c{c}\~ao qualquer senten\c{c}a pass\'ivel de possuir um dos valores l\'ogicos: verdadeiro ou falso.
\end{Definition}

Sempre que poss\'ivel, ap\'os uma defini\c{c}\~ao, apresentaremos alguns exemplos para ilustr\'a-la.

\begin{Example}
  Quais das seguintes senten\c{c}as podem ser consideradas proposi\c{c}\~oes?
  \begin{enumerate}
    \item Hoje \'e segunda-feira.
    \item $10 < 7$
    \item $x + 1 = 3$
    \item Como est\'a voc\^e?
    \item Ela \'e muito talentosa
    \item Existe vida em outros planetas.
  \end{enumerate}
  Neste exemplo, temos que a senten\c{c}a $1$ \'e uma proposi\c{c}\~ao, pois o dia de hoje pode ser ou n\~ao segunda-feira tornando essa frase
  verdadeira ou falsa. A senten\c{c}a $2$ \'e uma proposi\c{c}\~ao, pois temos que $10$ n\~ao \'e menor do que $7$. Logo, o valor l\'ogico dessa
  senten\c{c}a \'e igual a falso. A senten\c{c}a $3$ n\~ao \'e uma proposi\c{c}\~ao pois seu valor l\'ogico depende do valor atribu\'ido a 
  vari\'avel $x$. Se $x = 2$, temos que a senten\c{c}a $3$ \'e verdadeira. A mesma senten\c{c}a $3$ \'e falsa para qualquer outro valor de $x$. 
  Logo, como n\~ao \'e poss\'ivel determinar de maneira \'unica o valor l\'ogico da senten\c{c}a $3$, essa n\~ao \'e considerada uma 
  proposi\c{c}\~ao. 
  A senten\c{c}a $4$ n\~ao \'e uma proposi\c{c}\~ao pois n\~ao \'e poss\'ivel atribuir um valor verdadeiro ou falso para uma pergunta.
  A senten\c{c}a $5$ n\~ao \'e uma proposi\c{c}\~ao pois ``ela'' n\~ao est\'a especificada. Portanto, o fato de ``ela'' ser talentosa ou n\~ao
  depende de quem \'e ``ela''. Logo, essa senten\c{c}a n\~ao \'e uma proposi\c{c}\~ao.
  A senten\c{c}a $6$ \'e uma proposi\c{c}\~ao pois o fato de existir vida em outros planetas pode ser verdadeiro ou falso.
\end{Example}

No conceito de proposi\c{c}\~ao est\~ao impl\'icitas duas propriedades fundamentais da l\'ogica cl\'assica:
\begin{itemize}
  \item O princ\'ipio da n\~ao contradi\c{c}\~ao: Nenhuma proposi\c{c}\~ao \'e verdadeira e falsa simultaneamente.
  \item O princ\'ipio do terceiro exclu\'ido: Toda proposi\c{c}\~ao \'e verdadeira ou falsa.
\end{itemize}

Por\'em n\~ao \'e dif\'icil perceber que existem proposi\c{c}\~oes que s\~ao compostas por outras proposi\c{c}\~oes menores. Considere a
seguinte frase: `Gatos s\~ao peludos e elefantes pesados`. Esta \'e formada por duas proposi\c{c}\~oes distintas, a saber: 1) Gatos s\~ao peludos; 
2) Elefantes s\~ao pesados. Deste exemplo podemos perceber que \'e poss\'ivel combinar proposi\c{c}\~oes menores para formar outras. Desta maneira,
podemos classificar proposi\c{c}\~oes como sendo simples ou compostas que ser\~ao definidas a seguir.

\begin{Definition}[Proposi\c{c}\~ao simples e composta]
   Dizemos que uma proposi\c{c}\~ao \'e simples se essa n\~ao puder ser decomposta em proposi\c{c}\~oes menores e composta caso
   essa possa ser divida em uma ou mais proposi\c{c}\~oes. 
\end{Definition}
\begin{Example}
  Classifique as seguintes proposi\c{c}\~oes como simples ou compostas. Caso a proposi\c{c}\~ao em quest\~ao seja composta, identifique
  as proposi\c{c}\~oes simples que a comp\~oe.
  \begin{enumerate}
    \item Di\'ogenes trabalha como carteiro.
    \item Jo\~aozinho n\~ao conta mentiras.
    \item O bandido \'e franc\^es.
    \item Se Cl\'eber ganhar elei\c{c}\~ao, ent\~ao os impostos ser\~ao reduzidos.
    \item O processador \'e r\'apido mas a impressora \'e lenta.
    \item Se Jo\~ao correr vai ficar cansado.
  \end{enumerate}
  A primeira proposi\c{c}\~ao \'e simples, pois n\~ao pode ser dividida em proposi\c{c}\~oes menores. Isto \'e, n\~ao poss\'ivel decompor a frase
  em ``peda\c{c}os'' de maneira que estes possam ter valores l\'ogicos verdadeiro ou falso. A proposi\c{c}\~ao 2) \'e composta, pois possui como
  compontente a proposi\c{c}\~ao `Jo\~aozinho conta mentiras`. A proposi\c{c}\~ao 3) \'e simples. A proposi\c{c}\~ao 4) \'e composta e esta \'e 
  formada pelas seguintes proposi\c{c}\~oes simples: `Cl\'eber ganhou a elei\c{c}\~ao` e `Os impostos ser\~ao reduzidos`. A proposi\c{c}\~ao 5)
  tamb\'em \'e composta e \'e formada pelas proposi\c{c}\~oes: `O processador \'e r\'apido` e `A impressora \'e lenta`. Finalmente, a 
  proposi\c{c}\~ao 6) \'e tamb\'em composta e formada por `Jo\~ao corre` e `Jo\~ao fica cansado`.
\end{Example}

Proposi\c{c}\~oes compostas combinam proposi\c{c}\~oes simples utilizando conectivos que a partir do valor l\'ogico das proposi\c{c}\~oes simples
permitem obter o valor da proposi\c{c}\~ao composta em quest\~ao. Existem um n\'umero infinito de conectivos l\'ogicos poss\'iveis. Iremos nos
ater aos usualmente utilizados na matem\'atica. Estes s\~ao definidos a seguir.

\begin{Definition}[Conectivos]
  Os cinco principais conectivos da l\'ogica cl\'assica expressam as seguintes no\c{c}\~oes descritas informalmente abaixo:
  \begin{itemize}
    \item \textit{Nega\c{c}\~ao}: A proposi\c{c}\~ao afirma que certa proposi\c{c}\~ao n\~ao \'e verdadeira, ou seja, \'e falsa 
          (de acordo com o princ\'ipio do terceiro exclu\'ido). Exemplo: \textit{O c\'eu n\~ao est\'a nublado hoje} diz que a
          proposi\c{c}\~ao \textit{O c\'eu est\'a nublado hoje} \'e falsa.
    \item \textit{Conjun\c{c}\~ao}: Afirma que a proposi\c{c}\~ao em quest\~ao s\'o \'e verdadeira quando as duas proposi\c{c}\~oes que
          que a comp\~oe s\~ao tamb\'em verdadeiras. Exemplo: \textit{O dia est\'a lindo, embora nublado} diz que \textit{o dia est\'a lindo}
          e \textit{nublado}.
    \item \textit{Disjun\c{c}\~ao}: Afirma que pelo menos uma dentre duas proposi\c{c}\~oes \'e verdadeira. Exemplo: \textit{Diocreciano estuda
          muito ou \'e inteligente} diz que \textit{Diocreciano estudo muito} ou \textit{Diocreciano \'e inteligente} ou 
          \textit{Diocreciano estudo muito e \'e inteligente}. 
    \item \textit{Condicional}: Afirma que caso uma certa proposi\c{c}\~ao seja verdadeira, uma outra tamb\'em o \'e, ou seja, n\~ao \'e o caso
          que a primeira possa ser verdadeira e a outra falsa. Exemplo: \textit{Se hoje chover, n\~ao irei \`a pra\c{c}a} diz que se a 
          proposi\c{c}\~ao \textit{hoje ir\'a chover} for verdadeira ent\~ao a proposi\c{c}\~ao \textit{n\~ao irei a pra\c{c}a} tamb\'em ser\'a
          verdadeira.
     \item \textit{Bicondicional}: Afirma que uma proposi\c{c}\~ao \'e verdadeira exatamente nos casos que uma outra tamb\'em o \'e.
           Exemplo: \textit{o n\'umero \'e par se e somente se seu quadrado tamb\'em \'e par} diz que as proposi\c{c}\~oes
           \textit{o n\'umero \'e par} e \textit{o quadrado do n\'umero \'e par} s\~ao ambas verdadeiras ou ambas falsas.
  \end{itemize}
\end{Definition}

Algumas palavras da l\'ingua portuguesa s\~ao frequentemente utilizadas em proposi\c{c}\~oes para denotar conectivos. A tabela \ref{table:1}
apresenta algumas destas palavras e quais conectivos estas representam. Nesta tabela utilizamos as vari\'aveis A e B para denotar proposi\c{c}\~oes
quaisquer.

\begin{table}
  \begin{tabular}{|c|l|}
    \hline
    \textbf{Conectivo L\'ogico}  & \textbf{Express\~ao em Portugu\^es} \\ \hline
     Conjun\c{c}\~ao             & A e B; A mas B; A tamb\'em B ; A al\'em disso B\\ \hline
     Disjun\c{c}\~ao             & A ou B\\ \hline
    \multirow{7}{*}{Condicional} 
    & Se A, ent\~ao B \\ 
    & A implica B     \\ 
    & A logo, B \\ 
    & A s\'o se B \\
    & A somente se B\\
    & B segue de A \\
    & A \'e uma condi\c{c}\~ao suficiente para B\\
    & basta A para B \\
    & B \'e uma condi\c{c}\~ao necess\'aria para A \\ \hline
    \multirow{2}{*}{Bicondicional} 
    & A se e somente se B \\
    & A \'e condi\c{c}\~ao necess\'aria e suficiente para B \\ \hline
    \multirow{3}{*}{Nega\c{c}\~ao}
    & n\~ao A \\
    & \'E falso que A\\
    & N\~ao \'e verdade que A \\ \hline
  \end{tabular}
  \centering
  \caption{Relacionando palavras do portugu\^es com conectivos l\'ogicos}
  \label{table:1}
\end{table}

\begin{Example}
  Quais s\~ao os conectivos presentes nas seguintes proposi\c{c}\~oes compostas?
    \begin{enumerate}
    \item Jo\~aozinho n\~ao conta mentiras.
    \item Se Cl\'eber ganhar elei\c{c}\~ao, ent\~ao os impostos ser\~ao reduzidos.
    \item O processador \'e r\'apido mas a impressora \'e lenta.
    \item Amanh\~a irei \`a pra\c{c}a ou ao supermercado.
    \item Pagarei todas minhas d\'ividas se e somente se meu sal\'ario sair.
  \end{enumerate}
  Neste exemplo, temos que o conectivo presente na proposi\c{c}\~ao 1) \'e a nega\c{c}\~ao e a proposi\c{c}\~ao 2 \'e formada por um condicional.
  A proposi\c{c}\~ao 3) \'e formada pelo conectivo de conjun\c{c}\~ao. Por sua vez, a proposi\c{c}\~ao 4) \'e formada pelo conectivo 
  de disjun\c{c}\~ao e a proposi\c{c}\~ao 5 pelo bicondicional.
\end{Example}

Apesar da tabela \ref{table:1} ser um guia \'util na identifica\c{c}\~ao de conectivos, certamente ela n\~ao \'e exaustiva. Al\'em disso,
diversas senten\c{c}as da l\'ingua portuguesa n\~ao podem ser representadas utilizando apenas esses tipos de composi\c{c}\~ao. Usualmente
elementos que n\~ao possuem uma correspond\^encia direta com a l\'ogica proposicional podem ser ``despresados'' durante a modelagem em
quest\~ao. Outro ponto referente a modelagem utilizando l\'ogica proposicional \'e que o conceito de proposi\c{c}\~ao simples e composta
\'e relativo ao problema a ser representado. Por exemplo, considere a seguinte proposi\c{c}\~ao: \textit{5 n\~ao \'e um n\'umero par}. Esta
proposi\c{c}\~ao pode ser considerada composta --- formada pela nega\c{c}\~ao de \textit{5 \'e um n\'umero par} --- ou considerada uma 
proposi\c{c}\~ao simples, indivis\'ivel. A tarefa de determinar a ``granularidade'' do que deve ser considerado como proposi\c{c}\~ao
simples varia de problema para problema. Visando tornar esse tipo de conceito uniforme, nesta apostila adotaremos como conven\c{c}\~ao que
uma proposi\c{c}\~ao simples \'e uma proposi\c{c}\~ao que n\~ao pode ser dividida em proposi\c{c}\~oes menores. Desta maneira, a proposi\c{c}\~ao
\textit{5 n\~ao \'e um n\'umero par} ser\'a considerada uma proposi\c{c}\~ao composta.

\subsection{Exerc\'icios}\label{cap1:ex1}

\begin{enumerate}
  \item Para cada uma das senten\c{c}as a seguir, apresente as proposi\c{c}\~oes simples que a comp\~oe e os conectivos nela envolvidos.
  \begin{enumerate}
     \item Jo\~ao \'e pol\'itico, mas \'e honesto.
     \item Jo\~ao \'e honesto, mas seu irm\~ao n\~ao \'e.
     \item Vir\~ao a festa Jo\~ao ou sua irm\~a, al\'em da m\~ae.
     \item A estrela do espet\'aculo n\~ao canta, dan\c{c}a nem representa.
     \item Sempre que o trem apita, Jo\~ao sai correndo.
     \item Caso Jo\~ao n\~ao perca dinheiro no jogo, ele vai a festa.
     \item Jo\~ao vai ser multado, a menos que diminua a velocidade ou a rodovia n\~ao tenha radar.
     \item Uma condi\c{c}\~ao suficiente para que um n\'umero natural $n$ seja primo \'e que este seja \'impar.
     \item Jo\~ao vai ao teatro somente se estiver em cartaz uma com\'edia.
     \item Se voc\^e for Brasileiro, gosta de futebol a menos que tor\c{c}a para o Tabajara ou \'Ibis.
     \item A propina ser\'a paga exatamente nas situa\c{c}\~oes em que o deputado votar como instru\'ido por Jo\~ao.
 \end{enumerate}
\end{enumerate}

\subsection{Formalizando Senten\c{c}as}

Considere as seguintes proposi\c{c}\~oes compostas: 
\begin{enumerate}
   \item  O dia est\'a lindo, embora nublado.
   \item  O dia est\'a ensolarado e Jos\'e est\'a feliz.
\end{enumerate}
Ao observarmos estas duas proposi\c{c}\~oes, podemos dizer que estas possuem estrutura equivalente, pois ambas s\~ao formadas por 
duas proposi\c{c}\~oes simples e pelo conectivo de conjun\c{c}\~ao. Desta forma, podemos representar estas proposi\c{c}\~oes compostas
de maneira mais compacta substituindo as proposi\c{c}\~oes simples que as comp\~oe por vari\'aveis. A tabela seguinte apresenta a vari\'avel
associada a uma determinada proposi\c{c}\~ao simples para as frases anteriores.
\begin{table}[h]
  \begin{tabular}{c|l}
    Vari\'avel & Proposi\c{c}\~ao Simples \\ \hline
    $A$        & O dia est\'a lindo \\ 
    $B$        & O dia est\'a nublado \\
    $C$        & O dia est\'a ensolarado \\
    $D$        & Jos\'e est\'a feliz\\
  \end{tabular}
  \centering
\end{table}

Utilizando a tabela anterior, podemos as senten\c{c}as em quest\~ao podem ser representadas da seguinte maneira:
\begin{enumerate}
  \item $A$ e $B$
  \item $C$ e $D$
\end{enumerate}
Apesar do uso de vari\'aveis ter eliminado grande parte dos detalhes que n\~ao s\~ao relevantes para estrutura das proposi\c{c}\~oes em quest\~ao,
ainda utilizamos o portugu\^es para representar os conectivos l\'ogicos utilizados em proposi\c{c}\~oes compostas. Visando tornar a nota\c{c}\~ao
para representa\c{c}\~ao de proposi\c{c}\~oes uniforme, adotaremos os seguintes s\'imbolos para conectivos l\'ogicos, em que $A$ e $B$ denotam
 proposi\c{c}\~oes quaisquer:
\begin{table}[h]
  \begin{tabular}{|l|c|}
    \hline
    Conectivo & S\'imbolo \\ \hline
    Nega\c{c}\~ao & $\neg A$ \\ \hline
    Conjun\c{c}\~ao & $A \land B$ \\ \hline
    Disjun\c{c}\~ao & $A \lor B$ \\ \hline
    Condicional & $A \to B$ \\ \hline
    Bicondicional & $A \leftrightarrow B$ \\ \hline
  \end{tabular}
  \centering
  \caption{Nota\c{c}\~ao para conectivos l\'ogicos}
  \label{table:2}
\end{table}

Utilizando a nota\c{c}\~ao presente na tabela \ref{table:2}, temos que as senten\c{c}as anteriores seriam representadas pelas seguintes f\'ormulas
$A \land B$  e $C \land D$.

\subsection{Exerc\'icios}

\begin{enumerate}
  \item Escreva cada uma das proposi\c{c}\~oes compostas a seguir utilizando a nota\c{c}\~ao simb\'olica introduzida nesta se\c{c}\~ao.
  \begin{enumerate}
    \item Se Jane vencer ou perder, ir\'a ficar cansada.
    \item Rosas s\~ao vermelhas ou violetas s\~ao azuis.
    \item Se elefantes podem subir em \'arvores, 3 \'e um n\'umero irracional.
    \item \'E proibido fumar cigarros ou charutos.
    \item N\~ao \'e verdade que se $\pi > 0$ se e somente se $\pi > 1$.
    \item Se as laranjas s\~ao amarelas, ent\~ao os morangos s\~ao vermelhos.
    \item \'E falso que se Montreal \'e a capital do Canad\'a, ent\~ao a pr\'oxima copa ser\'a realizada no Brasil.
  \end{enumerate}
  \item Represente utilizando nota\c{c}\~ao simb\'olica as proposi\c{c}\~oes do exerc\'icio 1 da se\c{c}\~ao \ref{cap1:ex1}.
\end{enumerate}

\section{Sintaxe da L\'ogica Proposicional}

Tanto no Portugu\^es quanto na matem\'atica e nas linguagens de programa\c{c}\~ao, existem regras que determinam quando uma determinada 
senten\c{c}a \'e ou n\~ao v\'alida na linguagem em quest\~ao. Como exemplo, em linguagens de programa\c{c}\~ao, a express\~ao $(2 + 3$ \'e
considerada sintaticamente inv\'alida devido a falta do s\'imbolo $)$ nesta express\~ao. Em liguagens de programa\c{c}\~ao h\'a a necessidade
de verifica\c{c}\~ao sint\'atica, pois estamos interessados no signficado (execu\c{c}\~ao) das senten\c{c}as (programas) em quest\~ao e, 
formalmente, n\~ao h\'a como atribuir sem\^antica a senten\c{c}as sintaticamente incorretas.

Para definir quais senten\c{c}as da l\'ogica proposicional s\~ao pass\'iveis de atribuirmos um significado preciso, iremos definir o conjunto
de \textit{f\'ormulas bem formadas} da l\'ogica proposicional. Neste texto o termo f\'ormula (da l\'ogica proposicional) denotar\'a f\'ormulas
bem formadas, a menos que seja explicitamente dito o contr\'ario.

\begin{Definition}[F\'ormulas Bem Formadas]\label{propsyn}
O conjunto $\mathcal{F}$ de f\'ormulas bem formadas da l\'ogica proposicional \'e definido recursivamente da seguinte maneira:
\begin{enumerate}
  \item As constantes l\'ogicas $\top,\bot \in \mathcal{F}$ e denotam verdadeiro e falso respectivamente. 
  \item Seja $\mathcal{V}$ o conjunto (infinito) de vari\'aveis proposicionais. Ent\~ao $\mathcal{V} \subseteq \mathcal{F}$.
  \item Se $\alpha,\beta \in \mathcal{F}$, ent\~ao:
  \begin{enumerate}
    \item $\neg \alpha \in \mathcal{F}$.
    \item $\alpha \circ \beta \in \mathcal{F}$, em que $\circ \in \{\land,\lor,\to,\leftrightarrow\}$.
    \item $(\alpha)\in\mathcal{F}$.
  \end{enumerate}
\end{enumerate}
Todos os elementos de $\mathcal{F}$ podem ser constru\'idos pelas regras anteriores.
\end{Definition}
Apresentaremos alguns exemplos de f\'ormulas da l\'ogica e como estas
podem ser constru\'idas utilizando a defini\c{c}\~ao \ref{propsyn}.
\begin{Example}
Considere as seguintes fórmulas da lógica proposicional:
\begin{enumerate}
  \item $\neg (A \lor \top)$
  \item $A \to \neg A$
\end{enumerate}
A f\'ormula 1) pode ser constru\'ida da seguinte maneira:
Primeiramente, pelas regras 1 e 2 temos que a vari\'avel $A$ e a
constante $\top$ s\~ao f\'ormulas da lógica e, portanto, pela regra
3-b temos que $A \lor \top$. Uma vez que $A \lor \top$ \'e uma
f\'ormula, temos, pela regra 3-a, temos que $\neg (A \lor \top)$.

Por sua vez, a f\'ormula $A \to \neg A$ pode ser formada da seguinte
forma: Pela regra 2, temos que a vari\'avel $A$ \'e uma
f\'ormula. Pela regra 3-a, temos que $\neg A$ \'e uma f\'ormula e,
finalmente, por 3-b, temos que $A \to \neg A$.

Por\'em, as seguintes express\~oes n\~ao podem ser consideradas
f\'ormulas pois, n\~ao podem ser constru\'idas de acordo com a
defini\c{c}\~ao \label{propsyn}: $A \lor \neg B \land$ e $A \to
\neg$. A primeira n\~ao pode ser considerada uma f\'ormula pois,
pela regra 3-b), o operador $\land$ precisa de dois par\^ametro. O
mesmo problema ocorre com a segunda f\'ormula, pois de acordo com
a regra 3-a), o operador $\neg$ precisa de um par\^ametro.
\end{Example}

Uma boa maneira de visualizar como as regras da defini\c{c}\~ao
\ref{propsyn} s\~ao utilizadas para construir uma f\'ormula \'e
utilizando uma \'arvore, que representa, graficamente, a estrutura do
termo (fórmula) em quest\~ao. 

N\~ao apresentaremos uma defini\c{c}\~ao
rigorosa de como uma \'arvore \'e formada a partir da sintaxe de uma
linguagem formal, visto que esse tema \'e parte da ementa da
disciplina Fundamentos Te\'oricos da Computa\c{c}\~ao. Neste texto,
utilizaremos esse conceito de maneira informal.

Como um primeiro exemplo, vamos apresentar uma \'arvore para ilustrar
a estrutura da f\'ormula $\neg A \lor B$:

