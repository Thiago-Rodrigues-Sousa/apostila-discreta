\chapter{L\'ogica Proposicional}

\begin{chapquote}
`Uma vez pessoa me disse: `Me conven\c{c}a de que l\'ogica \'e \'util.' --- `Voc\^e deseja que eu prove isso?', respondi.
--- `Sim.', ele respondeu. --- `Ent\~ao, eu devo produzir um argumento que comprove este fato?' --- Ele concordou. --- `Ent\~ao,
como voc\^e saber\'a que eu n\~ao produzi um argumento falacioso?' --- Ele nada disse. --- `Veja, voc\^e acaba de se convencer
de que a l\'ogica \'e necess\'aria, uma vez que sem ela voc\^e n\~ao \'e capaz de saber se esta \'e necess\'aria ou n\~ao.'
Epicteto 1, Discursos 3 II xxv.
\end{chapquote}

\section{Motiva\c{c}\~ao}

A l\'ogica prov\^e um ferramental para o racioc\'inio sobre matem\'atica, algoritmos
e circuitos digitais. A aplicabilidade da l\'ogica permeia diversas \'areas da computa\c{c}\~ao,
citaremos aqui apenas alguns exemplos:

\begin{itemize}
  \item \textbf{Engenharia de Software}: considera-se uma boa pr\'atica especificar um sistema antes
        de iniciar a sua codifica\c{c}\~ao. Uma das v\'arias t\'ecnicas de especifica\c{c}\~ao de
        software \'e o uso de l\'ogica.
  \item \textbf{Aplica\c{c}\~oes de Miss\~ao Cr\'itica}: dizemos que uma determinada aplica\c{c}\~ao
        \'e de miss\~ao cr\'itica se essa est\'a relacionada a algum risco (de vida, elevados preju\'izos financeiros, etc.).
        Em tais aplica\c{c}\~oes, a utiliza\c{c}\~ao de testes para garantir o funcionamento adequado
        n\~ao \'e suficiente. O que espera-se \'e uma prova de corretude do programa em quest\~ao, isso \'e uma
        demonstra\c{c}\~ao de que o programa comporta-se de acordo com sua especifica\c{c}\~ao em todas as
        situa\c{c}\~oes poss\'iveis. A l\'ogica \'e a fundamenta\c{c}\~ao matem\'atica de demonstra\c{c}\~oes de
        corre\c{c}\~ao de programas.
   \item \textbf{Recupera\c{c}\~ao de informa\c{c}\~ao}: em m\'aquinas de busca para Web, utiliza-se l\'ogica para
         especificar como propriedades que classificam uma determinada p\'agina como relevante ou n\~ao com base
         em seu conte\'udo.
   \item \textbf{Circuitos Digitais e Arquitetura de Computadores:} l\'ogica \'e a linguagem utilizada para descrever
         sinais produzidos e recebidos como entrada por componentes eletr\^onicos. Um problema comum no projeto de
         circuitos eletr\^onicos \'e determinar uma vers\~ao equivalente, por\'em mais eficiente, de um circuito l\'ogico.
         T\'ecnicas para solu\c{c}\~ao desse problema s\~ao baseadas em algoritmos eficientes para o processamento de
         f\'ormulas da l\'ogica.
   \item \textbf{Bancos de dados}: um recurso fundamental de qualquer sistema gerenciador de bancos de dados \'e uma linguagem
         simples e expressiva para recuperar informa\c{c}\~oes nele armazenadas. A utiliza\c{c}\~ao de recursos baseados em
         l\'ogica \'e a chave para a expressividade de linguagens para consultas a bancos de dados.
\end{itemize}

Al\'em das \'areas citadas anteriormente, a l\'ogica \'e fundamental no estudo e no projeto de linguagens de programa\c{c}\~ao
e da teoria de computabilidade.

Neste cap\'itulo, discutiremos as dificuldades presentes na utiliza\c{c}\~ao do Portugu\^es para expressar racioc\'inio
l\'ogico e como contorn\'a-las utilizando l\'ogica formal. Existem diversos tipos de l\'ogicas formais, cada uma com
uma aplica\c{c}\~ao espec\'ifica. Inicialmente consideraremos uma l\'ogica bem simples chamada de l\'ogica proposicional.
Primeiramente, iremos analisar a sintaxe da linguagem da l\'ogica proposicional e depois consideraremos tr\^es sistemas
matem\'aticos para racioc\'inio sobre f\'ormulas da l\'ogica proposicional: tabelas verdade, dedu\c{c}\~ao natural e \'algebra
Booleana.

\emph{Tabelas verdade} definem o significado dos conectivos l\'ogicos e como eles podem ser utilizados para calcular os
valores de express\~oes e provar que duas proposi\c{c}\~oes s\~ao logicamente equivalentes. Como tabelas verdade
expressam diretamente o siginificado de propoposi\c{c}\~oes, dizemos que essas s\~ao uma abordagem baseada em sem\^antica para l\'ogica.
Tabelas verdade s\~ao de simples entendimento, por\'em n\~ao s\~ao \'uteis na solu\c{c}\~ao de problemas reais devido ao seu tamanho.

\emph{Dedu\c{c}\~ao Natural} \'e uma formaliza\c{c}\~ao de princ\'ipios b\'asicos de racioc\'inio l\'ogico utilizado no cotidiano.
A dedu\c{c}\~ao natural prov\^e um conjunto de regras de infer\^encia que especificam exatamente quais fatos podem ser deduzidos
a partir de um conjunto de fatos dados. Em dedu\c{c}\~ao natural n\~ao h\'a a no\c{c}\~ao de `valor l\'ogico` de proposi\c{c}\~oes;
j\'a que tudo no sistema est\'a encapsulado em suas regras de infer\^encia. Conforme veremos posteriormente, essas regras s\~ao baseadas
na estrutura das proposi\c{c}\~oes envolvidas, a dedu\c{c}\~ao natural \'e uma abordagem puramente sint\'atica para a l\'ogica. Diversas
t\'ecnicas utilizadas em pesquisas na \'area de linguagens de programa\c{c}\~ao s\~ao baseadas em sistemas l\'ogicos que s\~ao de alguma
maneira relacionados \`a dedu\c{c}\~ao natural.

\emph{\'Algebra Booleana} \'e uma abordagem para formaliza\c{c}\~ao da l\'ogica baseada em um conjunto de equa\c{c}\~oes --- as leis
da \'algebra Booleana --- para especificar que certas proposi\c{c}\~oes s\~ao iguais a outras. A \'algebra Booleana \'e uma abordagem
axiom\'atica, similar a \'algebra elementar e geometria, pois prov\^e um conjunto de leis para manipular proposi\c{c}\~oes. T\'ecnicas
alg\'ebricas para a l\'ogica s\~ao fundamentais para o projeto de circuitos digitais.

\section{Introdu\c{c}\~ao \`a L\'ogica Formal}\label{cap1:sec1}

A l\'ogica formal foi inicialmente concebida na gr\'ecia antiga onde fil\'osofos desejavam ser capazes de analisar argumentos
em linguagem natural. Os gregos eram fascinados pela id\'eia de que alguns argumentos eram sempre verdadeiros e outros sempre
falsos. Por\'em, eles rapidamente perceberam que o racioc\'inio l\'ogico \'e dif\'icil de ser analisado usando linguagens naturais
como o Grego (ou o Portugu\^es!). Isso se deve principalmente devido \`as \emph{ambiguidades} inerentes \`as linguagens naturais.
Uma das maneiras de se evitar essas dificuldades \'e o uso de vari\'aveis que denominaremos \emph{vari\'aveis proposicionais}.

Suponha que um conhecido lhe diga `O dia est\'a ensolarado e estou feliz`. Aparentemente essa frase possui interpreta\c{c}\~ao
\'obvia, mas ao observ\'a-la com cuidado percebe-se que o significado dessa n\~ao \'e t\~ao evidente. Talvez essa pessoa goste
de dias ensolarados e fica contente quando esse fato ocorre. Note que existe uma conex\~ao entre as duas partes da senten\c{c}a,
dessa forma, a palavra `e` presente na frase `O dia est\'a ensolarado e estou feliz` significa `e, portanto`. Por\'em, essa an\'alise
depende de nossa experi\^encia em relacionar o clima com a felicidade das pessoas. Considere agora o seguinte exemplo:
`Gatos s\~ao peludos e elefantes pesados`. Essa senten\c{c}a possui a mesma estrutura do exemplo anterior, mas ningu\'em ir\'a tentar
relacionar o peso de elefantes com a quantidade de pelos de gatos. Neste caso, a palavra `e` significa `e, tamb\'em`. Pode-se perceber que a
palavra `e` possui diversos significados sutis, e escolhemos o significado apropriado usando nosso conhecimento do mundo \`a nossa volta.
Perceba que as duas simples frases de exemplo consideradas ilustram as dificuldades de interpreta\c{c}\~ao que podem surgir ao se utilizar
uma linguagem natural. As dificuldades em se dar um significado preciso a frases em linguagem natural n\~ao se restringem a somente como
intepretar a palavra `e`. O estudo preciso da sem\^antica de senten\c{c}as expressas em linguagem natural \'e objeto de estudo da lingu\'istica
e da filosofia.

Ao inv\'es de tentarmos o imposs\'ivel --- expressar, de maneira precisa,  racioc\'inio l\'ogico em linguagem natural  --- n\'os iremos separar
a estrutura l\'ogica de um argumento de todas as conota\c{c}\~oes da l\'ingua portuguesa. Faremos isso utilizando \textbf{proposi\c{c}\~oes}, que
s\~ao definidas a seguir.

\begin{Definition}[Proposi\c{c}\~ao]
  Definimos por proposi\c{c}\~ao qualquer senten\c{c}a pass\'ivel de possuir um dos valores l\'ogicos: verdadeiro ou falso.
\end{Definition}

Sempre que poss\'ivel, ap\'os uma defini\c{c}\~ao, apresentaremos alguns exemplos para ilustr\'a-la.

\begin{Example}
  Quais das seguintes senten\c{c}as podem ser consideradas proposi\c{c}\~oes?
  \begin{enumerate}
    \item Hoje \'e segunda-feira.
    \item $10 < 7$
    \item $x + 1 = 3$
    \item Como est\'a voc\^e?
    \item Ela \'e muito talentosa
    \item Existe vida em outros planetas.
  \end{enumerate}
  Neste exemplo, temos que a senten\c{c}a $1$ \'e uma proposi\c{c}\~ao, pois o dia de hoje pode ser ou n\~ao segunda-feira tornando essa frase
  verdadeira ou falsa. A senten\c{c}a $2$ \'e uma proposi\c{c}\~ao, pois temos que $10$ n\~ao \'e menor do que $7$. Logo, o valor l\'ogico dessa
  senten\c{c}a \'e igual a falso. A senten\c{c}a $3$ n\~ao \'e uma proposi\c{c}\~ao pois seu valor l\'ogico depende do valor atribu\'ido a
  vari\'avel $x$. Se $x = 2$, temos que a senten\c{c}a $3$ \'e verdadeira. A mesma senten\c{c}a $3$ \'e falsa para qualquer outro valor de $x$.
  Logo, como n\~ao \'e poss\'ivel determinar de maneira \'unica o valor l\'ogico da senten\c{c}a $3$, essa n\~ao \'e considerada uma
  proposi\c{c}\~ao.
  A senten\c{c}a $4$ n\~ao \'e uma proposi\c{c}\~ao pois n\~ao \'e poss\'ivel atribuir um valor verdadeiro ou falso para uma pergunta.
  A senten\c{c}a $5$ n\~ao \'e uma proposi\c{c}\~ao pois ``ela'' n\~ao est\'a especificada. Portanto, o fato de ``ela'' ser talentosa ou n\~ao
  depende de quem \'e ``ela''. Logo, essa senten\c{c}a n\~ao \'e uma proposi\c{c}\~ao.
  A senten\c{c}a $6$ \'e uma proposi\c{c}\~ao pois o fato de existir vida em outros planetas pode ser verdadeiro ou falso.
\end{Example}

No conceito de proposi\c{c}\~ao est\~ao impl\'icitas duas propriedades fundamentais da l\'ogica cl\'assica:
\begin{itemize}
  \item O princ\'ipio da n\~ao contradi\c{c}\~ao: Nenhuma proposi\c{c}\~ao \'e verdadeira e falsa simultaneamente.
  \item O princ\'ipio do terceiro exclu\'ido: Toda proposi\c{c}\~ao \'e verdadeira ou falsa.
\end{itemize}

Por\'em n\~ao \'e dif\'icil perceber que existem proposi\c{c}\~oes que s\~ao compostas por outras proposi\c{c}\~oes menores. Considere a
seguinte frase: ``Gatos s\~ao peludos e elefantes pesados''. Esta \'e formada por duas proposi\c{c}\~oes distintas, a saber: 1) Gatos s\~ao peludos;
2) Elefantes s\~ao pesados. Deste exemplo podemos perceber que \'e poss\'ivel combinar proposi\c{c}\~oes menores para formar outras. Desta maneira,
podemos classificar proposi\c{c}\~oes como sendo simples ou compostas que ser\~ao definidas a seguir.

\begin{Definition}[Proposi\c{c}\~ao simples e composta]
   Dizemos que uma proposi\c{c}\~ao \'e simples se essa n\~ao puder
   ser decomposta em proposi\c{c}\~oes menores. Por sua vez, uma
   proposi\c{c}\~ao \'e composta caso
   essa possa ser dividida em uma ou mais proposi\c{c}\~oes.
\end{Definition}
\begin{Example}
  Classifique as seguintes proposi\c{c}\~oes como simples ou compostas. Caso a proposi\c{c}\~ao em quest\~ao seja composta, identifique
  as proposi\c{c}\~oes simples que a comp\~oe.
  \begin{enumerate}
    \item Di\'ogenes \'e carteiro.
    \item Jo\~aozinho n\~ao conta mentiras.
    \item O bandido \'e franc\^es.
    \item Se Cl\'eber ganhar elei\c{c}\~ao, ent\~ao os impostos ser\~ao reduzidos.
    \item O processador \'e r\'apido mas a impressora \'e lenta.
    \item Se Jo\~ao correr vai ficar cansado.
  \end{enumerate}
  A primeira proposi\c{c}\~ao \'e simples, pois n\~ao pode ser dividida em proposi\c{c}\~oes menores. Isto \'e, n\~ao poss\'ivel decompor a frase
  em ``peda\c{c}os'' de maneira que estes possam ter valores l\'ogicos verdadeiro ou falso. A proposi\c{c}\~ao 2) \'e composta, pois possui como
  componente a proposi\c{c}\~ao ``Jo\~aozinho conta mentiras''. A proposi\c{c}\~ao 3) \'e simples. A proposi\c{c}\~ao 4) \'e composta e esta \'e
  formada pelas seguintes proposi\c{c}\~oes simples: ``Cl\'eber ganhou a elei\c{c}\~ao'' e ``Os impostos ser\~ao reduzidos''. A proposi\c{c}\~ao 5)
  tamb\'em \'e composta e \'e formada pelas proposi\c{c}\~oes: ``O processador \'e r\'apido'' e ``A impressora \'e lenta''. Finalmente, a
  proposi\c{c}\~ao 6) \'e tamb\'em composta e formada por ``Jo\~ao corre'' e ``Jo\~ao fica cansado''.
\end{Example}

Proposi\c{c}\~oes compostas combinam proposi\c{c}\~oes simples utilizando conectivos que a partir do valor l\'ogico das proposi\c{c}\~oes simples
permitem obter o valor da proposi\c{c}\~ao composta em quest\~ao. Existem um n\'umero infinito de conectivos l\'ogicos poss\'iveis. Iremos nos
ater aos usualmente utilizados na matem\'atica. Estes s\~ao definidos a seguir.

\begin{Definition}[Conectivos]
  Os cinco principais conectivos da l\'ogica cl\'assica expressam as seguintes no\c{c}\~oes descritas informalmente abaixo:
  \begin{itemize}
    \item \textit{Nega\c{c}\~ao}: A proposi\c{c}\~ao afirma que certa proposi\c{c}\~ao n\~ao \'e verdadeira, ou seja, \'e falsa
          (de acordo com o princ\'ipio do terceiro exclu\'ido). Exemplo: \textit{O c\'eu n\~ao est\'a nublado hoje} diz que a
          proposi\c{c}\~ao \textit{O c\'eu est\'a nublado hoje} \'e falsa.
    \item \textit{Conjun\c{c}\~ao}: Afirma que a proposi\c{c}\~ao em quest\~ao s\'o \'e verdadeira quando as duas proposi\c{c}\~oes que
          a comp\~oe s\~ao tamb\'em verdadeiras. Exemplo: \textit{O dia est\'a lindo, embora nublado} diz que \textit{o dia est\'a lindo}
          e \textit{nublado}.
    \item \textit{Disjun\c{c}\~ao}: Afirma que pelo menos uma dentre duas proposi\c{c}\~oes \'e verdadeira. Exemplo: \textit{Diocreciano estuda
          muito ou \'e inteligente} diz que \textit{Diocreciano estudo muito} ou \textit{Diocreciano \'e inteligente} ou
          \textit{Diocreciano estudo muito e \'e inteligente}.
    \item \textit{Condicional}: Afirma que caso uma certa proposi\c{c}\~ao seja verdadeira, uma outra tamb\'em o \'e, ou seja, n\~ao \'e o caso
          que a primeira possa ser verdadeira e a outra falsa. Exemplo: \textit{Se hoje chover, n\~ao irei \`a pra\c{c}a} diz que se a
          proposi\c{c}\~ao \textit{hoje ir\'a chover} for verdadeira ent\~ao a proposi\c{c}\~ao \textit{n\~ao irei a pra\c{c}a} tamb\'em ser\'a
          verdadeira.
     \item \textit{Bicondicional}: Afirma que uma proposi\c{c}\~ao \'e verdadeira exatamente nos casos que uma outra tamb\'em o \'e.
           Exemplo: \textit{o n\'umero \'e par se e somente se seu quadrado tamb\'em \'e par} diz que as proposi\c{c}\~oes
           \textit{o n\'umero \'e par} e \textit{o quadrado do n\'umero \'e par} s\~ao ambas verdadeiras ou ambas falsas.
  \end{itemize}
\end{Definition}

Algumas palavras da l\'ingua portuguesa s\~ao frequentemente utilizadas em proposi\c{c}\~oes para denotar conectivos. A tabela \ref{table:1}
apresenta algumas destas palavras e quais conectivos estas representam. Nesta tabela utilizamos as vari\'aveis A e B para denotar proposi\c{c}\~oes
quaisquer.

\begin{table}
  \begin{tabular}{|c|l|}
    \hline
    \textbf{Conectivo L\'ogico}  & \textbf{Express\~ao em Portugu\^es} \\ \hline
     Conjun\c{c}\~ao             & A e B; A mas B; A tamb\'em B ; A al\'em disso B\\ \hline
     Disjun\c{c}\~ao             & A ou B\\ \hline
    \multirow{7}{*}{Condicional}
    & Se A, ent\~ao B \\
    & A implica B     \\
    & A logo, B \\
    & A s\'o se B \\
    & A somente se B\\
    & B segue de A \\
    & A \'e uma condi\c{c}\~ao suficiente para B\\
    & basta A para B \\
    & B \'e uma condi\c{c}\~ao necess\'aria para A \\ \hline
    \multirow{2}{*}{Bicondicional}
    & A se e somente se B \\
    & A \'e condi\c{c}\~ao necess\'aria e suficiente para B \\ \hline
    \multirow{3}{*}{Nega\c{c}\~ao}
    & n\~ao A \\
    & \'E falso que A\\
    & N\~ao \'e verdade que A \\ \hline
  \end{tabular}
  \centering
  \caption{Relacionando palavras do portugu\^es com conectivos l\'ogicos}
  \label{table:1}
\end{table}

\begin{Example}
  Quais s\~ao os conectivos presentes nas seguintes proposi\c{c}\~oes compostas?
    \begin{enumerate}
    \item Jo\~aozinho n\~ao conta mentiras.
    \item Se Cl\'eber ganhar elei\c{c}\~ao, ent\~ao os impostos ser\~ao reduzidos.
    \item O processador \'e r\'apido mas a impressora \'e lenta.
    \item Amanh\~a irei \`a pra\c{c}a ou ao supermercado.
    \item Pagarei todas minhas d\'ividas se e somente se meu sal\'ario sair.
  \end{enumerate}
  Neste exemplo, temos que o conectivo presente na proposi\c{c}\~ao 1) \'e a nega\c{c}\~ao e a proposi\c{c}\~ao 2 \'e formada por um condicional.
  A proposi\c{c}\~ao 3) \'e formada pelo conectivo de conjun\c{c}\~ao. Por sua vez, a proposi\c{c}\~ao 4) \'e formada pelo conectivo
  de disjun\c{c}\~ao e a proposi\c{c}\~ao 5 pelo bicondicional.
\end{Example}

Apesar da tabela \ref{table:1} ser um guia \'util na identifica\c{c}\~ao de conectivos, certamente ela n\~ao \'e exaustiva. Al\'em disso,
diversas senten\c{c}as da l\'ingua portuguesa n\~ao podem ser representadas utilizando apenas esses tipos de composi\c{c}\~ao. Usualmente
elementos que n\~ao possuem uma correspond\^encia direta com a l\'ogica proposicional podem ser ``despresados'' durante a modelagem em
quest\~ao. Outro ponto referente a modelagem utilizando l\'ogica proposicional \'e que o conceito de proposi\c{c}\~ao simples e composta
\'e relativo ao problema a ser representado. Por exemplo, considere a seguinte proposi\c{c}\~ao: \textit{5 n\~ao \'e um n\'umero par}. Esta
proposi\c{c}\~ao pode ser considerada composta --- formada pela nega\c{c}\~ao de \textit{5 \'e um n\'umero par} --- ou considerada uma
proposi\c{c}\~ao simples, indivis\'ivel. A tarefa de determinar a ``granularidade'' do que deve ser considerado como proposi\c{c}\~ao
simples varia de problema para problema. Visando tornar esse tipo de conceito uniforme, nesta apostila adotaremos como conven\c{c}\~ao que
uma proposi\c{c}\~ao simples \'e uma proposi\c{c}\~ao que n\~ao pode ser dividida em proposi\c{c}\~oes menores. Desta maneira, a proposi\c{c}\~ao
\textit{5 n\~ao \'e um n\'umero par} ser\'a considerada uma proposi\c{c}\~ao composta.

\subsection{Exerc\'icios}\label{cap1:ex1}

\begin{enumerate}
  \item Para cada uma das senten\c{c}as a seguir, apresente as proposi\c{c}\~oes simples que a comp\~oe e os conectivos nela envolvidos.
  \begin{enumerate}
     \item Jo\~ao \'e pol\'itico, mas \'e honesto.
     \item Jo\~ao \'e honesto, mas seu irm\~ao n\~ao \'e.
     \item Vir\~ao a festa Jo\~ao ou sua irm\~a, al\'em da m\~ae.
     \item A estrela do espet\'aculo n\~ao canta, dan\c{c}a nem representa.
     \item Sempre que o trem apita, Jo\~ao sai correndo.
     \item Caso Jo\~ao n\~ao perca dinheiro no jogo, ele vai a festa.
     \item Jo\~ao vai ser multado, a menos que diminua a velocidade ou a rodovia n\~ao tenha radar.
     \item Uma condi\c{c}\~ao suficiente para que um n\'umero natural $n$ seja primo \'e que este seja \'impar.
     \item Jo\~ao vai ao teatro somente se estiver em cartaz uma com\'edia.
     \item Se voc\^e for Brasileiro, gosta de futebol a menos que tor\c{c}a para o Tabajara ou \'Ibis.
     \item A propina ser\'a paga exatamente nas situa\c{c}\~oes em que o deputado votar como instru\'ido por Jo\~ao.
 \end{enumerate}
\end{enumerate}

\subsection{Formalizando Senten\c{c}as}

Considere as seguintes proposi\c{c}\~oes compostas:
\begin{enumerate}
   \item  O dia est\'a lindo, embora nublado.
   \item  O dia est\'a ensolarado e Jos\'e est\'a feliz.
\end{enumerate}
Ao observarmos estas duas proposi\c{c}\~oes, podemos dizer que estas possuem estrutura equivalente, pois ambas s\~ao formadas por
duas proposi\c{c}\~oes simples e pelo conectivo de conjun\c{c}\~ao. Desta forma, podemos representar estas proposi\c{c}\~oes compostas
de maneira mais compacta substituindo as proposi\c{c}\~oes simples que as comp\~oe por vari\'aveis. A tabela seguinte apresenta a vari\'avel
associada a uma determinada proposi\c{c}\~ao simples para as frases anteriores.
\begin{table}[h]
  \begin{tabular}{c|l}
    Vari\'avel & Proposi\c{c}\~ao Simples \\ \hline
    $A$        & O dia est\'a lindo \\
    $B$        & O dia est\'a nublado \\
    $C$        & O dia est\'a ensolarado \\
    $D$        & Jos\'e est\'a feliz\\
  \end{tabular}
  \centering
\end{table}

Utilizando a tabela anterior, as senten\c{c}as em quest\~ao podem ser representadas da seguinte maneira:
\begin{enumerate}
  \item $A$ e $B$
  \item $C$ e $D$
\end{enumerate}
Apesar do uso de vari\'aveis ter eliminado grande parte dos detalhes que n\~ao s\~ao relevantes para estrutura das proposi\c{c}\~oes em quest\~ao,
ainda utilizamos o portugu\^es para representar os conectivos l\'ogicos utilizados em proposi\c{c}\~oes compostas. Visando tornar a nota\c{c}\~ao
para representa\c{c}\~ao de proposi\c{c}\~oes uniforme, adotaremos os seguintes s\'imbolos para conectivos l\'ogicos, em que $A$ e $B$ denotam
 proposi\c{c}\~oes quaisquer:
\begin{table}[h]
  \begin{tabular}{|l|c|}
    \hline
    Conectivo & S\'imbolo \\ \hline
    Nega\c{c}\~ao & $\neg A$ \\ \hline
    Conjun\c{c}\~ao & $A \land B$ \\ \hline
    Disjun\c{c}\~ao & $A \lor B$ \\ \hline
    Condicional & $A \to B$ \\ \hline
    Bicondicional & $A \leftrightarrow B$ \\ \hline
  \end{tabular}
  \centering
  \caption{Nota\c{c}\~ao para conectivos l\'ogicos}
  \label{table:2}
\end{table}

Utilizando a nota\c{c}\~ao presente na tabela \ref{table:2}, temos que as senten\c{c}as anteriores seriam representadas pelas seguintes f\'ormulas
$A \land B$  e $C \land D$.

\subsection{Exerc\'icios}

\begin{enumerate}
  \item Escreva cada uma das proposi\c{c}\~oes compostas a seguir utilizando a nota\c{c}\~ao simb\'olica introduzida nesta se\c{c}\~ao.
  \begin{enumerate}
    \item Se Jane vencer ou perder, ir\'a ficar cansada.
    \item Rosas s\~ao vermelhas ou violetas s\~ao azuis.
    \item Se elefantes podem subir em \'arvores, 3 \'e um n\'umero irracional.
    \item \'E proibido fumar cigarros ou charutos.
    \item N\~ao \'e verdade que se $\pi > 0$ se e somente se $\pi > 1$.
    \item Se as laranjas s\~ao amarelas, ent\~ao os morangos s\~ao vermelhos.
    \item \'E falso que se Montreal \'e a capital do Canad\'a, ent\~ao a pr\'oxima copa ser\'a realizada no Brasil.
  \end{enumerate}
  \item Represente utilizando nota\c{c}\~ao simb\'olica as proposi\c{c}\~oes do exerc\'icio 1 da se\c{c}\~ao \ref{cap1:ex1}.
\end{enumerate}

\section{Sintaxe da L\'ogica Proposicional}

Tanto no Portugu\^es quanto na matem\'atica e nas linguagens de programa\c{c}\~ao, existem regras que determinam quando uma determinada
senten\c{c}a \'e ou n\~ao v\'alida na linguagem em quest\~ao. Como exemplo, em linguagens de programa\c{c}\~ao, a express\~ao ``$(2 + 3$'' \'e
considerada sintaticamente inv\'alida devido a falta do s\'imbolo ``$)$'' nesta express\~ao. Em liguagens de programa\c{c}\~ao h\'a a necessidade
de verifica\c{c}\~ao sint\'atica, pois estamos interessados no signficado (execu\c{c}\~ao) das senten\c{c}as (programas) em quest\~ao e,
formalmente, n\~ao h\'a como atribuir sem\^antica a senten\c{c}as sintaticamente incorretas.

Para definir quais senten\c{c}as da l\'ogica proposicional s\~ao pass\'iveis de atribuirmos um significado preciso, iremos definir o conjunto
de \textit{f\'ormulas bem formadas} da l\'ogica proposicional. Neste texto o termo f\'ormula (da l\'ogica proposicional) denotar\'a f\'ormulas
bem formadas, a menos que seja explicitamente dito o contr\'ario.

\begin{Definition}[F\'ormulas Bem Formadas]\label{propsyn}
O conjunto $\mathcal{F}$ de f\'ormulas bem formadas da l\'ogica proposicional \'e definido recursivamente da seguinte maneira:
\begin{enumerate}
  \item As constantes l\'ogicas $\top,\bot \in \mathcal{F}$ e denotam verdadeiro e falso respectivamente.
  \item Seja $\mathcal{V}$ o conjunto (infinito) de vari\'aveis proposicionais. Ent\~ao $\mathcal{V} \subseteq \mathcal{F}$.
  \item Se $\alpha,\beta \in \mathcal{F}$, ent\~ao:
  \begin{enumerate}
    \item $\neg \alpha \in \mathcal{F}$.
    \item $\alpha \circ \beta \in \mathcal{F}$, em que $\circ \in \{\land,\lor,\to,\leftrightarrow\}$.
    \item $(\alpha)\in\mathcal{F}$.
  \end{enumerate}
\end{enumerate}
Todos os elementos de $\mathcal{F}$ podem ser constru\'idos pelas regras anteriores.
\end{Definition}
Apresentaremos alguns exemplos de f\'ormulas da l\'ogica e como estas
podem ser constru\'idas utilizando a defini\c{c}\~ao \ref{propsyn}.
\begin{Example}
Considere as seguintes fórmulas da lógica proposicional:
\begin{enumerate}
  \item $\neg (A \lor \top)$
  \item $A \to \neg A$
\end{enumerate}
A f\'ormula 1) pode ser constru\'ida da seguinte maneira:
Primeiramente, pelas regras 1 e 2 temos que a vari\'avel $A$ e a
constante $\top$ s\~ao f\'ormulas da lógica e, portanto, pela regra
3-b temos que $A \lor \top$. Uma vez que $A \lor \top$ \'e uma
f\'ormula, temos, pela regra 3-a, temos que $\neg (A \lor \top)$.

Por sua vez, a f\'ormula $A \to \neg A$ pode ser formada da seguinte
forma: Pela regra 2, temos que a vari\'avel $A$ \'e uma
f\'ormula. Pela regra 3-a, temos que $\neg A$ \'e uma f\'ormula e,
finalmente, por 3-b, temos que $A \to \neg A$.

Por\'em, as seguintes express\~oes n\~ao podem ser consideradas
f\'ormulas pois, n\~ao podem ser constru\'idas de acordo com a
defini\c{c}\~ao \label{propsyn}: $A \lor \neg B \land$ e $A \to
\neg$. A primeira n\~ao pode ser considerada uma f\'ormula pois,
pela regra 3-b), o operador $\land$ precisa de dois par\^ametro. O
mesmo problema ocorre com a segunda f\'ormula, pois de acordo com
a regra 3-a), o operador $\neg$ precisa de um par\^ametro.
\end{Example}

Em matem\'atica, \'e usual o uso de par\^enteses para impor uma ordem
de avalia\c{c}\~ao sobre express\~oes. Como exemplo, o resultado da express\~ao $(2
+ 3)\times 5$ \'e obtido calculando-se primeiro a soma para s\'o
ent\~ao efetuarmos a multiplica\c{c}\~ao. Em f\'ormulas da l\'ogica
proposicional, par\^enteses podem ser utilizados para determinar a
ordem de avalia\c{c}\~ao de uma certa express\~ao.
Por\'em, para permitir uma melhor legibilidade, utilizaremos uma ordem de preced\^encia entre
os conectivos para evitar o excesso de par\^enteses. O conectivo de
maior preced\^encia \'e o de nega\c{c}\~ao ($\neg$). O pr\'oximo
conectivo de maior preced\^encia \'e a conjun\c{c}\~ao ($\land$)
seguido da disjun\c{c}\~ao ($\lor$). Finalmente, os dois conectivos de menor
preced\^encia s\~ao o condicional ($\to$) e o bicondicional
($\leftrightarrow$), sendo o \'ultimo o de menor preced\^encia.
Usando essa ordem de preced\^encia, temos que a f\'ormula $A \land  B
\to C$ deve ser entendida como $(A \land B) \to C$, uma vez que a
conjun\c{c}\~ao possui maior preced\^encia que o condicional ($\to$).

Outra maneira de evitar o excesso de par\^enteses \'e a
utiliza\c{c}\~ao de crit\'erios de associatividade de
operadores. Neste texto vamos considerar que os operadores de
conjun\c{c}\~ao e disjun\c{c}\~ao associam \`a esquerda, isto \'e,
temos que $A \land B \land C \land D$ denota a mesma express\~ao que
$((A \land B) \land C) \land D$. Por sua vez, os conectivos
condicional e bicondicional associam \`a direita, logo, temos que
$A \to B \to C \to D$ representa a mesma express\~ao  que
$A \to (B \to (C \to D))$.


\subsection{Exerc\'icios}

\begin{enumerate}
  \item Para cada uma dos termos a seguir, use a defini\c{c}\~ao
    de f\'ormulas bem formadas (defini\c{c}\~ao \ref{propsyn}) para
    justificar o porqu\^e estes podem ser consideradas f\'ormulas bem
    formadas.
   \begin{enumerate}
       \item $\neg A \land B \to C$
       \item $(A \to B) \land \neg (A \lor B \to C)$
       \item $A \to B \to C \leftrightarrow \bot$
       \item $A \land \neg A \to B$
       \item $A \lor B \land C$
   \end{enumerate}
   \item Para cada umas das f\'ormulas seguintes, acrescente
     par\^enteses de maneira que n\~ao seja necess\'ario utilizar as
     regras de preced\^encia entre os conectivos da l\'ogica proposicional.
   \begin{enumerate}
       \item $\neg A \land B \to C$
       \item $(A \to B) \land \neg (A \lor B \to C)$
       \item $A \to B \to C \leftrightarrow \bot$
       \item $A \land \neg A \to B$
       \item $A \lor B \land C$
   \end{enumerate}
   \item Para cada uma das f\'ormulas seguintes, elimine os par\^enteses
   desnecess\'arios.
  \begin{enumerate}
	\item $((A\lor B)\lor (C\lor D))$
	\item $(A\rightarrow (B\rightarrow (A\land B)))$
	\item $\neg(A \lor (B\land C))$
	\item $\neg(A \land (B\lor C))$
  \end{enumerate}
\end{enumerate}

\section{Sem\^antica da L\'ogica Proposicional}\label{propsem}

As f\'ormulas da l\'ogica proposicional, descritas na se\c{c}\~ao
\ref{propsyn}, apesar de possu\'irem uma defini\c{c}\~ao sint\'atica,
ainda n\~ao possuem um significado matematicamente preciso. Nesta
se\c{c}\~ao apresentaremos a sem\^antica de f\'ormulas da l\'ogica
proposicional, que foi inicialmente concebida por Alfred Tarski na
primeira metade do s\'eculo XX \cite{halmos57}.

Conforme apresentado no cap\'itulo \ref{cap1}, uma forma de
atribuirmos sem\^antica a linguagens formais \'e definindo uma
fun\c{c}\~ao cujo dom\'inio \'e o conjunto de termos da linguagem
em quest\~ao e cujo contradom\'inio \'e um conjunto com
significado conhecido formalmente. Para a sem\^antica da l\'ogica
proposicional, consideraremos como dom\'inio o conjunto de f\'ormulas
bem formadas, $\mathcal{F}$, e contradom\'inio o conjunto formado
apenas pelos valores verdadeiro e falso, $\{T,F\}$.

Tradicionalmente, a fun\c{c}\~ao que descreve a sem\^antica de
f\'ormulas da l\'ogica proposicional \'e apresentada utilizando
tabelas verdade, que descrevem o significado de conectivos em termos
dos valores l\'ogicos das subf\'ormulas que o comp\~oe, ou seja,
a sem\^antica deve ser definida de acordo com a estrutura da sintaxe
das f\'ormulas.

As pr\'oximas subse\c{c}\~oes definem o significado de
cada um dos componentes da defini\c{c}\~ao de f\'ormulas bem formadas
da l\'ogica proposicional.


\subsection{Sem\^antica de constantes e vari\'aveis}

A sem\^antica das f\'ormulas $\top\in\mathcal{F}$ e
$\bot\in\mathcal{F}$ \'e dada pelas constantes $T$ e $F$,
respectivamente. Para vari\'aveis, a sem\^antica deve considerar as
possibilidades de valores l\'ogicos que podem ser assumidos por esta
vari\'avel. Para uma vari\'avel $A$ qualquer, temos que seu
significado pode ser um dos valores: verdadeiro ou falso. Este fato
\'e representado pela tabela verdade seguinte:
\begin{table}[h]
  \begin{tabular}{|c|}
        \hline
        $A$\\
        \hline
         F \\
         \hline
         T \\ \hline
  \end{tabular}
  \centering
\end{table}


\subsection{Sem\^antica da nega\c{c}\~ao ($\neg$)}

O significado de uma f\'ormula $\neg \alpha$, em que $\alpha$ \'e uma
f\'ormula da l\'ogica proposicional \'e dada pela seguinte tabela
verdade:
\begin{table}[h]
  \begin{tabular}{|c|c|}
    \hline
    $\alpha$ & $\neg \alpha$ \\ \hline
    $F$         & $T$ \\ \hline
    $T$         & $F$ \\ \hline
  \end{tabular}
  \centering
\end{table}
A primeira linha da tabela verdade da nega\c{c}\~ao diz que se uma
f\'ormula $\alpha$ possui o valor falso ($F$) ent\~ao sua
nega\c{c}\~ao ser\'a o valor $T$, verdadeiro. De maneira similar,
temos que se $\alpha$ possuir o valor falso, $\neg \alpha$ possuir\'a
o valor verdadeiro, conforme especificado na segunda linha da tabela
verdade anterior.

\subsection{Sem\^antica da conjun\c{c}\~ao ($\land$)}

Dadas duas f\'ormulas quaisquer, $\alpha,\beta$, temos que
$\alpha\land\beta$ s\'o possuir\'a o valor verdadeiro quando tanto
$\alpha$ e $\beta$ forem verdadeiros. Esta interpreta\c{c}\~ao para a
conjun\c{c}\~ao \'e dada pela tabela a seguir:
\begin{table}[h]
  \begin{tabular}{|c|c|c|}
    \hline
    $\alpha$ & $\beta$ & $\alpha \land \beta$\\ \hline
    $F$         & $F$        & $F$ \\ \hline
    $F$         & $T$        & $F$ \\ \hline
    $T$         & $F$        & $F$ \\ \hline
    $T$         & $T$        & $T$ \\ \hline
   \end{tabular}
  \centering
\end{table}
Note que a tabela verdade para a conjun\c{c}\~ao \'e formada por
quatro linhas que correspondem \`as maneira de atribuir valores
verdadeiro e falso para as subf\'ormulas $\alpha$ e $\beta$.

Al\'em disso, a tabela verdade reflete o significado informal da
conjun\c{c}\~ao, a saber: 1) basta que $\alpha$ ou $\beta$ seja falso
para que $\alpha\land\beta$ seja falso; e 2) $\alpha\land\beta$ ser\'a
verdadeiro apenas quando $\alpha$ e $\beta$ tamb\'em forem verdadeiros
simultanemante.

\subsection{Sem\^antica da disjun\c{c}\~ao ($\lor$)}

A f\'ormula $\alpha\lor\beta$ ser\'a verdadeira quando uma ou ambas
das f\'ormulas $\alpha$ ou $\beta$ forem verdadeiras. Disso segue que
a \'unica maneira de $\alpha\lor\beta$ possu\'irem o valor falso \'e
quando tanto $\alpha$ quanto $\beta$ forem falsos.

\begin{table}[h]
  \begin{tabular}{|c|c|c|}
    \hline
    $\alpha$ & $\beta$ & $\alpha \lor \beta$\\ \hline
    $F$         & $F$        & $F$ \\ \hline
    $F$         & $T$        & $T$ \\ \hline
    $T$         & $F$        & $T$ \\ \hline
    $T$         & $T$        & $T$ \\ \hline
   \end{tabular}
  \centering
\end{table}

\subsection{Sem\^antica do condicional ($\to$)}

O conectivo condicional, tamb\'em conhecido como implica\c{c}\~ao
l\'ogica, possui a mais peculiar sem\^antica dentre os conectivos
usuais da l\'ogica proposicional. A peculiaridade da defini\c{c}\~ao
sem\^antica da implica\c{c}\~ao l\'ogica decorre do fato de que este
conectivo \'e utilizado para representar afirmativas do tipo
``se-ent\~ao'', mas seu significado difere um pouco da
interpreta\c{c}\~ao cotidiana deste tipo de senten\c{c}a.

Para quaisquer f\'ormulas $\alpha$ e $\beta$, temos que
$\alpha\to\beta$ denota ``se $\alpha$ ent\~ao $\beta$''; $\alpha$
implica $\beta$ ou ainda ``n\~ao \'e o caso que $\alpha$ \'e
verdadeiro e $\beta$ falso''. Assim, $\alpha\to\beta$ representa que
n\~ao \'e poss\'ivel que $\alpha$ seja verdadeiro sem que $\beta$
tamb\'em o seja. Em outras palavras, ou $\alpha$ \'e falso ou $\alpha$
e $\beta$ s\~ao ambos verdadeiros.

A tabela verdade para
implica\c{c}\~ao segue diretamente da discuss\~ao anterior.

\begin{table}[h]
  \begin{tabular}{|c|c|c|}
    \hline
    $\alpha$ & $\beta$ & $\alpha \to \beta$\\ \hline
    $F$         & $F$        & $T$ \\ \hline
    $F$         & $T$        & $T$ \\ \hline
    $T$         & $F$        & $F$ \\ \hline
    $T$         & $T$        & $T$ \\ \hline
   \end{tabular}
  \centering
\end{table}

A partir da tabela anterior, podemos concluir os seguintes fatos
\'uteis sobre a implica\c{c}\~ao:
\begin{enumerate}
   \item Se $\alpha$ \'e falso, ent\~ao $\alpha\to\beta$ \'e
     verdadeiro, independente do valor de $\beta$.
   \item Se $\beta$ \'e verdadeiro, ent\~ao $\alpha\to\beta$ \'e
     verdadeiro, independente do valor de $\alpha$.
   \item A \'unica situa\c{c}\~ao em que $\alpha\to\beta$ possui o
     valor falso acontece quanto $\alpha$ \'e verdadeiro e $\beta$, falso.
\end{enumerate}

\subsection{Sem\^antica do bicondicional ($\leftrightarrow$)}

Escrevemos $\alpha\leftrightarrow\beta$ para representar que $\alpha$
e $\beta$ s\~ao ambas verdadeiras ou ambas falsas. Desta forma, temos
que $\alpha\leftrightarrow\beta$ ir\'a possuir o valor falso somente
quando o valor l\'ogico de $\alpha$ e $\beta$ for diferente. Estes
fatos s\~ao descritos formalmente na tabela verdade deste conectivo
que \'e apresentada a seguir.

\begin{table}[h]
  \begin{tabular}{|c|c|c|}
    \hline
    $\alpha$ & $\beta$ & $\alpha \leftrightarrow \beta$\\ \hline
    $F$         & $F$        & $T$ \\ \hline
    $F$         & $T$        & $F$ \\ \hline
    $T$         & $F$        & $F$ \\ \hline
    $T$         & $T$        & $T$ \\ \hline
   \end{tabular}
  \centering
\end{table}


\subsection{Construindo tabelas verdade para f\'ormulas}

A construç\~ao da tabela verdade de uma
f\'ormula $\alpha$ \'e feita calculando o valor desta de acordo com a
tabela de cada um dos conectivos nela presente e os valores l\'ogicos
das vari\'aveis nela presentes.

Considera-se uma boa pr\'atica, para construir tabelas verdade,
adicionar colunas para ``resultados intermedi\'arios'' de uma
f\'ormula. A no\c{c}\~ao de resultado intermedi\'ario de uma f\'ormula
\'e definida de maneira precisa usando o conceito de
subf\'ormula. Intuitivamente, o conjunto de subf\'ormulas \'e formado
por todas as f\'ormulas bem formadas que comp\~oes um dado termo $\alpha$.


\begin{Definition}[Subf\'ormula]\label{cap1sub}
Dada uma f\'ormula $\alpha$, o conjunto de subf\'ormulas de $\alpha$, $sub(\alpha)$,
\'e definido recursivamente da seguinte maneira:
\[
sub(\alpha) = \left\{
    \begin{array}{ll}
         \{\alpha\} & \text{se }\alpha = \top\text{ ou }\alpha =
         \bot\text{ ou }\alpha \text{ \'e uma vari\'avel}\\
         \{\neg \beta\} \cup T & \text{se }\alpha = \neg
         \beta\text{ e } T = sub(\beta)\\
         \{\beta \circ \rho\} \cup T \cup T' & \text{se }\alpha =
         \beta\circ\rho \text{,
         }\circ\in\{\land,\lor,\to,\leftrightarrow\}, \\
         & T =
         sub(\beta)\text{ e }T' =sub(\rho)
     \end{array}
                      \right.
\]
\end{Definition}
\begin{Example}
De acordo com a defini\c{c}\~ao \ref{cap1sub}, temos que o conjunto de
subf\'ormulas de $A \land B \to \bot$ \'e $\{A\land B \to \bot,A\land B,\bot,A,B\}$. Evidentemente, temos
que as vari\'aveis $A$, $B$, a constante $\bot$ e $A \land B \to\bot$
pertecem ao conjunto de subf\'ormulas de $sub(A \land B
\to\bot)$. Como $A\land B$ \'e um subtermo de $sub(A \land B \to
\bot)$, temos que $A\land B$ tamb\'em pertence a
$sub(A \land B \to\bot)$.
\end{Example}
Como $sub(\alpha)$ \'e o conjunto de subf\'ormulas de $\alpha$, estas
n\~ao possuem uma ordem. Para melhorar a leitura de tabelas verdade,
podemos ordenar colunas de tabelas verdade de acordo com o tamanho de
f\'ormulas. O conceito de tamanho de f\'ormulas \'e definido
formalmente pela seguinte fun\c{c}\~ao recursiva:
\[
tamanho(\alpha)=\left\{
  \begin{array}{ll}
    0       & \text{se }\alpha = \bot\text{ ou }\alpha = \top\\
    1      & \text{se }\alpha\text{ \'e uma vari\'avel}\\
    n +1 & \text{se }\alpha = \neg\beta\text{ e }n = tamanho(\beta)\\
    n + n' + 1 & \text{se }\alpha = \beta\circ\rho\text{,
    }\circ\in\{\land,\lor,\to,\leftrightarrow\}\\
                     & n = tamanho(\beta)\text{ e }n' = tamanho(\rho)
  \end{array}
                            \right .
\]
Logo, podemos escrever o conjunto $sub(A \land B \to\bot)$ ordenado
por tamanho da seguinte maneira: $\{\bot,A,B,A\land B,A\land B \to \bot\}$.

A tabela verdade de uma f\'ormula pode ser constru\'ida tomando por
colunas cada uma das sub\'ormulas ordenadas de maneira crescente de
acordo com o tamanho. Desta forma, temos que a tabela verdade para
$A\land B\to \bot$ \'e:
\[
\begin{array}{|c|c|c|c|c|}
  \hline
  \bot & A & B & A\land B & A\land B \to \bot \\ \hline
   F     & F  & F &      F         &  T                            \\
   F     & F & T &       F         &   T                           \\
   F     & T & F &       F         &     T                         \\
   F     & T & T &       T         &      F                       \\ \hline
\end{array}
\]
Observe que as linhas da tabela s\~ao preenchidas da seguinte maneira:
Todas as linhas para a constante $\bot$ s\~ao preenchidas com o valor
$F$. As linhas para vari\'aveis s\~ao formadas por todas as
combina\c{c}\~oes de valores verdadeiro-falso para estas. Logo, se uma
f\'ormula possuir $n$ vari\'aveis, esta ter\'a uma tabela com $2^n$
linhas, representando as duas possibilidades (verdadeiro e falso) para
cada uma de suas $n$ vari\'aveis.

Al\'em disso, obtemos valores das linhas de acordo com a tabela
verdade do conectivo da f\'ormula da coluna. Para facilitar, o
resultado de colunas posteriores pode ser obtido a partir dos
resultados de colunas anteriores.

\begin{Example}\label{extable}
Considere a tarefa de construir a tabela verdade para a f\'ormula $((A
\to B)\land \neg B) \to \neg A$. Para isso, devemos determinar o conjunto de
subf\'ormulas de $((A\to B)\land \neg B) \to \neg A$ e orden\'a-lo de acordo
com o tamanho. Ao fazer isso, obtemos o seguinte conjunto:
\[
\{A,B,\neg A,\neg B,A\to B, (A\to B)\land \neg B,((A\to B)\land \neg
B)\to \neg A\}
\]
Agora basta construir a tabela verdade:
\[
\begin{array}{|c|c|c|c|c|c|c|}
  \hline
A & B & \neg A & \neg B & A \to B & (A \to B) \land \neg B & ((A\to
B)\land \neg B)\to \neg A\\ \hline
F & F &   T  & T & T & T & T\\
F & T &  T   & F & F  & F  & T\\
T & F &   F    &  T  &  F   & F  & T \\
T & T &   F    &  F  &  T   & F   & T \\ \hline
\end{array}
\]
\end{Example}
Note que a f\'ormula do exemplo \ref{extable} \'e sempre verdadeira
independente do valor l\'ogico atribu\'ido \`as suas vari\'aveis. Isso
nos permite classificar f\'ormulas de acordo com sua tabela
verdade. Esse ser\'a o assunto da pr\'oxima se\c{c}\~ao.

\subsection{Classificando F\'ormulas}

O objetivo desta se\c{c}\~ao \'e descrever uma maneira de classificar
f\'ormulas da l\'ogica proposicional de acordo com o valor de sua
tabela verdade.

\begin{Definition}[Tautologia e Contradi\c{c}\~ao]
Dizemos que uma f\'ormula da l\'ogica \'e uma tautologia se esta \'e
sempre verdadeira independente do valor l\'ogico de suas
vari\'aveis. Por sua vez, uma f\'ormula \'e uma contradi\c{c}\~ao se
esta \'e sempre falsa independente do valor de suas vari\'aveis.
\end{Definition}
\begin{Example}
A f\'ormula $A\land B\to A \lor B$ \'e uma tautologia, pois \'e sempre
verdadeira independente dos valores das vari\'aveis $A$ e $B$, como
mostrado pela tabela verdade seguinte:
\[
\begin{array}{|c|c|c|c|c|}
\hline
A & B & A \land B & A \lor B & A \land B \to A \lor B \\ \hline
F & F & F & F & T \\
F & T & F & T & T \\
T & F & F & T & T \\
T & T & T & T & T \\ \hline
\end{array}
\]
Como exemplo de uma contradi\c{c}\~ao considere a f\'ormula
$(A\to B) \land (A \land \neg B)$ e sua tabela verdade:
\[
\begin{array}{|c|c|c|c|c|c|}
  \hline
  A & B & \neg B & A \land \neg B & A \to B & (A \to B) \land (A \land
  \neg B) \\ \hline
  F & F & T & F & T & F \\
  F & T & F & F & T & F \\
  T & F & T & T & F & F \\
  T & T & F & F & T & F \\ \hline
\end{array}
\]
\end{Example}
\begin{Definition}[F\'ormula Satisfat\'ivel, False\'avel e Contigente]
  Uma f\'ormula da l\'ogica \'e dita ser satisfat\'ivel se existe uma
  maneira de atribuir valores l\'ogicos \`as suas vari\'aveis de
  maneira a torn\'a-la verdadeira. Uma f\'ormula \'e dita ser
  false\'avel se existe uma maneira de atribuir valores l\'ogicos \`as
  suas vari\'aveis de maneira a torn\'a-la falsa. Finalmente, dizemos
  que uma f\'ormula \'e contingente se esta \'e satisfat\'ivel e
  false\'avel simultaneamente.
\end{Definition}
\begin{Example}
Para ilustrar os conceitos de f\'ormula satisfat\'ivel, false\'avel e
contingente, considere a f\'ormula $A \lor B \to \neg A$ e sua tabela
verdade:
\[
\begin{array}{|c|c|c|c|c|}
  \hline
  A & B & \neg A & A \lor B & A \lor B \to \neg A \\ \hline
  F & F & T & F & T \\
  F & T & T & T & T \\
  T & F & F & T & F \\
  T & T & F & T & F \\ \hline
\end{array}
\]
Com isso temos que $A \lor B \to \neg A$ \'e satisfaz\'ivel, pois para
$A = F$ e $B = F$ temos que esta f\'ormula \'e verdadeira. De maneira
similar, para $A = T$ e $B = T$ temos que esta f\'ormula \'e falsa e,
portanto, false\'avel. Como, $A \lor B \to \neg A$ \'e satisfaz\'ivel
e false\'avel temos que esta pode ser classificada como contingente.
\end{Example}

Uma das aplica\c{c}\~oes dos conceitos anteriores \'e a determinar
quando duas f\'ormulas $\alpha,\beta$ da l\'ogica proposicional s\~ao
equivalentes. Dizemos que duas f\'ormulas s\~ao equivalentes se estas
possuem o mesmo valor l\'ogico para a mesma atribui\c{c}\~ao de
valores \`as suas vari\'aveis, isto \'e se $\alpha \leftrightarrow
\beta$ \'e uma tautologia.

\subsection{Limita\c{c}\~oes de tabelas verdade}\label{limitacao-tabela-verdade}

Tabelas verdade s\~ao um m\'etodo simples para determinar a
satisfazibilidade de f\'ormulas da l\'ogica proposicional, pois estas
denotam de maneira direta o significado de conectivos e
f\'ormulas. Por\'em, a simplicidade de tabelas verdade possui um
grande complicador: estas possuem tamanho exponencial sobre o n\'umero
de vari\'aveis em uma dada f\'ormula.

Em exemplos anteriores, mostramos tabelas verdade para f\'ormulas que
possu\'iam duas vari\'aveis. Todas estas tabelas possu\'iam 4
linhas. Considere a seguinte tabela para uma f\'ormula com $3$
vari\'aveis:
\[
\begin{array}{|c|c|c|c|c|}
  \hline
  A & B & C & A \land B & A \land B \land C \\ \hline
  F  & F & F & F & F \\
  F & F & T & F & F \\
  F & T & F & F & F \\
  F  & T & T & F & F \\
  T & F  & F & F & F \\
  T & T & F & T & F \\
  T & F & T & F & F \\
  T & T & T & T & T\\ \hline
\end{array}
\]
esta possui 8 linhas. De forma geral, a tabela verdade para f\'ormulas
contendo $n$ vari\'aveis possuir\'a $2^n$ linhas, o que limita a
utilização de tabelas verdades  para soluç\~oes de problemas
pr\'aticos.

\subsection{Consequ\^encia l\'ogica}\label{consequencia-logica}

A no\c{c}\~ao de consequ\^encia l\'ogica \'e um dos mais importantes
conceitos no estudo de l\'ogica. Informalmente, a consequ\^encia
l\'ogica expressa quando um argumento l\'ogico \'e considerado
v\'alido. Dizemos que argumentos s\~ao v\'alidos se sua conclus\~ao
\'e uma consequ\^encia de suas premissas (tamb\'em chamadas de hip\'oteses), em que tanto a conclus\~ao
quanto as premissas s\~ao proposi\c{c}\~oes. A pr\'oxima
defini\c{c}\~ao descreve de maneira precisa o que \'e uma
consequ\^encia l\'ogica.

\begin{Definition}[Consequ\^encia L\'ogica]
Dizemos que uma f\'ormula $\alpha$ \'e consequ\^encia l\'ogica de um
conjunto de f\'ormulas $\Gamma$, $\Gamma \models \alpha$, se, e
somente se sempre que toda f\'ormula em $\Gamma$ for verdadeira,
$\alpha$ tamb\'em o \'e. Isto \'e, se
\[
     \left(\bigwedge_{\varphi\in\Gamma}\varphi\right)\to \alpha
\]
\'e uma tautologia.
\end{Definition}

Uma maneira para determinarmos se uma f\'ormula $\alpha$ \'e
consequ\^encia l\'ogica de um conjunto $\Gamma$ \'e construir uma
tabela verdade. O seguinte exemplo ilustra esse uso de tabelas
verdade.
\begin{Example}
Considere as seguintes proposi\c{c}\~oes:
\begin{itemize}
  \item Se hoje for segunda-feira, irei a reuni\~ao.
  \item Hoje \'e segunda-feira.
   \item Hoje Irei a reuni\~ao.
\end{itemize}
Note que essas proposi\c{c}\~oes podem ser modeladas pelas seguintes
f\'ormulas:
\begin{itemize}
  \item $A \to B$
  \item $A$
  \item $B$
\end{itemize}
em que a vari\'avel $A$ denota ``Hoje \'e segunda-feira'' e $B$,
``Hoje irei a reuni\~ao''.

De acordo com a interpreta\c{c}\~ao usual da l\'ingua portuguesa,
temos que ``Hoje irei a reuni\~ao'' \'e uma consequ\^encia de ``Se
hoje for segunda-feira, irei a reuni\~ao'' e ``hoje \'e
segunda-feira''. Mas ser\'a que a defini\c{c}\~ao formal de
consequ\^encia l\'ogica, coincide com a no\c{c}\~ao usual de
consequ\^encias dedutivas utilizadas coloquialmente na l\'ingua
portuguesa?

Considerando as f\'ormulas $A\to B$, $A$ e $B$ que
modelam estas proposi\c{c}\~oes citadas, temos que se  $B$ \'e uma
consequ\^encia de $A \to B$ e $A$ se $[(A\to B)\land A]\to B$ \'e uma
tautologia, o que pode ser verificado pela tabela verdade abaixo:
\begin{table}[h]
\begin{tabular}{|c|c|c|c|c|}
\hline
$A$ & $B$ & $A\to B$ & $(A\to B)\land A$ & $[(A\to B)\land A]\to B$\\
\hline
\F   & \F    &     \T       &      \F                     & \T \\
\F   & \T    &     \T       &     \F                      & \T \\
\T   & \F    &     \F       &     \F                      & \T \\
\T   & \T   &     \T        &    \T                       & \T \\ \hline
\end{tabular}
\centering
\end{table}

Desta maneira, temos que a f\'ormula $B$ \'e uma consequ\^encia
l\'ogica de $A \to B$ e $A$.
\end{Example}

Evidentemente, utilizar tabelas verdade para determinar
consequ\^encias l\'ogicas possui o inconveniente de que tabelas
verdade s\~ao exponenciais no n\'umero de vari\'aveis presentes em uma
f\'ormula, logo, mesmo para senten\c{c}as envolvendo poucas
proposi\c{c}\~oes, o uso de tabelas verdade para verificar a validade
de argumentos \'e impratic\'avel, como j\'a apresentado na se\c{c}\~ao
\ref{limitacao-tabela-verdade}. Na se\c{c}\~ao
\ref{deducao-natural-proposicional}, apresentaremos o sistema de
dedu\c{c}\~ao natural para l\'ogica proposicional que permite
verificar consequ\^encias l\'ogicas sem a constru\c{c}\~ao de uma
tabela verdade.

\subsection{Exerc\'icios}

\begin{enumerate}
        \item Obtenha o conjunto de subf\'ormulas de cada f\'ormula a
          seguir utilizando a defini\c{c}\~ao \ref{cap1sub}.
        \begin{enumerate}
           \item $P\lor Q \rightarrow Q \lor P$
           \item $((P\land Q)\lor (P\land R))\leftrightarrow(P\land (Q\lor R))$
           \item $(P\rightarrow Q) \land P\land \neg Q$
           \item $(P \rightarrow Q)\land \neg P \rightarrow Q$
        \end{enumerate}
	\item Construa tabelas verdade para as f\'ormulas a seguir e
          classifique-as como sendo tautologias, conting\^encias ou contradi\c{c}\~oes:
	\begin{enumerate}
		\item $(A\rightarrow B)\leftrightarrow\neg A\lor B$
		\item $(A\land B)\lor C\rightarrow A\land(B\lor C)$
		\item $A\land\neg (\neg A\lor \neg B)$
		\item $A\land B\rightarrow\neg A$
		\item $(A\rightarrow B)\rightarrow[(A\lor C)\rightarrow (B\lor C)]$
		\item $A\rightarrow(B\rightarrow A)$
		\item $(A\land B)\leftrightarrow(\neg B\lor \neg A)$
	\end{enumerate}
        \item Determine se as seguintes f\'ormulas s\~ao ou n\~ao
          equivalentes.
         \begin{enumerate}
		\item $P\leftrightarrow Q$ e $(P\rightarrow Q)\land(\neg P\rightarrow \neg Q)$
		\item $(P\land\neg Q)\lor (\neg P\land Q)$ e $(P\lor Q)\land\neg(P\land Q)$
          \end{enumerate}
          \item Suponha que voc\^e possua um algoritmo que a partir de
            uma f\'ormula $\alpha$ da l\'ogica responda sim se esta
            \'e satisfaz\'ivel e não, caso contr\'ario. Explique
            como usar esse algoritmo para determinar se uma f\'ormula
            \'e uma:
            \begin{enumerate}
              \item Tautologia
              \item Contradi\c{c}\~ao
             \end{enumerate}
\end{enumerate}


\section{Dedu\c{c}\~ao Natural para L\'ogica Proposicional}\label{deducao-natural-proposicional}

A dedu\c{c}\~ao natural \'e um sistema formal que permite a
dedu\c{c}\~ao de consequ\^encias l\'ogicas sem a necessidade de
substituir vari\'aveis por valores l\'ogicos ou avaliar express\~oes.
O formalismo de dedu\c{c}\~ao natural \'e intensivamente estudado por
cientistas da computa\c{c}\~ao, uma vez que este \'e o formalismo
subjacente a ferramentas para verifica\c{c}\~ao de provas por
computador como Coq.

De maneira simples, a dedu\c{c}\~ao natural consiste de um conjunto de
regras que permitem estabelecer a validade de argumentos representados
como sequentes, que s\~ao definidos a seguir.

\begin{Definition}[Sequente]
Sejam $\alpha_1,...,\alpha_n,\varphi$ f\'ormulas bem formadas da
l\'ogica proposicional. A nota\c{c}\~ao
$\alpha_1,...,\alpha_n\,\vdash\,\varphi$ \'e denominada de sequente e
representa que $\varphi$ pode ser deduzida a partir de
$\alpha_1,...,\alpha_n$ utilizando as regras da dedu\c{c}\~ao natural.
\end{Definition}

Como argumentos s\~ao formados por premissas e uma conclus\~ao, temos
que no sequente
\[\alpha_1,...,\alpha_n\,\vdash\,\varphi\]
o conjunto formado pelas f\'ormulas $\alpha_i$, $1\leq i \leq n$,
s\~ao as premissas e
$\varphi$ a conclus\~ao do argumento representado.

Para determinar a validade de argumentos utilizando dedu\c{c}\~ao
natural, devemos ser capazes de inferir a conclus\~ao a partir das
premissas, utilizando as regras da dedu\c{c}\~ao natural. Regras da
dedu\c{c}\~ao natural s\~ao expressas escrevendo as premissas acima de
uma linha horizontal que as separam da conclus\~ao.

\[
      \infer
           {\text{Conclus\~ao}}
          {\text{F\'ormula}_1,...,\text{F\'ormula}_n}
\]
Esta nota\c{c}\~ao expressa, intuitivamente, que se formos capazes de
determinar a validade de cada uma das f\'ormulas $\text{Formula}_i,\:
1\leq i \leq n$, ent\~ao a Conclus\~ao tamb\'em ser\'a verdadeira.

A maioria das regras da dedu\c{c}\~ao natural podem ser dividas em duas categorias.
Regras de introdu\c{c}\~ao s\~ao aquelas nas quais um novo conectivo
\'e inclu\'ido na f\'ormula da conclus\~ao e s\~ao utilizadas para
construir express\~oes mais complexas a partir de outras mais simples.
Por sua vez, regras de elimina\c{c}\~ao possuem como premissa uma
f\'ormula com um certo conectivo e este \'e removido da
conclus\~ao. Estas regras s\~ao utilizadas para decompor express\~oes
complexas em express\~oes mais simples.

Visando simplificar a quantidade de regras para a dedu\c{c}\~ao
natural, utilizaremos apenas os conectivos de disju\c{c}\~ao ($\lor$),
conjun\c{c}\~ao ($\land$), implica\c{c}\~ao ($\to$) e a constante
($\bot$). Esta conven\c{c}\~ao n\~ao compromete a expressividade da
l\'ogica pois os conectivos de nega\c{c}\~ao, bicondicional e a
constante $\top$ podem ser definidos da seguinte maneira:
\[
\begin{array}{lcl}
   \neg A & \equiv & A \to \bot \\
   A \leftrightarrow B & \equiv & (A \to B) \land (B \to A) \\
  \top   & \equiv & \bot \to \bot
\end{array}
\]

\'E f\'acil verificar, utilizando tabelas verdade, que as
abrevia\c{c}\~oes anteriores s\~ao realmente equivalentes.
As pr\'oximas se\c{c}\~oes descrevem cada uma das regras da
dedu\c{c}\~ao natural apresentando exemplos de sua utiliza\c{c}\~ao.

\subsection{Regra para identidade ($\Id$)}

A primeira regra da dedu\c{c}\~ao natural expressa um fato bastante
\'obvio: se voc\^e deseja provar que uma f\'ormula $\alpha$
\'e verdadeira e $\alpha$ \'e uma das f\'ormulas presentes no conjunto
de hip\'oteses, ent\~ao voc\^e pode conclu\'i-la utilizando a regra
$\Id$, que \'e apresentada a seguir.

\[
\infer[\Id]
         {\Gamma \vdash \alpha}
         {\alpha \in \Gamma}
\]

Por\'em, a utiliza\c{c}\~ao do conjunto de hip\'oteses $\Gamma$, pode
ser omitida, para facilitar a leitura das dedu\c{c}\~oes. Usando esta
nota\c{c}\~ao simplificada, podemos expressar a regra $\Id$, da
seguinte maneira:

\[
\infer[\Id]
         {\alpha}
         {}
\]

Note que na vers\~ao ``simplificada'' da regra, todas as refer\^encias
ao conjunto de hip\'oteses $\Gamma$ foram removidas, por\'em, s\'o
podemos utilizar esta regra se a f\'ormula $\alpha$ pertencer ao
conjunto de hip\'oteses.

\subsection{Regras para a conjun\c{c}\~ao ($\land$)}

\subsubsection{Introdu\c{c}\~ao da conjun\c{c}\~ao $\andI$}

De maneira simples, a regra de introdu\c{c}\~ao da conjun\c{c}\~ao
($\{\land\,I\}$), diz que se for poss\'ivel deduzir uma f\'ormula $\alpha$, a
partir de um conjunto de hip\'oteses (premissas) $\Gamma$ e tamb\'em
for poss\'ivel deduzir $\beta$ a partir deste mesmo conjunto de
hip\'oteses $\Gamma$, ent\~ao a partir de $\Gamma$ \'e poss\'ivel
inferir $\alpha\land \beta$.  Isto \'e expresso de maneira concisa pela
seguinte regra:
\[
    \infer[\andI]
             {\Gamma\,\vdash\,\alpha \land \beta}
             {\Gamma\,\vdash\,\alpha & \Gamma\,\vdash\, \beta}
\]
Como j\'a dito anteriormente, omitimos o conjunto de hip\'oteses $\Gamma$ obtendo a
seguinte vers\~ao simplificada desta regra:
\[
    \infer[\andI]
             {\alpha \land \beta}
             {\alpha & \beta}
\]
Para uma melhor compreens\~ao de como construir demonstra\c{c}\~oes
utilizando dedu\c{c}\~ao natural utilizando estas regras, considere os
seguintes exemplos.

\begin{Example}
    Como um primeiro exemplo, considere a tarefa de determinar a
    validade do seguinte sequente: $P, Q \,\vdash\, P \land Q$. Temos
    que neste sequente o conjunto de hip\'oteses \'e
    $\Gamma=\{P,Q\}$ e a conclus\~ao $P\land Q$. Este sequente pode
    ser provado utilizando a regra $\andI$, conforme a dedu\c{c}\~ao a seguir:

    \[
         \infer[\andI]
                  {P \land Q}
                  {P & Q}
    \]

    A mesma dedu\c{c}\~ao deixando o conjunto de hip\'oteses
    expl\'icito \'e apresentada abaixo:

    \[
         \infer[\andI]
                  {\{P,Q\}\,\vdash\,P \land Q}
                  {
                    \infer[\Id]
                             {\{P,Q\} \,\vdash\,P}
                             {P \in \{P,Q\}}
                    &
                    \infer[\Id]
                             {\{P,Q\} \,\vdash\,Q}
                             {Q \in \{P,Q\}}
                  }
    \]
    Note que o uso do conjunto de hip\'oteses torna as dedu\c{c}\~oes
    mais verbosas e, portanto, dificulta a leitura. Por isso, vamos
    utilizar a vers\~ao simplificada das regras, exceto em alguns
    exemplos como este.
\end{Example}

\begin{Example}
   Considere a tarefa de demonstrar a validade do seguinte sequente:
   $P,\,Q,\,R\,\vdash\,(P\land Q)\land R$. Para obter a conclus\~ao a
   partir das hip\'oteses, temos que utilizar a regra $\andI$ duas
   vezes, uma para deduzir $P \land Q$ e outra para deduzir $(P\land
   Q)\land R$, conforme apresentado na dedu\c{c}\~ao seguinte:
  \[
      \infer[\andI]
               {(P\land Q)\land R}
               {
                   \infer[\andI]
                            {P\land Q}
                            {
                              \infer[\Id]
                                       {P}{}
                               &
                               \infer[\Id]
                                        {Q}{}
                            }
                    &
                    \infer[\Id]
                            {R}{}
               }
  \]
\end{Example}

\subsubsection{Elimina\c{c}\~ao da conjun\c{c}\~ao  $\andEE$,
  $\andED$}

O conectivo de conjun\c{c}\~ao ($\land$) possui duas regras de
elimina\c{c}\~ao. Estas regras expressam o fato de que se sabemos que
$\alpha \land \beta$ \'e verdadeira, ent\~ao $\alpha$ \'e verdadeira e
$\beta$ \'e verdadeiro. A regra de elimina\c{c}\~ao da conjun\c{c}\~ao
\`a esquerda ($\andEE$) permite concluir $\alpha$ a partir de
$\alpha\land \beta$, isto \'e, mantemos a f\'ormula \`a esquerda. Por
sua vez, a regra de elimina\c{c}\~ao \`a direita permite concluir a
f\'ormula \`a direita do conectivo $\land$. Ambas as regras s\~ao
apresentadas a seguir.

\[
\begin{array}{cc}
     \infer[\andEE]
              {\alpha}
              {\alpha \land \beta}
      &
     \infer[\andED]
              {\beta}
              {\alpha \land \beta}
\end{array}
\]

Os exemplos seguintes ilustram a utiliza\c{c}\~ao destas regras.

\begin{Example}
     Utilizando as regras para conjun\c{c}\~ao podemos provar que $P,
     Q \land R \vdash P \land R$ \'e um sequente v\'alido. Note que
     para isso, utilizaremos as regras de introdu\c{c}\~ao e
     elimina\c{c}\~ao a esquerda para a conjun\c{c}\~ao, conforme
     ilustrado a seguir.
     \[
         \infer[\andI]
                  {P \land R}
                  {
                    \infer[\Id]
                            {P}{}
                     &
                     \infer[\andED]
                              {R}
                              {Q \land R}
                  }
     \]
\end{Example}
\begin{Example}
     Outro sequente que podemos provar utilizando as regras para
     conjun\c{c}\~ao \'e $P\land (Q\land R) \vdash (P \land Q) \land
     R$, cuja dedu\c{c}\~ao \'e apresentada a seguir:

    \[
         \infer[\andI]
                 {(P \land Q) \land R}
                {
                    \infer[\andI]
                             {P\land Q}
                             {
                                 \infer[\andEE]
                                          {P}
                                          {
                                             \infer[\Id]{P\land
                                               (Q\land R)} {}
                                         }
                                  &
                                  \infer[\andEE]
                                           {Q}
                                           {
                                              \infer[\andED]
                                                       {Q\land R}
                                                       {
                                                         \infer[\Id]{P\land
                                                           (Q\land
                                                           R)}{}
                                                       }
                                           }
                             }
                   &
                   \infer[\andED]
                           {R}
                           {
                              \infer[\andED]
                                       {Q \land R}
                                       {
                                         \infer[\Id]{P\land (Q\land
                                           R)}{}
                                        }
                           }
                }
    \]
    Inicialmente, utilizamos a regra $\andI$ para deduzir $(P\land
    Q)\land R$, a partir de $P\land Q$ e $R$. A dedu\c{c}\~ao de
    $P\land Q$ utiliza $\andI$ e tr\^es elimina\c{c}\~oes da
    conjun\c{c}\~ao sobre a hip\'otese $P \land (Q\land R)$. Para a
    dedu\c{c}\~ao de $R$, utilizamos duas elimina\c{c}\~oes da
    conjun\c{c}\~ao sobre $P\land (Q\land R)$.
\end{Example}

\subsection{Regras para a implica\c{c}\~ao ($\to$)}

\subsubsection{Elimina\c{c}\~ao da implica\c{c}\~ao ($\to\,E$)}

Em nosso cotidiano, provavelmente a regra de dedu\c{c}\~ao que mais
utilizamos \'e a regra de elimina\c{c}\~ao da implica\c{c}\~ao,
$\impE$. Esta regra afirma que se conseguirmos deduzir que $\alpha \to
\beta$ \'e verdade e que $\alpha$ \'e verdade, ent\~ao, utilizando a
regra $\impE$, podemos deduzir que $\beta$ possui o valor
verdadeiro. Esta regra \'e apresentada a seguir:

\[
    \infer[\impE]{\beta}
                        {\alpha \to \beta & \alpha}
\]
A regra de elimina\c{c}\~ao da implica\c{c}\~ao, $\impE$, \'e tamb\'em
conhecida como \textit{modus ponens}.
O pr\'oximo exemplo apresenta uma simples aplica\c{c}\~ao desta
regra.
\begin{Example}
  O sequente $A \to B, B\to C, A \vdash A \land C$ possui a seguinte
  demonstra\c{c}\~ao:
  \[
       \infer[\andI]{A \land C}
                          {
                            \infer[\Id]{A}{}
                            &
                            \infer[\impE]{C}
                                     {
                                       \infer[\Id]{B\to C}
                                                      {}
                                        &
                                        \infer[\impE]{B}
                                                            {
                                                              \infer[\Id]{A
                                                              \to B}{}
                                                              &
                                                              \infer[\Id]{A}{}
                                                            }
                                     }
                          }
  \]
\end{Example}

\subsubsection{Introdu\c{c}\~ao da implica\c{c}\~ao ($\impI$)}

A regra de introdu\c{c}\~ao da implica\c{c}\~ao, $\impI$, especifica
que para deduzirmos uma f\'ormula $\alpha\to\beta$, a partir de um
conjunto de hip\'oteses $\Gamma$, devemos obter uma prova de $\beta$
utilizando $\alpha$ como uma hip\'otese adicional. Esta regra \'e
apresentada abaixo:

\[
     \infer[\impI]{\Gamma \vdash \alpha \to \beta}
                        {\Gamma \cup \{\alpha\}\vdash \beta}
\]

Note que o efeito de utilizar a regra $\impI$ \'e adicionar o lado
esquerdo da implica\c{c}\~ao a ser deduzida como uma hip\'otese
adicional. O pr\'oximo exemplo ilustra a utiliza\c{c}\~ao desta regra.

\begin{Example}
   Considere a tarefa de deduzir que $\vdash A \land B \to A$. Note
   que de acordo com a regra $\impI$, devemos transformar o sequente
   $\vdash A \land B \to A$, no sequente $A \land B \vdash A$. Por sua
   vez, o sequente $A \land B \vdash A$ pode ser deduzido de maneira
   imediata utilizando $\andEE$. A dedu\c{c}\~ao completa \'e
   apresentada a seguir.
   \[
       \infer[\impI]{\vdash A \land B \to A}
                          {
                            \infer[\andEE]{\{A\land B\} \vdash A}
                                                 {
                                                   \infer[\Id]{\{A\land
                                                     B\} \vdash A
                                                     \land B}
                                                     {A\land B \in
                                                       \{A\land B\}}
                                                 }
                          }
   \]
Neste exemplo, pode-se perceber a utilidade do s\'imbolo $\vdash$,
tornar expl\'icita a separa\c{c}\~ao das hip\'oteses e da conclus\~ao
de um sequente. Antes de utilizarmos a regra $\impI$, o conjunto de
hip\'oteses deste sequente era vazio, isto \'e este sequente n\~ao
possu\'ia hip\'oteses. Usar a regra $\impI$ nos permitiu incluir o
lado esquerdo de $A\land B \to A$ ($A \land B$) no conjunto de
hip\'oteses, possibilitanto assim, o t\'ermino desta dedu\c{c}\~ao.
\end{Example}

Note que ao observarmos a dedu\c{c}\~ao do exemplo anterior, esta permite-nos pensar
que $A \land B$ \'e uma hip\'otese deste sequente, visto que aplicamos
a regra $\Id$ para deduz\'i-la. Por\'em, a f\'ormula $A \land B$ \'e
uma hip\'otese de ``visibilidade local'', cujo \'unico prop\'osito \'e
possibilitar a demonstra\c{c}\~ao do sequente $A\land B \vdash
A$. Assim que obtemos a dedu\c{c}\~ao desejada, a hip\'otese adicional
pode ser ``descartada'', isto \'e, eliminada do conjunto de
hip\'oteses do sequente em quest\~ao. Em nosso exemplo, a visibilidade
da hip\'otese adicional $A \land B$ \'e toda a dedu\c{c}\~ao acima do
uso da regra $\impI$.

Em dedu\c{c}\~oes maiores, manter, de maneira consistente, quais
hip\'oteses tempor\'arias est\~ao ou n\~ao visi\'iveis em um dado
ponto da demonstra\c{c}\~ao pode ser uma tarefa complicada. Uma
solu\c{c}\~ao para isso \'e manter o conjunto de hip\'oteses em todo
ponto da dedu\c{c}\~ao, mas como j\'a argumentamos diversas vezes
neste texto, isso prejudica o entendimento das demonstra\c{c}\~oes.
Visando
facilitar a legibilidade das dedu\c{c}\~oes, vamos numerar cada
hip\'otese tempor\'aria e indicar com o mesmo n\'umero a regra que a
introduziu. Isto permitir\'a definir a visibilidade de uma
hip\'otese como sendo toda a dedu\c{c}\~ao ``acima'' da regra que a
introduziu. Al\'em disso, omitiremos o conjunto de hip\'oteses da
regra de introdu\c{c}\~ao da implica\c{c}\~ao, escrevendo-a da
seguinte forma simplificada:
\[
     \infer[\impI]{\alpha\to \beta}
                        {\alpha \vdash \beta}
\]
em que a nota\c{c}\~ao ``$\alpha \vdash \beta$'', denota ``deduzir
$\beta$ utilizando $\alpha$ como hip\'otese adicional''.
Utilizando a conven\c{c}\~ao de numera\c{c}\~ao de hip\'oteses locais
e vers\~ao simplificada da regra $\impI$, a dedu\c{c}\~ao do
exemplo anterior, ficaria como:

   \[
       \infer[\impI^1]{A \land B \to A}
                          {
                            \infer[\andEE]{A}
                                                 {
                                                   \infer[\Id]{A
                                                     \land B^1}
                                                     {}
                                                 }
                          }
   \]
Note que a visibilidade de $A\land B$ \'e delimitada pela regra
$\impI$ que foi numerada com o valor $1$. Este mesmo valor foi
utilizado para marcar a utiliza\c{c}\~ao de $A\land B$ quando da
utiliza\c{c}\~ao da regra $\Id$, para explicitar o uso de uma
hip\'otese tempor\'aria.

A seguir apresentamos mais dois exemplos para estas regras.

\begin{Example}
Neste exemplo, mostraremos que se sabe-se que $A \to B$ e $B \to C$
s\~ao verdadeiras, ent\~ao $A \to C$ tamb\'em ser\'a verdadeira. Tal
fato \'e expresso pelo seguinte sequente: $\{A\to B,B\to C\}\vdash A
\to C$. A dedu\c{c}\~ao \'e apresentada abaixo:
\[
     \infer[\impI^1]
              {A \to C}
              {
                \infer[\impE]
                         {C}
                         {
                           \infer[\Id]{B\to C}{} &
                           \infer[\impE]
                                    {B}
                                    {
                                      \infer[\Id]{A\to B}{} &
                                      \infer[\Id]{A^1}{}
                                    }
                         }
              }
\]
\end{Example}
O pr\'oximo exemplo apresenta um resultado quase imediato utilizando a
dedu\c{c}\~ao anterior, este resultado \'e conhecido em muitos livros
de l\'ogica como \textit{modus tollens}.
\begin{Example}
O \textit{modus tollens} especifica que se $A\to B$ e $\neg B$ s\~ao
f\'ormulas verdadeiras, ent\~ao, $\neg A$ tamb\'em deve ser
verdadeira. Note que $\neg B \equiv B\to \bot$ . Ent\~ao, usando o
resultado do exemplo anterior, temos que
a partir de $A\to B$ e $B \to \bot$ podemos deduzir $A \to \bot$.
Evidentemente, podemos deduzir este resultado sem apelar para o
exemplo anterior. Deixamos essa dedu\c{c}\~ao como um exerc\'icio para
o leitor.
\end{Example}

\subsection{Regras para a disjun\c{c}\~ao ($\lor$)}

\subsubsection{Introdu\c{c}\~ao da disjun\c{c}\~ao $\orIE$, $\orID$}

As regras para introdu\c{c}\~ao da disjun\c{c}\~ao estabelecem
condi\c{c}\~oes que devem ser satisfeitas para que possamos deduzir
uma f\'ormula contendo o conectivo $\lor$. Caso $\alpha$ seja
verdadeiro, temos que  $\alpha \lor \beta$ e $\beta\lor \alpha$
tamb\'em devem ser verdadeiros, para qualquer f\'ormula $\beta$.
Como basta uma das f\'ormuas ser verdadeira para que toda a
disjun\c{c}\~ao tamb\'em o seja, temos duas regras para introduzir o
conectivo $\lor$, apresentadas a seguir:
\[
\begin{array}{cc}
    \infer[\orIE]{\Gamma \vdash \alpha \lor \beta}
                      {\Gamma \vdash \alpha}
    &
    \infer[\orID]{\Gamma \vdash \alpha \lor \beta}
                      {\Gamma \vdash \beta}
\end{array}
\]
Assim como em regras anteriores, omitiremos o conjunto de hip\'oteses
$\Gamma$, obtendo as seguintes formas simplificadas das regras anteriores:
\[
\begin{array}{cc}
    \infer[\orIE]{\alpha \lor \beta}
                      {\alpha}
    &
    \infer[\orID]{\alpha \lor \beta}
                      {\beta}
\end{array}
\]
O pr\'oximo exemplo ilustra a utiliza\c{c}\~ao destas regras.
\begin{Example}
Considere a tarefa de demonstrar o seguinte sequente: $\{P \land Q\}
\vdash P \lor Q$. Como a conclus\~ao deste sequente possui o conectivo
$\lor$, podemos iniciar sua prova utilizando uma das regras de
introdu\c{c}\~ao da disjun\c{c}\~ao, conforme ilustrado na
dedu\c{c}\~ao abaixo:
\[
     \infer[\orID]{P \lor Q}
                        {
                          \infer[\andED]{Q}
                                                {
                                                  \infer[\Id]{P\land Q}{}
                                                }
                        }
\]
\end{Example}
Por\'em, esta n\~ao \'e a \'unica maneira de se demonstrar esse
sequente. Podemos deduz\'i-lo iniciando com a regra $\orIE$, conforme
apresentado a seguir:
\[
     \infer[\orIE]{P \lor Q}
                        {
                          \infer[\andEE]{P}
                                                {
                                                  \infer[\Id]{P\land Q}{}
                                                }
                        }
\]
Como existem duas demonstra\c{c}\~oes para esse sequente, qual destas
seria a correta? A resposta \'e simples: Ambas! O fato de um sequente
admitir mais de uma demonstra\c{c}\~ao permite-nos ``escolher'' entre
qualquer uma destas. Isso quer dizer, que podemos considerar
diferentes dedu\c{c}\~oes de um sequente como sendo ``iguais''. Este
fato de considerar diferentes dedu\c{c}\~oes de um mesmo sequente como
sendo iguais \'e conhecido como \textit{irrelev\^ancia de
  provas}\footnote{Do ingl\^es: Proof Irrelevance.}.  Note que, como
podemos considerar ambas as provas como sendo equivalentes, n\~ao h\'a
necessidade de se construir ambas ou de se escolher uma em detrimento
da outra.

\subsubsection{Elimina\c{c}\~ao da disjun\c{c}\~ao $\orE$}

A regra de elimina\c{c}\~ao da disjun\c{c}\~ao especifica o que pode
ser deduzido a partir do fato de que $\alpha \lor \beta$ \'e
verdadeira. Note que, se $\alpha\lor \beta$ \'e uma f\'ormula
verdadeira, n\~ao podemos concluir diretamente que $\alpha$ ou $\beta$
tamb\'em devem ser verdadeiras. Isto decorre do significado da
disjun\c{c}\~ao. Se $\alpha \lor \beta$ \'e verdadeira, temos que
$\alpha$ pode ser verdadeira ou $\beta$ pode ser verdadeira ou
ambas\footnote{Lembre-se da tabela verdade para a disjun\c{c}\~ao!}!

Contudo, se sabemos que $\alpha \lor \beta$ \'e verdadeira e que uma
f\'ormula $\gamma$ pode ser inferida a partir de $\alpha$ e tamb\'em
de $\beta$, podemos ent\~ao deduzir que $\gamma$ deve ser verdadeira.
Estas id\'eias s\~ao ilustradas pela regra de elimina\c{c}\~ao da
disjun\c{c}\~ao, $\orE$, apresentada a seguir:
\[
   \infer[\orE]{\Gamma \vdash \gamma}
                    {\Gamma \vdash \alpha \lor \beta &
                      \Gamma\cup\{\alpha\} \vdash \gamma &
                      \Gamma\cup\{\beta\} \vdash \gamma}
\]
Note que, assim como a regra $\impI$, a elimina\c{c}\~ao da
disjun\c{c}\~ao permite a inclus\~ao de novas hip\'oteses. Novamente,
utilizaremos a conven\c{c}\~ao de numerar as hip\'oteses tempor\'arias
de maneira que sua visibilidade na demonstra\c{c}\~ao fique
evidente. Eliminando as ocorr\^encias do conjunto de hip\'oteses
$\Gamma$, podemos reescrever a regra $\orE$, da seguinte maneira:
\[
   \infer[\orE]{\gamma}
                    {\alpha \lor \beta &
                      \alpha \vdash \gamma &
                      \beta \vdash \gamma}
\]
\begin{Example}
Neste exemplo, considere a tarefa de demonstrar o seguinte sequente: $\{A\lor B, A \to C, B\to C\}
\vdash C$. Para deduzir $C$, utilizaremos a regra $\orE$ sobre $A \lor
B$ para obter hip\'oteses que possibilitem deduzir $C$ a partir das
implicações $A\to C$ e $B\to C$. Esta dedução é apresentada abaixo:
\[
    \infer[\orE^1]{C}
                  {\infer[\Id]{A \lor B}{} &
                    \infer[\impE]{C}
                             {
                                \infer[\Id]{A\to C}{} & A^1
                             }
                     &
                     \infer[\impE]{C}
                              {
                                 \infer[\Id]{B\to C}{} & B^1
                              }
                  }
\]
\end{Example}
\begin{Example}
Vamos considerar um dedução utilizando $\orE$ um pouco mais complexa:
provar o sequente $\{(A \land B) \lor (A \land C)\}\vdash b \lor C$.

Para demonstrar o sequente anterior, utilizaremos a regra $\orE$ sobre
a hipótese $(A \land B)\lor (A \land C)$ e utilizaremos as hipóteses
introduzidas por esta regra para deduzir $B\lor C$.

\[
\infer[\orE^1]{B \lor C}
        {
          \infer[\Id]{(A \land B)\lor (A\land C)}{} &
          \infer[\orIE]{B \lor C}
                  {
                     \infer[\andED]{B}
                             {
                               \infer[\Id]{A\land B^1}{}
                             }
                  } &
          \infer[\orID]{B \lor C}
                  {
                     \infer[\andED]{C}
                             {
                               \infer[\Id]{A\land C^1}{}
                             }
                  }
        }
\]

\end{Example}

\subsection{Contradição}

A regra da contradição especifica que podemos deduzir \emph{qualquer
  fórmula} a partir de uma dedução de $\bot$ (falso).

\[
\infer[\ctr]{\Gamma \vdash \alpha}
                 {\Gamma \vdash \bot}
\]
Esta regra expressa a ``inutilidade'' de uma hipótese falsa, pois caso
$\bot$ seja dedutível, então qualquer fórmula pode ser deduzida. Os
próximos exemplos apresentam aplicações desta regra.

\begin{Example}
O sequente $\{A, \neg A \}\vdash B$ é provável utilizando as regras
$\ctr$, $\impE$ e $\Id$, conforme a dedução abaixo:

\[
\infer[\ctr]{B}
        {
           \infer[\impE]{\bot}
                    {
                         \infer[\Id]{\neg A}{} &
                         \infer[\Id]{A}{}
                    }
        }
\]
Note que implicitamente esta demonstração utiliza o fato de que $\neg
A$ é equivalente a $A \to \bot$ para utilizar a regra $\impE$.
\end{Example}
O próximo exemplo mostra uma propriedade do conectivo $\lor$: se
$A\lor B$ é verdadeiro e $\neg A$ também o é, temos que
necessariamente $B$ deve ser verdadeiro.
\begin{Example}
O sequente $\{A \lor B,\neg A\}\vdash B$ possui a seguinte
demonstração:
\[
\infer[\orE^1]{B}
         {
           \infer[\Id]{A\lor B}{} &
           \infer[\ctr]{B}
                  {
                    \infer[\impE]{\bot}
                            {
                              \infer[\Id]{\neg A}{} &
                              \infer[\Id]{A^1}{}
                            }
                  } &
           \infer[\Id]{B^1}{}
         }
\]
\end{Example}

\subsection{Reductio ad Absurdum}

A regra \emph{Reduction ad Absurdum} (redução ao absurdo) especifica
que se conseguirmos deduzir $\bot$ a partir de $\neg \alpha$ então $\alpha$ deve
ser uma fórmula verdadeira.

\[
    \infer[\raa]{\Gamma \vdash \alpha}
                     {\Gamma\cup\{\neg \alpha\}\vdash \bot}
\]

A regra $\raa$ é a formalização lógica de um conceito amplamente
utilizando em matemática: o de prova por contradição. Se desejamos
demonstrar que $\alpha$ é verdadeiro, basta supor que este é falso e a
partir desta suposição obter um resultado absurdo (contradição).

Utilizando esta regra, podemos deduzir algumas demonstrações que não
seriam possíveis utilizando outras regras da dedução natural. O
seguinte exemplo, ilustra essa situação.

\begin{Example}
O seguinte sequente somente pode ser demonstrado utilizando a regra
$\raa$: $\vdash \neg\neg A \to A$. A demonstração deste sequente é
apresentada a seguir:
\[
     \infer[\impI^1]{\neg\neg A \to A}
           {
             \infer[\raa^2]{A}
                      {
                        \infer[\impE]{\bot}
                                 {
                                   \infer[\Id]{\neg\neg A^1}{} &
                                   \infer[\Id]{\neg A^2}{}
                                 }
                      }
           }
\]
Note que ao utilizarmos a regra de $\raa$, adquirimos como hipótese
adicional $\neg A$ que possibilita a utilização da regra $\impE$, que
conclui a demonstração.
\end{Example}

\subsection{Exerc\'icios}

\begin{enumerate}
	\item Prove os seguintes sequentes usando dedu\c{c}\~ao
          natural. Tente demonstrá-los sem utilizar a regra $\raa$.
	\begin{enumerate}
		\item $\{(P\land Q)\land R,\, S\land T\}\vdash\,Q\land S$
		\item $\{(P\land Q)\land R\}\,\vdash\,(P\land R)\lor Z$
		\item $\{Q\rightarrow (P\rightarrow R),\, \neg R,\, Q\,\} \vdash\,\neg P$
		\item $\{P\}\,\vdash\, Q\rightarrow(P\land Q)$
		\item $\{(P\rightarrow R)\land (Q\rightarrow R),\, P\land Q\}\,\vdash\, Q\land R$
		\item $\{P\rightarrow Q, R\rightarrow S\}\vdash (P\lor R)\rightarrow (Q\lor S)$
		\item $\{Q\rightarrow R\}\vdash (P\rightarrow Q)\rightarrow(P\rightarrow R)$
		\item $\{(P\land Q)\lor(P\land R)\}\vdash P\land(Q\lor
                  R)$
                  \item $\{\neg(A \lor B)\}\vdash \neg A \land \neg B$
                  \item $\{\neg A \land \neg B\}\vdash \neg (A\lor B)$
                  \item $\{\neg(A \land B)\}\vdash \neg A \lor \neg B$
                  \item $\{\neg A \lor \neg B\}\vdash \neg (A\land B)$
	\end{enumerate}
        \item Demonstre os seguintes sequentes. Nestes sequentes você
          terá que utilizar a regra $\raa$ para deduzí-los.
        \begin{enumerate}
             \item $\{A \to B\}\vdash \neg A \lor B$
             \item $\vdash (\neg B \to \neg A)\to (A \to B)$
             \item $\vdash (A \to B)\to (\neg A \to B) \to B$
             \item $\vdash ((A \to B)\to A)\to A$
        \end{enumerate}
\end{enumerate}

\section{Álgebra Booleana para Lógica Proposicional}

Até o presente momento, apresentamos duas abordagens para o estudo da
lógica proposicional: uma baseada na semântica, utilizando tabelas
verdade e, uma abordagem sintática utilizando regras de inferência da
dedução natural. Além destas duas abordagens, existe uma terceira,
a álgebra booleana, que é uma abordagem axiomática para o estudo da
lógica.

A álgebra booleana consiste de um conjunto de leis que estabelecem
quando duas fórmulas podem ser consideradas
logicamente equivalentes. A próxima definição apresenta as condições
para que duas fórmulas sejam consideradas equivalentes.

\begin{Definition}[Equivalência Lógica]
Dizemos que duas
fórmulas $\alpha$ e $\beta$ são equivalentes, $\alpha \equiv \beta$, se estas possuem o mesmo valor lógico para
uma mesma atribuição de valores às suas variáveis. Podemos verificar
se $\alpha$ e $\beta$ são equivalentes se a
fórmula $\alpha\leftrightarrow\beta$ é uma tautologia.
\end{Definition}

\begin{Example}
As fórmulas $\neg \neg A$ e $A$ são equivalentes,
o que pode ser verificado pela seguinte tabela verdade:
\begin{table}[h]
  \begin{tabular}{|c|c|c|c|} \hline
      $A$ & $\neg A$ & $\neg \neg A$ & $\neg\neg A\leftrightarrow A$
      \\ \hline
      $F$ & $T$ & $F$ & $T$\\
      $T$ & $F$ & $T$ & $T$ \\ \hline
   \end{tabular}
   \centering
\end{table}
\end{Example}
Evidentemente, o uso de tabelas verdade para determinar a equivalência
lógica de duas fórmulas possui o inconveniente de que o número de
linhas de uma tabela verdade é exponencial no número de variáveis de
uma fórmula. O objetivo desta seção é apresentar a álgebra booleana
que permite determinar se duas fórmulas são equivalentes sem a
utilização de tabelas verdade.

A álgebra booleana é uma forma de raciocínio algébrico sobre
fórmulas, o que, de maneira simples, permite: 1) mostrar que duas
fórmulas são iguais por meio de uma sequência de igualdades \footnote{A
noção de ``sequência de igualdades'' é formalizada em termos da
seguinte propriedade, denominada \emph{transitividade}: se $a= b$ e $b
= c$ então $a = c$.} e; 2) se $x = y$ e você possui uma expressão que
possui ocorrências de $x$, você pode substituir ocorrências de
$x$ por $y$ nesta expressão. Essa última propriedade é conhecida como
\emph{indiscernibilidade de valores iguais}\footnote{Essa regra é
  comumente citada na comunidade de lógica e teoria de tipos como
  regra de Leibniz, que pode ser expressa da seguinte maneira: Seja
  $P$ uma propriedade qualquer, se sabemos que $x = y$ e que a
  propriedade $P$ é verdadeira para $x$, então esta também deve ser
  para o valor $y$.}. Como exemplo, considere
que $x = y + 2$ e que $z = 2 \times x + 5$, usando a propriedade de
indiscernibilidade de iguais, temos que
\[
\begin{array}{lcl}
z & = & \\
2 \times x + 5 & = & \{\text{pela def. de }z\}\\
2\times (y + 2) & = & \{\text{por }x = y + 2\}\\
2\times y + 4
\end{array}
\]
Note que neste exemplo, envolvendo aritmética, apresentamos uma
justificativa para cada passo da dedução de que as fórmulas $z$ e
$2\times y + 4$ são equivalentes. Considera-se uma boa prática rotular
cada passo de uma equação com a justificativa que permite concluir a
próxima expressão da cadeia de igualdades. Adotaremos essa convenção
durante a apresentação do conteúdo de álgebra booleana.

\subsection{Leis da Álgebra Booleana}

A álgebra boolena consiste de um conjunto de equações que descreve
propriedades algébricas de proposições. Estas equações são normalmente
chamadas de ``leis'', uma vez que estas são aceitas como verdadeiras a
priori. Dizemos que uma proposição é uma lei se esta é sempre
verdadeira, independente dos valores lógicos atribuídos às suas
variáveis\footnote{Note que, desta maneira, toda tautologia pode ser
  vista como uma lei.}.

As leis da álgebra boolena são análogas às leis da álgebra
convencional. Existem leis que especificam que certos valores agem
como elementos neutros, outras para dizer que certas operações são
associativas e que algumas operações distribuem sobre outras. Na
álgebra convencional, temos que a adição é associativa, isto é que
para quaisquer valores numéricos $x$, $y$ e $z$ temos que $x + (y + z)
= (x + y) + z$. A multiplicação, por sua vez, distribui com respeito a
adição, isto é, para $x$, $y$ e $z$ temos que $x \times (y + z) = (x \times
y) + (x \times z)$.  Leis similares existem para os conectivos da
lógica proposicional. As próximas seções apresentarão estas regras.

\subsection{Leis Envolvendo Constantes}

As leis envolvendo constantes especificam como as constantes lógicas
interagem com os conectivos $\land$ e $\lor$.

\begin{table}[h]
    \begin{tabular}{|cccl|}
        \hline
             $\alpha \land \bot$ & $\equiv$ & $\bot$ &
             $\{\land-\text{null}\}$\\
             $\alpha \lor \top$ & $\equiv$ & $\top$ &
             $\{\lor-\text{null}\}$\\
             $\alpha \land \top$ & $\equiv$ & $\alpha$ & $\{\land-\text{identidade}\}$\\
             $\alpha \lor \bot$ & $\equiv$ & $\alpha$ & $\{\lor-\text{identidade}\}$\\
        \hline
    \end{tabular}
    \centering
\end{table}

O seguinte exemplo mostra como estas leis podem ser utilizadas para
demonstrar a equivalência de duas fórmulas.

\begin{Example}
As fórmulas $(A \lor \bot)\land(B \lor \top)$ e $A$ são
equivalentes, o que pode ser demonstrado pela seguinte dedução
algébrica:
\[
\begin{array}{lcl}
(A \lor \bot)\land(B \lor \top) & \equiv &\{\lor-\text{identidade}\} \\
A \land (B\lor \top) & \equiv & \{\lor-\text{null}\}\\
A \land \top & \equiv & \{\land-\text{null}\}\\
A & &
\end{array}
\]
\end{Example}

A partir do exemplo anterior, podemos perceber a estrutura de uma
demonstração de equivalência entre duas fórmulas. Como o objetivo é
demonstrar uma igualdade, a dedução consiste de uma sequência de
igualdades. A sequência inicia com o lado esquerdo da igualdade que
desejamos deduzir e termina com o lado direito. Além disso, perceba
que cada passo da dedução é justificado pelo nome da regra utilizada.


\subsection{Leis Elementares dos Conectivos $\land$ e $\lor$}

As leis seguintes descrevem que os conectivos $\land$ e $\lor$ são
idempotentes, associativos e comutativos. Se $\alpha$ é uma fórmula e
$\circ$ um conectivo binário, dizemos que $\circ$ é idempotente se
$\alpha\circ \alpha = \alpha$. Por sua vez, dizemos que $\circ$ é
associativo se, para fórmulas $\alpha,\beta$ e $\gamma$, temos que
$\alpha \circ (\beta\circ \gamma) = (\alpha \circ \beta)\circ
\gamma$. Finalmente, dizemos que $\circ$ é comutativo, se a seguinte
igualdade é verdadeira, para fórmulas $\alpha$ e $\beta$: $\alpha
\circ \beta = \beta \circ \alpha$.

A tabela seguinte apresenta essas propriedades, em termos para os
conectivos $\land$ e $\lor$.

\begin{table}[h]
    \begin{tabular}{|cccl|}
         \hline
             $\alpha \land \alpha$ & $\equiv$ & $\alpha$ &
             $\{\land-\text{idempotente}\}$\\
             $\alpha \lor \alpha$ & $\equiv$ & $\alpha$ &
             $\{\lor-\text{idempotente}\}$\\
             $\alpha \land \beta$ & $\equiv$ & $\beta \land \alpha$ &
             $\{\land-\text{comutativo}\}$\\
             $\alpha \lor \beta$ & $\equiv$ & $\beta \lor \alpha$ &
             $\{\lor-\text{comutativo}\}$\\
             $\alpha\land(\beta\land\gamma)$ & $\equiv$ &  $(\alpha \land
             \beta)\land\gamma$ & $\{\land-\text{associativo}\}$\\
             $\alpha\lor(\beta\lor\gamma)$ & $\equiv$ &  $(\alpha \lor
             \beta)\lor\gamma$ & $\{\lor-\text{associativo}\}$\\
         \hline
    \end{tabular}
     \centering
\end{table}

O próximo exemplo mostram como utilizar essas regras para
demonstrar uma equivalência.

\begin{Example}
As fórmulas $(\bot \land A)\lor B$ e $B$ são equivalentes, o que pode
ser confirmado pela seguinte demonstração:
\[
\begin{array}{lcl}
(\bot\land A)\lor B & \equiv & \{\land-\text{comutativo}\}\\
(A\land \bot)\lor B & \equiv & \{\land-\text{null}\}\\
\bot \lor B & \equiv & \{\lor-\text{comutativo}\}\\
B\lor \bot & \equiv & \{\lor-\text{identidade}\}\\
B
\end{array}
\]
\end{Example}

Encerraremos esta seção apresentando um conjunto de leis que descreve
o relacionamento dos conectivos $\land$ e $\lor$ com a negação lógica
e leis que mostram que estes conectivos distribuem um sobre o outro.
Em álgebra, sabemos que a multiplicação distribui sobre a adição, isto
é, para quaisquer valores numéricos $a,b$ e $c$ temos que $a \times (b
+ c) = (a\times b) + (a\times c)$. A tabela seguinte apresenta estas
leis:

\begin{table}[h]
  \begin{tabular}{|cccl|}
    \hline
      $\alpha\land (\beta \lor \gamma)$ & $\equiv$ & $(\alpha \land
      \beta)\lor(\alpha \land \gamma)$ &
      $\{\land-\text{distribui}-\lor\}$\\
        $\alpha\lor (\beta \land \gamma)$ & $\equiv$ & $(\alpha \lor
      \beta)\land(\alpha \lor \gamma)$ & $\{\lor-\text{distribui}-\land\}$\\
      $\neg (\alpha\land \beta)$ & $\equiv$ & $\neg \alpha \lor \neg \beta$ &
      $\{\text{DeMorgan}-\land\}$\\
      $\neg (\alpha\lor \beta)$ & $\equiv$ & $\neg \alpha \land \neg \beta$ &
      $\{\text{DeMorgan}-\lor\}$\\
      \hline
  \end{tabular}
  \centering
\end{table}

As duas últimas regras apresentadas são conhecidas como leis de
DeMorgan e estas possuem explicações intuitivas. Por exemplo,
$\neg (\alpha \land \beta)$ especifica que ``não é verdade que
$\alpha$ e $\beta$ são simultaneamente verdadeiros'' logo, temos que
ou $\alpha$ é falso ou $\beta$ é falso. Pode-se explicar a regra
$\text{DeMorgan}-\lor$ de maneira similar.

\subsection{Leis Envolvendo a Negação}

As leis algébricas relacionadas com o conectivo de negação são bem
diretas e refletem o significado deste conectivo:

\begin{table}[h]
  \begin{tabular}{|cccl|}
    \hline
    $\neg \top$ & $\equiv$ & $\bot$ & $\{\text{negação-}\top\}$\\
    $\neg \bot$ & $\equiv$ & $\top$ & $\{\text{negação-}\bot\}$\\
    $\alpha\land\neg\alpha$ & $\equiv$ & $\bot$ &
    $\{\text{complemento-}\land\}$\\
    $\alpha\lor\neg\alpha$ & $\equiv$ & $\top$ &
    $\{\text{complemento-}\lor\}$\\
    $\neg(\neg\alpha)$ & $\equiv$ & $\alpha$ & $\{\text{dupla-negação}\}$\\
    \hline
  \end{tabular}
  \centering
\end{table}

Utilizando as leis apresentadas até o momento, podemos demonstrar a
equivalência $A\land \neg (B \lor A) \equiv \bot$.

\begin{Example}
A fórmula $A \land \neg (B \lor A)$ é equivalente a $\bot$, conforme
demonstrado abaixo:
\[
    \begin{array}{lcl}
      A \land \neg (B \lor A) & \equiv & \{\text{DeMorgan}-\lor\}\\
      A \land \neg B \land \neg A & \equiv & \{\land-\text{comutativo}\}\\
      A \land \neg A \land \neg B & \equiv & \{\text{complemento}-\land\}\\
      \bot \land \neg B & \equiv & \{\land-\text{comutativo}\}\\
      \neg B \land \bot & \equiv & \{\land-\text{null}\}\\
      \bot & &
    \end{array}
\]
\end{Example}

\subsection{Leis Envolvendo a Implicação e Bicondicional}

As leis algébricas para a implicação e bicondicional mostram como
expressar esses conectivos em termos de outros.

\begin{table}[h]
  \begin{tabular}{|cccl|}
    \hline
        $\alpha \to \beta$ & $\equiv$ & $\neg \alpha \lor \beta$ &
        $\{\text{implicação}\}$\\
        $\alpha \leftrightarrow \beta$ & $\equiv$ & $(\alpha \to \beta)\land
        (\beta \to \alpha)$ & \{\text{bicondicional}\}\\
    \hline
  \end{tabular}
  \centering
\end{table}
Usando as equivalências até aqui apresentadas, podemos demonstrar
alguns resultados conhecidos da lógica, como por exemplo, a
contrapositiva de uma implicação, que é apresentada no próximo
exemplo.
\begin{Example}
A fórmula $A\to B$ é equivalente a $\neg B \to \neg A$, conforme
demonstrado a seguir:
\[
\begin{array}{lcl}
  A \to B & \equiv & \{\text{implicação}\}\\
 \neg A \lor B & \equiv & \{\text{dupla-negação}\}\\
\neg A \lor \neg(\neg B) & \equiv & \{\lor-\text{comutativo}\}\\
 \neg(\neg B) \lor \neg A & \equiv & \{\text{implicação}\}\\
 \neg B \to \neg A & &\\
\end{array}
\]
Note que no último passo, utilizamos a regra da implicação, que
especifica que  $\alpha \to \beta \equiv \neg \alpha \lor \beta$, sobre a
fórmula $\neg(\neg B)\lor \neg A$. Neste caso, temos que $\alpha =
\neg B$ e $\beta = \neg A$, o que nos permite deduzir que
$\neg(\neg B)\lor \neg A = \neg B \to \neg A$.
\end{Example}

Uma das aplicações da álgebra boolena é permitir expressar algumas
funções lógicas (conectivos) em termos de outros. Por exemplo,
utilizando as leis de DeMorgan, podemos expressar o conectivo $\land$
em termos de $\lor$ e $\neg$, conforme apresentado abaixo:
\[
\begin{array}{lc}
  A \land B & \equiv \\
\neg\neg A \land \neg\neg B & \equiv \\
\neg(\neg A \lor \neg B)
\end{array}
\]
Uma vez que podemos expressar conectivos em termos de outros, cabe
perguntar se existe um conjunto ``mínimo'' de conectivos a partir dos
quais é possível definir todos os outros. A próxima definição
formaliza este conceito.

\begin{Definition}[Conjunto Completo de Conectivos]
Seja
$\mathcal{C}\subseteq\{\bot,\neg,\lor,\land,\to,\leftrightarrow\}$ um
conjunto de conectivos. Dizemos que $\mathcal{C}$ é completo para
$\{\bot,\neg,\lor,\land,\to,\leftrightarrow\}$ se é possível expressar
todos os conectivos não presentes em $\mathcal{C}$ em termos dos
conectivos presentes no conjunto $\mathcal{C}$ e variáveis.
\end{Definition}

\begin{Example}
O conjunto $\{\neg,\lor\}$ é completo, pois é possível expressar todos
os outros conectivos da lógica utilizando apenas $\neg$, $\lor$ e
variáveis, conforme apresentado abaixo:
\begin{enumerate}
     \item A constante $\top$ pode ser representada como $\alpha\lor\neg\alpha$.
     \item A constrante $\bot$ pode ser representada como $\neg (\alpha
       \lor \neg \alpha)$. Sabe-se que $\alpha \lor \neg \alpha\equiv \top$, pela regra
       $\{\lor-\text{null}\}$, e que $\bot \equiv \neg \top$. Logo, $\bot \equiv \neg
       (\alpha \lor \neg \alpha)$, para qualquer fórmula $\alpha$.
     \item Conectivo de conjunção pode ser represento por $\neg$ e
       $\lor$ da seguinte maneira, em que $\alpha$ e $\beta$ são
       fórmulas quaisquer:
\[
\begin{array}{lc}
  \alpha \land \beta & \equiv \\
\neg\neg \alpha \land \neg\neg \beta & \equiv \\
\neg(\neg \alpha \lor \neg \beta)
\end{array}
\]
    \item A implicação lógica possui representação direta:
      $\alpha\to\beta \equiv \neg\alpha\lor\beta$.
    \item Finalmente, representamos o conectivo bicondicional da
      seguinte forma:
\[
\begin{array}{lc}
    \alpha\leftrightarrow\beta & \equiv\\
   (\alpha \to \beta)\land (\beta \to \alpha) & \equiv \\
   (\neg \alpha \lor \beta) \land (\neg \beta\lor \alpha)
\end{array}
\]
    Agora, utilizaremos o fato de que deduzimos em um item anterior
    que  $A \land B \equiv \neg (\neg A \lor \neg B)$ e, consideraremos que
   $A = \neg\alpha \lor \beta$ e $B = \neg \beta \lor \alpha$. Com
   isso, obtemos:
\[
\neg(\neg (\neg\alpha \lor \beta)\lor \neg (\neg \beta \lor \alpha))
\]
que é a representação do conectivo bicondicional em termos de $\neg$ e $\lor$.
\end{enumerate}
\end{Example}

\subsection{Exercícios}

\begin{enumerate}
	\item Prove as seguintes equival\^encias usando racioc\'inio \'algebrico:
	\begin{enumerate}
		\item $(A\lor B)\land B\equiv B$
		\item $(\neg A\land B)\lor (A\land\neg B)\equiv (A\lor B)\land \neg (A\land B)$
		\item $((A\rightarrow B)\rightarrow A)\rightarrow A\equiv T$
	\end{enumerate}
        \item Mostre que o conjunto $\{\neg,\land\}$ é completo para
          os conectivos
          $\{\bot,\neg,\lor,\land,\to,\leftrightarrow\}$.
        \item Mostre que o conjunto $\{\neg,\to\}$ é completo para os
          conectivos $\{\bot,\neg,\lor,\land,\to,\leftrightarrow\}$.
        \item Descreva como podemos determinar que uma fórmula é uma
          tautologia utilizando leis da álgebra booleana.
        \item O conectivo de negação conjunta,
          $\alpha\downarrow\beta$, é definido como verdadeiro sempre
          que $\alpha$ ou $\beta$ são falsos.
         \begin{enumerate}
              \item Apresente a tabela verdade para
                $\alpha\downarrow\beta$.
              \item Mostre que o conjunto $\{\bot,\downarrow\}$ é completo
                para os conectivos $\{\bot,\neg,\lor,\land,\to,\leftrightarrow\}$.
         \end{enumerate}
\end{enumerate}


\section{Formas Normais}

A lógica proposicional possui diversas aplicações práticas na ciência
da computação. Porém, algumas destas aplicações exigem que as fórmulas
possuam uma certa estrutura. Nesta seção apresentaremos duas formas
normais que são amplamente utilizadas: a forma normal disjuntiva,
aplicada em minimização de fórmulas lógicas e a formal normal
conjuntiva, utilizada como entrada para algoritmos para teste de
satisfazibilidade.

\subsection{Forma Normal Conjuntiva}

A definição seguinte apresenta condições para que uma dada fórmula bem
formada da lógica proposicional seja considerada uma fórmula na forma
normal conjuntiva.

\begin{Definition}[Forma Normal Conjuntiva]\label{FNC}
Definimos o conjunto de fórmulas da lógica proposicional na forma
normal conjuntiva (FNC), da seguinte maneira:
\begin{enumerate}
  \item As constantes lógicas $\bot$ e $\top$ são fórmulas na forma
    normal conjuntiva.
  \item Seja $\mathcal{V}$ o conjunto de todas as variáveis da lógica
    proposicional. Seja $\alpha \in \mathcal{V}$ e
    $\neg\alpha\in\mathcal{V}$ são fórmulas na forma normal
    conjuntiva. Dá-se o nome de literal a fórmulas que são variáveis
    ou negação de variáveis.
  \item Seja $\{l_1,...,l_n\}$ um conjunto de $n\geq 0$
    literais. Então, \[\bigvee_{i=1}^nl_i\] é uma fórmula na forma
    normal conjuntiva. Dá-se o nome de cláusula a fórmulas que
    consistem apenas de uma disjunção de literais.
  \item Seja $\{C_1,...,C_n\}$ um conjunto de $n\geq 0$
    cláusulas. Então, \[\bigwedge_{i=1}^nC_i\] é uma fórmula na forma
    normal conjuntiva.
\end{enumerate}
\end{Definition}
Os exemplos a seguir ilustram o conceito de fórmulas na forma normal conjuntiva.
\begin{Example}
São exemplos de fórmulas na forma normal conjuntiva:
\begin{itemize}
     \item $\bot,\top,\alpha,\neg\alpha$, em que $\alpha$ é uma
       variável.
     \item Sejam $x_1$, $x_2$ e $x_3$ variáveis da lógica
       proposicional. Então $\neg x_1 \lor x_2 \lor x_3$ e $x_1\lor
       \neg x_2 \lor \neg x_3$ são cláusulas e, portanto, são, cada
       uma, fórmulas na forma normal conjuntiva.
     \item Sejam $\neg x_1 \lor x_2 \lor x_3$, $x_1\lor
       \neg x_2 \lor \neg x_3$ e $x_4 \lor \neg x_5$ cláusulas. Então, a fórmula
       $(\neg x_1 \lor x_2 \lor x_3)\land (x_1\lor
       \neg x_2 \lor \neg x_3) \land (x_4 \lor \neg x_5)$ está na forma normal conjuntiva.
\end{itemize}
Os próximos exemplos mostram fórmulas que não estão na forma normal
conjuntiva.
\begin{itemize}
    \item $A \to B \land (C \leftrightarrow B)$, não está na FNC, pois possui o conectivo de
      implicação e bicondicional.
    \item $\neg (A \land B)$, não está na FNC, pois a negação não está
      associada somente a variáveis.
    \item $A\lor(B \land C)$, não está na FNC, pois, de acordo com a
      definição \ref{FNC}, o conectivo $\lor$ ocorre apenas em
      cláusulas e não no nível mais externo da fórmula.
\end{itemize}
\end{Example}
A definição \ref{FNC} mostra que apenas um subconjunto das fórmulas
bem formadas da lógica proposicional pode ser considerada na
FNC. Porém, este fato não constitui uma limitação já que toda fórmula
bem formada da lógica proposicional pode ser convertida para FNC,
seguindo-se o algoritmo seguinte.
\begin{algorithm}
  \begin{algorithmic}[1]
     \Require{Uma fórmula bem formada $\alpha$}
     \For{Todas as subfórmulas $\beta,\gamma,\varphi$ de $\alpha$}
         \State{Eliminar bicondicionais usando $\beta \leftrightarrow
           \gamma = (\beta\to\gamma)\land (\gamma\to\beta)$}
         \State{Eliminar implicações usando $\beta \to \gamma = \neg
           \beta \lor \gamma$}
         \State{Empurrar as negações usando as leis de
           DeMorgan\[\begin{array}{lcl}\neg(\beta\land \gamma) & = &
             \neg \beta \lor \neg \gamma\\ \neg(\beta \lor \gamma) & =
             & \neg \beta \land \neg \gamma\end{array}\]  até que
           estas fiquem associadas a variáveis}
         \State{Elimine duplas negações usando $\neg \neg \beta  =
           \beta$}
         \State{Aplique, enquanto possível, a distributividade do
           $\lor$ sobre $\land$:\[\beta\lor(\gamma \land \varphi) =
           (\beta\lor \gamma)\land(\beta \lor \varphi))\]}
     \EndFor
     \State{A fórmula resultante estará na forma normal conjuntiva.}
  \end{algorithmic}
  \caption{Convertendo para a Forma Normal Conjuntiva}
  \label{fncconvert}
\end{algorithm}
O próximo exemplo mostra como converter uma fórmula da lógica para a
forma normal conjuntiva.
\begin{Example}
Mostraremos como converter a fórmula $A\to (B\land C)$ para a forma
normal conjuntiva. Para isso, executaremos, passo a passo, o algoritmo
\ref{fncconvert}.
\begin{enumerate}
  \item O passo 1 do algoritmo é desnecessário, já que a fórmula $A
    \to (B\land C)$ não possui bicondicionais.
  \item No passo 2, eliminamos a implicação:
    \[A \to (B\land C) = \neg A \lor (B\land C)\]
  \item No passo 3, não há nada a fazer já que a negação está
    associada somente a variáveis.
   \item No passo 4, não há nada a fazer já que não há dupla negações.
   \item No passo 5, temos que distribuir o $\lor$ sobre o $\land$,
     obtendo: \[\neg A \lor (B\land C) = (\neg A \lor B)\land (\neg A
     \lor C)\]
\end{enumerate}
\end{Example}
\begin{Example}
Mostraremos como converter a fórmula $\neg (A\leftrightarrow B)$ para
a forma normal conjuntiva, passo a passo.
\begin{enumerate}
  \item No passo 1, eliminamos o bicondicional
    obtendo: \[\neg(A\leftrightarrow B) = \neg[(A \to
    B)\land (B\to A)]\]
  \item No passo 2, eliminamos a implicação obtendo:
   \[\neg[(A \to
    B)\land (B\to A)  = \neg[(\neg A \lor B) \land (\neg B \lor A)]\]
  \item No passo 3, movemos as negações utilizando as leis de
    DeMorgan:
   \[
       \begin{array}{lc}
         \neg[(\neg A \lor B) \land (\neg B \lor A)] & =\\
         \neg (\neg A \lor B) \lor \neg (\neg B \lor A) & = \\
         (\neg\neg A \land \neg B) \lor (\neg\neg B\land\neg A)
       \end{array}
   \]
   \item No passo 4, eliminamos as duplas negações:
  \[         (\neg\neg A \land \neg B) \lor (\neg\neg B\land\neg A) =
  (A \land \neg B) \lor (B\land\neg A)\]
  \item Finalmente, no passo 5 distribuímos o $\lor$ sobre $\land$:
  \[
      \begin{array}{lc}
        (A \land \neg B) \lor (B\land\neg A) & = \\
        ((A\land \neg B) \lor B) \land ((A \land \neg B) \lor \neg A)
        & = \\
        ((A \lor B) \land (\neg B \lor B)) \land ((A \lor \neg A)
        \land (\neg B \lor \neg A))
      \end{array}
  \]
\end{enumerate}
Obtendo assim a fórmula equivalente a $\neg (A\leftrightarrow B)$ na
forma normal conjuntiva.
\end{Example}


\subsection{Forma Normal Disjuntiva}


A próxima definição especifica as condições para que uma fórmula
esteja na forma normal disjuntiva (FND).

\begin{Definition}[Forma Normal Disjuntiva]\label{FND}
Definimos o conjunto de fórmulas da lógica proposicional na forma
normal disjuntiva  (FND), da seguinte maneira:
\begin{enumerate}
  \item As constantes lógicas $\bot$ e $\top$ são fórmulas na forma
    normal disjuntiva.
  \item Seja $\mathcal{V}$ o conjunto de todas as variáveis da lógica
    proposicional. Seja $\alpha \in \mathcal{V}$ e
    $\neg\alpha\in\mathcal{V}$ são fórmulas na forma normal
    disjuntiva. Dá-se o nome de literal a fórmulas que são variáveis
    ou negação de variáveis.
  \item Seja $\{l_1,...,l_n\}$ um conjunto de $n\geq 0$
    literais. Então, \[\bigwedge_{i=1}^nl_i\] é uma fórmula na forma
    normal disjuntiva. Dá-se o nome de cláusula dual a fórmulas que
    consistem apenas de uma conjunção de literais.
  \item Seja $\{C_1,...,C_n\}$ um conjunto de $n\geq 0$
    cláusulas duais. Então, \[\bigvee_{i=1}^nC_i\] é uma fórmula na forma
    normal disjuntiva.
\end{enumerate}
\end{Definition}
O seguinte algoritmo pode ser utilizado para converter qualquer
fórmula da lógica proposicional para a forma normal disjuntiva.

\begin{algorithm}
  \begin{algorithmic}[2]
     \Require{Uma fórmula bem formada $\alpha$}
     \For{Todas as subfórmulas $\beta,\gamma,\varphi$ de $\alpha$}
         \State{Eliminar bicondicionais usando $\beta \leftrightarrow
           \gamma = (\beta\to\gamma)\land (\gamma\to\beta)$}
         \State{Eliminar implicações usando $\beta \to \gamma = \neg
           \beta \lor \gamma$}
         \State{Empurrar as negações usando as leis de
           DeMorgan\[\begin{array}{lcl}\neg(\beta\land \gamma) & = &
             \neg \beta \lor \neg \gamma\\ \neg(\beta \lor \gamma) & =
             & \neg \beta \land \neg \gamma\end{array}\]  até que
           estas fiquem associadas a variáveis}
         \State{Elimine duplas negações usando $\neg \neg \beta  =
           \beta$}
         \State{Aplique, enquanto possível, a distributividade do
           $\land$ sobre $\lor$:\[\beta\land(\gamma \lor \varphi) =
           (\beta\land \gamma)\lor(\beta \land\varphi))\]}
     \EndFor
     \State{A fórmula resultante estará na forma normal disjuntiva.}
  \end{algorithmic}
  \caption{Convertendo para a Forma Normal Disjuntiva}
  \label{fndconvert}
\end{algorithm}
A seguir, os próximos exemplos mostram a conversão de duas fórmulas
para a forma normal disjuntiva utilizando o algoritmo
\ref{fndconvert}.

\begin{Example}
Mostraremos como converter a fórmula $A\to (B\land C)$ para a forma
normal disjuntiva. Para isso, executaremos, passo a passo, o algoritmo
\ref{fndconvert}.
\begin{enumerate}
  \item O passo 1 do algoritmo é desnecessário, já que a fórmula $A
    \to (B\land C)$ não possui bicondicionais.
  \item No passo 2, eliminamos a implicação:
    \[A \to (B\land C) = \neg A \lor (B\land C)\]
  \item No passo 3, não há nada a fazer já que a negação está
    associada somente a variáveis.
   \item No passo 4, não há nada a fazer já que não há dupla negações.
   \item No passo 5, não há nada a fazer já que não há conjunções a
     serem distribuídas sobre disjunções.
\end{enumerate}
\end{Example}
\begin{Example}
Mostraremos como converter a fórmula $\neg (A\leftrightarrow B)$ para
a forma normal disjuntiva, passo a passo.
\begin{enumerate}
  \item No passo 1, eliminamos o bicondicional
    obtendo: \[\neg(A\leftrightarrow B) = \neg[(A \to
    B)\land (B\to A)]\]
  \item No passo 2, eliminamos a implicação obtendo:
   \[\neg[(A \to
    B)\land (B\to A)  = \neg[(\neg A \lor B) \land (\neg B \lor A)]\]
  \item No passo 3, movemos as negações utilizando as leis de
    DeMorgan:
   \[
       \begin{array}{lc}
         \neg[(\neg A \lor B) \land (\neg B \lor A)] & =\\
         \neg (\neg A \lor B) \lor \neg (\neg B \lor A) & = \\
         (\neg\neg A \land \neg B) \lor (\neg\neg B\land\neg A)
       \end{array}
   \]
   \item No passo 4, eliminamos as duplas negações:
  \[         (\neg\neg A \land \neg B) \lor (\neg\neg B\land\neg A) =
  (A \land \neg B) \lor (B\land\neg A)\]
  \item No passo 5, não há o que fazer pois não há conjunções para
    serem distribuídas sobre disjunções. Logo, a fórmula
   \[(A \land \neg B) \lor (B\land\neg A)\]
   está na forma normal disjuntiva.
\end{enumerate}
\end{Example}

\subsection{Exercícios}

\begin{enumerate}
        \item Para cada uma das fórmulas a seguir, apresente fórmulas
          equivalentes na forma normal conjuntiva e disjuntiva.
	\begin{enumerate}
		\item $(A\land B)\lor C\rightarrow A\land(B\lor C)$
		\item $A\land\neg (\neg A\lor \neg B)$
		\item $A\land B\rightarrow\neg A$
		\item $(A\rightarrow B)\rightarrow[(A\lor C)\rightarrow (B\lor C)]$
		\item $A\rightarrow(B\rightarrow A)$
		\item $(A\land B)\leftrightarrow(\neg B\lor \neg A)$
	\end{enumerate}
\end{enumerate}

\ifbool{COQ}{
\section{Lógica Proposicional no Assistente de Provas Coq}

Até o presente momento, estudamos a lógica proposicional de um ponto
de vista matemático. Nesta seção, utilizaremos o assistente de provas
Coq para consolidar os conceitos estudados nesta ferramenta
computacional.

\subsection{Sintaxe da Lógica Proposicional em Coq}\label{proplogiccoqsyntax}

Nesta seção descrevemos como representar a sintaxe de fórmulas da
lógica proposicional em \texttt{Coq}. Para isso, utilizaremos um tipo de dados
para representar fórmulas bem formadas da lógica. A
estrutura deste tipo é apenas uma transcrição da definição do conjunto
de fórmulas bem formadas.

\begin{lstlisting}
Inductive Formula : Set :=
   | Falsum  : Formula
   | Verum : Formula
   | Var : nat -> Formula
   | Not : Formula -> Formula
   | And  : Formula -> Formula -> Formula
   | Or    : Formula -> Formula -> Formula
   | Implies : Formula -> Formula -> Formula
   | Iff  : Formula -> Formula -> Formula.
\end{lstlisting}
O significado de cada um dos construtores deste tipo é direto. Os
valores \texttt{Verum} e \texttt{Falsum} representam as constantes
$\top$ e $\bot$, respectivamente.

Variáveis são representadas pelo
construtor \texttt{Var} que espera como parâmetro um
número natural que representa a variável em questão. A representação de
variáveis usando números naturais é uma prática comum na formalização de propriedades de
linguagens de programação e outros formalismos baseados em lógica e,
por isso, adotamos essa representação.

Os demais construtores representam os conectivos da lógica
proposicional: o construtor \texttt{Not} representa $(\neg)$;
\texttt{And} denota $(\land)$; \texttt{Or} , o conectivo $(\lor)$; \texttt{Implies},
$\to$ e \texttt{Iff}, $(\leftrightarrow)$. Note que não incluímos
explicitamente uma maneira de expressar expressões entre
parênteses. Isso se deve que utilizaremos a própria sintaxe de Coq
para representar fórmulas e, com isso, utilizamos parênteses da
linguagem de Coq para impor precedência entre subexpressões de uma
dada fórmula. O próximo exemplo ilustrará esse conceito.

\begin{Example}
Vamos representar a fórmula $\neg A \to B$ utilizando o tipo de dados
\texttt{Formula}. Utilizando a convenção de precedência entre os
conectivos da lógica, temos que a fórmula $\neg A \to B$ deve ser
interpretada como $(\neg A) \to B$, pois a negação possui precedência
maior que a implicação lógica. Desta forma, a seguinte definição,
representa esta fórmula em Coq:
\begin{lstlisting}
Definition exemplo1 : Formula :=
              Implies (Not (Var 0)) (Var 1).
\end{lstlisting}
Note que representamos a variável $A$ pela fórmula \texttt{Var 0} e
$B$ por \texttt{Var 1}.
\end{Example}

Evidentemente, escrever fórmulas utilizando o tipo de dados
\texttt{Formula} é tedioso. Neste sentido, o código fonte que
acompanha este material, define uma sintaxe alternativa para definir
valores do tipo \texttt{Formula}. A sintaxe alternativa é descrita
pela tabela a seguir:

\begin{table}[h]
  \begin{tabular}{|c|c|}
     \hline
         Sintaxe Convencional & Sintaxe Alternativa \\ \hline
         \texttt{Var}                 & \texttt{\#} \\
         \texttt{Not}                & \verb|:~| \\
         \texttt{And}                & \texttt{\&}\\
         \texttt{Or}                  & \texttt{|}\\
         \texttt{Implies}          & \texttt{->>}\\
         \texttt{Iff}                  & \texttt{<<->>}\\
     \hline
  \end{tabular}
  \centering
  \caption{Sintaxe alternativa para valores do tipo \texttt{Formula}.}
\end{table}
Utilizando a sintaxe alternativa, podemos escrever fórmulas de maneira
similar a notação matemática, escrevendo conectivos de forma infixa. A
seguir, apresentamos como a fórmula do exemplo anterior pode ser
escrita usando a sintaxe alternativa.
\begin{lstlisting}
Definition exemplo1' : Formula := :~ (# 0) ->> (# 1).
\end{lstlisting}
A principal função de uma definição formal da sintaxe de uma
linguagem é evitar a escrita de termos que não ``fazem sentido'', isto
é, que não possuem uma semântica definida. Ao utilizarmos um tipo de
dados Coq para representação da sintaxe de fórmulas, garantimos que
somente fórmulas que obedecem as regras sintáticas da linguagem podem
ser escritas. Isto é, utilizamos o verificador de tipos de Coq para
garantir que somente termos sintaticamente corretos possam ser
escritos. O próximo exemplo ilustra essa utilização de Coq.

\begin{Example}
Considere o termo $\neg (A \lor)$ que não é uma fórmula bem formada da
lógica proposicional, já que o conectivo $(\lor)$ está aplicado apenas
a um argumento. Este termo poderia ser escrito em Coq como:
\begin{lstlisting}
Definition exemplo2 : Formula := :~ ((# 0) |).
\end{lstlisting}
Porém, o compilador rejeita esta definição apresentando uma mensagem
de erro.
\end{Example}

\subsection{Semântica da Lógica Proposicional em Coq}

Nesta seção descreveremos como representar em Coq a semântica da
lógica proposicional. Primeiramente, serão definidas funções para
representar cada um dos conectivos da lógica proposicional e, em
seguida, será apresentada uma maneira de se calcular tabela verdades
para fórmulas.

\subsubsection{Representando o significado dos conectivos}\label{conectivoscoq}

De maneira simplista, conectivos da lógica podem ser vistos como
funções cujo domínio são valores booleanos ou pares de tais valores e
contra-domínio, um valor booleano. Por exemplo, o conectivo de negação
possui como domínio e contra-domínio o conjunto de valores
booleanos. Desta forma, podemos representá-lo como a seguinte
definição em Coq:
\begin{lstlisting}
Definition negB (a : bool) : bool :=
        if a then false else true.
\end{lstlisting}
No trecho de código anterior temos que a função que representa a
negação (\texttt{negB}) recebe um parâmetro de nome \texttt{a} e
retorna o valor lógico apropriado com base no valor de \texttt{a}.

Todos os demais conectivos da lógica são binários, isto é, possuem
dois parâmetros e, portanto, serão representados como funções que
possuem dois argumentos e retornam um valor booleano. Abaixo,
apresentamos a definição para a conjunção:
\begin{lstlisting}
Definition andB (a b : bool) : bool :=
    match a, b with
      | true , b' => b'
      | false , _ => false
    end.
\end{lstlisting}
A definição da função \texttt{andB} expressa diretamente o significado
de $a \land b$, caso $a$ seja verdadeiro, o resultado é o valor de
$b$, caso seja falso, o resultado também será falso. Abaixo,
apresentamos as definições dos conectivos $\lor$, $\to$ e
$\leftrightarrow$ que são análogas.
\begin{lstlisting}
Definition orB (a b : bool) : bool :=
    match a , b with
      | false, b' => b'
      | true, _   => true
    end.

Definition implB(a b : bool) : bool :=
    match a , b with
      | true , false => false
      | _ , _        => true
    end.

Definition iffB (a  b : bool) : bool :=
    match a , b with
      | false, false => true
      | true , true  => true
      | _ , _ => false
    end.
\end{lstlisting}
Tendo definido o significado de cada um dos conectivos lógicos, basta
definir uma função que atribui significado à sintaxe definida na seção
\ref{proplogiccoqsyntax}.

\subsubsection{Semântica de fórmulas}

Para definirmos um algoritmo para cálculo de tabelas verdade para
fórmulas da lógica em Coq, devemos primeiramente mostrar uma
função que, dada uma fórmula e uma atribuição de valores às variáveis
desta, calcula o valor booleano da fórmula para a atribuição em
questão.

Para isso, primeiramente, vamos definir um tipo para representar atribuições de
valores booleanos a variáveis:
\begin{lstlisting}
Definition Assign : Set := list bool.
\end{lstlisting}
Representaremos atribuições por uma lista de valores booleanos. Como
variáveis são representadas por números naturais, o valor lógico
presente em uma atribuição para a variável $n$ estará na $n$-ésima
posição desta lista. Além disso, utilizaremos a função
\texttt{nth}\footnote{A função \texttt{nth} está definida na biblioteca \texttt{List}.}
que, a partir de um número natural, uma lista e um valor padrão de
mesmo tipo que os elementos da lista em questão, retorna o $n$-ésimo
elemento desta lista ou o valor padrão, caso o índice passado seja
inválido.

A função que calcula o valor lógico de uma fórmula,
\texttt{eval}, é apresentada a seguir:
\begin{lstlisting}
Fixpoint eval (a : Assign)(f : Formula) : bool :=
    match f with
      | Falsum => false
      | Verum => true
      | Var v => nth v a false
      | Not f' => negB (eval a f')
      | And l r => andB (eval a l) (eval a r)
      | Or l r => orB (eval a l) (eval a r)
      | Implies l r => implB (eval a l) (eval a r)
      | Iff l r => iffB (eval a l) (eval a r)
    end.
\end{lstlisting}
Exceto para o caso de variáveis, o significado de cada uma das
equações é imediato. O resultado de avaliar as constantes \texttt{Falsum} e
\texttt{Verum} é \texttt{false} e \texttt{true}, respectivamente. Para
variáveis, utilizamos a função \texttt{nth} para obter o valor lógico
associado a esta. Finalmente, para cada um dos conectivos, calculamos
o resultado de avaliar suas subfórmulas e o valor lógico final é dado
pelas funções semânticas dos conectivos definidas na seção
\ref{conectivoscoq}. A seguir, apresentaremos um exemplo de como
utilizar a função \texttt{eval} para calcular o valor de uma fórmula
para uma certa atribuição.
\begin{Example}
Considere a tarefa de calcular o valor lógico da fórmula $\neg A \to (B
\land C)$ quando $A$ possui o valor $T$, $B$ o valor $T$ e $C$ o valor
$F$ utilizando a função \texttt{eval}.

Primeiramente, vamos definir um valor do tipo \texttt{Assign} que
representa a atribuição de valores para as variáveis desta fórmula.
Como variáveis representam a posição de seus valores lógicos na
atribuição, representaremos a variável $A$ pelo número $0$, $B$ por
$1$ e $C$ por $2$. Logo, a atribuição pode ser definida pela seguinte
lista de valores booleanos:
\begin{lstlisting}
Definition assignEx40 : Assign :=
          true :: true :: false :: nil.
\end{lstlisting}
A fórmula em questão é definida da seguinte maneira, utilizando o tipo
\texttt{Formula}:
\begin{lstlisting}
Definition formulaEx40 : Formula :=
         Implies (Not (Var 0)) (And (Var 1) (Var 2)).
\end{lstlisting}
Finalmente, podemos solicitar que o interpretador de Coq execute a
função \texttt{eval} sobre a fórmula anterior usando o seguinte
comando:
\begin{lstlisting}
Eval compute in eval assignEx40 formulaEx40.
\end{lstlisting}
que exibirá \texttt{true : bool} como resultado desta execução.
\end{Example}
Evidentemente, enumerar manualmente todas as atribuições e utilizar a
função \texttt{eval} para obter o valor lógico de uma fórmula é
impraticável. Neste sentido, no código Coq para este capítulo é
definida a função \texttt{truthTable}, que a partir de uma fórmula
calcula a tabela verdade desta. Omitiremos sua definição por esta
envolver conceitos de programação funcional que estão fora do escopo
deste trabalho. O leitor interessado pode acessar sua implementação no
código fonte (disponível on-line) que acompanha este material.

Note que o interpretador de Coq demora um tempo considerável para
calcular a tabela verdade de fórmulas simples. Isso se deve ao fato de
que o interpretador de Coq não é muito eficiente na realização de
computações (cálculos). Coq é uma ferramenta que foi construída com o
intuito de verificar deduções lógicas.

\subsection{Dedução Natural para Lógica Proposicional em Coq}



\subsection{Álgebra Booleana para Lógica Proposicional em Coq}

\subsection{Exercícios}

\begin{enumerate}
    \item Represente as seguintes fórmulas bem formadas da lógica
      proposicional como valores do tipo \texttt{Formula}:
   \begin{enumerate}
       \item $\neg A \land B \to C$
       \item $(A \to B) \land \neg (A \lor B \to C)$
       \item $A \to B \to C \leftrightarrow \bot$
       \item $A \land \neg A \to B$
       \item $A \lor B \land C$
   \end{enumerate}
\end{enumerate}

}{}

\section{Considerações Metamatemáticas}

A metamatemática consiste em utilizar técnicas matemáticas para o
estudo da própria matemática. Neste capítulo, apresentamos a sintaxe,
semântica e um sistema de provas para a lógica proposicional: a dedução
natural, que nos permite demonstrar consequências lógicas.

Apesar da dedução natural possuir uma semântica intuitiva, não apresentamos como este
se relacionam com a semântica da lógica proposicional. A relação de um certo sistema de
provas para um formalismo e a semântica deste é dada por propriedades
conhecidas como correção e completude. Essas propriedades expressam o
relacionamento de um sistema de provas com o conceito de validade
semântica do formalismo em questão. Para o caso da lógica
proposicional, podemos considerar que o conceito de validade é
exatamente o conceito de tautologia.

Desta forma, estamos interessados em saber:
\begin{enumerate}
    \item Se sempre que uma fórmula for dedutível no sistema de prova
      em questão, então esta é válida ---  Essa propriedade é
      conhecida como correção.
    \item Se toda fórmula válida possui uma dedução no sistema de
      prova --- Essa propriedade é conhecida como completude.
\end{enumerate}
A dedução natural é um sistema de
prova correto e completo para a lógica proposicional. Essas
propriedades são enunciadas a seguir.
\begin{Theorem}[Correção da dedução natural]
Seja $\alpha$ uma fórmula bem formada qualquer da lógica
proposicional. Se $\vdash\,\alpha$, então $\models\,\alpha$
\end{Theorem}
Por sua vez, a completude
especifica que toda tautologia é demonstrável utilizando o sistema de
dedução natural.
\begin{Theorem}[Completude da dedução natural]
Seja $\alpha$ uma fórmula bem formada qualquer da lógica
proposicional. Se $\models\,\alpha$, então $\vdash\,\alpha$.
\end{Theorem}
Note que representamos o fato de uma fórmula $\alpha$ ser uma
tautologia utilizando o conceito de consequência lógica,
$\models\,\alpha$.

Infelizmente, não possuímos as ferramentas necessárias para demonstrar
esses resultados. Para provar estes teoremas precisamos utilizar
indução matemática, que será abordada posteriormente.

\section{Notas Bibliográficas}

Existem diversos bons livros que abordam a lógica
proposicional. Citaremos apenas alguns:
