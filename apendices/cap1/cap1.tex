\section{Conceitos Preliminares}

\subsection{1.1.3 Exerc\'icios}

	\begin{enumerate}
	  \item

	  \begin{enumerate}
	    \item
	    \[
	    \begin{array}{ll}
	    & \llbracket \suc (\suc (\suc \zero)) \rrbracket\\
	    = & 1 + \llbracket \suc (\suc (zero)) \rrbracket \\
	    = & 1 + 1 + \llbracket \suc (zero) \rrbracket \\
	    = & 1 + 1 + 1 + \llbracket zero \rrbracket \\
            = & 3
	    \end{array}
	    \]

	    \item
	    \[
	    \begin{array}{llcl}
	    & \length\,(\zero ::\zero :: [\,])& & \\
	    = &1 + \length \,(\zero :: [\,]) & & \{\text{pela equa\c{c}\~ao }(2)\}\\
	    = & 1 + (1 + \length \,[\,])  & & \{\text{pela equa\c{c}\~ao }(2)\}\\
	    = &1 + (1 + 0) & & \{\text{pela equa\c{c}\~ao }(1)\}\\
	    = &2
	    \end{array}
	    \]

	    \item
	    \[
	    \begin{array}{llcl}
	    & (\zero :: (\suc\,\,\zero) :: [\,]) \text{ ++ } ((\suc\,\,\zero) :: \zero :: [\,]) & & \\
	    = & \zero :: (((\suc\,\,\zero) :: [\,]) \text{ ++ } ((\suc\,\,\zero) :: \zero::[\,])) & & \{\text{pela equa\c{c}\~ao }(2)\}\\
	    = & \zero :: ((\suc\,\,\zero) :: ( [\,] \text{ ++ } ((\suc\,\,\zero) :: \zero::[\,]))) & & \{\text{pela equa\c{c}\~ao }(2)\}\\
	    = & \zero :: ((\suc\,\,\zero) :: ((\suc\,\,\zero) :: \zero::[\,])) & & \{\text{pela equa\c{c}\~ao }(1)\}\\
	    \end{array}
	    \]
	  \end{enumerate}

	  \item  A defini\c{c}\~ao de concatena\c{c}\~ao de listas \'e uma fun\c{c}\~ao pois satisfaz os crit\'erios de totalidade e termina\c{c}\~ao, isso se dá devido ao fato de que, pelas regras $(1) \, e \, (2)$ definimos a sem\^antica recursivamente sobre a primeira lista cobrindo todas as possibilidades de constru\c{c}\~oes dos termos da sintaxe(sejam elas uma lista vazia ou uma lista n\~ao vazia), e tamb\'em porque a defini\c{c}\~ao e aplicada recursivamente sobre os subtermos da primeira lista e tendo como caso base a regra $(1)$\,.

	  \item
	  \[
	  \begin{array}{lcl}
	  \textit{nsubtermos } (\tconst\, n)  & = & 1 \\
	  \textit{nsubtermos } (\tplus\,e_1\,e_2) & = & 1 +
          (\textit{nsubtermos }\,e_1) + (\textit{nsubtermos }\,e_2) \\
	  \textit{nsubtermos }(\ttimes\,e_1\,e_2) & = & 1
          + (\textit{nsubtermos }\,e_1) + (\textit{nsubtermos }\,e_2) \\
	  \end{array}
	  \]

	  Temos a definição de uma função que retorna a quantidade de subtermos de uma expressão arimética de linguagem $\TExp$. Isso porque pelo princípio da totalidade a definição abrange todo possível valor de entrada na função, seja ele $(\tconst\, n)$, $(\tplus\,e_1\,e_2)$, $(\ttimes\,e_1\,e_2)$ e pelo principio da terminação a cada chamada recursiva da função o número de subtermos diminui tendendo a chegar no caso base que teria como entrada uma constante do tipo $(\tconst\, n)$


	  \item
	  \begin{enumerate}
	  \item Caso a definição de concatenação de listas fosse apresentada somente com a seguinte regra:
	  	  \[
	  	  \begin{array}{c}
	  	  	(x :: xs) \text{ ++ }  ys  =  x :: (xs\text{ ++ } ys)
	  	  \end{array}
	  	  \]
	  não poderíamos chamá-la de função devido ao fato de que esta definição não segue o princípio da totalidade, uma vez que não há procedimento para o caso de lista vazia.

	 \item Tomando o caso dos números naturais como base, não poderíamos criar uma função semântica somente com a seguinte regra:
	 	 \[
	 	 \begin{array}{c}
	 	 	\llbracket \suc\,\,n\rrbracket  =  \llbracket \suc\,n \rrbracket + 1
	 	\end{array}
	 	\]
	 já que ela não segue o principio da terminação sendo que a
         cada chamada recursiva é feita sobre um termo que não é menor
         que o parâmetro original de entrada.
	 \end{enumerate}

	\end{enumerate}

\ifbool{COQ}{
\subsection{1.2.3 Exerc\'icios}

	\begin{enumerate}
		\item

		\begin{lstlisting}
		Fixpoint evalExp (e: Exp) : nat :=
	     match e with
	      | Const n => NatSem n
	      | Plus (h)(t) => (evalExp h) + (evalExp t)
	      | Times (h)(t)=> (evalExp h) * (evalExp t)
	    end.
	    \end{lstlisting}
		Note que para esta fun\c{c}\~ao funcionar, se faz necess\'ario a declara\c{c}\~ao das seguintes defini\c{c}\~oes antes do corpo da fun\c{c}\~ao em quest\~ao.

		\begin{lstlisting}
		Inductive Nat : Set :=
		  | Zero : Nat
		  | Succ : Nat -> Nat.
		\end{lstlisting}

		\begin{lstlisting}
		Inductive Exp : Set :=
		  | Const : Nat -> Exp
		  | Plus : Exp -> Exp -> Exp
		  | Times : Exp -> Exp -> Exp .
		\end{lstlisting}

		\begin{lstlisting}
		Fixpoint NatSem (n : Nat) : nat :=
			match n with
			| Zero  => 0
			| Suc n' => 1 + NatSem n'
		end.
		\end{lstlisting}


		\item
		\begin{lstlisting}
		Inductive List (T : Set ) : Set :=
	 	  | nil : List T
		  | cons : T -> List T -> List T.
		\end{lstlisting}

		\begin{lstlisting}
		Definition tail {T} (l : List T) : List T :=
		  match l with
		  | nil => l
		  | cons h t => t
		  end.
		\end{lstlisting}

		\item

		\begin{lstlisting}
		Inductive List (T : Set ) : Set :=
		  | nil : List T
		  | cons : T -> List T -> List T.
		\end{lstlisting}


		\begin{lstlisting}
		Fixpoint last {T} (default : T) (l : List T) : T :=
		  match l with
		  | nil => default
		  | cons h t => last t
		  | cons t nil => t
		  end.
		\end{lstlisting}

	\end{enumerate}
}{}