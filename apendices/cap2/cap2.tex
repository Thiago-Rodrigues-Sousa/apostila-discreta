\section{L\'ogica Proposicional}

\subsection{2.2.1 Exerc\'icios}

	\begin{enumerate}
		\item 
			\begin{enumerate}
				\item
				 \[\begin{array}{ll}    
				 \text{Jo\~ao \'e pol\'itico} & \text{Conjun\c{c}\~ao (mas) }\\
				 \text{Jo\~ao \'e honesto} & \\
				 \end{array}
				 \]
				 
				 \item
				 \[\begin{array}{ll}    
 				 \text{Jo\~ao \'e honesto} & \text{Conjun\c{c}\~ao (mas), Nega\c{c}\~ao (n\~ao) }\\
 				 \text{O irm\~ao de Jo\~ao \'e honesto} & \\
 				 \end{array}
 				 \]
 				 
 				 \item
 				 \[\begin{array}{ll}    
  				 \text{Jo\~ao vir\'a a festa} & \text{Disjun\c{c}\~ao (ou), Conjun\c{c}\~ao (al\'em) }\\
  				 \text{A irm\~a de Jo\~ao vir\'a a festa.} & \\
  				 \text{A m\~ae de Jo\~ao vir\'a a festa.} & \\
  				 \end{array}
  				 \]
				 
				 \item
  				 \[\begin{array}{ll}    
   				 \text{A estrela do espet\'aculo canta.} & \text{Nega\c{c}\~ao (n\~ao) , Conjun\c{c}\~ao + Nega\c{c}\~ao (nem)} \\
   				 \text{A estrela do espet\'aculo dan\c{c}a.} & \\
   				 \text{A estrela do espet\'aculo representa.} & \\
   				 \end{array}
   				 \]
				 
				 \item
   				 \[\begin{array}{ll}    
  				 \text{O trem apita.} & \text{Condicional (Sempre que \ldots) } \\
  				 \text{Jo\~ao sai correndo.} & \\
  				 \end{array}
  				 \]
				 
				 \item
    			 \[\begin{array}{ll}    
   				 \text{Jo\~ao perde dinheiro no jogo.} & \text{Condicional (Caso \ldots), Nega\c{c}\~ao (n\~ao) } \\
   				 \text{Jo\~ao vai a festa.} & \\
   				 \end{array}
   				 \]
				  
				 \item
     			 \[\begin{array}{ll}    
  				 \text{Jo\~ao vai ser multado.} & \text{Condicional (a menos que \ldots), Disjun\c{c}\~ao (ou), Nega\c{c}\~ao (n\~ao) } \\
  				 \text{Jo\~ao diminui a velocidade.} & \\
  				 \text{A rodovia tem radar.} & \\
  				 \end{array}
  				 \]
				 				  
				\item
   			    \[\begin{array}{ll}    
				\text{Um n\'umero natural \'e primo.} & \text{Condicional (Uma condi\c{c}\~ao suficiente \ldots)} \\
				\text{Um n\'umero natural \'e \'impar.} & \\
				\end{array}
				\]
				 
				\item
			    \[\begin{array}{ll}    
				\text{Jo\~ao vai ao teatro.} & \text{Condicional (somente se \ldots)} \\
				\text{Uma com\'edia est\'a em cartaz.} & \\
				\end{array}
				\]
				  
				\item
			    \[\begin{array}{ll}    
				\text{Voc\^e \'e Brasileiro.} & \text{Condicional   (Se\ldots), Bicondicional + Nega\c{c}\~ao (a menos \ldots)} \\
				\text{Voc\^e gosta de futebol.} & \\
				\text{Voc\^e tor\c{c}e para o Tabajara.} & \\
				\text{Voc\^e tor\c{c}e para o \'Ibis.} & \\
				\end{array}
				\]  
				
				\item
			    \[\begin{array}{ll}    
				\text{A propina ser\'a paga.} & \text{Bicondicional (exatamente nas situa\c{c}\~oes em que \ldots)} \\
				\text{O deputado vota como instru\'ido por Jo\~ao.} & \\
				\end{array}
				\]
	 
			\end{enumerate}

	\end{enumerate}


	\subsection{2.2.3 Exerc\'icios}
	
		\begin{enumerate}
			\item
			\begin{enumerate}
							
			\item  $(A \lor B) \to C$
			\[\begin{array}{c|l} 		
			\text{Vari\'avel} & \text{Proposi\c{c}\~ao Simples} \\ \hline
   			$A$        & \text{Jane vence}\\ 
   			$B$        & \text{Jane perde}\\
   			$C$        & \text{Jane fica cansada} \\									
			\end{array}
			\]			
										
			\item  $A \lor B$
			\[\begin{array}{c|l} 		
			\text{Vari\'avel} & \text{Proposi\c{c}\~ao Simples} \\ \hline
   			$A$        & \text{Rosas s\~ao vermelhas}\\ 
   			$B$        & \text{Violetas s\~ao azuis}\\											
			\end{array}
			\]
		
			\item  $A \to B$
			\[\begin{array}{c|l} 
			\text{Vari\'avel} & \text{Proposi\c{c}\~ao Simples} \\ \hline
			$A$        & \text{Elefantes podem subir em \'arvores}\\ 
			$B$        & \text{3 \'e um n\'umero irracional}\\
			\end{array}
			\]
					
			\item $A \lor B$
			\[\begin{array}{c|l} 
			\text{Vari\'avel} & \text{Proposi\c{c}\~ao Simples} \\ \hline
			$A$        & \text{\'E proibido fumar cigarros}\\ 
			$B$        & \text{3 \'e um n\'umero irracional}\\
			\end{array}
			\]												
			
			\item $ \neg \,(A \to B )$
			\[\begin{array}{c|l} 
			\text{Vari\'avel} & \text{Proposi\c{c}\~ao Simples} \\ \hline
			$A$        & \pi > 0\\ 
			$B$        & \pi > 1\\
			\end{array}
			\]
			
			\item $ A \to B $
			\[\begin{array}{c|l} 
			\text{Vari\'avel} & \text{Proposi\c{c}\~ao Simples} \\ \hline
			$A$        & \text{As laranjas s\~ao amarelas}\\ 
			$B$        & \text{Os morangos s\~ao vermelhos}\\
			\end{array}
			\]			
						
			\item $ \neg ( A \to B) $
			\[\begin{array}{c|l} 
			\text{Vari\'avel} & \text{Proposi\c{c}\~ao Simples} \\ \hline
			$A$        & \text{Montreal \'e a capital do Canad\'a}\\ 
			$B$        & \text{A pr\'oxima copa ser\'a realizada no Brasil}\\
			\end{array}
			\]
	
			\end{enumerate}
	

	
			\item
				\begin{enumerate}
				
				\item $ A \land B $
				\[\begin{array}{c|l} 
				\text{Vari\'avel} & \text{Proposi\c{c}\~ao Simples} \\ \hline
				$A$        & \text{Jo\~ao \'e pol\'itico}\\ 
				$B$        & \text{Jo\~ao \'e honesto}\\
				\end{array}
				\]
					
				\item $ A \land \neg B $
				\[\begin{array}{c|l} 
				\text{Vari\'avel} & \text{Proposi\c{c}\~ao Simples} \\ \hline
				$A$        & \text{Jo\~ao \'e honesto}\\ 
				$B$        & \text{O irm\~ao de jo\~ao \'e honesto}\\
				\end{array}
				\]

				\item $ (A \lor B ) \land C $
				\[\begin{array}{c|l} 
				\text{Vari\'avel} & \text{Proposi\c{c}\~ao Simples} \\ \hline
				$A$        & \text{Jo\~ao vir\'a a festa}\\ 
				$B$        & \text{A irm\~a de Jo\~ao vir\'a a festa}\\
				$C$ 	   & \text{A m\~ae de Jo\~ao vir\'a a festa} \\
				\end{array}
				\]
				
				\item $ \neg A \land \neg B \land \neg C $
				\[\begin{array}{c|l} 
				\text{Vari\'avel} & \text{Proposi\c{c}\~ao Simples} \\ \hline
				$A$        & \text{A estrela do espet\'aculo canta}\\ 
				$B$        & \text{A estrela do espet\'aculo dan\c{c}a}\\
				$C$ 	   & \text{A estrela do espet\'aculo representa} \\
				\end{array}
				\]
				
				\item $ A \to B $
				\[\begin{array}{c|l} 
				\text{Vari\'avel} & \text{Proposi\c{c}\~ao Simples} \\ \hline
				$A$        & \text{O trem apita}\\ 
				$B$        & \text{Jo\~ao sai correndo}\\
				\end{array}
				\]
 
				\item $ \neg A \to B $
				\[\begin{array}{c|l} 
				\text{Vari\'avel} & \text{Proposi\c{c}\~ao Simples} \\ \hline
				$A$        & \text{Jo\~ao perde dinheiro no jogo}\\ 
				$B$        & \text{Jo\~ao vai a festa}\\
				\end{array}
				\] 
				
				\item $ A \to \neg (B \lor \neg C) $
				\[\begin{array}{c|l} 
				\text{Vari\'avel} & \text{Proposi\c{c}\~ao Simples} \\ \hline
				$A$        & \text{Jo\~ao vai ser multado}\\ 
				$B$        & \text{Jo\~ao diminui a velocidade}\\
				$C$        & \text{A rodovia tem radar}\\
				\end{array}
				\]
				
				\item $ A \to B $
				\[\begin{array}{c|l} 
				\text{Vari\'avel} & \text{Proposi\c{c}\~ao Simples} \\ \hline
				$A$        & \text{Um n\'umero natural \'e primo}\\ 
				$B$        & \text{Um n\'umero natural \'e \'impar}\\
				\end{array}
				\]		
						
				\item $ A \to B $
				\[\begin{array}{c|l} 
				\text{Vari\'avel} & \text{Proposi\c{c}\~ao Simples} \\ \hline
				$A$        & \text{Jo\~ao vai ao teatro}\\ 
				$B$        & \text{Uma com\'edia est\'a em cartaz}\\
				\end{array}
				\]		
									
				\item $ (A \to B) \to \neg (C \lor D) $
				\[\begin{array}{c|l} 
				\text{Vari\'avel} & \text{Proposi\c{c}\~ao Simples} \\ \hline
				$A$        & \text{Voc\^e \'e Brasileiro}\\ 
				$B$        & \text{Voc\^e gosta de futebol}\\
				$C$        & \text{Voc\^e tor\c{c}e para o Tabajara}\\ 
				$D$        & \text{Voc\^e tor\c{c}e para o \'Ibis}\\
				\end{array}
				\]		
								
				\item $ A \leftrightarrow B $
				\[\begin{array}{c|l} 
				\text{Vari\'avel} & \text{Proposi\c{c}\~ao Simples} \\ \hline
				$A$        & \text{A propina ser\'a paga}\\ 
				$B$        & \text{O deputado vota como instru\'ido por Jo\~ao}\\
				\end{array}
				\]		
							
				\end{enumerate}
	
		\end{enumerate}

\subsection{2.3.1 Exerc\'icios}

	\begin{enumerate}
	
			\item 
			\begin{enumerate}		
					\item 
					Pela regra 2 temos que as vari\'aveis $A$, $B$ e $C$ s\~ao f\'ormulas da l\'ogica, com isso pela regra 3\,-a temos que $\neg A$ tamb\'em faz parte do conjunto de f\'ormulas. Pela regra 3\,-b temos $\neg A \land B$. Novamente por 3\,-b podemos concluir $\neg A \land B \to C $. 
					
					\item
					Pela regra 2 temos que as vari\'aveis $A$, $B$ e $C$ s\~ao f\'ormulas da l\'ogica. Por 3\,-b temos $A \to B$ e $A \lor B$, com isso, novamente por 3\,-b conclu\'imos $A \lor B \to C$. Pela regra 3\,-a temos $\neg(A \lor B \to C)$ e por fim, pela regra 3\,b chegamos a $(A \to B) \land \neg(A \lor B \to C)$.
					
					\item
					Pelas regras 1 e 2 temos que as vari\'aveis $A$, $B$, $C$ e a constante $ \bot $ pertence ao conjunto de f\'ormulas da l\'ogica. Tomando como base a regra 3\,-b temos $B \to C$ que por sua vez podemos concluir $A \to B \to C $ e novamente por 3\,-b conclu\'imos  $A \to B \to C \leftrightarrow \bot $.
					
					\item
					Pela regra 2 temos que as vari\'aveis $A$, $B$ s\~ao f\'ormulas da l\'ogica, com isso pela regra 3\,-a temos que $\neg A$ tamb\'em faz parte do conjunto de f\'ormulas. Por 3\,-b pode-se construir $\neg A \to B $, novamente por 3\,-b $ A \land \neg A \to B $.
					
					\item
					Pela regra 2 temos que as vari\'aveis $A$, $B$ e $C$ s\~ao f\'ormulas da l\'ogica. Por 3\,-b temos $B \land C$ e com isso conclu\'imos $A \lor B \land C$.			
			\end{enumerate}
			
			\item
			\begin{enumerate}
				\item $(\neg A \land B) \to C$
				\item $(A \to B) \land \neg((A \lor B) \to C)$
				\item $(A \to (B \to C)) \leftrightarrow \bot$
				\item $(A \land \neg A) \to B$
				\item $A \lor (B \land C)$
			\end{enumerate}
			
			
			\item
				\begin{enumerate}
					\item $(A \lor B) \lor (C \lor D)$
					\item $A \to B \to (A \land B)$
					\item $\neg (A \lor B \land C)$
					\item $\neg (A \land (B \lor C))$
				\end{enumerate}
	
	\end{enumerate}

\subsection{2.4.10 Exerc\'icios}


	\begin{enumerate}
	
	
			\item
			\begin{enumerate}
			
			%LETRA A -------------------------------------------->
			\item Conting\^encia	
			\[\begin{array}{|c|c|c|c|c|c|}
			\hline
			 A & B & A \to B & A \lor B & \neg(A \lor B)& (A \to B) \leftrightarrow \neg(A \lor B) \\ \hline
			T & T & T & T & F & F \\
			T & F & F & T & F & F \\
			F & T & T & T & F & F \\
			F & F & T & F & T & T \\
			\hline
			\end{array}
			\]	
			
			
			%LETRA B -------------------------------------------->
			\item Conting\^encia		
			\[\begin{array}{|c|c|c|c|c|c|}
			\hline
			A & B & C & A \land B & (A \land B) \lor C & B \lor C \\ \hline
			T & T & T & T & T & T \\
			T & T & F & T & T & T \\
			T & F & T & F & T & T \\
			T & F & F & F & F & F \\
			F & T & T & F & T & T \\
			F & T & F & F & F & T \\
			F & F & T & F & T & T \\
			F & F & F & F & F & F \\
			\hline
			\end{array}
			\]
			
			\[\begin{array}{|c|c|}
			\hline
			A \land (B \lor C) & (A \land B) \lor C \to A \land (B \lor C)  \\ \hline
			T & T \\
			T & T \\
			T & T \\
			F & T \\
			F & F \\
			F & T \\
			F & F \\
			F & T \\
			\hline
			\end{array}
			\]
			
			%------------------------------------------------------>
			
		
			%LETRA C -------------------------------------------->
			\item Conting\^encia		
			\[\begin{array}{|c|c|c|c|c|}
			\hline
			A & B & \neg A & \neg B & \neg A \lor \neg B \\ \hline
			T & T & F & F & F \\
			T & F & F & T & T \\
			F & T & T & F & T \\
			F & F & T & T & T \\
			\hline
			\end{array}
			\]
			
			
			\[\begin{array}{|c|c|}
			\hline
			\neg(\neg A \lor \neg B) & A \land \neg(\neg A \lor \neg B) \\ \hline
			T & T \\
			F & F \\
			F & F \\
			F & F \\
			\hline
			\end{array}
			\]
			%------------------------------------------------------>
						
					
			%LETRA D -------------------------------------------->
			\item Conting\^encia
			\[\begin{array}{|c|c|c|c|c|}
			\hline
			A & B & A \land B & \neg A & A \land B \to \neg A \\ \hline
			T & T & T & F & F\\
			T & F & F & F & T\\
			F & T & F & T & T\\
			F & F & F & T & T\\
			\hline
			\end{array}
			\]
			
			
			%LETRA E -------------------------------------------->
			\item Tautologia
			\[\begin{array}{|c|c|c|c|c|c|}
			\hline
			A & B & C & A \to B & A \lor C & B \lor C \\ \hline
			T & T & T & T & T & T \\
			T & T & F & T & T & T \\
			T & F & T & F & T & T \\
			T & F & F & F & T & F \\
			F & T & T & T & T & T \\
			F & T & F & T & F & T \\
			F & F & T & T & T & T \\
			F & F & F & T & F & F \\
			\hline
			\end{array}
			\]
			
			\[\begin{array}{|c|c|}
			\hline
			(A \lor C) \to (B \lor C) & (A \to B) \to [(A \lor C) \to (B \lor C)] \\ \hline
			T & T \\
			T & T \\
			T & T \\
			F & T \\
			T & T \\
			T & T \\
			T & T \\
			T & T \\
			\hline
			\end{array}
			\]
			%------------------------------------------------------>
			
			%LETRA F -------------------------------------------->
			\item Tautologia
			\[\begin{array}{|c|c|c|c|}
			\hline
			A & B & B \to A & A \to (B \to A) \\ \hline
			T & T & T & T \\
			T & F & T & T \\
			F & T & F & T \\
			F & F & F & T \\
			\hline
			\end{array}
			\]
			
			%LETRA G -------------------------------------------->
			\item Contradi\c{c}\~ao
			\[\begin{array}{|c|c|c|c|c|}
			\hline
			A & B & A \land B & \neg B & \neg A \\ \hline
			T & T & T & F & F \\
			T & F & F & T & F \\
			F & T & F & F & T \\
			F & F & F & T & T \\
			\hline
			\end{array}
			\]
			
			\[\begin{array}{|c|c|}
			\hline
			(\neg B \lor \neg A) & (A \land B) \leftrightarrow (\neg B \lor \neg A) \\ \hline
			F & F \\
			T & F \\
			T & F \\
			T & F \\
			\hline
			\end{array}
			\]
			
			%------------------------------------------------------>			
			\end{enumerate}
			
			\item
			\begin{enumerate}
				
				%2a
				\item 
				\[\begin{array}{|c|c|c|c|}
				\hline
				P & Q & P \leftrightarrow Q & P \to Q \\ \hline
				T & T & T & T \\
				T & F & F & F \\
				F & T & F & T \\
				F & F & T & T \\
				\hline
				\end{array}
				\]
				
				\[\begin{array}{|c|c|c|c|}
				\hline
				\neg P & \neg Q & (\neg P \to \neg Q) & (P \to Q) \land (\neg P \to \neg Q) \\ \hline
				F & F & T & T \\
				F & T & T & T \\
				T & F & F & F \\
				T & T & T & T \\
				\hline
				\end{array}
				\]
				
				Pela an\'alise das tabelas anteriores podemos concluir que  $P \leftrightarrow Q$ e $(P \to Q) \land (\neg P \to \neg Q)$ n\~ao s\~ao f\'ormulas equivalentes da l\'ogica proposicional, pois apresentam resultados diferentes na solu\c{c}\~ao da tabela verdade.
				
				\item
				\[\begin{array}{|c|c|c|c|c|c|c|}
				\hline
				P & Q & \neg P & \neg Q & P \land \neg Q & \neg P \land Q & (P \land \neg Q) \lor (\neg P \land Q) \\ \hline
				T & T & F & F & F & F & F \\
				T & F & F & T & T & F & T \\
				F & T & T & F & F & T & T \\
				F & F & T & T & F & F & F \\
				
				
				\hline
				\end{array}
				\]
				
				\[\begin{array}{|c|c|c|c|}
				\hline
				P \lor Q & P \land Q & \neg (P \land Q)& (P \lor Q) \land \neg (P \land Q) \\ \hline
				T & T & F & F \\
				T & F & T & T \\
				T & F & T & T \\
				F & F & T & F \\
				\hline
				\end{array}
				\]
				
				$(P \land \neg Q) \lor (\neg P \land Q)$ e $(P \lor Q) \land \neg (P \land Q)$ s\~ao f\'ormulas equivalentes da l\'ogica proposicional, j\'a que apresentam resultados iguais pela tabela verdade.		
			\end{enumerate}
			
			\item 
			\begin{enumerate}
				
				\item Quando dizemos que uma f\'ormula e satisfaz\'ivel isso indica que pelo menos um resultado do conjunto de valores poss\'iveis daquela f\'ormula \'e verdadeiro. Supondo que a f\'ormula a ser analisada seja uma tautologia, a nega\c{c}\~ao da mesma nos dar\'a uma contradi\c{c}\~ao. Portanto se passarmos a nega\c{c}\~ao da f\'ormula para o algoritmo e este retornar falso conclu\'imos que a nega\c{c}\~ao n\~ao \'e satisfaz\'ivel e portanto uma contradi\c{c}\~ao, o que confirma a nossa hip\'otese de que a f\'ormula \'e uma tautologia.
				
				\item Se passarmos a f\'ormula para o algoritmo e este nos retornar falso indica que a f\'ormula n\~ao \'e satisfaz\'ivel e portanto n\~ao possui nenhum resultado no conjunto de valores poss\'iveis com o valor verdadeiro, com isso temos a defini\c{c}\~ao de contradi\c{c}\~ao.
				
			\end{enumerate}
			
			
	\end{enumerate}